\documentclass[11pt,a4paper]{article}
\usepackage[utf8]{inputenc}
\usepackage[T1]{fontenc}
\usepackage{lmodern}
\usepackage{amsmath,amssymb,amsthm}
\usepackage{geometry}
\usepackage{pgfplots,pgfplotstable}
\pgfplotsset{compat=1.18}
\geometry{margin=1in}

\title{Phase II: Hecke-Modulated Yukawa Ratios (No-Fit, CSV-Driven)}
\author{UBT Project}
\date{\today}

\newtheorem{lemma}{Lemma}

\begin{document}
\maketitle

\section*{Summary}
We extend the Phase I modular construction by introducing a \emph{minimal, no-fit} Hecke modulation factor
\begin{equation}
M_n(\tau;p) \;=\; \frac{\Theta(p\,n\tau)/\Theta(n\tau)}{\Theta(p\,\tau)/\Theta(\tau)},\qquad \Theta=\theta_2\theta_3\theta_4,\ \ \tau=i\kappa,
\end{equation}
which is normalized so that $M_1=1$ and has net weight~0. The Phase II candidate Yukawa ratios are
\begin{equation}
\frac{y_n}{y_1} \;=\; R_n(\tau)\,M_n(\tau;p),\qquad n\in\{3,9\},
\end{equation}
where $R_n$ are the Phase I modular ratios. No experimental inputs and no fitted parameters are used.
All values in this document are imported from CSV.

\section*{CSV Results}
\pgfplotstableread[col sep=comma]{../validation/yukawa_ratios_theta_hecke.csv}\Data
\pgfplotstabletypeset[
  columns/kappa/.style={column name={$\kappa$}},
  columns/p/.style={column name={$p$}},
  columns/R3/.style={column name={$R_3$}},
  columns/R9/.style={column name={$R_9$}},
  columns/M3/.style={column name={$M_3$}},
  columns/M9/.style={column name={$M_9$}},
  columns/Y3/.style={column name={$Y_3=R_3M_3$}},
  columns/Y9/.style={column name={$Y_9=R_9M_9$}},
  fixed,precision=10,
  every head row/.style={before row=\hline, after row=\hline},
  every last row/.style={after row=\hline}
]{\Data}

\section*{Remarks}
This ``minimal'' Hecke modulation is a mathematically motivated, weight-neutral ratio that introduces $p$-dependence while preserving $M_1=1$.
It is not the full Hecke operator action; Phase III will incorporate normalized Hecke eigenforms and selection rules from the $\Theta$-sector so that
the exponents arise from first principles. Until then, this CSV reports a clean, reproducible \emph{candidate} for the Yukawa hierarchy in the UBT framework.

\end{document}
