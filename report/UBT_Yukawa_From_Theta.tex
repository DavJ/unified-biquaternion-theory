\documentclass[11pt,a4paper]{article}
\usepackage[utf8]{inputenc}
\usepackage[T1]{fontenc}
\usepackage{lmodern}
\usepackage{amsmath,amssymb,amsthm}
\usepackage{geometry}
\usepackage{pgfplots,pgfplotstable}
\pgfplotsset{compat=1.18}
\geometry{margin=1in}

\title{Phase I: Yukawa Hierarchies from Jacobi Theta Invariants (No-Fit)}
\author{UBT Project}
\date{\today}

\newtheorem{lemma}{Lemma}
\newtheorem{theorem}{Theorem}

\begin{document}
\maketitle

\section*{Goal}
We derive a \emph{no-fit} modular construction for \emph{dimensionless} Yukawa \emph{ratios} for leptons
\((y_1,y_3,y_9)\) using Jacobi theta constants at a purely imaginary modulus \(\tau=i\kappa\) (with \(\kappa>0\)),
and Hecke-sector label \(p\). Absolute scales are \emph{not} introduced in Phase I; only ratios are reported in CSV and consumed by documents.

\section*{Setup}
Let \(q = e^{i\pi\tau}\). For \(\tau=i\kappa\) we have \(q=e^{-\pi\kappa}\in(0,1)\).
The Jacobi theta constants at \(z=0\) are (standard conventions):
\begin{align}
\theta_2(0|\tau) &= 2 \sum_{n=0}^{\infty} q^{(n+\frac12)^2},\quad
\theta_3(0|\tau) = 1 + 2\sum_{n=1}^{\infty} q^{n^2},\quad
\theta_4(0|\tau) = 1 + 2\sum_{n=1}^{\infty} (-1)^n q^{n^2}.
\end{align}
They carry modular weight \(1/2\). Hence the symmetric product \(\Theta(\tau):=\theta_2\theta_3\theta_4\) has weight \(3/2\).

\section*{Weight-0 Yukawa Ratios}
Define the winding-modified modulus \(\tau_n := n\tau\) for \(n\in\{1,3,9\}\).
Consider the \emph{dimensionless} modular ratio
\begin{equation}
R_n(\tau) \;:=\; \frac{\theta_2(0|\tau_n)\,\theta_3(0|\tau_n)\,\theta_4(0|\tau_n)}{\theta_2(0|\tau)\,\theta_3(0|\tau)\,\theta_4(0|\tau)}.
\end{equation}
\begin{lemma}[Weight neutrality]
Each \(R_n\) has net modular weight \(0\): numerator and denominator carry equal weight \(3/2\).
\end{lemma}
\begin{proof}
Immediate from the transformation law of theta constants and linearity of the winding map \(\tau\mapsto n\tau\).
\end{proof}

\noindent
\textbf{Phase I Ansatz (no-fit):} Yukawa ratios are generated by a Hecke-modulated modular functional of \(\tau\),
\begin{equation}
\frac{y_n}{y_1} \;=\; \mathcal{H}_p(\tau)^{\sigma_n}\, R_n(\tau)^{\gamma},\qquad n\in\{3,9\},
\end{equation}
with fixed rational exponents \(\sigma_n,\gamma\) determined by UBT field content. In Phase I we set \(\mathcal{H}_p\equiv 1\), \(\gamma=1\)
to isolate the \emph{pure modular} contribution and report \(\frac{y_n}{y_1}\approx R_n(\tau)\) as a \emph{theoretical candidate}—no parameters are fitted.

\section*{CSV-Driven Results}
We compute \(R_3(\tau),R_9(\tau)\) for user-specified \(\kappa\) and \(p\) and store them in
\texttt{validation/yukawa\_ratios\_theta.csv}. Documents import values \emph{only} from CSV.

\bigskip
\pgfplotstableread[col sep=comma]{../validation/yukawa_ratios_theta.csv}\Yukawa
\pgfplotstabletypeset[
  columns/kappa/.style={column name={$\kappa$}},
  columns/p/.style={column name={$p$}},
  columns/R3/.style={column name={$R_3(\tau)$}},
  columns/R9/.style={column name={$R_9(\tau)$}},
  fixed,precision=10,
  every head row/.style={before row=\hline, after row=\hline},
  every last row/.style={after row=\hline}
]{\Yukawa}

\section*{Outlook (Phase II)}
Hecke modulation \(\mathcal{H}_p(\tau)\) will be introduced via normalized Hecke eigenforms so that
\(\sigma_n\) emerges from selection rules in the \(\Theta\)-sector. Absolute masses then follow from the
UBT EWSB mechanism or an equivalent scale-setting relation. No experimental mass inputs will be used.

\end{document}
