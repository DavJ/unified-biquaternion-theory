
\subsubsection{Bekenstein Bridge (Variant B): UBT Complex-Time/KMS Sketch}

\paragraph{Goal.}
This subsection outlines a research program to \emph{derive} a Bekenstein-like
entropy-displacement relation from the complex-time structure $\tau=t+i\psi$
and the local near-horizon (Rindler/KMS) limit. This is a sketch, not a completed proof.

\paragraph{Step B1 (Local Rindlerization).}
In the vicinity of any spacetime point, consider an accelerated frame defining
an approximate local Rindler horizon. The corresponding thermal behavior in QFT
is encoded by a KMS condition, leading to an effective temperature scale.

\paragraph{Step B2 (Complex time and KMS periodicity).}
In thermal/KMS states, imaginary time is periodic with period
\[
  \beta = \frac{\hbar}{k_B T}.
\]
With $\tau=t+i\psi$, interpret $\psi$ as the imaginary-time coordinate and
posit that locally the dynamics admits an approximate KMS periodicity in $\psi$:
\[
  \psi \sim \psi + \beta.
\]
Combined with Unruh scaling in the Rindler frame, this yields
\[
  T \sim \frac{\hbar a}{2\pi k_B c},
\]
where the $2\pi$ factor is tied to the Euclidean continuation of boost/Rindler time.

\paragraph{Step B3 (UBT entropic functional as modular generator candidate).}
Use the complex entropic functional
\[
  \mathcal{S}(x) = k_B \operatorname{Tr}(\log\Theta(x)) = k_B \log\det\Theta(x),
\]
and define the physical entropic scalar by real projection:
\[
  S_\Theta(x) = 2\,\mathrm{Re}(\mathcal{S}(x)) = 2k_B \log|\det\Theta(x)|.
\]
Hypothesis: In the local Rindler limit, variations of $S_\Theta$ with respect to
transverse displacements $\Delta x$ encode information flow across the causal screen.

\paragraph{Step B4 (Entropy-displacement relation).}
Seek a derivation of
\[
  \Delta S_{\text{test}} \stackrel{?}{=} 2\pi k_B \frac{E}{\hbar c}\,\Delta x
\]
from the combination of:
(i) KMS periodicity in $\psi$,
(ii) local boost (Rindler) generators,
and (iii) coupling of test-body energy $E$ to $\Theta$-induced entropic gradients.

\paragraph{Step B5 (Chronofactor vs local spacetime).}
The global chronofactor is treated as a background scaling mode,
while the local Rindler/KMS construction is performed in the locally projected 4D frame.
The program is to show that the local $2\pi$ (boost) factor is universal, independent
of the global chronofactor, while the chronofactor sets cosmological background scales (e.g. $H$).

\paragraph{Deliverable of Variant B.}
Produce a minimal lemma of the form:
\emph{If UBT admits local KMS periodicity in $\psi$ in the accelerated frame, then a Bekenstein-like
entropy displacement law follows with the universal $2\pi$ factor.}
This requires specifying the precise test-body coupling and the definition of the causal screen
in UBT language.
