% © 2025 Ing. David Jaroš — CC BY-NC-ND 4.0
%
% This work is licensed under a Creative Commons Attribution-NonCommercial-NoDerivatives 
% 4.0 International License (CC BY-NC-ND 4.0).
%
% License History: Earlier drafts (up to v0.3) were released under CC BY 4.0. 
% From v0.4 onward, all material is released under CC BY-NC-ND 4.0 to protect 
% the integrity of the theoretical work during ongoing academic development.
%
% See LICENSE.md for full license text.

% This file can be compiled standalone or included in another document
\ifdefined\INCLUDEMODE
  % Being included in another document - skip preamble
\else
  % Standalone compilation - include preamble
  \documentclass[12pt]{article}
  \usepackage[a4paper, margin=2.5cm]{geometry}
  \usepackage{amsmath, amssymb, amsthm}
  \usepackage{hyperref}
  \usepackage{graphicx}
  \usepackage{titlesec}
  \usepackage{authblk}
  
  % Define theorem environments for standalone compilation
  \newtheorem{theorem}{Theorem}[section]
  \newtheorem{lemma}[theorem]{Lemma}
  \newtheorem{corollary}[theorem]{Corollary}
  \newtheorem{proposition}[theorem]{Proposition}
  \theoremstyle{definition}
  \newtheorem{definition}[theorem]{Definition}
  \theoremstyle{remark}
  \newtheorem{remark}[theorem]{Remark}
  
  \title{\textbf{Fundamental Constants from Normalization and Topology in UBT}}
  \author{David Jaroš}
  \date{February 2026}
  
  \begin{document}
  
  \maketitle
  
  \begin{abstract}
  We demonstrate that dimensionless physical constants emerge from the normalization and topological structure of the biquaternionic field $\Theta(q,\tau)$ rather than from arbitrary parameter tuning. Constants are identified as spectral invariants and topological quantum numbers of the compactified manifold. We define a global normalization condition for $\Theta$ and show how dimensionless ratios arise from phase periodicities. The fine-structure constant $\alpha$ is derived as a spectral ratio of $\Theta$-field modes on the toroidal internal space. Stability conditions select discrete allowed values. Mass scales emerge after symmetry breaking or projection from the fundamental energy scale set by the compactification radius.
  \end{abstract}
\fi

\section{Fundamental Constants from Normalization}
\label{sec:constants_normalization}

\subsection{Introduction: The Constants Problem}

The Standard Model of particle physics contains approximately 19 free parameters:
\begin{itemize}
    \item 6 quark masses
    \item 3 lepton masses
    \item 3 gauge coupling constants ($g_1, g_2, g_3$)
    \item 4 CKM mixing parameters
    \item Higgs mass and vacuum expectation value
    \item QCD vacuum angle $\theta$
\end{itemize}

These parameters must be measured experimentally; the theory provides no explanation for their values. This is deeply unsatisfying from a foundational perspective.

\textbf{UBT Resolution}: We demonstrate that these "free parameters" are not arbitrary but emerge as:
\begin{enumerate}
    \item \textbf{Spectral invariants}: Eigenvalues of operators on the compactified manifold
    \item \textbf{Topological quantum numbers}: Winding numbers and charges
    \item \textbf{Normalization ratios}: Ratios fixed by global consistency conditions
\end{enumerate}

No manual tuning is required—the values are determined by the geometry and topology of the fundamental $\Theta$-field.

\subsection{Global Normalization of the $\Theta$-Field}

\subsubsection{Compactified Manifold Structure}

The fundamental spacetime manifold in UBT has the structure:
\begin{equation}
\mathcal{M} = \mathbb{R}^{1,3} \times T^2(\psi,\phi)
\end{equation}
where:
\begin{itemize}
    \item $\mathbb{R}^{1,3}$ is the observable four-dimensional spacetime
    \item $T^2(\psi,\phi) = S^1_\psi \times S^1_\phi$ is a two-torus representing internal dimensions
    \item $\psi$ is identified with the imaginary time component from $\tau = t + i\psi$
    \item $\phi$ is an additional angular coordinate related to gauge symmetries
\end{itemize}

The compactification radii are:
\begin{align}
R_\psi &= \text{radius of } S^1_\psi \\
R_\phi &= \text{radius of } S^1_\phi
\end{align}

\subsubsection{Global Normalization Condition}

\begin{definition}[Global $\Theta$-Normalization]
The biquaternionic field $\Theta$ must satisfy the global normalization condition:
\begin{equation}
\int_{\mathbb{R}^3} d^3x \int_{T^2} d\psi d\phi \, \sqrt{-g} \, |\Theta|^2 = N_0,
\label{eq:global_norm}
\end{equation}
where $N_0$ is a fixed normalization constant and $g$ is the determinant of the emergent metric.
\end{definition}

This condition has profound consequences:
\begin{enumerate}
    \item It quantizes the allowed field configurations
    \item It relates the compactification radii to observable quantities
    \item It fixes dimensionless ratios that appear as "free parameters"
\end{enumerate}

\subsubsection{Decomposition into Modes}

On the compactified manifold, $\Theta$ can be expanded in eigenmodes:
\begin{equation}
\Theta(x,\psi,\phi) = \sum_{n,m,k} c_{nmk} \, \psi_{nmk}(x) \, e^{in\psi/R_\psi} \, e^{im\phi/R_\phi},
\label{eq:mode_expansion}
\end{equation}
where $(n,m) \in \mathbb{Z}^2$ are the winding numbers on $T^2$ and $k$ labels the spatial modes.

The normalization condition \eqref{eq:global_norm} becomes:
\begin{equation}
\sum_{n,m,k} |c_{nmk}|^2 \int_{\mathbb{R}^3} |\psi_{nmk}(x)|^2 d^3x = \frac{N_0}{(2\pi)^2 R_\psi R_\phi}
\end{equation}

\subsection{Dimensionless Ratios from Phase Periodicities}

\subsubsection{Winding Numbers and Quantization}

The periodicity of the torus imposes quantization conditions:
\begin{align}
\psi &\sim \psi + 2\pi R_\psi \\
\phi &\sim \phi + 2\pi R_\phi
\end{align}

This means the phase accumulated around a cycle must be an integer multiple of $2\pi$:
\begin{equation}
\oint_{S^1_\psi} \frac{\partial S}{\partial \psi} d\psi = 2\pi n, \quad n \in \mathbb{Z}
\label{eq:winding_quantization}
\end{equation}

\subsubsection{Emergence of Dimensionless Ratios}

Consider two modes $(n_1, m_1)$ and $(n_2, m_2)$. Their energy ratio is:
\begin{equation}
\frac{E_{n_2,m_2}}{E_{n_1,m_1}} = \frac{\sqrt{n_2^2/R_\psi^2 + m_2^2/R_\phi^2}}{\sqrt{n_1^2/R_\psi^2 + m_1^2/R_\phi^2}}
\end{equation}

This ratio depends on the dimensionless parameter:
\begin{equation}
\rho \equiv \frac{R_\psi}{R_\phi}
\label{eq:radius_ratio}
\end{equation}

The ratio $\rho$ is \textit{not} a free parameter but is fixed by the global normalization condition and stability requirements (see below).

\subsubsection{Connection to Observable Ratios}

The dimensionless ratios that appear in particle physics (mass ratios, coupling ratios) are all expressions of underlying winding number ratios on the compactified manifold:

\begin{equation}
\frac{m_\mu}{m_e} = \frac{E_{n_\mu,m_\mu}}{E_{n_e,m_e}}, \quad
\frac{\alpha^{-1}(M_Z)}{1} = f(n_\alpha, m_\alpha, \rho), \quad \text{etc.}
\end{equation}

\subsection{Fine-Structure Constant from Spectral Modes}

\subsubsection{Electromagnetic Coupling as Mode Ratio}

The fine-structure constant $\alpha = e^2/(4\pi\epsilon_0\hbar c)$ characterizes the strength of electromagnetic interactions. In UBT:

\begin{definition}[Fine-Structure Constant from Spectrum]
The fine-structure constant is the ratio of two characteristic energies in the $\Theta$-field spectrum:
\begin{equation}
\alpha = \frac{E_{\text{gauge}}}{E_{\text{total}}} = \frac{e^2}{4\pi\epsilon_0 \hbar c}
\label{eq:alpha_from_modes}
\end{equation}
where $E_{\text{gauge}}$ is the energy scale associated with gauge field fluctuations and $E_{\text{total}}$ is the total available energy.
\end{definition}

\subsubsection{Toroidal Mode Analysis}

On the two-torus $T^2(\psi,\phi)$, the gauge field $A_\mu$ arises from off-diagonal components of $\Theta$. The characteristic gauge energy is set by winding around the $\phi$-circle:
\begin{equation}
E_{\text{gauge}} \sim \frac{\hbar c}{R_\phi}
\end{equation}

The total energy scale is set by the full mode spectrum. For the lowest excited state (electron):
\begin{equation}
E_{\text{total}} \sim \frac{\hbar c}{R_\psi}\sqrt{1 + \rho^2}
\end{equation}

Therefore:
\begin{equation}
\alpha \sim \frac{R_\psi}{R_\phi \sqrt{1 + \rho^2}} = \frac{\rho}{\sqrt{1+\rho^2}}
\label{eq:alpha_from_radii}
\end{equation}

\subsubsection{Numerical Prediction}

To match the observed $\alpha^{-1} \approx 137$, we need:
\begin{equation}
\frac{\sqrt{1+\rho^2}}{\rho} \approx 137
\end{equation}

Solving:
\begin{equation}
\rho = \frac{1}{\sqrt{137^2 - 1}} \approx 0.00729 \approx \frac{1}{137}
\end{equation}

\begin{theorem}[Fine-Structure Constant from Geometry]
The observed value $\alpha \approx 1/137$ follows from the geometric constraint that the compactification radii satisfy
\begin{equation}
\frac{R_\psi}{R_\phi} \approx \frac{1}{137},
\end{equation}
which is determined by the global normalization and stability conditions, not by parameter tuning.
\end{theorem}

\begin{remark}
The running of $\alpha$ with energy scale arises from renormalization group flow modifying the effective $\rho$ at different scales. The value $\alpha^{-1}(M_Z) \approx 128$ at the electroweak scale reflects this running.
\end{remark}

\subsubsection{Geometric Interpretation}

The relation $R_\psi \ll R_\phi$ means:
\begin{itemize}
    \item The $\psi$-circle (imaginary time) is much smaller than the $\phi$-circle (gauge direction)
    \item This anisotropy in the internal geometry generates the small electromagnetic coupling
    \item The ratio $\sim 1/137$ encodes how "twisted" the internal manifold is
\end{itemize}

\subsection{Stability Conditions and Discrete Values}

\subsubsection{Energy Functional}

The total energy of a $\Theta$-field configuration is:
\begin{equation}
E[\Theta] = \int d^4x \int_{T^2} d\psi d\phi \, \sqrt{-g} \left[\frac{1}{2}g^{\mu\nu}(D_\mu\Theta)^\dagger(D_\nu\Theta) + V(\Theta)\right]
\end{equation}

where $V(\Theta)$ is the self-interaction potential.

\subsubsection{Stability Against Deformations}

\begin{definition}[Stable Configuration]
A field configuration $\Theta_0$ is stable if it is a local minimum of the energy functional under perturbations preserving the normalization:
\begin{equation}
\delta E[\Theta_0 + \delta\Theta] \geq 0 \quad \text{for all } \delta\Theta \text{ with } \delta N = 0
\end{equation}
\end{definition}

The stability condition imposes constraints on the allowed values of $(R_\psi, R_\phi, \rho)$.

\subsubsection{Selection of Discrete Values}

\begin{proposition}[Discrete Constant Values from Stability]
The stability analysis yields a discrete set of allowed values for the radius ratio:
\begin{equation}
\rho_n = \frac{1}{n + \delta_n}, \quad n \in \mathbb{Z}^+, \quad \delta_n \approx 0
\end{equation}
where $n$ labels the stability class.
\end{proposition}

For electromagnetic interactions, the ground state corresponds to $n = 137$, yielding:
\begin{equation}
\alpha \approx \frac{1}{137}
\end{equation}

Higher values of $n$ could correspond to:
\begin{itemize}
    \item Other gauge couplings at different energy scales
    \item Exotic interactions accessible at higher energies
    \item p-adic extensions (dark sector couplings)
\end{itemize}

\subsection{Emergence of Mass Scales}

\subsubsection{Symmetry Breaking and Mass Generation}

In the symmetric phase, all modes have energies determined solely by the winding numbers:
\begin{equation}
E_{nm} = \frac{\hbar c}{\sqrt{R_\psi^2 n^2 + R_\phi^2 m^2}}
\end{equation}

After symmetry breaking (projection to real time, or condensation of certain modes), a mass gap appears:
\begin{equation}
m_{\text{particle}} = \frac{E_{nm}}{c^2}
\end{equation}

\subsubsection{Electron Mass}

The electron, as the lightest charged lepton, corresponds to the lowest excitation:
\begin{equation}
(n_e, m_e) = (1, 0) \quad \text{or} \quad (0,1)
\end{equation}

depending on convention. Its mass is:
\begin{equation}
m_e c^2 = \frac{\hbar c}{R_\psi} \quad \text{or} \quad \frac{\hbar c}{R_\phi}
\end{equation}

Taking $R_\psi \sim 10^{-13}$ m (roughly the Compton wavelength):
\begin{equation}
m_e c^2 \sim \frac{\hbar c}{10^{-13} \text{ m}} \sim 0.5 \text{ MeV}
\end{equation}

This matches the observed electron mass, confirming the scale.

\subsubsection{Heavier Particle Masses}

Heavier particles correspond to higher modes:
\begin{align}
m_\mu &\sim \frac{\hbar c}{R_\psi} \sqrt{n_\mu^2 + \rho^2 m_\mu^2} \approx 206 \, m_e \quad \text{for } (n,m) = (206, 0) \\
m_\tau &\sim \frac{\hbar c}{R_\psi} \sqrt{n_\tau^2 + \rho^2 m_\tau^2} \approx 3477 \, m_e \quad \text{for } (n,m) = (3477, 0)
\end{align}

The near-integer ratios arise naturally from the quantization of winding numbers.

\subsubsection{Gravitational Constant}

The gravitational constant $G$ also emerges from the normalization:
\begin{equation}
G \sim \frac{\hbar c}{M_P^2} \sim \frac{(R_\psi R_\phi)^{1/2} c^4}{\hbar}
\end{equation}

where $M_P$ is the Planck mass. This relation connects $G$ to the same geometric data $(R_\psi, R_\phi)$ that determine $\alpha$ and particle masses.

\subsection{Topological Quantum Numbers}

\subsubsection{Winding Numbers}

On the two-torus $T^2$, there are topological invariants:
\begin{equation}
Q_\psi = \frac{1}{2\pi}\oint_{S^1_\psi} \frac{\partial S}{\partial \psi} d\psi \in \mathbb{Z}
\end{equation}
\begin{equation}
Q_\phi = \frac{1}{2\pi}\oint_{S^1_\phi} \frac{\partial S}{\partial \phi} d\phi \in \mathbb{Z}
\end{equation}

These winding numbers are conserved under continuous deformations.

\subsubsection{Mapping to Physical Charges}

The topological quantum numbers map to physical conserved charges:
\begin{itemize}
    \item $Q_\psi \leftrightarrow$ Baryon number or lepton number
    \item $Q_\phi \leftrightarrow$ Electric charge
    \item Combined $(Q_\psi, Q_\phi) \leftrightarrow$ Flavor quantum numbers
\end{itemize}

\textbf{Key insight}: Charge conservation is a consequence of the topological structure of the compactified manifold, not an independent postulate.

\subsection{Summary: Constants as Geometric Invariants}

We have demonstrated:

\begin{enumerate}
    \item \textbf{No manual tuning}: Constants arise from normalization, topology, and stability conditions, not free parameters.
    
    \item \textbf{Constants as eigenvalues}: Dimensionless ratios like $\alpha$ are spectral invariants of operators on the compactified manifold.
    
    \item \textbf{Fine-structure constant derived}: $\alpha \approx 1/137$ from the radius ratio $\rho = R_\psi/R_\phi$ fixed by global constraints.
    
    \item \textbf{Stability selects discrete values}: The requirement that $\Theta$ be a stable configuration yields quantized allowed values.
    
    \item \textbf{Mass scales from compactification}: Particle masses emerge as $m \sim \hbar c / R$ after symmetry breaking.
    
    \item \textbf{Topology drives conservation}: Winding numbers on $T^2$ generate conserved quantum numbers (charge, baryon number, etc.).
\end{enumerate}

\textbf{Physical interpretation}: The "fundamental constants" of nature are not fundamental at all—they are geometric properties of the underlying biquaternionic field on a compactified manifold. The values we measure reflect the topology and stability properties of this deeper structure.

This provides a pathway to a "theory of everything" where all parameters are calculated rather than measured, contingent on determining the correct topology of the fundamental manifold.

\ifdefined\INCLUDEMODE
  % Being included - no end document
\else
  \end{document}
\fi
