% ============== Two-Loop Beta Function and Diagrammatics ===============
\section{Two-Loop Beta Function in the CT Scheme}
\label{sec:beta-ct-two-loop}

\subsection{Diagram classes}
The contributing two-loop topologies to the charge renormalization in CT are:
(i) vacuum polarization (sunset), (ii) double-bubble, (iii) nested fermion self-energies,
and (iv) vertex corrections satisfying Ward identities.

\subsection{Integral representation}
Each topology is written as a sum of dimensionally-regularized integrals with CT propagators.
A canonical basis of master integrals is chosen via IBP relations; divergences are isolated
in \(1/\epsilon^k\). Finite remainders define \(\mathcal R_{\mathrm{UBT}}\) after subtractions.

\subsection{Beta function}
Let \(Z_e\) be the charge renormalization. In CT-\(\overline{\mathrm{MS}}\),
\[
\beta(e) = \mu\frac{de}{d\mu} = e\,\left[ -\frac{1}{2}\mu \frac{d}{d\mu}\log Z_e^2 \right]_{\text{finite}}.
\]
The two-loop term is obtained from the \(1/\epsilon\) single poles of the two-loop graphs
and counterterm insertions. Gauge-parameter cancellation follows from Ward identities.

\subsection{Consistency checks}
(i) In the QED/real-time limit the known small two-loop value is recovered.
(ii) \(\mu\)-dependence cancels in the observable \(B\) at the stated order.
% =======================================================================
