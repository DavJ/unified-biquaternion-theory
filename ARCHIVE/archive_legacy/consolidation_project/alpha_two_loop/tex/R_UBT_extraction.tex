% =================== Extraction of R_UBT (Two-Loop) ====================
\section{Extraction of \texorpdfstring{$\mathcal R_{\mathrm{UBT}}$}{R_UBT}}
\label{sec:rubt-extraction}

\paragraph{Note:} This section provides technical details on the extraction of 
$\mathcal{R}_{\mathrm{UBT}}$ from two-loop vacuum polarization calculations. For the main 
theoretical result establishing $\mathcal{R}_{\mathrm{UBT}} = 1$ under standard assumptions, 
see Appendix~CT (Section~\ref{app:ct-baseline-R1}).

\subsection{Definition}
We define \(\mathcal R_{\mathrm{UBT}}\) as the finite, renormalized ratio
\[
\mathcal R_{\mathrm{UBT}} := \frac{\Pi^{(2)}_{\text{CT, finite}}(q^2\!=\!0;\mu)}{\Pi^{(2)}_{\text{QED, finite}}(q^2\!=\!0;\mu)}\times \mathcal N_{\text{CT}\to\text{QED}},
\]
with \(\Pi^{(2)}\) the two-loop vacuum polarization scalar function and \(\mathcal N_{\text{CT}\to\text{QED}}\)
a normalization ensuring the QED/real-time limit is unity. Alternative equivalent definitions
via charge renormalization yield the same factor by Ward identities.

\subsection{Baseline result: $\mathcal{R}_{\mathrm{UBT}} = 1$}
Under the standard assumptions detailed in Appendix~CT (Section~\ref{app:ct-baseline-R1}):
\begin{itemize}
  \item \textbf{A1 (Geometry fixed)}: $N_{\mathrm{eff}}$ and $R_\psi$ determined without tunable parameters
  \item \textbf{A2 (CT scheme)}: Dimensional regularization, Ward identities, QED limit
  \item \textbf{A3 (Observable)}: Thomson-limit extraction, gauge independence
\end{itemize}
we have the rigorous result (Theorem~\ref{thm:RUBT-equals-one}):
\[
\boxed{\mathcal{R}_{\mathrm{UBT}} = 1}.
\]
This is the \textbf{fit-free baseline}. Any deviation requires explicit calculation of CT-specific 
effects beyond the standard assumptions.

\subsection{Gauge and scheme independence at fixed order}
At the stated order, gauge-parameter (\(\xi\)) drops out of the extracted \(\mathcal R_{\mathrm{UBT}}\),
and residual \(\mu\)-dependence cancels within \(B\). Proof: (a) \(Z_1=Z_2\) in CT; (b) longitudinal
photon parts vanish; (c) counterterms remove scheme artifacts up to finite reparametrizations
that cancel in \(B\).

\subsection{QED limit}
For \(\psi\to 0\), \(\mathcal R_{\mathrm{UBT}}\to 1\). Deviations quantify genuine CT effects.

\subsection{Baseline value at two loops}
By CT reduction to real-time QED, Ward identities, and Thomson-limit normalization (A1–A3),
the finite renormalized ratio at \(q^2=0\) equals unity:
\[
\boxed{\ \mathcal R_{\mathrm{UBT}}=1\ }\qquad\text{(Appendix~\ref{app:ct-baseline-R1}).}
\]
Any claim of \(\mathcal R_{\mathrm{UBT}}\neq 1\) requires an explicit, gauge-invariant CT two-loop computation that reproduces QED in the real-time limit.
% =======================================================================
