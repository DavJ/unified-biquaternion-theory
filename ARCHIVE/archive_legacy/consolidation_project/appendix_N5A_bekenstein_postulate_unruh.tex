
\subsubsection{Bekenstein Bridge (Variant A): Postulate + Unruh Closure}

\paragraph{Postulate (Bekenstein entropy change for a test body).}
For a test particle of energy $E$ displaced by $\Delta x$ relative to a local
causal screen, assume the saturated Bekenstein form:
\[
  \Delta S_{\text{test}} = 2\pi k_B \frac{E}{\hbar c}\,\Delta x.
\]
In the non-relativistic limit $E \approx mc^2$:
\[
  \Delta S_{\text{test}} = 2\pi k_B \frac{mc}{\hbar}\,\Delta x.
\]
This is introduced as a compatibility assumption with gravitational thermodynamics.

\paragraph{Unruh temperature and Newton's second law.}
Using the Unruh temperature
\[
  T = \frac{\hbar a}{2\pi k_B c},
\]
and the entropic-force relation
\[
  F\,\Delta x = T\,\Delta S_{\text{test}},
\]
yields
\[
  F = ma.
\]

\paragraph{Connection to the UBT entropic potential.}
Define the UBT background entropic scalar
\[
  S_\Theta(x) = 2k_B \log|\det\Theta(x)|.
\]
In the weak-field regime (where $\lambda S_\Theta \ll 1$) with an exponential map
\[
  g_{tt}=-\exp(2\lambda S_\Theta)\approx -(1+2\lambda S_\Theta),
\]
identify the Newtonian potential as
\[
  \Phi(x)=\lambda c^2 S_\Theta(x).
\]
Consistency with the standard Newtonian limit implies that for a spherical source
\[
  \Phi(r)=-\frac{GM}{r}
  \quad\Rightarrow\quad
  S_\Theta(r)=-\frac{GM}{\lambda c^2}\frac{1}{r}.
\]

\paragraph{Scope.}
Variant A closes the entropic route to Newton's second law and establishes
an explicit matching condition for the gravitational potential. A derivation
of the Bekenstein form from UBT is not claimed here.
