% ================== Appendix P7: Two-Loop Roadmap & Normalization ==================
% Requires: amsmath, amssymb, amsthm packages already loaded by the main doc.

\section{Two-Loop Roadmap and Normalization for \texorpdfstring{$\mathcal R_{\mathrm{UBT}}$}{R_UBT}}
\label{app:two-loop-roadmap}

\subsection*{Purpose}
This appendix fixes the \emph{definition}, \emph{normalization}, and \emph{checks} for the
two-loop factor $\mathcal R_{\mathrm{UBT}}$ that modifies the one-loop backbone of the $B$-formula.
It also lists the exact diagram topologies required at two loops and the Ward-identity constraints
that must hold in the complex-time scheme.

\subsection*{Definition (perturbative series)}
We define $\mathcal R_{\mathrm{UBT}}$ at a renormalization scale $\mu$ as a bona fide perturbative series:
\begin{equation}
\boxed{
\mathcal R_{\mathrm{UBT}}(\mu)
= 1 + c_1\!\left(\frac{\alpha(\mu)}{\pi}\right)
      + c_2\!\left(\frac{\alpha(\mu)}{\pi}\right)^{\!2}
      + \mathcal O(\alpha^3)\!,
}
\label{eq:RUBT-series}
\end{equation}
where $c_1,c_2$ are scheme-dependent coefficients determined by the complex-time prescription
and the UBT Lagrangian choices. The one-loop backbone for $B$ then reads
\begin{equation}
\boxed{
B(\mu) \;=\; \frac{2\pi N_{\mathrm{eff}}}{3\,R_\psi}\;
\mathcal R_{\mathrm{UBT}}(\mu).
}
\end{equation}
Any earlier use of a scale-independent constant (e.g.\ $\mathcal R_{\mathrm{UBT}}\!\approx\!1.84$) should be
interpreted as a \emph{numerical placeholder at some} $\mu$, not as a universal two-loop correction.

\subsection*{Two-loop topologies (must include)}
All computations use dimensional regularization in $d=4-2\epsilon$:
\begin{itemize}
  \item \textbf{Vacuum polarization (sunset)}: photon self-energy with an internal fermion loop insertion (Källén--Sabry type).
  \item \textbf{Double-bubble}: two disjoint fermion bubbles connected by photon propagators.
  \item \textbf{Vertex two-loop corrections}: required to verify Ward--Takahashi identities ($Z_1=Z_2$).
  \item \textbf{Fermion self-energy (nested)}: contributes via counterterms to charge renormalization.
  \item \textbf{Counterterm insertions}: close the renormalization at two loops consistently with the scheme.
\end{itemize}

\subsection*{Ward identities and charge renormalization}
Gauge invariance enforces the Ward--Takahashi identity $Z_1=Z_2$, so the renormalization of the charge
is governed by the photon field renormalization $Z_3$ extracted from the vacuum polarization.
Any complex-time deformation must preserve these identities; otherwise $\mathcal R_{\mathrm{UBT}}$ is not physically meaningful.

\subsection*{QED-limit normalization (baseline check)}
In the limit that the complex-time prescription reduces to standard QED,
the running of $\alpha$ (mass-independent scheme) obeys the universal two-loop form
\begin{equation}
\frac{d\alpha}{d\ln\mu}
= \frac{2N_f}{3\pi}\,\alpha^2
+ \frac{N_f}{2\pi^2}\,\alpha^3
+ \mathcal O(\alpha^4),
\label{eq:qed-two-loop}
\end{equation}
so the UBT-specific coefficients must reduce as
\begin{equation}
\boxed{
\lim_{\text{UBT}\to\text{QED}} c_1 = 0,
\qquad
\lim_{\text{UBT}\to\text{QED}} c_2 = 0.
}
\end{equation}
Equivalently, in the QED limit $\mathcal R_{\mathrm{UBT}}(\mu)\to 1 + \mathcal O(\alpha^3)$ for the $B$-backbone,
since the prefactor $B$ has no counterpart in pure QED.

\subsection*{Deliverables and pass/fail criteria}
\begin{enumerate}
  \item \textbf{Diagrammatics $\to$ master integrals}: List generated two-loop graphs, perform IBP reduction to master integrals, and document the complex-time contour rules.
  \item \textbf{Renormalization}: Provide $Z_3$ and verify $Z_1=Z_2$ at two loops in the complex-time scheme.
  \item \textbf{Extraction of $c_2$}: Present $c_2$ (and $c_1$ if nonzero) explicitly, with intermediate checks.
  \item \textbf{QED-limit test}: Reproduce Eq.~\eqref{eq:qed-two-loop} and show $c_1,c_2\to 0$ in that limit.
\end{enumerate}
A result that returns a \emph{scale-independent} multiplier for $\mathcal R_{\mathrm{UBT}}$ is not acceptable as a genuine two-loop effect; two-loop contributions scale as powers of $\alpha(\mu)$.
% ================== End Appendix P7 ==================
