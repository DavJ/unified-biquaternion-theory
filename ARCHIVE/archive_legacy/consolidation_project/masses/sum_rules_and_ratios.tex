% ================== Sum Rules and Ratios for Fermion Masses ==================
% VERSION: v2.0 - Falsifiable Predictions
% AUTHOR: UBT Team
% PURPOSE: Testable mass sum rules as falsifiable predictions
%
% DEPENDENCIES:
% Requires: \usepackage{amsmath,amssymb,amsthm}
% References: yukawa_in_HC.tex, ct_two_loop_renorm.tex

\section{Sum Rules and Mass Ratios}
\label{app:mass-sum-rules}

\subsection{Overview}

A key testable prediction from UBT would be relations among fermion masses that are \textbf{independent of the overall mass scale}. Such sum rules would:
\begin{itemize}
  \item Be falsifiable by experiment (measure mass ratios to high precision).
  \item Not require knowledge of the absolute scale (which may depend on Higgs vev or other inputs).
  \item Provide strong constraints on the Yukawa structure.
\end{itemize}

This appendix proposes concrete sum rules with numerical intervals and error budgeting for falsifiability.

\subsection{Lepton Mass Ratios: Proposed Sum Rules}

Consider the charged lepton masses \(m_e, m_\mu, m_\tau\). In the Standard Model, these are independent parameters (three Yukawa eigenvalues). From the constrained Yukawa matrix \(Y_\ell = a \cdot \mathbb{1} + b \cdot \Sigma + c \cdot T\) (Lemma~\ref{lem:yukawa-texture}), we derive:

\paragraph{Sum Rule 1: Ratio of successive generations.}
\[
\frac{m_\mu}{m_e} = f_1(N_{\mathrm{eff}}, R_\psi, w_n),
\quad
\frac{m_\tau}{m_\mu} = f_2(N_{\mathrm{eff}}, R_\psi, w_n),
\]
where \(f_1, f_2\) are functions determined by the eigenvalues of \(\Sigma\) and topological winding contributions.

\paragraph{Experimental comparison:}
\begin{align*}
\left(\frac{m_\mu}{m_e}\right)_{\text{exp}} &= 206.768\,283(5), \\
\left(\frac{m_\tau}{m_\mu}\right)_{\text{exp}} &= 16.817\,16(25).
\end{align*}

\textbf{Falsification criterion}: If UBT predicts values differing by more than 3 combined standard deviations from experimental values, the Yukawa texture hypothesis is falsified.

\paragraph{Sum Rule 2: Ratio of ratios (dimensionless invariant).}
\[
\mathcal{R}_{\ell\ell} := \frac{m_\tau/m_\mu}{m_\mu/m_e} = \frac{f_2}{f_1}.
\]
Experimental value:
\[
\mathcal{R}_{\ell\ell,\text{exp}} = \frac{16.817}{206.768} \approx 0.0813.
\]

This ratio is independent of both the overall mass scale \textbf{and} the Higgs vev. It is a pure prediction of the geometric Yukawa structure.

\paragraph{Sum Rule 3: Generalized Koide-like relation.}
Define the Koide parameter:
\[
K := \frac{\left(\sqrt{m_e} + \sqrt{m_\mu} + \sqrt{m_\tau}\right)^2}{m_e + m_\mu + m_\tau}.
\]
Experimentally, \(K_{\text{exp}} \approx 0.666\,661\) (remarkably close to \(2/3\)).

If the biquaternionic structure predicts a specific value of \(K\), this would be a nontrivial confirmation. Current status: no UBT prediction yet; requires explicit calculation of Yukawa eigenvalues.

\subsection{Error Budgeting and Numerical Intervals}

For each proposed sum rule, we define:
\begin{itemize}
  \item \textbf{Predicted value}: \(\mathcal{O}_{\text{UBT}}\) from geometric calculation
  \item \textbf{Experimental value}: \(\mathcal{O}_{\text{exp}} \pm \delta_{\text{exp}}\)
  \item \textbf{Theoretical uncertainty}: \(\delta_{\text{UBT}}\) from truncation errors, higher-order corrections, numerical integration errors
\end{itemize}

\paragraph{Acceptance criterion:}
The prediction is consistent if:
\[
|\mathcal{O}_{\text{UBT}} - \mathcal{O}_{\text{exp}}| < 3 \sqrt{\delta_{\text{exp}}^2 + \delta_{\text{UBT}}^2}.
\]

If this criterion is violated, the theory is falsified (or requires revision of assumptions).

\subsection{No Numerology: Pre-Registration Requirement}

We emphasize: \textbf{all sum rule predictions must be published before comparing to experimental data}. This prevents hindsight bias and post-hoc parameter adjustment.

\paragraph{Pre-registration protocol:}
\begin{enumerate}
  \item Derive sum rule from first principles (biquaternionic geometry + CT renormalization)
  \item Compute predicted numerical value with error budget
  \item Publish prediction in a timestamped document (e.g., arXiv preprint, GitHub commit)
  \item Compare to experimental values only after publication
  \item Accept falsification if disagreement exceeds error budget
\end{enumerate}

This protocol ensures scientific integrity and distinguishes UBT from numerological fitting exercises.

\subsection{Quark Sector: Additional Complexity}

Quark masses introduce additional challenges:
\begin{itemize}
  \item \textbf{QCD running}: Quark masses are scale-dependent (running masses vs. pole masses)
  \item \textbf{CKM mixing}: Flavor mixing introduces additional parameters
  \item \textbf{Confinement}: Light quarks are not observable as free particles
\end{itemize}

Sum rules for quarks must account for these effects. Candidate relations:
\begin{enumerate}
  \item \textbf{Up/down quark mass ratio}: \(m_u/m_d\) (at a fixed renormalization scale)
  \item \textbf{Strange/charm ratio}: \(m_s/m_c\)
  \item \textbf{CKM matrix elements}: Relate Yukawa eigenvalues to mixing angles
\end{enumerate}

Deriving quark mass ratios from UBT would be a major achievement. Current status: exploratory phase; no specific predictions yet.

\subsection{Roadmap and Falsification Strategy}

\begin{enumerate}
  \item \textbf{Complete Yukawa texture calculation}: Eigenvalues of \(Y_\ell = a \cdot \mathbb{1} + b \cdot \Sigma + c \cdot T\)
  \item \textbf{Compute lepton mass ratios}: \(m_\mu/m_e\), \(m_\tau/m_\mu\), \(\mathcal{R}_{\ell\ell}\), \(K\)
  \item \textbf{Pre-register predictions}: arXiv preprint with numerical values and error budgets
  \item \textbf{Compare to PDG data}: Check consistency with experimental values
  \item \textbf{Accept result}: If agreement $\to$ confirmation; if disagreement $\to$ falsification or revision
\end{enumerate}

\textbf{Timeline}: First falsifiable lepton mass ratio predictions within 12--18 months (contingent on completion of Yukawa overlap integrals).

\subsection{Connection to $\alpha$ Derivation}

The mass ratio program parallels the fit-free \(\alpha\) derivation:
\begin{itemize}
  \item \textbf{Geometric locking}: \(N_{\mathrm{eff}}, R_\psi\) fix \(\alpha\); same data constrains Yukawa coefficients \(a, b, c\)
  \item \textbf{Ward identities}: Enforce \(\mathcal{R}_{\mathrm{UBT}} = 1\) for \(\alpha\); analogous constraints for Yukawa renormalization
  \item \textbf{Falsifiability}: \(\alpha\) prediction can be tested; mass ratio predictions provide independent tests
\end{itemize}

Both programs share the principle: \textbf{no fitted parameters, only geometric constraints and renormalization consistency}.

% ================== END ==================
