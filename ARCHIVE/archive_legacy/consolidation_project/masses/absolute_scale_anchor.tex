% ================== Absolute Scale Anchor for Fermion Masses ==================
% VERSION: v1.0 - Program Sketch
% AUTHOR: UBT Team
% PURPOSE: Candidate anchor for fixing overall fermion mass scale from geometry
%
% DEPENDENCIES:
% Requires: \usepackage{amsmath,amssymb,amsthm}
% References: Appendix CT (geometric locking), yukawa_in_HC.tex, ct_two_loop_renorm.tex

\section{Absolute Scale Anchor from Geometric Normalization}
\label{app:absolute-scale-anchor}

\subsection{Overview}

Even if the biquaternionic structure fixes all \textbf{mass ratios} (e.g., $m_\mu/m_e$, $m_\tau/m_\mu$) from first principles, there remains the question of the \textbf{absolute mass scale}. In the Standard Model, this scale is set by the Higgs vacuum expectation value (vev) $v \approx 246$ GeV, which is an empirical input.

For UBT to be fully predictive, we must derive the absolute fermion mass scale from geometric considerations without introducing fitted parameters. This appendix outlines two candidate anchors.

\subsection{Candidate Anchor 1: Invariant Measure Normalization}

The same geometric normalization that fixes $N_{\mathrm{eff}}$ and $R_\psi$ for the $\alpha$ derivation may also constrain the absolute fermion mass scale.

\paragraph{Recall: $\alpha$ derivation.}
In Appendix CT, we showed that the effective mode count $N_{\mathrm{eff}}$ is determined by:
\begin{itemize}
  \item Lorentz-invariant measure on the Hermitian slice
  \item Spectral domain $\Omega$ with boundary conditions $\mathcal B$
  \item Thomson-limit normalization at $q^2 = 0$
\end{itemize}

The normalization $R_\psi$ was fixed by requiring:
\[
\int_0^{2\pi} e^{B(\psi)-2A(\psi)} |\xi_0(\psi)|^2 \, d\psi = 1,
\]
where $A, B$ are warp factors and $\xi_0$ is the zero-mode profile (Lemma~\ref{lem:Rpsi-fixed} in Appendix CT).

\paragraph{Extension to Yukawa sector.}
Consider the overlap integral for Yukawa couplings on the Hermitian slice:
\[
Y_{ij} \sim \int_{\mathcal H} \psi_L^i(q) \, \phi(q) \, \psi_R^j(q) \, d\mu(q),
\]
where:
\begin{itemize}
  \item $\psi_L^i, \psi_R^j$ are left/right fermion mode functions
  \item $\phi$ is the Higgs field profile
  \item $d\mu(q)$ is the invariant measure on the Hermitian slice $\mathcal H$
\end{itemize}

If the normalization of $d\mu$ is the \textbf{same} as the one used for $N_{\mathrm{eff}}$ and $R_\psi$, then the overall scale of $Y_{ij}$ (and hence fermion masses) is tied to the geometric normalization.

\paragraph{Concrete proposal.}
Define the overall Yukawa scale by:
\[
Y_0 := \frac{\Lambda_{\mathbb{H}_{\mathbb{C}}}}{\sqrt{R_\psi}},
\]
where $\Lambda_{\mathbb{H}_{\mathbb{C}}}$ is a characteristic energy scale from the biquaternionic geometry (e.g., the inverse of a geometric length scale).

Then fermion masses are:
\[
m_f = Y_0 \times r_f \times v_{\text{eff}},
\]
where:
\begin{itemize}
  \item $r_f$ is a dimensionless ratio fixed by Yukawa texture (from involutions and CT renormalization)
  \item $v_{\text{eff}}$ is the effective Higgs vev, possibly related to $R_\psi$ or another geometric scale
\end{itemize}

\paragraph{Acceptance criterion.}
This anchor is acceptable if:
\begin{enumerate}
  \item $\Lambda_{\mathbb{H}_{\mathbb{C}}}$ is determined uniquely by the Hermitian slice construction (no free parameter)
  \item The relation $Y_0 \sim \Lambda/\sqrt{R_\psi}$ follows from dimensional analysis and normalization consistency
  \item Predicted fermion masses match experimental values within error budget (after fixing ratios separately)
\end{enumerate}

\subsection{Candidate Anchor 2: Topological Constraint}

An alternative approach uses topological invariants to fix the absolute scale.

\paragraph{Hopfion mass formula (electron example).}
In earlier UBT literature, the electron mass was related to a topological soliton (Hopfion) in the $\Theta$-field:
\[
m_e \sim \frac{\hbar}{c R_\psi} \times n_H,
\]
where $n_H$ is a winding number.

While this gave approximately correct numerical results ($m_e \approx 0.510$ MeV), it was not a fully first-principles derivation because:
\begin{enumerate}
  \item The Hopfion ansatz was phenomenological (not derived from field equations)
  \item The relation to Yukawa couplings was not established
  \item The vev structure was not included consistently
\end{enumerate}

\paragraph{Improved topological anchor.}
A rigorous version would:
\begin{enumerate}
  \item Derive fermion zero-modes as topological solitons from the UBT field equations
  \item Relate the soliton energy to the Yukawa coupling eigenvalues
  \item Fix the overall mass scale by requiring consistency between:
  \begin{itemize}
    \item Soliton energy $E_{\text{soliton}}$
    \item Yukawa mass $m_Y = Y \times v_{\text{eff}}$
    \item Topological charge normalization
  \end{itemize}
\end{enumerate}

If successful, this would yield an absolute mass scale without empirical input, analogous to the fit-free $\alpha$ derivation.

\subsection{Acceptance Criteria for Any Anchor}

Any proposed absolute scale anchor must satisfy:
\begin{enumerate}
  \item \textbf{No fitted parameters:} All quantities are derived from $\mathbb H_{\mathbb C}$ geometry, renormalization consistency, or topological invariants.
  \item \textbf{Scheme independence:} The result must not depend on arbitrary choices in the renormalization scheme (modulo irrelevant finite reparametrizations).
  \item \textbf{Consistency with ratios:} The absolute scale must be compatible with mass ratios derived separately from Yukawa texture constraints.
  \item \textbf{Falsifiability:} Predicted masses must be compared to experimental values; disagreement falsifies the theory.
\end{enumerate}

\subsection{Relation to $\alpha$ Derivation}

The absolute scale problem for fermion masses is analogous to the pipeline function $F(B)$ in the $\alpha$ derivation. We established:
\[
\alpha^{-1} = F\!\left(\frac{2\pi N_{\mathrm{eff}}}{3R_\psi}\right),
\]
where $B = \frac{2\pi N_{\mathrm{eff}}}{3R_\psi}$ is fixed by geometric locking, but $F$ itself is not yet derived from first principles.

Similarly, for fermion masses:
\[
m_f = Y_0 \times r_f \times v_{\text{eff}},
\]
where $r_f$ may be fixed by Yukawa constraints, but $Y_0$ and $v_{\text{eff}}$ require geometric anchoring.

\paragraph{Priority.}
Deriving $F(B)$ and $Y_0$ from first principles are both open problems in UBT. Progress on either front would strengthen the theory's predictive power.

\subsection{Current Status and Next Steps}

\textbf{Status:} Two candidate anchors proposed; neither fully developed.

\textbf{Next steps:}
\begin{enumerate}
  \item Formalize the invariant measure normalization on the Hermitian slice
  \item Compute overlap integrals for Yukawa couplings with canonical normalization
  \item Derive topological soliton solutions from UBT field equations (not just ansätze)
  \item Compare predictions to experimental fermion masses
  \item Publish predictions \textbf{before} comparing to data (to avoid hindsight bias)
\end{enumerate}

\textbf{Timeline estimate:} 24--36 months for first absolute mass predictions.

\subsection{Ethical Note: No Post-Hoc Fitting}

We emphasize again: any absolute scale anchor must be derived \textbf{before} comparing to experimental values. Adjusting the anchor after seeing data would constitute parameter fitting and undermine the scientific integrity of UBT.

If the derived scale disagrees with experiment, we must:
\begin{itemize}
  \item Acknowledge the disagreement openly
  \item Re-examine the geometric construction
  \item Revise or abandon the mass derivation program if no resolution is found
\end{itemize}

This commitment to falsifiability is central to the UBT research program.

% ================== END ==================
