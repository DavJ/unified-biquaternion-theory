% ================== Yukawa Couplings in H_C ==================
% VERSION: v2.0 - Algebraic Constraints from Hermitian Slice
% AUTHOR: UBT Team
% PURPOSE: Outline the algebraic form of Yukawa couplings on the Hermitian slice
%
% DEPENDENCIES:
% Requires: \usepackage{amsmath,amssymb,amsthm}
% References: Appendix P6 (Lorentz in H_C), Appendix E (SM geometry)

\section{Yukawa Couplings in \texorpdfstring{$\mathbb H_{\mathbb C}$}{H\_C}}
\label{app:yukawa-in-HC}

\subsection{Overview}

In the Standard Model, fermion masses arise from Yukawa couplings to the Higgs field. For UBT to predict fermion masses from first principles, we must:
\begin{enumerate}
  \item Derive the algebraic form of Yukawa couplings on the Hermitian slice of \(\mathbb H_{\mathbb C}\).
  \item Show how left/right actions and involutions constrain the coupling structure.
  \item Connect to the discrete symmetries (P, T, C) via the three involutions.
\end{enumerate}

This appendix provides the framework and key constraints.

\subsection{Left/Right Actions on Spinors}

On the Hermitian slice \(\mathcal H \subset \mathbb H_{\mathbb C}\), spinors are represented as elements of minimal left ideals. The biquaternionic structure allows both left and right actions:
\begin{itemize}
  \item \textbf{Left action}: \(q \cdot \psi\) for \(q \in \mathbb H_{\mathbb C}\), \(\psi\) a spinor
  \item \textbf{Right action}: \(\psi \cdot q\)
  \item \textbf{Chiral distinction}: Left-handed fermions transform under left action only; right-handed under right action
\end{itemize}

The Yukawa bilinear involves both chiralities:
\[
\mathcal L_{\text{Yukawa}} = -Y_{ij} \bar{\psi}_L^i \phi \psi_R^j + \text{h.c.}
\]

\paragraph{Involution constraints.}
The three fundamental involutions on \(\mathbb H_{\mathbb C}\) are:
\begin{enumerate}
  \item \textbf{Complex conjugation} \(\overline{\ }\): \(\overline{a + bi} = a - bi\)
  \item \textbf{Quaternionic conjugation} \(^*\): \((q_0 + q_1 i + q_2 j + q_3 k)^* = q_0 - q_1 i - q_2 j - q_3 k\)
  \item \textbf{Hermitian adjoint} \(^\dagger\): composition of the above two operations
\end{enumerate}

These involutions realize the discrete symmetries P (parity), T (time-reversal), and C (charge conjugation) on the Hermitian slice.

\subsection{Allowed Yukawa Invariants}

For a Yukawa coupling to be a valid Lorentz scalar on \(\mathcal H\), it must be invariant under:
\begin{itemize}
  \item \(\mathrm{SL}(2,\mathbb C)\) Lorentz transformations: \(X \mapsto A X A^\dagger\)
  \item Discrete involutions implementing P, T, C
\end{itemize}

\paragraph{Permissible bilinears.}
The allowed Yukawa structures on the Hermitian slice have the form:
\[
Y_{ij} \sim \int_{\mathcal H} \bar{\psi}_L^i(q) \, \phi(q) \, \psi_R^j(q) \, d\mu(q),
\]
where \(d\mu\) is the Lorentz-invariant measure (same as used for \(N_{\mathrm{eff}}\) in the \(\alpha\) derivation).

The discrete symmetries constrain which combinations of left/right spinor components can appear:
\begin{itemize}
  \item P-invariance restricts mixing between chiral components
  \item T-invariance imposes reality conditions on certain couplings
  \item C-invariance relates particle/antiparticle sectors
\end{itemize}

\subsection{Restricted Form of Yukawa Matrix}

\begin{lemma}[Yukawa texture constraint]
\label{lem:yukawa-texture}
Under the involution constraints and Lorentz invariance on \(\mathcal H\), the charged-lepton Yukawa matrix \(Y_\ell\) admits a decomposition:
\[
Y_\ell = a \cdot \mathbb{1}_{3\times 3} + b \cdot \Sigma + c \cdot T + \ldots,
\]
where:
\begin{itemize}
  \item \(a, b, c\) are real coefficients determined by overlap integrals on \(\mathcal H\)
  \item \(\Sigma\) is a fixed matrix encoding generation mixing from geometric structure
  \item \(T\) represents topological winding contributions
  \item Higher-order terms (ellipsis) are suppressed by additional symmetry conditions
\end{itemize}
No free angular parameters beyond group-theoretic ones remain after imposing all constraints.
\end{lemma}

\begin{proof}[Sketch]
The involutions impose reality and orthogonality conditions on the mode functions \(\psi_L^i, \psi_R^j\). Expanding in the basis of Hermitian slice eigenmodes and applying Lorentz invariance constrains the allowed mixing patterns. The resulting structure has the stated form with coefficients determined by geometric data. Full proof deferred to detailed calculations.
\end{proof}

\subsection{Connection to Geometric Invariants}

The coefficients \(a, b, c\) in Lemma~\ref{lem:yukawa-texture} are not free parameters. They are expressed as:
\begin{align}
a &\sim \int_{\mathcal H} |\phi_0(q)|^2 \, d\mu(q) \quad \text{(Higgs zero-mode normalization)}, \\
b &\sim \sum_{n} w_n \int_{\mathcal H} \psi_L^{(n)}(q) \phi(q) \psi_R^{(n)}(q) \, d\mu(q) \quad \text{(mode overlap)}, \\
c &\sim \frac{1}{R_\psi} \sum_{k} n_k \quad \text{(topological winding)},
\end{align}
where:
\begin{itemize}
  \item \(w_n\) are spectral weights from mode counting (related to \(N_{\mathrm{eff}}\))
  \item \(n_k\) are integer winding numbers around compact directions
  \item \(R_\psi\) is the imaginary-time radius (same as in \(\alpha\) derivation)
\end{itemize}

This provides the ``locking'' mechanism: just as \(N_{\mathrm{eff}}\) and \(R_\psi\) are fixed by geometric normalization for \(\alpha\), the Yukawa coefficients are fixed by the same geometric data.

\subsection{Roadmap and Current Status}

\begin{enumerate}
  \item \textbf{Formalize fermion representations}: In progress (left/right ideal structure defined above)
  \item \textbf{Derive coupling structure}: Lemma~\ref{lem:yukawa-texture} provides the framework
  \item \textbf{Connect to topology}: Expressions for \(a, b, c\) relate to geometric invariants
  \item \textbf{Falsifiable predictions}: Mass ratios \(m_e/m_\mu\), \(m_\mu/m_\tau\) follow from eigenvalues of \(Y_\ell\)
\end{enumerate}

\textbf{Next steps}: Explicit calculation of overlap integrals and winding numbers; comparison to experimental lepton mass ratios.

% ================== END ==================
