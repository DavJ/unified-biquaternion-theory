% ================== RG Flows for Fermion Masses ==================
% AUTHOR: UBT Team
% PURPOSE: Renormalization-group flow equations and fixed-point structure

\section{Renormalization-Group Flows}
\label{sec:rg-flows}

\subsection{General RG Equations}

The running of Yukawa couplings in UBT follows from the renormalization-group equations
(RGE) for the coupling parameters. In the complex-time (CT) scheme with renormalization
scale \(\mu\), the one-loop RGE for a generic Yukawa coupling \(y\) is:
\begin{equation}
\label{eq:rge-yukawa}
\mu \frac{dy}{d\mu} = \beta_y(y, g_i, \lambda)
\end{equation}
where \(g_i\) are gauge couplings and \(\lambda\) is the Higgs self-coupling.

\subsection{CT Scheme Modifications}

The complex-time parameter \(\tau = t + i\psi\) introduces additional structure in the RG
flow. The beta function decomposes as:
\begin{equation}
\beta_y = \beta_y^{\text{real}} + i\psi \beta_y^{\text{phase}} + \mathcal{O}(\psi^2)
\end{equation}
In the real-time limit \(\psi \to 0\), we recover standard QED/SM running:
\begin{equation}
\beta_y^{\text{real}} = \frac{1}{16\pi^2}\left[
  \text{Tr}(Y^\dagger Y) y - \sum_i C_i g_i^2 y
\right] + \mathcal{O}(g^4)
\end{equation}

\subsection{Fixed Points and Geometric Constraints}

The biquaternionic structure may induce fixed points in the RG flow that are absent
in standard field theory. We consider:

\paragraph{Geometric fixed points.} If the Yukawa matrix aligns with geometric structures
on the Hermitian slice, the RG flow may stabilize:
\begin{equation}
\beta_y|_{\text{geom}} = 0 \quad \text{when} \quad y = y_{\text{crit}}(\mathcal{H}, \mathcal{B})
\end{equation}
where \(\mathcal{H}\) is the Hermitian slice and \(\mathcal{B}\) are boundary conditions.

\paragraph{Quasi-fixed points.} For hierarchical masses, the RG trajectories may approach
quasi-fixed point behavior in certain energy ranges, providing natural explanations for
observed mass ratios.

\subsection{Connection to Mass Sum Rules}

The fixed-point structure constrains possible mass ratios. Combined with the algebraic
constraints from Section~\ref{sec:yukawa-structure}, this leads to testable predictions
(Section~\ref{sec:mass-sum-rules}).

The explicit calculation of \(\beta_y^{\text{phase}}\) and the identification of geometric
fixed points requires numerical analysis of the mode structure on \(\mathbb{H}_{\mathbb{C}}\),
which is deferred to future work.

% =======================================================================
