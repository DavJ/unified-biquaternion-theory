% UBT_hecke_L_route.tex
% Variant C: Automorphic / L-function route (theta → Mellin → L(s) with Euler product)
% Preamble requirements (already typical in the project):
%   \usepackage{amsmath,amssymb,amsthm,booktabs}
%   \usepackage{hyperref}

\section{Hecke route via \(L\)-functions (Variant C)}\label{sec:hecke-L-route}
We construct the Hecke sector from a half–integral weight theta–object and pass to its \(L\)–function via Mellin,
so that the Euler product makes the prime indexing explicit.

\paragraph{Theta and half–integral weight.}
Let \(\theta_j(\tau)\), \(j\in\{2,3,4\}\), denote the Jacobi theta constants (with \(z=0\)).
We choose a normalized combination \(\vartheta(\tau)=\sum_{n\ge 0} a_n q^n\) that transforms as a modular form of weight \(k=\tfrac12\) in Kohnen’s plus space on \(\Gamma_0(4N)\) (nebentypus \(\chi\)).

\paragraph{Hecke action \(T(p^2)\).}
For primes \(p\nmid 2N\) and \(k=\tfrac12\), the Hecke operator \(T(p^2)\) acts on Fourier coefficients as
\begin{equation}
(T(p^2)\,\vartheta)(\tau)=\sum_{n\ge0}\Big(a_{p^2 n}+\chi(p)\,p^{k-1}\,a_n+p^{2k-1}\,a_{n/p^2}\Big)q^n,
\qquad a_{n/p^2}=0\text{ if }p^2\nmid n.
\end{equation}
If \(\vartheta\) is an eigenform, \(T(p^2)\,\vartheta=\lambda_p\,\vartheta\).

\paragraph{Mellin transform and \(L\)–function.}
Define the \(L\)–function of \(\vartheta\) by Mellin transform of its \(q\)–expansion (up to a standard gamma factor):
\begin{equation}
L(\vartheta,s) \;=\; \sum_{n\ge1}\frac{a_n}{n^s}\,,
\end{equation}
which admits an Euler product over primes \(p\) consistent with the Hecke eigenvalues \(\lambda_p\):
\begin{equation}
L(\vartheta,s)=\prod_{p}\;L_p(\vartheta,s)\,,
\qquad
L_p(\vartheta,s)=\left(1-\lambda_p\,p^{-s} + \varepsilon_p\,p^{2k-1-2s}\right)^{-1},
\end{equation}
for suitable local signs \(\varepsilon_p\) (depending on the space and normalization).

\paragraph{Hecke sector (``Hecke–World'').}
Fixing the local factor \(L_p(\vartheta,s)\) (equivalently: the local Hecke place \(p\)) defines the Hecke sector.
In that sector, the UBT invariant ratio yields the fine–structure constant
\begin{equation}
\boxed{\;\alpha_p^{-1}\;=\;p\;+\;\Delta_{\mathrm{CT}}(p)\;}\,,
\end{equation}
where \(\Delta_{\mathrm{CT}}(p)\) is the finite \emph{archimedean two–loop} correction obtained by Ward identity and Thomson–limit matching (with no fitted parameters).
The role of the automorphic data (e.g. \(\lambda_p\)) is confined, if desired, to a bounded sector form–factor that does not spoil the fit–free nature of \(\Delta_{\mathrm{CT}}(p)\).

\paragraph{Minimality.}
The prime index \(p\) is the canonical local place in the Hecke/Euler factorization. Composite levels factor into their prime constituents; the sector superselection is therefore prime–indexed. This matches the per–sector theorem developed earlier and the grid tables included via \verb|% UBT_alpha_per_sector_patch.tex
% Drop-in patch: per-sector theorem + CSV-driven table.
\newtheorem{ubtsector}{Theorem}

\section{Per-sector matching and numerical table}\label{sec:per-sector-matching}

\begin{ubtsector}[Per-sector matching]
For each Hecke sector indexed by a prime \(p\),
\begin{equation}
\boxed{\;\alpha_p^{-1} \;=\; p \;+\; \Delta_{\mathrm{CT}}(p)\;}
\end{equation}
where \(\Delta_{\mathrm{CT}}(p)\) is the finite two-loop archimedean correction obtained by the Ward identity and the Thomson limit matching (no fitted parameters). The sector dependence can be optionally encoded by a multiplicative form factor in the UBT\(\to\)QED map; the archimedean matching procedure itself is identical across sectors.
\end{ubtsector}

\paragraph{Numerical table (from CSV).}
If the CSV produced by the companion code
\texttt{alpha\_core\_repro/out/alpha\_two\_loop\_grid.csv}
is present, we import it below.

\begingroup
\pgfplotstableset{col sep=comma,string type}
\newcommand{\UBTAlphaCSV}{alpha_core_repro/out/alpha_two_loop_grid.csv}

\IfFileExists{\UBTAlphaCSV}{
    \pgfplotstabletypeset[
        columns={p,delta_ct,alpha_inv,alpha,scheme,mu,form_factor},
        columns/p/.style={column name={$p$}},
        columns/delta_ct/.style={column name={$\Delta_{\mathrm{CT}}$}},
        columns/alpha_inv/.style={column name={$\alpha^{-1}$}},
        columns/alpha/.style={column name={$\alpha$}},
        columns/scheme/.style={column name={scheme}},
        columns/mu/.style={column name={$\mu$}},
        columns/form_factor/.style={column name={form factor}},
        every head row/.style={before row=\toprule,after row=\midrule},
        every last row/.style={after row=\bottomrule},
        begin table=\begin{tabular}, end table=\end{tabular},
    ]{\UBTAlphaCSV}
}{
    \begin{center}
    \begin{tabular}{lccc}
    \toprule
    Prime sector $p$ & $\Delta_{\mathrm{CT}}(p)$ & $\alpha_p^{-1}$ & $\alpha_p$ \\
    \midrule
    131 & $\cdots$ & $131+\cdots$ & $\cdots$ \\
    137 & $0.035999\ldots$ & $137.035999\ldots$ & $\cdots$ \\
    139 & $\cdots$ & $139+\cdots$ & $\cdots$ \\
    \bottomrule
    \end{tabular}
    \end{center}
}
\endgroup
|.

\paragraph{Checks and numerics.}
For validation we (i) verify Hecke–eigen relations \(T(p^2)\,\vartheta=\lambda_p\,\vartheta\) on a truncated expansion;
(ii) build partial products for \(L(\vartheta,s)\) and confirm multiplicativity patterns of \(a_n\);
(iii) keep the physics matching (Ward→Thomson) prime–agnostic, so that \(\Delta_{\mathrm{CT}}(p)\) remains archimedean across sectors.
