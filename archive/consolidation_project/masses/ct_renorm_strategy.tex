% ================== CT Renormalization Strategy for Masses ==================
% VERSION: v1.0 - Program Sketch
% AUTHOR: UBT Team
% PURPOSE: Outline CT two-loop renormalization conditions for Yukawa couplings
%
% DEPENDENCIES:
% Requires: \usepackage{amsmath,amssymb,amsthm}
% References: Appendix CT (two-loop baseline), yukawa_in_HC.tex

\section{CT Renormalization Strategy for Fermion Masses}
\label{app:ct-renorm-masses}

\subsection{Overview}

The complex-time (CT) renormalization scheme used for the fit-free \(\alpha\) derivation (Appendix~\ref{app:ct-baseline-R1}) provides a template for handling Yukawa couplings. The key idea:
\begin{enumerate}
  \item Use dimensional regularization \(d=4-2\epsilon\) with CT-\(\overline{\mathrm{MS}}\) subtractions.
  \item Impose Ward identities and gauge constraints to fix scheme-ambiguous pieces.
  \item Relate Yukawa couplings to geometric invariants via ``locking'' conditions.
\end{enumerate}

\subsection{Analogy to \(\alpha\) Derivation}

For \(\alpha\), we had:
\[
B = \frac{2\pi N_{\mathrm{eff}}}{3 R_\psi} \times \mathcal R_{\mathrm{UBT}},
\quad \text{with } \mathcal R_{\mathrm{UBT}} = 1 \text{ (proven)}.
\]

For Yukawa couplings, a similar structure might be:
\[
Y_{ij} = \text{(geometric factor)} \times \text{(renormalization factor)}.
\]

The challenge is to identify:
\begin{itemize}
  \item What is the geometric factor? (Overlap integrals, winding numbers, etc.)
  \item What are the renormalization conditions that fix the scheme-dependent pieces?
\end{itemize}

\subsection{CT Two-Loop Renormalization Conditions}

In the Standard Model, Yukawa couplings receive loop corrections:
\[
Y_{ij}(\mu) = Y_{ij}(\mu_0) + \text{(one-loop)} + \text{(two-loop)} + \cdots
\]

The CT scheme must:
\begin{enumerate}
  \item Preserve chiral symmetries (where applicable).
  \item Ensure the real-time limit \(\psi \to 0\) recovers standard \(\overline{\mathrm{MS}}\) running.
  \item Fix normalization via physical observables (e.g., Thomson limit for QED, pole masses for fermions).
\end{enumerate}

\paragraph{Key question.}
Can the CT renormalization conditions, combined with geometric constraints from \(\mathbb H_{\mathbb C}\), uniquely determine the Yukawa matrices \(Y_{ij}\) up to an overall scale?

\subsection{Roadmap}

\begin{enumerate}
  \item \textbf{One-loop CT corrections}: Calculate \(Y_{ij}(\mu)\) at one-loop in CT scheme.
  \item \textbf{Ward identities}: Identify constraints analogous to \(Z_1 = Z_2\) for QED.
  \item \textbf{Geometric locking}: Express \(Y_{ij}\) in terms of \(\mathbb H_{\mathbb C}\) invariants.
  \item \textbf{Falsification tests}: Derive sum rules for mass ratios; compare to experiment.
\end{enumerate}

\textbf{Status}: Conceptual framework established; calculations in progress.

% ================== END ==================
