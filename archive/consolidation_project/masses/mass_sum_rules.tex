% ================== Mass Sum Rules ==================
% AUTHOR: UBT Team
% PURPOSE: Candidate sum rules and relations testable against data

\section{Mass Sum Rules}
\label{sec:mass-sum-rules}

\subsection{General Approach}

Mass sum rules are algebraic relations among fermion masses that arise from symmetries
or geometric constraints of the underlying theory. In UBT, such relations may emerge from:

\begin{itemize}
  \item Hermiticity conditions on the Hermitian slice (Section~\ref{sec:yukawa-structure})
  \item Fixed-point structure of RG flows (Section~\ref{sec:rg-flows})
  \item Topological invariants of the biquaternionic field configuration
  \item Discrete symmetries preserved by the \(\mathbb{H}_{\mathbb{C}}\) formalism
\end{itemize}

\subsection{Candidate Relations}

We identify several candidate sum rules for future testing:

\paragraph{Lepton sector.} The charged lepton masses \(m_e, m_\mu, m_\tau\) may satisfy:
\begin{equation}
\label{eq:lepton-sum-rule}
\frac{m_\mu}{m_e} + \frac{m_\tau}{m_\mu} = C_{\ell} \times \frac{m_\tau}{m_e}
\end{equation}
where \(C_{\ell}\) is a geometric constant determined by the Hermitian slice structure.
This relation, if validated, would reduce one degree of freedom in the lepton sector.

\paragraph{Quark sector.} For quarks, generational mixing introduces additional complexity.
A simplified candidate relation for the up-type quarks is:
\begin{equation}
\label{eq:quark-sum-rule}
\log\left(\frac{m_t}{m_u}\right) = C_q \left[
  \log\left(\frac{m_c}{m_u}\right) + \log\left(\frac{m_t}{m_c}\right)
\right]
\end{equation}
with \(C_q\) another geometric constant. This logarithmic form reflects the multiplicative
nature of RG evolution.

\subsection{Experimental Validation Strategy}

Testing these sum rules requires:

\begin{enumerate}
  \item Calculation of \(C_{\ell}, C_q\) from the \(\mathbb{H}_{\mathbb{C}}\) geometry
  (currently in progress)
  \item Comparison with precision measurements of fermion masses at appropriate energy scales
  \item Statistical analysis to determine if observed deviations are within theoretical
  and experimental uncertainties
\end{enumerate}

\subsection{Connection to CKM Matrix}

The quark mass sum rules are intertwined with the Cabibbo-Kobayashi-Maskawa (CKM) matrix
through the diagonalization of the Yukawa matrices. Any relation among masses constrains
the possible flavor-mixing patterns. This connection provides additional testable predictions
and will be explored in detail once the geometric constants are computed.

\paragraph{Status.} The mass sum rules program is at an early stage. Equations
\eqref{eq:lepton-sum-rule} and \eqref{eq:quark-sum-rule} are \textbf{conjectures} based on
structural arguments. Explicit derivation of \(C_{\ell}, C_q\) from first principles is
required before empirical testing.

% =======================================================================
