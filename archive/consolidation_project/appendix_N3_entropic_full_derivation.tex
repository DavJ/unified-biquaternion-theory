% © 2025 Ing. David Jaroš — CC BY-NC-ND 4.0
%
% This work is licensed under a Creative Commons Attribution-NonCommercial-NoDerivatives 
% 4.0 International License (CC BY-NC-ND 4.0).

\subsection{Full Entropic Construction from the Complex Field $\Theta$}
\label{sec:entropic_full_derivation}

This section provides a complete algebraic derivation of the entropic formulation in UBT, establishing the relationship between the complex biquaternionic field $\Theta(q,\tau)$ and gravitational thermodynamics. We clearly distinguish between derived mathematical identities and physical interpretations that remain as hypotheses requiring empirical validation.

\subsubsection{Definition: Complex Entropic Functional}

\paragraph{Definition 1 (Complex entropic functional).}
Let $\Theta(q,\tau) \in \mathbb{C}^{4\times4}$ be the fundamental UBT field over complex time $\tau = t + i\psi$. We define the complex entropic functional as:
\begin{equation}
\mathcal{S}(q) := k_B \operatorname{Tr}(\log \Theta(q)).
\label{eq:complex_entropy_functional}
\end{equation}

Using the fundamental logarithm-determinant identity $\det \Theta = e^{\operatorname{Tr}(\log \Theta)}$, we equivalently write:
\begin{equation}
\mathcal{S}(q) = k_B \log \det \Theta(q).
\label{eq:entropy_determinant_form}
\end{equation}

\paragraph{Mathematical Status.} 
Equations~\eqref{eq:complex_entropy_functional} and~\eqref{eq:entropy_determinant_form} are \textbf{derived identities} from matrix algebra, valid for any invertible complex matrix $\Theta$.

\subsubsection{Relation to Determinant Form}

\paragraph{Proposition 1 (Relation to the determinant form).}
For the Hermitian positive-definite product $\Theta^\dagger \Theta > 0$, the real entropic scalar is related to the complex functional by:
\begin{equation}
S(q) := k_B \log \det(\Theta^\dagger \Theta) = 2\,\mathrm{Re}(\mathcal{S}(q)).
\label{eq:real_entropy_relation}
\end{equation}

\paragraph{Proof.}
Starting from the logarithm-determinant identity for products:
\begin{align}
\log \det(\Theta^\dagger \Theta) 
&= \operatorname{Tr}(\log(\Theta^\dagger \Theta)) \\
&= \operatorname{Tr}(\log \Theta^\dagger + \log \Theta) \\
&= \operatorname{Tr}(\log \Theta^\dagger) + \operatorname{Tr}(\log \Theta) \\
&= \operatorname{Tr}((\log \Theta)^\dagger) + \operatorname{Tr}(\log \Theta).
\end{align}

For any complex matrix $A$, we have $\operatorname{Tr}(A^\dagger) = \overline{\operatorname{Tr}(A)}$. Therefore:
\begin{equation}
\log \det(\Theta^\dagger \Theta) = \overline{\operatorname{Tr}(\log \Theta)} + \operatorname{Tr}(\log \Theta) = 2\,\mathrm{Re}(\operatorname{Tr}(\log \Theta)).
\end{equation}

Substituting equation~\eqref{eq:complex_entropy_functional}:
\begin{equation}
\log \det(\Theta^\dagger \Theta) = 2\,\mathrm{Re}(\mathcal{S}(q)/k_B).
\end{equation}

Multiplying by $k_B$ yields:
\begin{equation}
S(q) = k_B \log \det(\Theta^\dagger \Theta) = 2\,\mathrm{Re}(\mathcal{S}(q)).
\end{equation}

Alternatively, using $|\det \Theta|^2 = \det(\Theta^\dagger \Theta)$:
\begin{equation}
S(q) = 2k_B \log |\det \Theta|.
\label{eq:entropy_modulus}
\end{equation}
\qed

\paragraph{Mathematical Status.} 
Proposition 1 and equation~\eqref{eq:real_entropy_relation} are \textbf{derived identities} from matrix algebra and complex analysis.

\subsubsection{Physical Interpretation}

\paragraph{Interpretation (Real vs Imaginary Sectors).}
The physical entropic scalar $S(q)$ is determined entirely by the real sector:
\begin{equation}
S(q) = 2\,\mathrm{Re}(\mathcal{S}(q)) = 2k_B \log |\det \Theta|.
\end{equation}

The imaginary sector defines an internal phase structure:
\begin{equation}
\mathrm{Im}(\mathcal{S}) = k_B \,\mathrm{Im}(\log \det \Theta) = k_B \arg(\det \Theta).
\label{eq:phase_entropy}
\end{equation}

This phase component represents the internal angular structure of the determinant in the complex plane. While $\mathrm{Im}(\mathcal{S})$ does not contribute to the thermodynamic entropy $S$, it encodes phase coherence information relevant to quantum correlations.

\paragraph{Physical Status.} 
The interpretation of $S(q)$ as a gravitational entropy is a \textbf{hypothesis} linking the algebraic structure to thermodynamic observables. The role of the phase component in equation~\eqref{eq:phase_entropy} requires experimental validation.

\subsubsection{Justification for Real Part Projection in 4D Metric}

\paragraph{Projection Principle.}
In UBT, the physical 4D metric tensor is obtained from the real part of the biquaternionic field:
\begin{equation}
g_{\mu\nu}(q,t) = \mathrm{Re}[\Theta_{\mu\nu}(q,\tau)].
\label{eq:metric_projection}
\end{equation}

This projection ensures:
\begin{enumerate}
\item \textbf{Reality:} The metric $g_{\mu\nu}$ is real-valued, as required for measurable spacetime intervals.
\item \textbf{Hermiticity:} For Hermitian $\Theta_{\mu\nu} = \Theta_{\mu\nu}^\dagger$, the real part is automatically real.
\item \textbf{Observable coupling:} Ordinary matter couples only to real metric components $g_{\mu\nu}$.
\item \textbf{GR compatibility:} In the limit $\mathrm{Im}(\Theta_{\mu\nu}) \to 0$, we exactly recover Einstein's General Relativity.
\end{enumerate}

Similarly, the entropic scalar $S(q)$ must be real-valued for thermodynamic consistency:
\begin{equation}
S(q) = 2\,\mathrm{Re}(\mathcal{S}(q)) \in \mathbb{R}.
\end{equation}

\paragraph{Mathematical Status.}
The projection principle~\eqref{eq:metric_projection} is an \textbf{axiomatic choice} ensuring observational consistency. The real-part requirement for entropy is a \textbf{derived necessity} from thermodynamic principles.

\subsubsection{Limit Analysis I: $\det \Theta \to 0$ (Degeneracy)}

\paragraph{Algebraic Structure.}
If $\det \Theta \to 0$, then from equation~\eqref{eq:entropy_modulus}:
\begin{equation}
S(q) = 2k_B \log |\det \Theta| \to -\infty.
\label{eq:entropy_divergence}
\end{equation}

This corresponds to algebraic degeneracy: the matrix $\Theta$ loses rank, indicating that one or more eigenvalues vanish. Geometrically, the internal volume element defined by $\det \Theta$ collapses to zero.

\paragraph{Physical Interpretation (Hypothesis).}
We interpret $\det \Theta \to 0$ as defining a causal boundary structure analogous to a horizon:
\begin{itemize}
\item \textbf{Maximal algebraic order:} $S \to -\infty$ indicates minimal internal volume, not thermodynamic chaos.
\item \textbf{Horizon analogy:} Similar to black hole horizons where $g_{tt} \to 0$, the vanishing determinant defines a boundary in the biquaternionic field space.
\item \textbf{No curvature singularity:} The divergence occurs at the algebraic level, not necessarily implying divergent curvature scalars.
\end{itemize}

\paragraph{Status.} 
Equation~\eqref{eq:entropy_divergence} is a \textbf{derived result}. The physical interpretation as a horizon-like structure is a \textbf{hypothesis} requiring:
\begin{enumerate}
\item Calculation of curvature invariants near $\det \Theta = 0$
\item Analysis of causal structure and light cone behavior
\item Comparison with known horizon geometries
\end{enumerate}

\subsubsection{Limit Analysis II: $\det \Theta = 1$ (Unimodular Vacuum)}

\paragraph{Algebraic Structure.}
If $|\det \Theta| = 1$, then from equation~\eqref{eq:entropy_modulus}:
\begin{equation}
S(q) = 2k_B \log(1) = 0.
\label{eq:unimodular_entropy}
\end{equation}

This defines a \textbf{unimodular state}: the biquaternionic field preserves volume while potentially rotating in internal space. A pure phase determinant:
\begin{equation}
\det \Theta = e^{i\alpha}, \quad \alpha \in \mathbb{R},
\end{equation}
satisfies $|\det \Theta| = 1$, hence $S = 0$, while retaining nontrivial phase structure $\mathrm{Im}(\mathcal{S}) = k_B \alpha$.

\paragraph{Physical Interpretation.}
The unimodular condition $S = 0$ defines a vacuum equilibrium state:
\begin{itemize}
\item \textbf{Zero entropy:} The system has maximal internal order (minimum thermodynamic entropy).
\item \textbf{Volume preservation:} The determinant constraint $|\det \Theta| = 1$ is analogous to unimodular gravity formulations.
\item \textbf{Phase freedom:} The argument $\alpha = \arg(\det \Theta)$ represents internal gauge freedom.
\end{itemize}

\paragraph{Status.}
Equation~\eqref{eq:unimodular_entropy} is a \textbf{derived result}. The interpretation as vacuum equilibrium is a \textbf{theoretical framework} consistent with unimodular gravity approaches.

\subsubsection{Exponential Metric Mapping}

\paragraph{Motivation: Stability and Regularity.}
A naive linear weak-field expansion:
\begin{equation}
g_{tt} \approx -(1 + 2\phi/c^2), \quad \phi \propto S,
\end{equation}
becomes problematic as $S \to -\infty$:
\begin{equation}
g_{tt} \approx -(1 + 2S/S_0) \to +\infty \quad \text{(wrong sign!)}.
\end{equation}

This would produce a metric with incorrect signature and unphysical behavior near degeneracy points.

\paragraph{Exponential Mapping.}
A structurally consistent mapping is the exponential form:
\begin{equation}
g_{tt} = -\exp(2\lambda S), \quad \lambda > 0,
\label{eq:exponential_metric}
\end{equation}
where $\lambda$ is a dimensional coupling constant. This ensures:
\begin{enumerate}
\item \textbf{Sign preservation:} $g_{tt} < 0$ for all $S \in \mathbb{R}$ (timelike signature).
\item \textbf{Regular limits:} As $S \to -\infty$, we have $g_{tt} \to 0$ (horizon-like).
\item \textbf{Vacuum normalization:} At $S = 0$, we obtain $g_{tt} = -1$ (Minkowski).
\item \textbf{Positive scaling:} For $S > 0$, we have $g_{tt} < -1$ (modified vacuum).
\end{enumerate}

\paragraph{Limit Behavior.}
Under the exponential mapping~\eqref{eq:exponential_metric}:
\begin{equation}
\det \Theta \to 0 \Rightarrow S \to -\infty \Rightarrow g_{tt} \to 0,
\end{equation}
which is geometrically analogous to approaching a horizon ($g_{tt} = 0$ defines null surfaces) without producing curvature singularities at the algebraic level.

\paragraph{Status.}
The exponential mapping~\eqref{eq:exponential_metric} is a \textbf{theoretical proposal} ensuring mathematical consistency. The specific form and coupling $\lambda$ require:
\begin{enumerate}
\item Derivation from UBT field equations
\item Consistency with observational constraints
\item Analysis of higher-order corrections
\end{enumerate}

\subsubsection{Chronofactor and Two-Level Time Structure}

\paragraph{Complex Time Decomposition.}
In UBT, time has the structure $\tau = t + i\psi$, where:
\begin{itemize}
\item $t$ is the real (observable) time coordinate
\item $\psi$ is the imaginary (phase) time component
\end{itemize}

This leads to a two-level temporal structure:
\begin{enumerate}
\item \textbf{Global chronofactor:} Governs the overall evolution of the $\Theta$ field through complex time $\tau$.
\item \textbf{Local 4D metric:} Arises from the real projection $g_{\mu\nu} = \mathrm{Re}[\Theta_{\mu\nu}]$, depending only on real time $t$.
\end{enumerate}

\paragraph{Role of the Phase Sector.}
The imaginary time component $\psi$ influences:
\begin{itemize}
\item The complex evolution equation: $\frac{\partial \Theta}{\partial \tau}$ couples $t$ and $\psi$ dynamics.
\item Internal phase structure: Through equation~\eqref{eq:phase_entropy}, $\psi$ contributes to $\mathrm{Im}(\mathcal{S})$.
\item Quantum correlations: Phase coherence encoded in $\arg(\det \Theta)$.
\end{itemize}

However, the phase sector does \textbf{not directly enter} the projected metric $g_{\mu\nu}$, remaining invisible to classical observations. This preserves compatibility with General Relativity.

\paragraph{Chronofactor Interpretation.}
The global chronofactor can be understood as:
\begin{equation}
\Theta(q,\tau) = \Theta_0(q) \cdot \mathcal{C}(\tau), \quad \mathcal{C}(\tau) = e^{i\omega\tau},
\end{equation}
where $\mathcal{C}(\tau)$ represents a universal phase rotation in complex time. While $\mathcal{C}(\tau)$ governs the background scaling, the physical 4D geometry emerges from:
\begin{equation}
g_{\mu\nu}(q,t) = \mathrm{Re}[\Theta_{\mu\nu}(q,t+i\psi)] \big|_{\psi=\psi_0},
\end{equation}
evaluated at a fixed phase time $\psi_0$.

\paragraph{Status.}
The two-level time structure is a \textbf{fundamental feature} of UBT. The specific role of the chronofactor and its coupling to observable dynamics is a \textbf{theoretical framework} requiring:
\begin{enumerate}
\item Analysis of $\Theta$ field evolution equations
\item Study of phase-dependent corrections
\item Experimental signatures in quantum gravity regimes
\end{enumerate}

\subsubsection{Summary: Derived Results vs Hypotheses}

\paragraph{Derived Results (Algebraic Identities).}
The following are \textbf{mathematically derived} from the structure of $\Theta(q,\tau)$:
\begin{itemize}
\item Complex entropic functional: $\mathcal{S}(q) = k_B \log \det \Theta(q)$
\item Real entropy relation: $S(q) = 2\,\mathrm{Re}(\mathcal{S}(q)) = 2k_B \log |\det \Theta|$
\item Phase structure: $\mathrm{Im}(\mathcal{S}) = k_B \arg(\det \Theta)$
\item Limit behavior: $\det \Theta \to 0 \Rightarrow S \to -\infty$; $|\det \Theta| = 1 \Rightarrow S = 0$
\end{itemize}

\paragraph{Physical Interpretations (Hypotheses to be Tested).}
The following require empirical validation:
\begin{itemize}
\item $S(q)$ represents gravitational entropy (links algebra to thermodynamics)
\item $\det \Theta \to 0$ defines horizon-like causal boundaries
\item $|\det \Theta| = 1$ corresponds to vacuum equilibrium
\item Exponential metric mapping $g_{tt} = -\exp(2\lambda S)$ is physically realized
\item Phase sector $\mathrm{Im}(\mathcal{S})$ encodes quantum correlations
\item Chronofactor governs background evolution distinct from local metric
\end{itemize}

\paragraph{Research Program.}
To elevate these hypotheses to validated predictions:
\begin{enumerate}
\item Derive explicit field equations for $\Theta(q,\tau)$
\item Calculate curvature invariants near limiting cases
\item Compare predictions with observational data (gravitational waves, black holes)
\item Develop experimental signatures for phase sector effects
\end{enumerate}

This derivation establishes the mathematical foundations for entropic thermodynamics in UBT while clearly delineating what is algebraically proven versus what remains theoretically proposed.
