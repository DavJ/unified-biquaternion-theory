% VERSION: v17 Stable Release
% Appendix E3: Neutrino Masses and PMNS Mixing from Θ Seesaw
% Status: Core derivation - seesaw-ready extension

\section{Appendix E3: Neutrino Masses and PMNS Mixing from $\Theta$ Seesaw}
\label{app:neutrino_masses_seesaw}

\subsection{E3.1 Motivation}

The charged fermion mass framework developed in Appendix~\ref{app:fermion_masses_theta} extends naturally to neutrinos via the Type-I seesaw mechanism. This appendix derives:
\begin{enumerate}
\item Neutrino Dirac masses from $\Theta$-invariant Yukawa couplings
\item Majorana mass scale $M_R$ and its relation to $M_\Theta$
\item Light neutrino masses via canonical seesaw formula
\item PMNS mixing matrix from texture structure
\item Predictions for normal vs.\ inverted ordering
\end{enumerate}

\subsection{E3.2 Type-I Seesaw Framework}

Introduce right-handed neutrino singlets $N_R$ with Majorana mass term:
\begin{equation}
\mathcal{L}_{\text{mass}}^{\nu} = -\frac{1}{2} \overline{N_R^c} M_R N_R - \bar{L}_L \tilde{H} Y_\nu N_R + \text{h.c.}
\label{eq:seesaw_lagrangian}
\end{equation}

where:
\begin{itemize}
\item $L_L = (\nu_L, e_L)^T$ is the left-handed lepton doublet
\item $\tilde{H} = i\sigma_2 H^*$ is the conjugate Higgs doublet
\item $Y_\nu$ is the Dirac Yukawa matrix (to be derived from $\Theta$)
\item $M_R$ is the right-handed Majorana mass matrix
\end{itemize}

After electroweak symmetry breaking with $\langle H \rangle = v/\sqrt{2}$, the Dirac mass matrix is:
\begin{equation}
M_D = \frac{v}{\sqrt{2}} Y_\nu
\label{eq:dirac_mass}
\end{equation}

\subsection{E3.3 Canonical Seesaw Formula}

In the limit $M_R \gg M_D$, integrating out the heavy right-handed neutrinos yields light neutrino masses:
\begin{equation}
M_\nu = -M_D^T M_R^{-1} M_D
\label{eq:seesaw_formula}
\end{equation}

This is the canonical Type-I seesaw mechanism. The small neutrino masses arise naturally from the hierarchy $M_D \sim \mathcal{O}(100 \text{ GeV}) \ll M_R \sim \mathcal{O}(10^{14} \text{ GeV})$.

\subsection{E3.4 $\Theta$-Induced Dirac Yukawa Structure}

Following the formalism of Appendix~\ref{app:fermion_masses_theta}, the neutrino Dirac Yukawa matrix is constructed from $\Theta$-invariants:

\begin{equation}
Y_\nu = y_\nu^{(0)} \, \Big( a_1 X + a_2 Q + a_3 \mathcal{K} \Big) \cdot \Big(\mathbf{1} + \epsilon_\nu T_\nu\Big)
\label{eq:neutrino_yukawa}
\end{equation}

where:
\begin{itemize}
\item $y_\nu^{(0)}$ is a universal neutrino coupling (can differ from charged fermions)
\item $a_i$ are the same fixed coefficients as in Eq.~(E2.11)
\item $\epsilon_\nu \ll 1$ is a small flavor-breaking parameter
\item $T_\nu$ is a texture matrix paralleling the charged lepton sector
\end{itemize}

The Dirac mass matrix becomes:
\begin{equation}
M_D = M_\Theta^{(\nu)} \, Y_\nu, \qquad M_\Theta^{(\nu)} \equiv \frac{v}{\sqrt{2}} y_\nu^{(0)} (a_1 c_X + a_2 c_Q + a_3 c_K)
\label{eq:M_D_theta}
\end{equation}

\subsection{E3.5 Neutrino Flavor Texture}

We adopt a texture structure analogous to charged leptons but with different hierarchy parameters:

\begin{equation}
T_\nu = \begin{pmatrix}
0 & \varepsilon_\nu & 0 \\
\varepsilon_\nu & \delta_\nu & \eta_\nu \\
0 & \eta_\nu & 1
\end{pmatrix}
\label{eq:neutrino_texture}
\end{equation}

\paragraph{Key differences from charged leptons:}
\begin{itemize}
\item Smaller hierarchy: $\varepsilon_\nu \sim \mathcal{O}(0.1-0.3)$ vs.\ $\varepsilon_e \sim 0.05$
\item Larger mixing: $|\eta_\nu| \sim \mathcal{O}(0.2-0.5)$ to generate atmospheric mixing
\item Different $\delta_\nu$ to control solar mixing angle
\end{itemize}

\subsection{E3.6 Majorana Mass Scale}

For minimality, we take $M_R$ diagonal:
\begin{equation}
M_R = \text{diag}(M_1, M_2, M_3)
\label{eq:M_R_diagonal}
\end{equation}

\paragraph{Hierarchy ansatz:}
\begin{itemize}
\item \textbf{Degenerate case:} $M_1 \approx M_2 \approx M_3 \equiv M_R$ (simplest)
\item \textbf{Hierarchical case:} $M_1 : M_2 : M_3 \sim 1 : \lambda : \lambda^2$ with $\lambda \sim 3-10$
\end{itemize}

The degenerate case is adopted as the baseline, with $M_R$ as a single free parameter.

\subsection{E3.7 Light Neutrino Mass Matrix}

Substituting Eqs.~\eqref{eq:M_D_theta} and \eqref{eq:M_R_diagonal} into the seesaw formula \eqref{eq:seesaw_formula}:

\begin{equation}
M_\nu = -\frac{(M_\Theta^{(\nu)})^2}{M_R} \, Y_\nu^T Y_\nu
\label{eq:light_neutrino_mass}
\end{equation}

For the texture matrix \eqref{eq:neutrino_texture}, the product $Y_\nu^T Y_\nu$ yields:

\begin{equation}
Y_\nu^T Y_\nu \approx y_\nu^2 \begin{pmatrix}
\varepsilon_\nu^2 & \varepsilon_\nu \delta_\nu & \varepsilon_\nu \eta_\nu \\
\varepsilon_\nu \delta_\nu & \delta_\nu^2 + \eta_\nu^2 & \delta_\nu \eta_\nu + \eta_\nu \\
\varepsilon_\nu \eta_\nu & \delta_\nu \eta_\nu + \eta_\nu & \eta_\nu^2 + 1
\end{pmatrix}
\label{eq:YnuT_Ynu}
\end{equation}

where $y_\nu = a_1 c_X + a_2 c_Q + a_3 c_K$.

\subsection{E3.8 Mass Eigenvalues (Leading Order)}

Diagonalizing $M_\nu$ to leading order in small parameters:

\begin{align}
m_1 &\approx m_\nu \, \varepsilon_\nu^2, \label{eq:m1_nu} \\
m_2 &\approx m_\nu \, (\delta_\nu^2 + \eta_\nu^2)^{1/2}, \label{eq:m2_nu} \\
m_3 &\approx m_\nu \, (1 + \eta_\nu^2), \label{eq:m3_nu}
\end{align}

where the overall scale is:
\begin{equation}
m_\nu \equiv \frac{(M_\Theta^{(\nu)})^2}{M_R} y_\nu^2
\label{eq:neutrino_scale}
\end{equation}

\paragraph{Numerical estimate:}
For $M_\Theta^{(\nu)} \sim 100$ GeV, $y_\nu \sim 1$, and $M_R \sim 10^{14}$ GeV:
\begin{equation}
m_\nu \sim \frac{(100 \text{ GeV})^2 \times 1}{10^{14} \text{ GeV}} \sim 0.1 \text{ eV}
\end{equation}
consistent with atmospheric neutrino oscillations.

\subsection{E3.9 PMNS Mixing Matrix}

The PMNS matrix arises from the mismatch between charged lepton and neutrino mass bases:
\begin{equation}
U_{\text{PMNS}} = U_e^\dagger U_\nu
\label{eq:PMNS}
\end{equation}

where $U_e$ diagonalizes the charged lepton mass matrix and $U_\nu$ diagonalizes the neutrino mass matrix.

\paragraph{Leading-order angles:}
From the texture structure, the mixing angles are approximately:
\begin{align}
\sin^2\theta_{12} &\approx \frac{\delta_\nu^2}{\delta_\nu^2 + \eta_\nu^2 + 1} \sim 0.3, \label{eq:theta12} \\
\sin^2\theta_{23} &\approx \frac{\eta_\nu^2}{2} \sim 0.5, \label{eq:theta23} \\
\sin^2\theta_{13} &\approx \varepsilon_\nu^2 \sim 0.02. \label{eq:theta13}
\end{align}

These can be compared with experimental values:
\begin{align}
\sin^2\theta_{12}^{\text{exp}} &= 0.307 \pm 0.013, \\
\sin^2\theta_{23}^{\text{exp}} &= 0.546 \pm 0.021, \\
\sin^2\theta_{13}^{\text{exp}} &= 0.0220 \pm 0.0007.
\end{align}

\subsection{E3.10 Mass Ordering Predictions}

The sign of the mass-squared differences determines the ordering:

\paragraph{Normal ordering (NO):} $m_1 < m_2 < m_3$
\begin{equation}
\Delta m_{21}^2 \equiv m_2^2 - m_1^2 \approx m_\nu^2 \, \delta_\nu^2 \sim 7.5 \times 10^{-5} \text{ eV}^2
\end{equation}
\begin{equation}
\Delta m_{31}^2 \equiv m_3^2 - m_1^2 \approx m_\nu^2 \sim 2.5 \times 10^{-3} \text{ eV}^2
\end{equation}

\paragraph{Inverted ordering (IO):} $m_3 < m_1 < m_2$
This requires a different texture structure or negative eigenvalues in $M_R$, which is disfavored in the minimal setup.

\textbf{Prediction:} The minimal $\Theta$-induced texture favors \textbf{normal ordering}.

\subsection{E3.11 CP Violation Phase}

The Dirac CP phase $\delta_{\text{CP}}$ arises from complex entries in the texture matrix. In the minimal real texture \eqref{eq:neutrino_texture}, $\delta_{\text{CP}} = 0$ or $\pi$.

To generate non-trivial CP violation, one can:
\begin{enumerate}
\item Allow complex phases in $\epsilon_\nu$ or $\eta_\nu$
\item Introduce a second-order correction from $\Theta$-field phase dynamics
\item Use Majorana phases in $M_R$ (if non-diagonal)
\end{enumerate}

\textbf{Prediction:} Measuring $\delta_{\text{CP}}$ tests whether $\Theta$-induced phases are significant.

\subsection{E3.12 Parameter Counting}

\paragraph{Free parameters (neutrino sector):}
\begin{itemize}
\item $M_R$: Majorana scale (1 parameter)
\item $y_\nu^{(0)}$: Dirac coupling normalization (absorbed into $M_R$)
\item $\varepsilon_\nu, \delta_\nu, \eta_\nu$: Texture parameters (3 parameters)
\end{itemize}
\textbf{Total:} 4 parameters

\paragraph{Observables:}
\begin{itemize}
\item 2 mass-squared differences: $\Delta m_{21}^2, \Delta m_{31}^2$
\item 3 mixing angles: $\theta_{12}, \theta_{23}, \theta_{13}$
\item 1 CP phase: $\delta_{\text{CP}}$ (if complex texture)
\end{itemize}
\textbf{Total:} 6 observables

\textbf{Predictivity:} 6 observables - 4 parameters = \textbf{2 testable relations}

\subsection{E3.13 Sum Rules (Neutrino Sector)}

\paragraph{Sum Rule 1 (mass-angle correlation):}
\begin{equation}
\frac{\Delta m_{21}^2}{\Delta m_{31}^2} \approx \sin^2\theta_{12} \sim 0.3
\label{eq:neutrino_sumrule1}
\end{equation}

\paragraph{Sum Rule 2 (mixing hierarchy):}
\begin{equation}
\sin\theta_{13} \approx \varepsilon_\nu \sim \sqrt{\frac{m_1}{m_3}} \sim 0.15
\label{eq:neutrino_sumrule2}
\end{equation}

\subsection{E3.14 Connection to Charged Lepton Sector}

The neutrino texture parameters can be related to charged lepton parameters through:
\begin{equation}
\frac{\varepsilon_\nu}{\varepsilon_e} \sim \mathcal{O}(2-5), \qquad \frac{\eta_\nu}{\eta_e} \sim \mathcal{O}(3-5)
\label{eq:lepton_connection}
\end{equation}

This reflects the larger mixing in the neutrino sector compared to charged leptons.

\subsection{E3.15 Summary}

This appendix has established:
\begin{enumerate}
\item Neutrino Dirac masses from $\Theta$-invariant Yukawa couplings
\item Type-I seesaw mechanism with minimal Majorana scale $M_R$
\item Light neutrino masses with texture-induced hierarchies
\item PMNS mixing angles from texture structure
\item Prediction of normal mass ordering
\item 2 testable sum rules connecting masses and mixing angles
\end{enumerate}

Combined with Appendix~\ref{app:fermion_masses_theta}, the UBT framework provides a unified description of all fermion masses with:
\begin{itemize}
\item \textbf{14 free parameters} (1 scale $M_\Theta$, 1 Majorana scale $M_R$, 12 dimensionless texture parameters)
\item \textbf{19 observables} (9 masses + 4 CKM + 2 neutrino $\Delta m^2$ + 3 PMNS angles + 1 $\delta_{\text{CP}}$)
\item \textbf{5 testable predictions} (sum rules and texture relations)
\end{itemize}

\subsection{E3.16 References}

\begin{itemize}
\item Appendix~\ref{app:fermion_masses_theta}: Charged fermion mass derivation
\item Appendix~\ref{app:biquaternion_inner_product}: Normalization conventions
\item Implementation: \texttt{/scripts/fit\_flavour\_minimal.py} (neutrino sector extension)
\end{itemize}
