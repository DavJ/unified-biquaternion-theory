
\documentclass[12pt]{article}
\usepackage{amsmath,amssymb}
\usepackage{geometry}
\geometry{margin=1in}
\title{Solution to Priority P2: Deriving the Electron from the Unified Biquaternion Field}
\author{Unified Biquaternion Theory Team}
\date{\today}

\begin{document}

\maketitle

\section*{Objective}

To demonstrate how the electron, with correct quantum numbers (mass, charge, spin), emerges as a solution or mode of the unified biquaternionic field equation:
\[
\Box \Theta(q, \tau) + \mathcal{N}(\Theta) = 0
\]

\section*{1. Structure of the Unified Field}

We define the total field:
\[
\Theta(q, \tau) \in \mathbb{B}^{4 \times 4}
\]
with components:
\[
\Theta(q, \tau) = \Theta_e(q, \tau) + \Theta_g(q, \tau) + \cdots
\]
where $\Theta_e$ is the electron mode.

\section*{2. Ansatz for the Electron Mode}

Let us define the electron excitation as:
\[
\Theta_e(q, \tau) = \psi(q) \otimes s
\]
where $\psi(q)$ is a Dirac spinor and $s$ is a fixed internal vector in $\mathbb{B}^{4}$.

Assume time-dependence of the form:
\[
\psi(q) = u(p) e^{-i \omega \tau}
\]
This satisfies:
\[
i \partial_\tau \psi = \omega \psi \quad \Rightarrow \quad m = \frac{\hbar \omega}{c^2}
\]

\section*{3. Mass and Spin from the Unified Equation}

The field $\Theta_e$ obeys a projected equation:
\[
\Box \Theta_e + m^2 \Theta_e = 0
\]
and satisfies spin-$\frac{1}{2}$ algebra through commutators of its components:
\[
[\Theta^i, \Theta^j] \sim i \epsilon^{ijk} \Theta^k
\]
implying intrinsic angular momentum (spin).

\section*{4. Charge Quantization}

The coupling of $\Theta_e$ to the EM projection $\Theta_{\text{em}}$ yields:
\[
j^\mu = \bar{\psi} \gamma^\mu \psi
\]
consistent with the standard QED current.

\section*{5. Geometric Embedding}

The excitation $\Theta_e$ contributes to the stress-energy tensor:
\[
T_{\mu\nu} = \frac{1}{2} \Re \left( \partial_\mu \Theta_e^\dagger \partial_\nu \Theta_e \right)
\]
which sources the gravitational field in the Einstein equation.

\section*{Conclusion}

The electron appears as a harmonic excitation of the unified biquaternion field with:
\begin{itemize}
    \item Correct mass generation via internal time oscillation.
    \item Spin-$\frac{1}{2}$ behavior from algebraic structure.
    \item Electromagnetic coupling via projection.
    \item Gravitational interaction via stress-energy contribution.
\end{itemize}

This strongly supports the feasibility of UBT as a unification framework.


\section*{License}
This work is licensed under a Creative Commons Attribution 4.0 International License (CC BY 4.0).

\end{document}