
\documentclass[12pt,a4paper]{article}
\usepackage{amsmath,amssymb}
\usepackage[utf8]{inputenc}
\usepackage{geometry}
\geometry{margin=1in}

\title{Symbolic Derivation of Electric Charge in UBT}
\author{UBT Research Team}
\date{\today}

\begin{document}
\maketitle

\section{Starting Point}

We begin with the Lagrangian density for the $\Theta$ field in the Unified Biquaternion Theory (UBT), emphasizing the covariant derivative which includes gauge interactions:
\begin{equation}
    \mathcal{L} = \langle D_\mu \Theta, D^\mu \Theta \rangle - V(\Theta)
\end{equation}
with
\begin{equation}
    D_\mu = \partial_\mu + \omega_\mu^R + i g_0 A_\mu,
\end{equation}
where the imaginary part encodes the electromagnetic interaction through the gauge field $A_\mu$, and $g_0$ is the bare coupling constant to be identified with the electric charge $e$.

\section{Phase-Amplitude Decomposition}

We write the field $\Theta$ as:
\begin{equation}
    \Theta(q) = \rho(q) e^{i \phi(q)}
\end{equation}
Then
\begin{equation}
    \partial_\mu \Theta = \left( \partial_\mu \rho + i \rho \partial_\mu \phi \right) e^{i\phi}
\end{equation}

\section{Interaction Term}

Expanding the interaction term to leading order in $A_\mu$:
\begin{align}
    \mathcal{L}_{\text{int}} &= \langle \partial_\mu \Theta, i g_0 A^\mu \Theta \rangle + \langle i g_0 A_\mu \Theta, \partial^\mu \Theta \rangle \\
    &= i g_0 A^\mu \left( \langle \partial_\mu \Theta, \Theta \rangle - \langle \Theta, \partial_\mu \Theta \rangle \right)
\end{align}
Evaluating this:
\begin{equation}
    \langle \partial_\mu \Theta, \Theta \rangle = \rho \partial_\mu \rho + i \rho^2 \partial_\mu \phi,
    \quad
    \langle \Theta, \partial_\mu \Theta \rangle = \rho \partial_\mu \rho - i \rho^2 \partial_\mu \phi
\end{equation}
Thus:
\begin{equation}
    \mathcal{L}_{\text{int}} = i g_0 A^\mu (2 i \rho^2 \partial_\mu \phi) = -2 g_0 A^\mu \rho^2 \partial_\mu \phi
\end{equation}

\section{Identification of the Electric Current}

Compare with the standard QED interaction term $-e A_\mu j^\mu$, we identify:
\begin{equation}
    j^\mu = 2 \rho^2 \partial^\mu \phi, \quad \text{and} \quad e = g_0
\end{equation}

\section{Quantization Hypothesis}

We propose the quantization condition for the topological phase:
\begin{equation}
    \oint \partial_\mu \phi \, dx^\mu = 2\pi n \quad \Rightarrow \quad \int j^\mu dx_\mu = 4\pi \rho^2 n
\end{equation}
This implies that the total charge is quantized:
\begin{equation}
    Q = \int j^0 d^3x = 4\pi \rho^2 n
\end{equation}
Hence $e = g_0$ must be such that the resulting charge matches observed values. A precise topological mechanism may enforce $g_0 = e = \sqrt{\frac{4\pi}{n}}$ in natural units.

\section{Conclusion}

We symbolically derived the expression for the electric current and charge in UBT, grounded in the internal phase structure of the $\Theta$ field. Next steps involve:
\begin{itemize}
    \item Solving topologically constrained field configurations.
    \item Determining the winding number $n$ leading to $e \approx 1/137$.
\end{itemize}

\end{document}
