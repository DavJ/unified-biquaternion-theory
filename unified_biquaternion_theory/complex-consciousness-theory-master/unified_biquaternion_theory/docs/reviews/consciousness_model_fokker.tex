
\documentclass{article}
\usepackage{amsmath}
\begin{document}

\section*{Analýza Fokker-Planckovy Rovnice}

Fokker-Planckova rovnice pro pole $\chi$ v potenciálu $V(\chi)$ má tvar:
\[
\frac{\partial P}{\partial t} = \frac{\partial}{\partial \chi} \left( \frac{\partial V}{\partial \chi} P \right) + D \frac{\partial^2 P}{\partial \chi^2}.
\]

\subsection*{Stacionární řešení}
Pro $\partial P/\partial t = 0$:
\[
\frac{\partial}{\partial \chi} \left( \frac{\partial V}{\partial \chi} P \right) + D \frac{\partial^2 P}{\partial \chi^2} = 0.
\]
Řešení má tvar:
\[
P_{\text{stat}}(\chi) = \mathcal{N} \exp\left(-\frac{V(\chi)}{D}\right).
\]

\subsection*{Interpretace}
Maxima $P_{\text{stat}}(\chi)$ odpovídají stabilním mentálním stavům – systém s nejvyšší pravděpodobností osciluje mezi dvěma stavy podle hloubky potenciálových jamek a šumu $D$.


\section*{Author's Note}

This work was developed solely by Ing. David Jaroš.  
Large language models (ChatGPT-4o by OpenAI and Gemini 2.5 Pro by Google) were used strictly as assistive tools for calculations, LaTeX formatting, and critical review.  
All core ideas, equations, theoretical constructs and conclusions are the intellectual work of the author.

\end{document}

