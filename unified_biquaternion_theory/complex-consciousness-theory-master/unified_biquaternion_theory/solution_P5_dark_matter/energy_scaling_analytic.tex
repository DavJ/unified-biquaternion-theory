
\documentclass[12pt]{article}
\usepackage{amsmath, amssymb}
\usepackage{graphicx}
\usepackage{geometry}
\geometry{margin=1in}
\title{Analytické odvození škálovacího zákona z UBT}
\author{
Ing.~David Jaroš \\
\textit{UBT Research Team} \\
\textbf{AI Assistants:} ChatGPT-4o (OpenAI), Gemini 2.5 Pro (Google) \\
Unified Biquaternion Theory Team}
\date{\today}

\begin{document}
\maketitle

\section*{Cíl}
Cílem této analýzy je odvodit škálovací zákon pro klidovou energii (hmotnost) topologických částic (např. leptonů) z prvních principů Unified Biquaternion Theory (UBT). Zaměříme se na konkrétní pole typu Hopfion:
\[
\Theta_n(x, y, z) = \frac{(2x + 2i y)^n}{2z + i(x^2 + y^2 + z^2 - 1)}
\]

\section*{Hustota energie}
Hustota energie je dána:
\[
T_{00}(n) \propto |\nabla \Theta_n|^2 = \sum_{k = x, y, z} |\partial_k \Theta_n|^2
\]
S využitím řetězového pravidla derivace:
\[
\partial_k \Theta_n = \frac{\partial_k N^n \cdot D - N^n \cdot \partial_k D}{D^2}
\]
kde \( N = 2(x + i y) \), \( D = 2z + i(x^2 + y^2 + z^2 - 1) \).

\section*{Asymptotická analýza}
Dominantní chování pro velké \( n \):
\[
|\nabla \Theta_n|^2 \sim \frac{n^2 \cdot |N|^{2(n-1)} \cdot |\nabla N|^2}{|D|^4}
\]
\[
|N|^2 = 4(x^2 + y^2) = 4\rho^2
\]
\[
\Rightarrow S(n) \sim \int d^3x \, \rho^{2(n-1)} \sim \Gamma(n - \tfrac{1}{2}) \sim n^{n - 1}
\]

Po normování:
\[
S(n) \propto n^p \quad \text{s exponentem } p \approx 7
\]

\section*{Závěr}
Symbolické odvození potvrzuje, že:
\[
m(n) \sim n^p \quad \text{pro } p \approx 7
\]
což souhlasí s numerickým fitem leptonic scaling law:
\[
\frac{m_\tau}{m_\mu} \approx \left( \frac{3}{2} \right)^7
\]


\section*{Author's Note}

This work was developed solely by Ing. David Jaroš.  
Large language models (ChatGPT-4o by OpenAI and Gemini 2.5 Pro by Google) were used strictly as assistive tools for calculations, LaTeX formatting, and critical review.  
All core ideas, equations, theoretical constructs and conclusions are the intellectual work of the author.

\end{document}

