\documentclass{article}
\usepackage[utf8]{inputenc}
\title{P4 – Odvození jemné struktury z bikvaternionové teorie}
\author{
Ing.~David Jaroš \\
\textit{UBT Research Team} \\
\textbf{AI Assistants:} ChatGPT-4o (OpenAI), Gemini 2.5 Pro (Google) \\
}
\date{}

\begin{document}
\maketitle

\section*{Cíl}
Odvodit jemnou strukturu (fine-structure constant) $\alpha$ čistě z principů bikvaternionové teorie pole $\Theta(q, \tau)$.

\section*{Pozadí}
Většina teorií považuje $\alpha$ za empirickou konstantu, avšak v rámci UBT lze předpokládat, že $\alpha$ vzniká jako poměr frekvence vnitřního módu, škálovací konstanty a geometrických faktorů toroidální struktury.

\section*{Zadání}
Ukázat, že poměr $\alpha = \frac{e^2}{\hbar c}$ může být v této teorii odvozen symbolicky.


\section*{Author's Note}

This work was developed solely by Ing. David Jaroš.  
Large language models (ChatGPT-4o by OpenAI and Gemini 2.5 Pro by Google) were used strictly as assistive tools for calculations, LaTeX formatting, and critical review.  
All core ideas, equations, theoretical constructs and conclusions are the intellectual work of the author.

\end{document}

