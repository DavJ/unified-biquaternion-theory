\documentclass[12pt]{article}
\usepackage{amsmath,amssymb}
\title{P1 – Interpretation and Solutions of the Scalar Equation}
\author{
Ing.~David Jaroš \\
\textit{UBT Research Team} \\
\textbf{AI Assistants:} ChatGPT-4o (OpenAI), Gemini 2.5 Pro (Google) \\
Ing. David Jaroš}
\date{\today}
\begin{document}
\maketitle

\section*{Goal}
Explore the physical and mathematical meaning of the imaginary scalar equation derived from the unified field equation $\Theta(q, \tau)$.

\section*{Equation of Interest}
From the scalar-imaginary projection:
\[
\Im[\text{Sc}(\mathcal{D}_q \Theta(q, \tau))] = 0
\]

\section*{Strategy}
\begin{itemize}
\item Analyze its form under spherical symmetry and FRW cosmology.
\item Test if this reduces to a constraint on topology or yields new scalar fields.
\item Examine links to axion/dilaton or inflaton-like dynamics.
\end{itemize}

\section*{Next Steps}
\begin{itemize}
\item Write $\Theta$ under spherical symmetry.
\item Perform symbolic computation of imaginary scalar part.
\item Solve for specific boundary conditions.
\end{itemize}


\section*{Author's Note}

This work was developed solely by Ing. David Jaroš.  
Large language models (ChatGPT-4o by OpenAI and Gemini 2.5 Pro by Google) were used strictly as assistive tools for calculations, LaTeX formatting, and critical review.  
All core ideas, equations, theoretical constructs and conclusions are the intellectual work of the author.

\end{document}
