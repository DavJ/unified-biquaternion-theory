\documentclass[12pt]{article}
\usepackage{amsmath,amssymb}
\title{P2 – Electron Model from the $\Theta$ Field}
\author{
Ing.~David Jaroš \\
\textit{UBT Research Team} \\
\textbf{AI Assistants:} ChatGPT-4o (OpenAI), Gemini 2.5 Pro (Google) \\
Ing. David Jaroš}
\date{\today}
\begin{document}
\maketitle

\section*{Goal}
Demonstrate how a minimal electron-like solution can emerge from the internal structure of the unified $\Theta(q,\tau)$ field.

\section*{Approach}
\begin{itemize}
\item Use the internal spinor/tensor decomposition of $\Theta$.
\item Map quantum numbers (charge, spin, mass) to components.
\item Attempt derivation of mass term analogous to Dirac field in curved space.
\end{itemize}

\section*{Expected Outcome}
A plausible geometric derivation of electron properties as a topological excitation in $\Theta(q, \tau)$.


\section*{Author's Note}

This work was developed solely by Ing. David Jaroš.  
Large language models (ChatGPT-4o by OpenAI and Gemini 2.5 Pro by Google) were used strictly as assistive tools for calculations, LaTeX formatting, and critical review.  
All core ideas, equations, theoretical constructs and conclusions are the intellectual work of the author.

\end{document}
