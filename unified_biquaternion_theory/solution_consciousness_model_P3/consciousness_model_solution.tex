
\documentclass[12pt]{article}
\usepackage{amsmath,amssymb}
\usepackage{geometry}
\usepackage{graphicx}
\usepackage{hyperref}
\geometry{margin=1in}

\title{Solution to Priority P3: Minimal Toy Model of Consciousness in Unified Biquaternion Theory}
\author{David Jaroš}
\date{\today}

\begin{document}
\maketitle

\section*{Abstract}
This document proposes and analyzes a minimal toy model for the projection of conscious dynamics from the unified biquaternionic field \(\Theta(q, \tau)\), in accordance with Priority P3 from the opponent review. The model demonstrates how bistable perception and binary decisions may arise from oscillatory dynamics in the imaginary part of the field, under constraints related to Free Energy minimization.

\section{Theoretical Background}
In the Unified Biquaternion Theory (UBT), consciousness is encoded in the imaginary spinor component \(\chi(q)\) of the field \(\Theta(q, \tau) = \phi(q) + i \chi(q)\), where \(q\) is a biquaternionic coordinate and \(\tau = t + i \psi\) is complex time.

Following the complex Fokker–Planck equation and variational principle derived in previous documents, we assume that the field \(\chi(q)\) evolves in a way that tends to minimize an effective informational Free Energy functional \(F[\chi]\), subject to constraints induced by projections to spacetime and interactions with the scalar part \(\phi(q)\).

\section{Minimal Toy Model: Bistable Dynamics}
We propose a simplified model:
\begin{itemize}
    \item Let \(x \in \mathbb{R}\) represent a single observable perceptual variable (e.g. face A vs face B).
    \item Let \(\chi(x,t)\) evolve according to a bistable potential \(V(x)\) with two minima.
\end{itemize}

The dynamics are given by:
\[
\frac{\partial \chi}{\partial t} = -\frac{\delta F[\chi]}{\delta \chi}, \quad
F[\chi] = \int dx \left[ \frac{1}{2} \left( \frac{d\chi}{dx} \right)^2 + V(\chi(x)) \right]
\]

A canonical choice for the potential is:
\[
V(\chi) = \frac{1}{4} \chi^4 - \frac{1}{2} \chi^2
\]

This leads to spontaneous switching between two "perception states" \(\chi \approx \pm 1\), modulated by internal stochastic or deterministic oscillations in the imaginary time direction.

\section{Interpretation}
This minimal model demonstrates that:
\begin{itemize}
    \item A spinor component \(\chi(q)\) evolving under a Free Energy principle can exhibit bistable cognitive dynamics.
    \item Switching between attractors corresponds to phenomenological perceptual changes (e.g. face/vase illusion).
    \item This process emerges naturally from the structure of UBT, without ad hoc postulates.
\end{itemize}

\section{Next Steps}
Future refinements may include:
\begin{itemize}
    \item Extending to multiple perceptual dimensions.
    \item Coupling \(\chi(q)\) to classical field \(\phi(q)\).
    \item Simulating actual perceptual processes in biological-like networks.
\end{itemize}


\section*{License}
This work is licensed under a Creative Commons Attribution 4.0 International License (CC BY 4.0).

\end{document}