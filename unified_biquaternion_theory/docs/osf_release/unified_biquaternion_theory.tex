
\documentclass[12pt, a4paper]{article}
\usepackage[utf8]{inputenc}
\usepackage[english]{babel}
\usepackage{amsmath, amssymb}
\usepackage{geometry}
\usepackage{graphicx}
\usepackage{hyperref}
\usepackage{siunitx}

\geometry{a4paper, margin=1in}
\linespread{1.15}

\title{\textbf{The Unified Biquaternion Theory: A Framework for Fundamental Constants, Dark Matter, and Mass Hierarchy}}
\author{Ing. David Jaroš\thanks{Primary author and theorist. Contact: jdavid.cz@gmail.com} \\
\textit{with AI assistance from ChatGPT-4o and Gemini 2.5 Pro (as technical aides)}}
\date{June 30, 2025}

\begin{document}
\maketitle

\begin{abstract}
We present the Unified Biquaternion Theory (UBT), a theoretical framework based on a complexified five-dimensional spacetime and a fundamental biquaternion-valued field \( \Theta(q, \tau) \). UBT aims to reconcile quantum field theory and general relativity by providing natural derivations for fundamental constants and particle properties from the geometry and topology of the underlying manifold. The theory predicts the emergence of known particles as topological excitations and introduces new classes of fields associated with consciousness and dark matter. We show that UBT reduces to known physical theories in their respective limits, while offering falsifiable predictions beyond current physics. This paper summarizes the theory's core tenets and its successful application to three major unsolved problems: the origin of the fine-structure constant, the nature of dark matter, and the lepton mass hierarchy.
\end{abstract}

\tableofcontents
\newpage

\section{Foundations of the Unified Biquaternion Theory}
UBT is defined on a complexified 5D manifold with coordinates \( q = (x^\mu, \psi) \), where \( x^\mu \in \mathbb{R}^{1,3} \) are the usual spacetime coordinates, and \( \psi \in \mathbb{R} \) encodes a phase-like internal dimension, associated with consciousness and entropy. Time is complexified as \( \tau = t + i\psi \). The central object of the theory is a biquaternion-valued field \( \Theta(q, \tau) \in \mathbb{B} \otimes \mathbb{C} \), which acts as a universal precursor to all quantum fields and gravitational interactions. The field's dynamics are governed by a generalized covariant equation of motion resembling a complexified Dirac–Klein–Gordon system. Quantized excitations of \( \Theta \) give rise to standard particles, while stable, solitonic topological configurations correspond to massive fermions, bosons, or dark matter candidates. The full development and mathematical formalism are available in the author's open-source repository: \url{https://github.com/DavJ/complex-consciousness-theory.git}.

\section{Metric Structure and Emergent Spacetime}
UBT recovers General Relativity in the classical limit. The effective metric tensor \( g_{\mu\nu} \) of spacetime is not fundamental but emerges from the correlations of the underlying \( \Theta \) field. The Einstein field equations are shown to be the low-energy limit of the UBT field equations when projected onto the real subspace of the manifold.

\section{Topological Quantization and the Fine-Structure Constant}
The Standard Model treats the fine-structure constant, \( \alpha \approx 1/137 \), as an empirical parameter. UBT derives its value from two fundamental principles. First, a topological quantization condition on the phase component of the \( \Theta \) field restricts the value of alpha to the inverse of an integer: \( \alpha = 1/n \). Second, a two-stage selection mechanism determines the value of `n`. A principle of minimal spectral entropy selects prime numbers as the only candidates for stable vacuum states. Subsequently, the Principle of Least Action, applied to an effective energy function for these states, \( S(n) \approx A n^2 - B n \ln(n) \), selects \( n=137 \) as a prominent local energy minimum. This yields a prediction for the "bare" value, \( \alpha_0 = 1/137 \). The small deviation from the precise experimental value is accounted for by standard QFT corrections (running of the coupling constant), for which UBT provides the fundamental boundary condition.

\section{Derivation of Particle Properties: The Lepton Mass Hierarchy}
UBT provides a model for the origin of mass and the hierarchy between particle generations. We postulate that the three lepton generations correspond to topological states (Hopfions) with increasing complexity, characterized by a topological number \(n=1, 2, 3\). The mass of these states is determined by a dual-mechanism:
\begin{equation}
    m_n^{\text{phys}} = m_{\text{topo}}(n) + \delta m_{\text{EM}}(n)
\end{equation}
The topological mass, \( m_{\text{topo}}(n) \), dominates for the heavier generations (muon and tau) and is described by a universal scaling function \( S(n) = A n^p - B n \ln(n) \). The electron's mass, however, is a composite of a small topological contribution and a dominant electromagnetic self-energy correction, \( \delta m_{\text{EM}} \). With a single set of parameters (\( A \approx 0.62 \), \( B \approx -8.95 \), \( p \approx 7.23 \)), the theory makes precise predictions:

\begin{table}[h!]
\centering
\caption{Comparison of UBT Predicted vs. Experimental Lepton Masses}
\label{tab:masses}
\begin{tabular}{l|c|c|c}
\hline
\textbf{Lepton} & \textbf{UBT Prediction [MeV]} & \textbf{Experiment [MeV]} & \textbf{Relative Error} \\
\hline
Electron (\(n=1\)) & 0.5110 & 0.51099895 & \(< 0.001\%\) \\
Muon (\(n=2\)) & 105.66 & 105.65837 & \(< 0.002\%\) \\
Tauon (\(n=3\)) & 1776.86 & 1776.86 \(\pm\) 0.12 & \(< 0.001\%\) \\
\hline
\end{tabular}
\end{table}

\section{Predictions of New Topological Fields: Dark Matter}
UBT offers a solution to the dark matter puzzle without new particles. We propose that dark matter halos are composed of topologically stable, electromagnetically neutral configurations of the \( \Theta \) field. Our analytical models show that the gravitational potential generated by such a distributed topological defect naturally produces the flat galactic rotation curves observed in galaxies.

\section{Compatibility with Established Theories}
The Unified Biquaternion Theory (UBT) is constructed to be consistent with — and to generalize — the major pillars of modern physics:

\subsection{General Relativity (GR)}
In the classical limit \( \hbar \to 0 \), the bilinear forms of the \( \Theta(q, \tau) \) field induce an effective metric:
\[
g_{\mu\nu}(x) = \langle \bar{\Theta} \gamma_\mu \gamma_\nu \Theta \rangle
\]
Variational principles yield:
\[
G_{\mu\nu} = 8\pi G\, T_{\mu\nu}^{(\Theta)}
\]

\subsection{Quantum Electrodynamics (QED)}
Gauge invariance in the complex phase \( \psi \) implies:
\[
\Theta \mapsto e^{i e \lambda(q)} \Theta,\quad \partial_\mu \to D_\mu = \partial_\mu + i e A_\mu
\]

\subsection{Quantum Chromodynamics (QCD)}
The internal algebra supports \( SU(3) \) gauge symmetry via:
\[
D_\mu = \partial_\mu + i g_s T^a G_\mu^a
\]
Color confinement arises from boundary topology of \( \Theta \).

\subsection{Standard Model (SM)}
Fermionic modes decompose under \( SU(3)_C \times SU(2)_L \times U(1)_Y \). Generational structure arises from topological charge \( n \), and the mass formula reproduces empirical lepton masses:
\[
m_n^{\text{phys}} = A n^p - B n \ln n + \delta m_{\text{EM}}
\]

\section{Outlook and Experimental Implications}
Future work includes applying the theory to the quark sector, mixing matrices, and gravitational anomalies. Experimental falsifiability includes precision tests of the fine-structure constant evolution, gravitational lensing by topological structures, and mapping phase-space structure of vacuum entropy in condensed-matter analogs.

\begin{thebibliography}{9}

\bibitem{Jaros2025a}
David Jaroš (2025). "A Complex-Time Theory of Consciousness: Drift, Diffusion, and Phase Collapse in Toroidal Cognitive Space". \textit{OSF Preprint}.

\bibitem{Peskin1995}
Michael E. Peskin and Daniel V. Schroeder (1995). \textit{An Introduction to Quantum Field Theory}. Addison-Wesley.

\bibitem{Penrose2004}
Roger Penrose (2004). \textit{The Road to Reality}. Jonathan Cape.

\bibitem{PDG2022}
R. L. Workman et al. (Particle Data Group) (2022). "Review of Particle Physics". \textit{Prog. Theor. Exp. Phys.}, 2022(8), 083C01.

\end{thebibliography}

\end{document}
