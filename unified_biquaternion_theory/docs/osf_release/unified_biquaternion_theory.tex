
\documentclass[12pt, a4paper]{article}
\usepackage[utf8]{inputenc}
\usepackage[english]{babel}
\usepackage{amsmath, amssymb}
\usepackage{geometry}
\usepackage{graphicx}
\usepackage{hyperref}
\usepackage{siunitx}

\geometry{a4paper, margin=1in}
\linespread{1.15}

\title{\textbf{The Unified Biquaternion Theory: A Framework for Fundamental Constants, Dark Matter, and Mass Hierarchy}}
\author{Ing. David Jaro\v{s}\thanks{Primary author and theorist. Contact: jdavid.cz@gmail.com} \\
\textit{with AI assistance from ChatGPT-4o and Gemini 2.5 Pro (as technical aides)}}
\date{June 30, 2025}

\begin{document}
\maketitle

\begin{abstract}
We present the Unified Biquaternion Theory (UBT), a theoretical framework based on a complexified five-dimensional spacetime and a fundamental biquaternion-valued field \( \Theta(q, \tau) \). UBT aims to reconcile quantum field theory and general relativity by providing natural derivations for fundamental constants and particle properties from the geometry and topology of the underlying manifold. The theory predicts the emergence of known particles as topological excitations and introduces new classes of fields associated with consciousness and dark matter. We show that UBT reduces to known physical theories in their respective limits, while offering falsifiable predictions beyond current physics. This paper summarizes the theory's core tenets and its successful application to three major unsolved problems: the origin of the fine-structure constant, the nature of dark matter, and the lepton mass hierarchy.
\end{abstract}

\tableofcontents
\newpage

\section{Foundations of the Unified Biquaternion Theory}
UBT is defined on a complexified 5D manifold with coordinates \( q = (x^\mu, \psi) \), where \( x^\mu \in \mathbb{C}^{1,3} \) are complexified spacetime coordinates, where both temporal and spatial components may take complex values, and \( \psi \in \mathbb{R} \) encodes a phase-like internal dimension, associated with consciousness and entropy. Time is complexified as \( \tau = t + i\psi \). The central object of the theory is a biquaternion-valued field \( \Theta(q, \tau) \in \mathbb{B} \otimes \mathbb{C} \), which acts as a universal precursor to all quantum fields and gravitational interactions. The field's dynamics are governed by a generalized covariant equation of motion resembling a complexified Dirac--Klein--Gordon system. Quantized excitations of \( \Theta \) give rise to standard particles, while stable, solitonic topological configurations correspond to massive fermions, bosons, or dark matter candidates. The full development and mathematical formalism are available in the author's open-source repository: \url{https://github.com/DavJ/complex-consciousness-theory.git}.

\subsection*{Structure of the \( \Theta(q, \tau) \) Field and Its Dynamics}
The fundamental field \( \Theta(q, \tau) \) in UBT is a \textbf{biquaternion-valued spinor field}, defined as:
\[
\Theta(q, \tau) \in \mathbb{B} \otimes \mathbb{C}^4
\]
where \( \mathbb{B} \) is the algebra of biquaternions (complexified quaternions), and \( \mathbb{C}^4 \) denotes a four-component spinor structure. The coordinate \( q = (x^\mu, \psi) \) spans a 5D complexified manifold, and \( \tau = t + i\psi \) encodes the complexified time.

The field evolves according to a generalized covariant wave equation:
\begin{equation}
i \partial_\tau \Theta = \left( -i v^\mu \partial_\mu + D \nabla^2 \right) \Theta + V(\Theta)
\label{eq:theta_wave}
\end{equation}
Here:
\begin{itemize}
  \item \( v^\mu \partial_\mu \) represents the \textbf{drift term} or directional propagation (wave part),
  \item \( D \nabla^2 \) is a \textbf{diffusion-like} or \textbf{dispersive term}, associated with entropy and phase spreading,
  \item \( V(\Theta) \) is a self-interaction or effective mass potential.
\end{itemize}

This structure generalizes both \textbf{Dirac} and \textbf{Klein--Gordon} equations. In the free-field, low-dispersion limit, Eq.~\eqref{eq:theta_wave} reduces to a \textbf{Dirac equation}:
\begin{equation}
(i \gamma^\mu \partial_\mu - m)\, \Theta = 0
\label{eq:dirac}
\end{equation}
which governs the fermionic sector of the Standard Model.

\subsubsection*{Connection to Jacobi Theta Functions}
In specific representations or limits (e.g., toroidal boundary conditions), solutions to Eq.~\eqref{eq:theta_wave} exhibit modular behavior characteristic of \textbf{Jacobi theta functions}, defined as:
\[
\vartheta(z, \tau) = \sum_{n=-\infty}^\infty e^{\pi i n^2 \tau} e^{2\pi i n z}
\]
These functions describe \textbf{coherent topological excitations}, \textbf{phase-locked states}, and \textbf{vacuum modes} in the internal time \( \psi \), forming a mathematical backbone of spectral quantization in UBT.

More information: \url{https://en.wikipedia.org/wiki/Jacobi_theta_function}

\section{Scalar Constraint and Topological Defects}
Starting from the polar decomposition of the field:
\[
\Theta(q) = \rho(q)\, e^{i\phi(q)}
\]
we obtain a scalar constraint from the conservation laws:
\[
\eta^{\mu\nu} \partial_\mu \rho \, \partial_\nu \phi = 0
\]
This implies a Lorentz-invariant orthogonality condition between the gradient of the amplitude and the gradient of the phase.

\subsection*{Example: Spherically Symmetric Field}
Let:
\[
\rho = \rho(r), \quad \phi = \phi(t)
\]
Then the constraint becomes:
\[
\partial_\mu \rho \, \partial^\mu \phi = \left(\frac{d\rho}{dr}\right)^2 \cdot 0 + 0 \cdot \left(\frac{d\phi}{dt}\right)^2 = 0
\]
and is trivially satisfied.

\subsection*{Topological Vortex in 2D}
Consider a vortex configuration:
\[
\phi(\theta) = n\theta, \quad \rho(r) \to 0 \text{ as } r \to 0
\]
Then:
\[
\nabla \phi = \frac{n}{r} \hat{\theta}, \quad \nabla \rho \sim \rho'(r) \hat{r}
\]
Their scalar product is zero due to radial--angular orthogonality:
\[
\nabla \rho \cdot \nabla \phi = 0
\]
So the scalar constraint remains satisfied even in the presence of a topological defect at the origin. In this context, a \emph{topological defect} refers to a localized, nontrivial configuration of the phase that cannot be removed by continuous deformation---such as vortices, solitons, or domain walls.
