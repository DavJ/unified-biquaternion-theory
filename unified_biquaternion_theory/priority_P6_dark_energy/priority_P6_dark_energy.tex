
\documentclass[12pt]{article}
\usepackage{amsmath}
\usepackage{geometry}
\geometry{a4paper, margin=2.5cm}
\title{Priority P6 – Dark Energy as Vacuum Tension of the $\Theta$ Field}
\author{}
\date{}

\begin{document}
\maketitle

\section*{Hypothesis}

In the framework of the Unified Biquaternion Theory (UBT), we postulate that the observed cosmological constant $\Lambda$---typically associated with dark energy---is not a mysterious external parameter, but emerges naturally from the intrinsic vacuum energy density of the fundamental field $\Theta(q, \tau)$.

\paragraph{Key claim:} Even in the absence of excitations, the $\Theta$ field has a non-zero vacuum stress-energy tensor component:
\[
T_{\mu\nu}^{(\text{vac})} = -\rho_{\text{vac}} g_{\mu\nu}
\]

This implies a contribution to the Einstein field equations in the form:
\[
G_{\mu\nu} + \Lambda g_{\mu\nu} = 8\pi G T_{\mu\nu} \quad \Rightarrow \quad \Lambda = 8\pi G \rho_{\text{vac}}
\]

Unlike in standard quantum field theory (QFT), where the predicted vacuum energy overshoots by 120 orders of magnitude, here $\rho_{\text{vac}}$ is not a sum over zero-point fluctuations, but a geometric property of the $\Theta$ field itself.

\paragraph{Implication:} This resolves the cosmological constant problem by shifting the interpretation from a particle-physics origin to a topological and geometric origin inherent in the UBT framework.


\section*{License}
This work is licensed under a Creative Commons Attribution 4.0 International License (CC BY 4.0).

\end{document}