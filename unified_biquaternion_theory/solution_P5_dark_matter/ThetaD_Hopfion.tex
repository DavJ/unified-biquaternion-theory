
% ThetaD_Hopfion.tex
\documentclass[12pt]{article}
\usepackage{amsmath,amssymb}
\title{Construction of a Dark Mode Hopfion Solution in Unified Biquaternion Theory}
\author{UBT Project}
\date{\today}

\begin{document}

\maketitle

\section*{Abstract}
This document presents the analytical ansatz and geometric characteristics of a topologically nontrivial Hopfion solution \(\Theta_D\) within the Unified Biquaternion Theory (UBT), proposed as a candidate configuration for dark matter.

\section{Ansatz}
We define the Hopfion-like solution in stereographic coordinates for a map \( \Theta : \mathbb{R}^3 \to S^2 \subset \mathbb{C}^2 \) via the rational map:
\[
\Theta_D(x, y, z, t) = \frac{(2(x+iy))^p}{(2z + i(r^2 - 1))^q},
\]
where \( r^2 = x^2 + y^2 + z^2 \), and \(p, q \in \mathbb{Z}^+\) define the topological charge.

\section{Properties}
\begin{itemize}
    \item \textbf{Topological Charge:} The Hopf invariant \(H = pq\).
    \item \textbf{Energy Density:} Localized in a toroidal region around the core, where \(|\Theta_D|\) varies rapidly.
    \item \textbf{Electromagnetic Neutrality:} Imposed via projection onto gauge-neutral components of \(\Theta\).
\end{itemize}

\section{Stress-Energy Tensor}
The energy-momentum tensor \(T_{\mu\nu}\) is derived from the UBT Lagrangian:
\[
T_{\mu\nu} = \text{Re} \left[ \partial_\mu \Theta^\dagger \cdot \partial_\nu \Theta - \frac{1}{2} \eta_{\mu\nu} \left( \partial^\alpha \Theta^\dagger \cdot \partial_\alpha \Theta \right) \right],
\]
ensuring conserved gravitational energy.

\section{Next Steps}
To validate this configuration:
\begin{itemize}
    \item Numerically simulate the stability of \(\Theta_D\),
    \item Compute the resulting gravitational potential from \(T_{\mu\nu}\),
    \item Fit predicted rotation curves to observed galactic data.
\end{itemize}


\section*{License}
This work is licensed under a Creative Commons Attribution 4.0 International License (CC BY 4.0).

\end{document}