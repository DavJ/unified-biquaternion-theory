
\documentclass[12pt]{article}
\usepackage{amsmath}
\usepackage{geometry}
\geometry{margin=1in}
\title{Topologický původ temné hmoty a hierarchie hmotností v Unified Biquaternion Theory}
\author{David Jaroš}
\date{}

\begin{document}
\maketitle

\section*{Temná hmota jako topologický defekt pole $\Theta$}
Na základě dokumentů \texttt{ThetaD\_Hopfion.tex} a \texttt{ThetaD\_Gravity.tex} prezentujeme model, v němž temná hmota není nová částice, ale stabilní topologická konfigurace pole $\Theta$ (Hopfion). Tento mód:
\begin{itemize}
    \item Je elektromagneticky neutrální (neinteraguje s běžnou hmotou),
    \item Má nenulový topologický náboj (Hopfovo číslo),
    \item Vede na přirozené vysvětlení plochých rotačních křivek galaxií,
    \item Je konzistentní s gravitačními efekty pozorovanými na galaktických škálách.
\end{itemize}
Tento model přirozeně vysvětluje existenci temné hmoty bez nutnosti zavádět nové částice.

\section*{Hierarchie hmotností leptonských generací}
Hypotéza: Leptony (elektron, mion, tauon) odpovídají různým topologickým stavům pole $\Theta$ s rostoucí komplexitou:
\[
n = \text{Hopfovo číslo}, \quad n = 1, 2, 3
\]

\subsection*{Model hmotnosti:}
Navrhujeme univerzální formu pro hmotnosti generací:
\[
S(n) = A n^p - B n \ln(n)
\]
kde $S(n)$ je klidová energie (hmotnost) daného stavu. Tento model dobře fituje mion a tauon (n = 2, 3) s exponentem $p \approx 6.96$.

\subsection*{Elektron jako speciální případ}
Ukázali jsme, že hmotnost elektronu (n = 1) výrazně vybočuje z topologické škálovací závislosti. Navrhujeme, že dominantním mechanismem u elektronu je elektromagnetická vlastní energie vznikající interakcí s polem $A^\mu$. Hmotnost elektronu tak vzniká hybridně:
\[
m_e = S_\text{topo}(n=1) + S_\text{EM}
\]
kde $S_\text{topo}(1)$ je velmi malý a rozhodující složkou je korekce $S_\text{EM}$.

\section*{Závěr}
Temná hmota i hmotnostní spektrum leptonských generací lze vysvětlit jako topologické excitace pole $\Theta$. Model má přímé fyzikální predikce a propojuje gravitační, topologické i kvantové vlastnosti částic v rámci jednotné teorie.

\end{document}
