
% dark_matter_hopfion_solution.tex
\documentclass[12pt]{article}
\usepackage{amsmath,amssymb,geometry,graphicx}
\geometry{margin=1in}
\title{Solution Draft: Hopfion-Based Dark Matter Configuration in UBT}
\author{Unified Biquaternion Theory Project}
\date{\today}

\begin{document}

\maketitle

\section*{Abstract}
We explore a solution to the dark matter problem using a topological configuration—specifically a hopfion—of the unified biquaternionic field \( \Theta(q, \tau) \). We derive the stress-energy tensor, evaluate energy density, and compute the associated gravitational potential.

\section{Hopfion Ansatz}
Let the field \( \Theta_D(q, \tau) \) be defined using a complex scalar preimage \( \phi: \mathbb{R}^3 \to \mathbb{C} \cup \{\infty\} \) via the Hopf map. The simplest Hopfion may be written:

\[
\phi(x,y,z) = \frac{2(x + i y)}{2z + i(r^2 - 1)}, \quad r^2 = x^2 + y^2 + z^2
\]

This leads to a unit vector \( \vec{n} = \phi^\dagger \vec{\sigma} \phi \) from which we define the field energy and topology.

\section{Stress-Energy Tensor}
The effective stress-energy tensor derived from the Lagrangian \( \mathcal{L}[\Theta] \) is:

\[
T_{\mu\nu} = \partial_\mu \Theta^\dagger \partial_\nu \Theta - \frac{1}{2} g_{\mu\nu} \left( \partial^\lambda \Theta^\dagger \partial_\lambda \Theta \right)
\]

where \( \Theta \) is interpreted as a normalized 4-spinor bundle.

\section{Energy Profile}
The energy density \( \rho = T_{00} \) can be computed numerically or symbolically from the above expression. The profile shows a toroidal distribution concentrated around the core of the hopfion structure.

\begin{center}
\includegraphics[width=0.6\textwidth]{hopfion_profile.png}
\end{center}

\section{Gravitational Potential}
Assuming weak-field gravity, the Newtonian potential \( \Phi \) satisfies:

\[
\nabla^2 \Phi = 4\pi G \rho(\vec{x})
\]

which can be solved via Green’s function or Fourier transform methods.


\section*{License}
This work is licensed under a Creative Commons Attribution 4.0 International License (CC BY 4.0).

\end{document}