
\documentclass[12pt]{article}
\usepackage{amsmath, amssymb}
\usepackage{geometry}
\geometry{margin=1in}
\title{UBT Appendix 5: Vector Part of the Biquaternionic Field Equation}
\author{}
\date{}

\begin{document}
\maketitle

\section*{1. Introduction}

In this appendix, we examine the \textbf{vector part} of the unified field equation derived from the Biquaternion Gravity framework:
\begin{equation}
\mathbf{R}^\mu_a - \frac{1}{2} e^\mu_a \mathbf{R} = 0
\end{equation}
where all quantities are biquaternions.

Our goal is to extract and interpret the \textbf{vector part} of this equation, in analogy with how we previously analyzed the scalar and imaginary scalar components.

\section*{2. Decomposition into Vector Components}

Recall that any biquaternion $\mathbf{X}$ can be decomposed as:
\[
\mathbf{X} = \text{Scal}(\mathbf{X}) + \vec{V}_R(\mathbf{X}) + i \, \vec{V}_I(\mathbf{X}) + i \, \text{PScal}(\mathbf{X})
\]
We now apply the projection:
\[
\text{Re}(\vec{V}(\mathbf{X})) = \text{Re}(\mathbf{X}) - \text{Scal}(\mathbf{X})
\]
and similarly for the imaginary vector part.

Let us define:
\begin{align}
\vec{V}_R^\mu_a &= \text{ReVectorPart}(\mathbf{R}^\mu_a - \tfrac{1}{2} e^\mu_a \mathbf{R}) \\
\vec{V}_I^\mu_a &= \text{ImVectorPart}(\mathbf{R}^\mu_a - \tfrac{1}{2} e^\mu_a \mathbf{R})
\end{align}

Then the vector part equation is simply:
\begin{equation}
\vec{V}_R^\mu_a + i \vec{V}_I^\mu_a = 0
\end{equation}
which splits into two real vector equations:
\begin{align}
\vec{V}_R^\mu_a &= 0 \\
\vec{V}_I^\mu_a &= 0
\end{align}

\section*{3. Interpretation and Hypotheses}

\subsection*{3.1 Real Vector Equation}

The real vector equation $\vec{V}_R^\mu_a = 0$ may encode a constraint on torsion-free, metric-compatible geometries. It likely corresponds to Einstein-Cartan-like conditions or vectorial conservation laws.

\subsection*{3.2 Imaginary Vector Equation}

The imaginary vector part $\vec{V}_I^\mu_a = 0$ is especially intriguing. It may encode:
\begin{itemize}
  \item A generalized Maxwell-type field equation,
  \item A hidden vector field coupled to spacetime geometry,
  \item A remnant of conformal or chiral gauge symmetry.
\end{itemize}

In the simplified case $e_I = 0$, this equation may reduce to a divergence-type condition on $\omega_I$:
\[
\nabla_\mu \omega_I^{\mu ab} + \text{(nonlinear terms)} = 0
\]
resembling Yang-Mills field dynamics.

\section*{4. Future Directions}

\begin{itemize}
  \item Attempt explicit calculation of $\vec{V}_I^\mu_a$ in terms of $\omega_R$ and $\omega_I$.
  \item Test reductions in symmetric backgrounds (FLRW, Schwarzschild).
  \item Seek Lagrangian formulation that yields this vector equation as Euler-Lagrange equation.
\end{itemize}


\section*{License}
This work is licensed under a Creative Commons Attribution 4.0 International License (CC BY 4.0).

\end{document}