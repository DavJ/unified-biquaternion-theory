
\documentclass[12pt]{article}
\usepackage{amsmath,amssymb}
\usepackage{geometry}
\geometry{margin=1in}

\title{Appendix 2: Imaginary Scalar Equation}
\author{}
\date{}

\begin{document}

\maketitle

\section*{Motivation}

In Appendix 1, we demonstrated that the real scalar part of our biquaternionic field equation reduces to the vacuum Einstein equations. This validates the classical gravitational limit of our theory. We now proceed to examine the \emph{imaginary scalar part} of the same equation, which should yield new physics not captured by General Relativity.

\section*{Field Decomposition}

We decompose the tetrad $e^a_\mu$ and connection $\omega_\mu^{ab}$ into their real and imaginary parts:
\[
e^a_\mu = e^a_\mu{}^{(R)} + i\,e^a_\mu{}^{(I)}, \quad
\omega_\mu^{ab} = \omega_\mu^{ab}{}^{(R)} + i\,\omega_\mu^{ab}{}^{(I)}
\]
where the superscripts $(R)$ and $(I)$ indicate real and imaginary parts respectively, with all components valued in real quaternions.

\section*{Curvature and Contraction}

The curvature two-form constructed from the connection is:
\[
R_{\mu\nu}{}^{ab} = \partial_\mu \omega_\nu^{ab} - \partial_\nu \omega_\mu^{ab} + \omega_\mu^{ac} \omega_{\nu c}{}^b - \omega_\nu^{ac} \omega_{\mu c}{}^b
\]

Substituting the decomposition, we collect terms linear in $i$, and retain only the scalar (trace) part after contraction with $e^b_\nu$. We define:
\[
S := \text{ScalarPart} \left( e^b_\nu R_{\mu\nu ab} \right)
\]

Expanding and applying scalar extraction, we write:
\[
S = \text{Re}(S) + i\,\text{Im}(S)
\]
We are interested in the imaginary scalar part:
\[
\text{Im}(S) = \text{Imag} \left[ \text{ScalarPart} \left( e^b_\nu{}^{(R)} R_{\mu\nu ab}{}^{(I)} + e^b_\nu{}^{(I)} R_{\mu\nu ab}{}^{(R)} \right) \right]
\]

\section*{Resulting Equation}

The imaginary scalar field equation becomes:
\[
\text{Im} \left[ \text{ScalarPart} \left( e^b_\nu{}^{(R)} R_{\mu\nu ab}{}^{(I)} + e^b_\nu{}^{(I)} R_{\mu\nu ab}{}^{(R)} \right) \right] + \text{Im}(R) = 0
\]
where $R$ is the scalar curvature (trace over $e^\mu_a R_{\mu\nu}{}^{ab}$).

\section*{Interpretation}

This equation is nontrivial and describes a new dynamical constraint, potentially corresponding to a scalar field $\phi(x)$ that couples to curvature. The appearance of $e^{(I)}$ indicates a deviation from pure Riemannian geometry, possibly encoding quantum or informational degrees of freedom.

\section*{Next Steps}

\begin{itemize}
\item Analyze solutions of this equation on symmetric backgrounds (Minkowski, FRW).
\item Determine whether this scalar has mass, source, or propagates.
\item Explore its role in early universe cosmology or black hole interiors.
\end{itemize}


\section*{License}
This work is licensed under a Creative Commons Attribution 4.0 International License (CC BY 4.0).

\end{document}