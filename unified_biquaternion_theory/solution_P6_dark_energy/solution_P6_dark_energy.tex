
\documentclass[12pt]{article}
\usepackage{amsmath}
\usepackage{geometry}
\geometry{a4paper, margin=2.5cm}
\title{Solution P6 – Derivation of the Cosmological Constant from $\Theta$ Field Geometry}
\author{}
\date{}

\begin{document}
\maketitle

\section*{Field Equations in the Presence of $\Theta$ Vacuum Energy}

From the UBT framework, the total energy-momentum tensor contains a vacuum component due to the structure of the $\Theta(q,\tau)$ field:
\[
T_{\mu\nu}^{(\Theta)} = T_{\mu\nu}^{(\text{matter})} - \rho_{\text{vac}} g_{\mu\nu}
\]

Einstein's field equations read:
\[
G_{\mu\nu} = 8\pi G T_{\mu\nu}^{(\Theta)}
\]

Substituting:
\[
G_{\mu\nu} = 8\pi G \left( T_{\mu\nu}^{(\text{matter})} - \rho_{\text{vac}} g_{\mu\nu} \right)
\Rightarrow G_{\mu\nu} + 8\pi G \rho_{\text{vac}} g_{\mu\nu} = 8\pi G T_{\mu\nu}^{(\text{matter})}
\]

Comparing with the standard form:
\[
G_{\mu\nu} + \Lambda g_{\mu\nu} = 8\pi G T_{\mu\nu}^{(\text{matter})}
\]

We identify:
\[
\Lambda = 8\pi G \rho_{\text{vac}}
\]

\section*{Conclusion}

The cosmological constant arises naturally from the vacuum tension of the $\Theta$ field. This reinterpretation avoids the fine-tuning problem of QFT and aligns with observations, provided $\rho_{\text{vac}}$ is determined from the geometry/topology of $\Theta$.


\section*{License}
This work is licensed under a Creative Commons Attribution 4.0 International License (CC BY 4.0).

\end{document}