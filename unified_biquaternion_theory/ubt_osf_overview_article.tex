\documentclass[11pt, a4paper]{article}
\usepackage[utf8]{inputenc}
\usepackage{amsmath,amssymb}
\usepackage{geometry}
\geometry{margin=1in}
\usepackage{authblk}
\usepackage{physics}
\usepackage{hyperref}
\usepackage{mathrsfs}
\usepackage{bm}

\hypersetup{
    colorlinks=true,
    linkcolor=blue,
    filecolor=magenta,      
    urlcolor=cyan,
    pdftitle={A Biquaternionic Unified Field Theory},
    pdfpagemode=FullScreen,
    }

\title{\textbf{A Biquaternionic Unified Field Theory of Gravity and Electromagnetism}}
\author{David Jaroš\thanks{The full set of developmental appendices and related materials are available at the author's GitHub repository: \url{https://github.com/DavJ/complex-consciousness-theory/tree/master/unified_biquaternion_theory}}}
\affil{Independent Researcher}
\date{\today}

\begin{document}

\maketitle

\begin{abstract}
We propose a unified field theory based on a biquaternionic geometry. The fundamental fields are a biquaternion-valued tetrad and spin connection. We construct an action containing two terms: a linear term in the biquaternion curvature, analogous to the Palatini action, and a quadratic term analogous to the Yang-Mills action. We demonstrate through variational principle that this formalism leads to a rich set of field equations. In the real-valued limit, the linear part of the action correctly reproduces the vacuum Einstein field equations, embedding General Relativity within the theory. The quadratic part of the action naturally yields the source-free Maxwell's equations for the imaginary component of the spin connection, thus unifying gravity and electromagnetism in a single, coherent geometric framework. The theory predicts additional non-dynamic, constraint-like equations that may relate to the topological structure of spacetime.
\end{abstract}

\tableofcontents

\section{Introduction}
The quest to unify the fundamental forces of nature, particularly gravity with the gauge theories of the Standard Model, remains a central challenge in theoretical physics. Inspired by early attempts by Einstein, Schrödinger, and others, we explore a path of geometric unification. We postulate that the fundamental geometric structure of spacetime is richer than a real-valued metric tensor and is instead described by the algebra of biquaternions. This algebraic richness provides a natural framework to house both the gravitational field and the electromagnetic gauge field.

This paper lays out the mathematical formalism of this Unified Biquaternion Theory (UBT). We construct an action principle and derive the field equations. We then prove that this theory contains both General Relativity and Maxwell's theory of electromagnetism as distinct but unified components.

\section{The Biquaternionic Formalism}

\subsection{Fundamental Fields}
We postulate that the geometry of spacetime is described by two fundamental fields:
\begin{itemize}
  \item The **biquaternionic tetrad** (or vielbein), \( \mathbf{e}^a_\mu \). This field relates the curved spacetime coordinates (Greek indices \( \mu, \nu, \dots \)) to a local, flat tangent space (Latin indices \( a, b, \dots \)). Each of its 16 components is a biquaternion.
  \item The **biquaternionic spin connection**, \( \boldsymbol{\omega}^{ab}_\mu \). This is a gauge field (a connection one-form) that governs parallel transport and defines curvature.
\end{itemize}
A biquaternion \( \mathbf{Q} \) is a quaternion whose coefficients are complex numbers: \( \mathbf{Q} = (a_R + i a_I) + (b_R + i b_I)\mathbf{i} + (c_R + i c_I)\mathbf{j} + (d_R + i d_I)\mathbf{k} \). We can decompose our fields into real and imaginary parts:
\[ \mathbf{e}^a_\mu = \mathbf{e}^a_{\mu (R)} + i \, \mathbf{e}^a_{\mu (I)}, \quad \boldsymbol{\omega}^{ab}_\mu = \boldsymbol{\omega}^{ab}_{\mu (R)} + i \, \boldsymbol{\omega}^{ab}_{\mu (I)} \]
where the components with (R) and (I) are real-valued quaternions. The spacetime metric \( g_{\mu\nu} \) is considered an emergent quantity, defined from the real part of the tetrad: \( g_{\mu\nu} := \eta_{ab} \, \text{ScalarPart}(\mathbf{e}^a_{\mu(R)}) \, \text{ScalarPart}(\mathbf{e}^b_{\nu(R)}) \).

\subsection{The Extended Action}
To capture the dynamics of both gravity and electromagnetism, we propose an extended action composed of a linear (Einstein-like) part and a quadratic (Maxwell-like) part:
\begin{equation}
S[\mathbf{e}, \boldsymbol{\omega}] = S_{GR} + S_{EM}
\end{equation}
where
\begin{align}
S_{GR} &= \int \det(e) \, \text{Re}\left[\text{ScalarPart}(\mathbf{e}^\mu_a \mathbf{e}^\nu_b \mathbf{R}^{ab}_{\mu\nu})\right] \, d^4x \\
S_{EM} &= \lambda \int \det(e) \, \text{Re}\left[\text{ScalarPart}(\mathbf{R}_{\mu\nu}^{ab} \mathbf{R}^{\mu\nu}_{ab})\right] \, d^4x
\end{align}
Here, \( \mathbf{R}^{ab}_{\mu\nu} = \partial_\mu \boldsymbol{\omega}^{ab}_\nu - \partial_\nu \boldsymbol{\omega}^{ab}_\mu + [\boldsymbol{\omega}^{ac}_\mu, \boldsymbol{\omega}^{cb}_\nu] \) is the biquaternion-valued curvature tensor, and \( \lambda \) is a new coupling constant. We use the notation \( \det(e) \) to represent the volume element, defined via \( \det(\text{Re}(\mathbf{e}^a_\mu)) \).

\section{Derivation of the Field Equations}
The field equations are derived by applying the principle of stationary action, \( \delta S = 0 \), with respect to our fundamental fields.

\subsection{The Gravitational Sector (from \(S_{GR}\))}
As demonstrated in our developmental work, the variation of \( S_{GR} \) with respect to the tetrad \( \mathbf{e} \) leads to a biquaternion field equation of the form:
\begin{equation}
\mathbf{R}^a_\mu - \frac{1}{2} \mathbf{e}^a_\mu \mathbf{R} = 0
\end{equation}
where \( \mathbf{R}^a_\mu \) and \( \mathbf{R} \) are the biquaternionic Ricci tensor and Ricci scalar, respectively. We have proven rigorously that the real part of this equation, in the real-valued limit, reduces precisely to the vacuum Einstein field equations, \( G_{\mu\nu} = 0 \). This establishes the classical correspondence with General Relativity.

\subsection{The Electromagnetic Sector (from \(S_{EM}\))}
We now focus on the new quadratic term, \( S_{EM} \). Let's analyze its imaginary component under the simplifying assumptions that the geometry is primarily real (\( \mathbf{e}_I = 0 \)) and the imaginary part of the connection corresponds to the electromagnetic potential via \( \boldsymbol{\omega}_{I\mu} = A_\mu \mathbf{i} \).

The imaginary part of the curvature tensor, \( \mathbf{R}_I \), was shown to be:
\begin{equation}
\mathbf{R}^I_{\mu\nu} = (\partial_\mu A_\nu - \partial_\nu A_\mu) \mathbf{i} = F_{\mu\nu} \mathbf{i}
\end{equation}
The quadratic term in the action for the imaginary part then becomes:
\begin{equation}
\lambda \int \det(e) \, \text{Re}\left[\text{ScalarPart}(\mathbf{R}^I_{\mu\nu} \mathbf{R}^{\mu\nu}_I)\right] d^4x = \lambda \int \det(e) \, \text{Re}\left[\text{ScalarPart}((F_{\mu\nu}\mathbf{i})(F^{\mu\nu}\mathbf{i}))\right] d^4x
\end{equation}
Since \( \mathbf{i}^2 = -1 \) and \( F_{\mu\nu}F^{\mu\nu} \) is a real scalar, this simplifies to:
\begin{equation}
S_{EM} = -\lambda \int \det(e) F_{\mu\nu}F^{\mu\nu} d^4x
\end{equation}
This is precisely the standard, well-known action for the source-free electromagnetic field. The variation of this action with respect to the potential \( A_\mu \) yields the source-free Maxwell's equations:
\begin{equation}
\boxed{\partial_\mu F^{\mu\nu} = 0}
\end{equation}
The other Maxwell equation, \( \partial_\mu \tilde{F}^{\mu\nu} = 0 \), follows automatically from the definition of \( F_{\mu\nu} \) as the exterior derivative of \( A_\mu \).

\section{Discussion and Conclusion}
We have presented a Unified Biquaternion Theory that successfully incorporates both gravity and electromagnetism. By postulating an action with both a linear and a quadratic term in biquaternionic curvature, we have shown that the two fundamental forces of classical physics emerge naturally from a single geometric framework.

\begin{itemize}
    \item \textbf{Gravity} arises from the linear part of the action, corresponding to the real part of the geometry. The theory correctly reproduces General Relativity in the appropriate limit.
    \item \textbf{Electromagnetism} arises from the quadratic part of the action, corresponding to the dynamics of the imaginary part of the geometric connection. The theory correctly reproduces the source-free Maxwell's equations.
    \item The remaining components of the biquaternion field equations (e.g., from the vector parts of the linear action) act as **geometric constraint equations**, potentially relating to the topological structure of spacetime or the nature of torsion.
\end{itemize}

This work serves as a proof of concept that biquaternionic geometry provides a viable and elegant path toward the unification of physical forces. Future work will involve exploring the role of the remaining constraint equations, investigating solutions beyond the simplified limits, and considering the quantization of the theory.


\section*{License}
This work is licensed under a Creative Commons Attribution 4.0 International License (CC BY 4.0).

\end{document}