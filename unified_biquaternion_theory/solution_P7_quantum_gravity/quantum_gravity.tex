
\documentclass[12pt]{article}
\usepackage[a4paper, margin=2.5cm]{geometry}
\usepackage{amsmath, amssymb}
\usepackage{hyperref}
\usepackage{graphicx}
\usepackage{titlesec}
\usepackage{authblk}

\titleformat{\section}{\normalfont\Large\bfseries}{\thesection.}{0.5em}{}
\titleformat{\subsection}{\normalfont\large\bfseries}{\thesubsection.}{0.5em}{}

\title{\textbf{Quantum Gravity in the Unified Biquaternion Theory (UBT)}}
\author{David Jaroš}
\affil{\texttt{jdavid.cz@gmail.com}}
\date{July 2025}

\begin{document}

\maketitle

% THEORY_STATUS_DISCLAIMER.tex
% This file contains standard disclaimers to be included in UBT LaTeX documents
% to ensure proper scientific transparency about the theory's current status.
%
% Usage: % THEORY_STATUS_DISCLAIMER.tex
% This file contains standard disclaimers to be included in UBT LaTeX documents
% to ensure proper scientific transparency about the theory's current status.
%
% Usage: % THEORY_STATUS_DISCLAIMER.tex
% This file contains standard disclaimers to be included in UBT LaTeX documents
% to ensure proper scientific transparency about the theory's current status.
%
% Usage: \input{THEORY_STATUS_DISCLAIMER} or \input{../THEORY_STATUS_DISCLAIMER}

% Main theory status disclaimer (for general use)
\newcommand{\UBTStatusDisclaimer}{%
\begin{center}
\fbox{\begin{minipage}{0.95\textwidth}
\textbf{WARNING: RESEARCH FRAMEWORK IN DEVELOPMENT}

\medskip
\noindent The Unified Biquaternion Theory (UBT) is currently a \textbf{research framework in early development} (Year 5), not a validated scientific theory. Recent progress (November 2025) includes substantial mathematical formalization, but significant challenges remain:

\begin{itemize}
\item \textbf{Limited peer-review} (not yet externally validated, submission in progress)
\item \textbf{Mathematical foundations}: substantially complete but not yet peer-reviewed
\item \textbf{Testable predictions}: CMB analysis feasible (1-2 years), but most predictions unobservable
\item \textbf{SM gauge group}: now rigorously derived from geometry (Nov 2025)
\item \textbf{Fermion masses}: not yet calculated from first principles
\item \textbf{Complex time}: causality/unitarity partially addressed, active research ongoing
\item \textbf{Consciousness claims}: highly speculative, lack neuroscientific grounding
\end{itemize}

\noindent UBT generalizes Einstein's General Relativity (recovering GR equations in the real limit) but extends beyond validated physics. Treat as \textbf{exploratory research}, not established science.

\medskip
\noindent For detailed assessment and November 2025 updates, see: \texttt{UBT\_UPDATED\_SCIENTIFIC\_RATING\_2025.md}, \texttt{CHALLENGES\_STATUS\_UPDATE\_NOV\_2025.md}, and \texttt{REMAINING\_CHALLENGES\_DETAILED\_STATUS.md}
\end{minipage}}
\end{center}
}

% Consciousness-specific disclaimer
\newcommand{\ConsciousnessDisclaimer}{%
\begin{center}
\fbox{\begin{minipage}{0.95\textwidth}
\textbf{WARNING: SPECULATIVE HYPOTHESIS - CONSCIOUSNESS CLAIMS}

\medskip
\noindent The following content presents \textbf{speculative philosophical ideas} about consciousness that are \textbf{NOT currently supported} by neuroscience or experimental evidence. These ideas represent long-term research directions.

\medskip
\noindent \textbf{Critical Issues:}
\begin{itemize}
\item No operational definition of consciousness in physical terms
\item No connection to established neuroscience findings
\item No testable predictions for brain function or behavior
\item Parameters (psychon mass, coupling constants) completely unspecified
\item Hard problem of consciousness not solved
\end{itemize}

\medskip
\noindent \textbf{Readers should:}
\begin{itemize}
\item Consult established neuroscience for scientific understanding of consciousness
\item NOT make medical, therapeutic, or life decisions based on these speculations
\item Recognize this as exploratory theoretical work requiring decades of validation
\end{itemize}

\medskip
\noindent See \texttt{CONSCIOUSNESS\_CLAIMS\_ETHICS.md} for ethical guidelines and detailed discussion.
\end{minipage}}
\end{center}
}

% Fine-structure constant disclaimer (Updated November 2025)
\newcommand{\AlphaDerivationDisclaimer}{%
\begin{center}
\fbox{\begin{minipage}{0.95\textwidth}
\textbf{IMPORTANT: FINE-STRUCTURE CONSTANT STATUS (Nov 2025)}

\medskip
\noindent This document discusses the fine-structure constant $\alpha$ within UBT. \textbf{Updated status (November 2025):}

\begin{itemize}
\item \textbf{Dimensional consistency}: Now proven - all quantities have correct dimensions
\item \textbf{Emergent geometric normalization}: $\alpha$ arises from $\Theta$-field self-interaction
\item \textbf{Ratio B/A $\approx$ 20.3}: Determines $n_{opt} = 137$ with energy scale factoring out
\item \textbf{Framework where $\alpha$ might emerge}: Not ab initio parameter-free prediction
\item \textbf{Still contains one adjustable parameter}: B/A ratio not yet uniquely derived
\item \textbf{Honest classification}: Emergent normalization with phenomenological matching
\end{itemize}

\medskip
\noindent \textbf{What would constitute complete derivation:}
\begin{enumerate}
\item Calculate B/A ratio from first principles (without adjustment)
\item Derive discrete parameter N from symmetry/topology alone
\item Show why $\alpha^{-1} = 137.036$ (not just 137) emerges uniquely
\item Account for quantum corrections without additional assumptions
\end{enumerate}

\medskip
\noindent \textbf{Progress made}: Dimensional analysis complete, geometric origin clarified, honest about limitations. \textbf{Remaining challenge}: Derive B/A from first principles or list as input parameter. See \texttt{CHALLENGES\_STATUS\_UPDATE\_NOV\_2025.md} for details.
\end{minipage}}
\end{center}
}

% Short-form disclaimer for appendices
\newcommand{\SpeculativeContentWarning}{%
\noindent\textit{\textbf{Note:} This section contains speculative content that extends beyond experimentally validated physics. See repository documentation for theory status and limitations.}
\medskip
}

% GR Compatibility statement (positive statement about what IS established)
\newcommand{\GRCompatibilityNote}{%
\noindent\textbf{Note on General Relativity Compatibility:} The Unified Biquaternion Theory (UBT) \textbf{generalizes Einstein's General Relativity} by embedding it within a biquaternionic field defined over complex time $\tau = t + i\psi$. In the real-valued limit (where imaginary components vanish), UBT \textbf{exactly reproduces Einstein's field equations} for all curvature regimes. All experimental confirmations of General Relativity are therefore automatically compatible with UBT, as they probe the real sector where the theories are identical. UBT extends (not replaces) GR through additional degrees of freedom that may be relevant for dark sector physics and quantum corrections.
}
 or % THEORY_STATUS_DISCLAIMER.tex
% This file contains standard disclaimers to be included in UBT LaTeX documents
% to ensure proper scientific transparency about the theory's current status.
%
% Usage: \input{THEORY_STATUS_DISCLAIMER} or \input{../THEORY_STATUS_DISCLAIMER}

% Main theory status disclaimer (for general use)
\newcommand{\UBTStatusDisclaimer}{%
\begin{center}
\fbox{\begin{minipage}{0.95\textwidth}
\textbf{WARNING: RESEARCH FRAMEWORK IN DEVELOPMENT}

\medskip
\noindent The Unified Biquaternion Theory (UBT) is currently a \textbf{research framework in early development} (Year 5), not a validated scientific theory. Recent progress (November 2025) includes substantial mathematical formalization, but significant challenges remain:

\begin{itemize}
\item \textbf{Limited peer-review} (not yet externally validated, submission in progress)
\item \textbf{Mathematical foundations}: substantially complete but not yet peer-reviewed
\item \textbf{Testable predictions}: CMB analysis feasible (1-2 years), but most predictions unobservable
\item \textbf{SM gauge group}: now rigorously derived from geometry (Nov 2025)
\item \textbf{Fermion masses}: not yet calculated from first principles
\item \textbf{Complex time}: causality/unitarity partially addressed, active research ongoing
\item \textbf{Consciousness claims}: highly speculative, lack neuroscientific grounding
\end{itemize}

\noindent UBT generalizes Einstein's General Relativity (recovering GR equations in the real limit) but extends beyond validated physics. Treat as \textbf{exploratory research}, not established science.

\medskip
\noindent For detailed assessment and November 2025 updates, see: \texttt{UBT\_UPDATED\_SCIENTIFIC\_RATING\_2025.md}, \texttt{CHALLENGES\_STATUS\_UPDATE\_NOV\_2025.md}, and \texttt{REMAINING\_CHALLENGES\_DETAILED\_STATUS.md}
\end{minipage}}
\end{center}
}

% Consciousness-specific disclaimer
\newcommand{\ConsciousnessDisclaimer}{%
\begin{center}
\fbox{\begin{minipage}{0.95\textwidth}
\textbf{WARNING: SPECULATIVE HYPOTHESIS - CONSCIOUSNESS CLAIMS}

\medskip
\noindent The following content presents \textbf{speculative philosophical ideas} about consciousness that are \textbf{NOT currently supported} by neuroscience or experimental evidence. These ideas represent long-term research directions.

\medskip
\noindent \textbf{Critical Issues:}
\begin{itemize}
\item No operational definition of consciousness in physical terms
\item No connection to established neuroscience findings
\item No testable predictions for brain function or behavior
\item Parameters (psychon mass, coupling constants) completely unspecified
\item Hard problem of consciousness not solved
\end{itemize}

\medskip
\noindent \textbf{Readers should:}
\begin{itemize}
\item Consult established neuroscience for scientific understanding of consciousness
\item NOT make medical, therapeutic, or life decisions based on these speculations
\item Recognize this as exploratory theoretical work requiring decades of validation
\end{itemize}

\medskip
\noindent See \texttt{CONSCIOUSNESS\_CLAIMS\_ETHICS.md} for ethical guidelines and detailed discussion.
\end{minipage}}
\end{center}
}

% Fine-structure constant disclaimer (Updated November 2025)
\newcommand{\AlphaDerivationDisclaimer}{%
\begin{center}
\fbox{\begin{minipage}{0.95\textwidth}
\textbf{IMPORTANT: FINE-STRUCTURE CONSTANT STATUS (Nov 2025)}

\medskip
\noindent This document discusses the fine-structure constant $\alpha$ within UBT. \textbf{Updated status (November 2025):}

\begin{itemize}
\item \textbf{Dimensional consistency}: Now proven - all quantities have correct dimensions
\item \textbf{Emergent geometric normalization}: $\alpha$ arises from $\Theta$-field self-interaction
\item \textbf{Ratio B/A $\approx$ 20.3}: Determines $n_{opt} = 137$ with energy scale factoring out
\item \textbf{Framework where $\alpha$ might emerge}: Not ab initio parameter-free prediction
\item \textbf{Still contains one adjustable parameter}: B/A ratio not yet uniquely derived
\item \textbf{Honest classification}: Emergent normalization with phenomenological matching
\end{itemize}

\medskip
\noindent \textbf{What would constitute complete derivation:}
\begin{enumerate}
\item Calculate B/A ratio from first principles (without adjustment)
\item Derive discrete parameter N from symmetry/topology alone
\item Show why $\alpha^{-1} = 137.036$ (not just 137) emerges uniquely
\item Account for quantum corrections without additional assumptions
\end{enumerate}

\medskip
\noindent \textbf{Progress made}: Dimensional analysis complete, geometric origin clarified, honest about limitations. \textbf{Remaining challenge}: Derive B/A from first principles or list as input parameter. See \texttt{CHALLENGES\_STATUS\_UPDATE\_NOV\_2025.md} for details.
\end{minipage}}
\end{center}
}

% Short-form disclaimer for appendices
\newcommand{\SpeculativeContentWarning}{%
\noindent\textit{\textbf{Note:} This section contains speculative content that extends beyond experimentally validated physics. See repository documentation for theory status and limitations.}
\medskip
}

% GR Compatibility statement (positive statement about what IS established)
\newcommand{\GRCompatibilityNote}{%
\noindent\textbf{Note on General Relativity Compatibility:} The Unified Biquaternion Theory (UBT) \textbf{generalizes Einstein's General Relativity} by embedding it within a biquaternionic field defined over complex time $\tau = t + i\psi$. In the real-valued limit (where imaginary components vanish), UBT \textbf{exactly reproduces Einstein's field equations} for all curvature regimes. All experimental confirmations of General Relativity are therefore automatically compatible with UBT, as they probe the real sector where the theories are identical. UBT extends (not replaces) GR through additional degrees of freedom that may be relevant for dark sector physics and quantum corrections.
}


% Main theory status disclaimer (for general use)
\newcommand{\UBTStatusDisclaimer}{%
\begin{center}
\fbox{\begin{minipage}{0.95\textwidth}
\textbf{WARNING: RESEARCH FRAMEWORK IN DEVELOPMENT}

\medskip
\noindent The Unified Biquaternion Theory (UBT) is currently a \textbf{research framework in early development} (Year 5), not a validated scientific theory. Recent progress (November 2025) includes substantial mathematical formalization, but significant challenges remain:

\begin{itemize}
\item \textbf{Limited peer-review} (not yet externally validated, submission in progress)
\item \textbf{Mathematical foundations}: substantially complete but not yet peer-reviewed
\item \textbf{Testable predictions}: CMB analysis feasible (1-2 years), but most predictions unobservable
\item \textbf{SM gauge group}: now rigorously derived from geometry (Nov 2025)
\item \textbf{Fermion masses}: not yet calculated from first principles
\item \textbf{Complex time}: causality/unitarity partially addressed, active research ongoing
\item \textbf{Consciousness claims}: highly speculative, lack neuroscientific grounding
\end{itemize}

\noindent UBT generalizes Einstein's General Relativity (recovering GR equations in the real limit) but extends beyond validated physics. Treat as \textbf{exploratory research}, not established science.

\medskip
\noindent For detailed assessment and November 2025 updates, see: \texttt{UBT\_UPDATED\_SCIENTIFIC\_RATING\_2025.md}, \texttt{CHALLENGES\_STATUS\_UPDATE\_NOV\_2025.md}, and \texttt{REMAINING\_CHALLENGES\_DETAILED\_STATUS.md}
\end{minipage}}
\end{center}
}

% Consciousness-specific disclaimer
\newcommand{\ConsciousnessDisclaimer}{%
\begin{center}
\fbox{\begin{minipage}{0.95\textwidth}
\textbf{WARNING: SPECULATIVE HYPOTHESIS - CONSCIOUSNESS CLAIMS}

\medskip
\noindent The following content presents \textbf{speculative philosophical ideas} about consciousness that are \textbf{NOT currently supported} by neuroscience or experimental evidence. These ideas represent long-term research directions.

\medskip
\noindent \textbf{Critical Issues:}
\begin{itemize}
\item No operational definition of consciousness in physical terms
\item No connection to established neuroscience findings
\item No testable predictions for brain function or behavior
\item Parameters (psychon mass, coupling constants) completely unspecified
\item Hard problem of consciousness not solved
\end{itemize}

\medskip
\noindent \textbf{Readers should:}
\begin{itemize}
\item Consult established neuroscience for scientific understanding of consciousness
\item NOT make medical, therapeutic, or life decisions based on these speculations
\item Recognize this as exploratory theoretical work requiring decades of validation
\end{itemize}

\medskip
\noindent See \texttt{CONSCIOUSNESS\_CLAIMS\_ETHICS.md} for ethical guidelines and detailed discussion.
\end{minipage}}
\end{center}
}

% Fine-structure constant disclaimer (Updated November 2025)
\newcommand{\AlphaDerivationDisclaimer}{%
\begin{center}
\fbox{\begin{minipage}{0.95\textwidth}
\textbf{IMPORTANT: FINE-STRUCTURE CONSTANT STATUS (Nov 2025)}

\medskip
\noindent This document discusses the fine-structure constant $\alpha$ within UBT. \textbf{Updated status (November 2025):}

\begin{itemize}
\item \textbf{Dimensional consistency}: Now proven - all quantities have correct dimensions
\item \textbf{Emergent geometric normalization}: $\alpha$ arises from $\Theta$-field self-interaction
\item \textbf{Ratio B/A $\approx$ 20.3}: Determines $n_{opt} = 137$ with energy scale factoring out
\item \textbf{Framework where $\alpha$ might emerge}: Not ab initio parameter-free prediction
\item \textbf{Still contains one adjustable parameter}: B/A ratio not yet uniquely derived
\item \textbf{Honest classification}: Emergent normalization with phenomenological matching
\end{itemize}

\medskip
\noindent \textbf{What would constitute complete derivation:}
\begin{enumerate}
\item Calculate B/A ratio from first principles (without adjustment)
\item Derive discrete parameter N from symmetry/topology alone
\item Show why $\alpha^{-1} = 137.036$ (not just 137) emerges uniquely
\item Account for quantum corrections without additional assumptions
\end{enumerate}

\medskip
\noindent \textbf{Progress made}: Dimensional analysis complete, geometric origin clarified, honest about limitations. \textbf{Remaining challenge}: Derive B/A from first principles or list as input parameter. See \texttt{CHALLENGES\_STATUS\_UPDATE\_NOV\_2025.md} for details.
\end{minipage}}
\end{center}
}

% Short-form disclaimer for appendices
\newcommand{\SpeculativeContentWarning}{%
\noindent\textit{\textbf{Note:} This section contains speculative content that extends beyond experimentally validated physics. See repository documentation for theory status and limitations.}
\medskip
}

% GR Compatibility statement (positive statement about what IS established)
\newcommand{\GRCompatibilityNote}{%
\noindent\textbf{Note on General Relativity Compatibility:} The Unified Biquaternion Theory (UBT) \textbf{generalizes Einstein's General Relativity} by embedding it within a biquaternionic field defined over complex time $\tau = t + i\psi$. In the real-valued limit (where imaginary components vanish), UBT \textbf{exactly reproduces Einstein's field equations} for all curvature regimes. All experimental confirmations of General Relativity are therefore automatically compatible with UBT, as they probe the real sector where the theories are identical. UBT extends (not replaces) GR through additional degrees of freedom that may be relevant for dark sector physics and quantum corrections.
}
 or % THEORY_STATUS_DISCLAIMER.tex
% This file contains standard disclaimers to be included in UBT LaTeX documents
% to ensure proper scientific transparency about the theory's current status.
%
% Usage: % THEORY_STATUS_DISCLAIMER.tex
% This file contains standard disclaimers to be included in UBT LaTeX documents
% to ensure proper scientific transparency about the theory's current status.
%
% Usage: \input{THEORY_STATUS_DISCLAIMER} or \input{../THEORY_STATUS_DISCLAIMER}

% Main theory status disclaimer (for general use)
\newcommand{\UBTStatusDisclaimer}{%
\begin{center}
\fbox{\begin{minipage}{0.95\textwidth}
\textbf{WARNING: RESEARCH FRAMEWORK IN DEVELOPMENT}

\medskip
\noindent The Unified Biquaternion Theory (UBT) is currently a \textbf{research framework in early development} (Year 5), not a validated scientific theory. Recent progress (November 2025) includes substantial mathematical formalization, but significant challenges remain:

\begin{itemize}
\item \textbf{Limited peer-review} (not yet externally validated, submission in progress)
\item \textbf{Mathematical foundations}: substantially complete but not yet peer-reviewed
\item \textbf{Testable predictions}: CMB analysis feasible (1-2 years), but most predictions unobservable
\item \textbf{SM gauge group}: now rigorously derived from geometry (Nov 2025)
\item \textbf{Fermion masses}: not yet calculated from first principles
\item \textbf{Complex time}: causality/unitarity partially addressed, active research ongoing
\item \textbf{Consciousness claims}: highly speculative, lack neuroscientific grounding
\end{itemize}

\noindent UBT generalizes Einstein's General Relativity (recovering GR equations in the real limit) but extends beyond validated physics. Treat as \textbf{exploratory research}, not established science.

\medskip
\noindent For detailed assessment and November 2025 updates, see: \texttt{UBT\_UPDATED\_SCIENTIFIC\_RATING\_2025.md}, \texttt{CHALLENGES\_STATUS\_UPDATE\_NOV\_2025.md}, and \texttt{REMAINING\_CHALLENGES\_DETAILED\_STATUS.md}
\end{minipage}}
\end{center}
}

% Consciousness-specific disclaimer
\newcommand{\ConsciousnessDisclaimer}{%
\begin{center}
\fbox{\begin{minipage}{0.95\textwidth}
\textbf{WARNING: SPECULATIVE HYPOTHESIS - CONSCIOUSNESS CLAIMS}

\medskip
\noindent The following content presents \textbf{speculative philosophical ideas} about consciousness that are \textbf{NOT currently supported} by neuroscience or experimental evidence. These ideas represent long-term research directions.

\medskip
\noindent \textbf{Critical Issues:}
\begin{itemize}
\item No operational definition of consciousness in physical terms
\item No connection to established neuroscience findings
\item No testable predictions for brain function or behavior
\item Parameters (psychon mass, coupling constants) completely unspecified
\item Hard problem of consciousness not solved
\end{itemize}

\medskip
\noindent \textbf{Readers should:}
\begin{itemize}
\item Consult established neuroscience for scientific understanding of consciousness
\item NOT make medical, therapeutic, or life decisions based on these speculations
\item Recognize this as exploratory theoretical work requiring decades of validation
\end{itemize}

\medskip
\noindent See \texttt{CONSCIOUSNESS\_CLAIMS\_ETHICS.md} for ethical guidelines and detailed discussion.
\end{minipage}}
\end{center}
}

% Fine-structure constant disclaimer (Updated November 2025)
\newcommand{\AlphaDerivationDisclaimer}{%
\begin{center}
\fbox{\begin{minipage}{0.95\textwidth}
\textbf{IMPORTANT: FINE-STRUCTURE CONSTANT STATUS (Nov 2025)}

\medskip
\noindent This document discusses the fine-structure constant $\alpha$ within UBT. \textbf{Updated status (November 2025):}

\begin{itemize}
\item \textbf{Dimensional consistency}: Now proven - all quantities have correct dimensions
\item \textbf{Emergent geometric normalization}: $\alpha$ arises from $\Theta$-field self-interaction
\item \textbf{Ratio B/A $\approx$ 20.3}: Determines $n_{opt} = 137$ with energy scale factoring out
\item \textbf{Framework where $\alpha$ might emerge}: Not ab initio parameter-free prediction
\item \textbf{Still contains one adjustable parameter}: B/A ratio not yet uniquely derived
\item \textbf{Honest classification}: Emergent normalization with phenomenological matching
\end{itemize}

\medskip
\noindent \textbf{What would constitute complete derivation:}
\begin{enumerate}
\item Calculate B/A ratio from first principles (without adjustment)
\item Derive discrete parameter N from symmetry/topology alone
\item Show why $\alpha^{-1} = 137.036$ (not just 137) emerges uniquely
\item Account for quantum corrections without additional assumptions
\end{enumerate}

\medskip
\noindent \textbf{Progress made}: Dimensional analysis complete, geometric origin clarified, honest about limitations. \textbf{Remaining challenge}: Derive B/A from first principles or list as input parameter. See \texttt{CHALLENGES\_STATUS\_UPDATE\_NOV\_2025.md} for details.
\end{minipage}}
\end{center}
}

% Short-form disclaimer for appendices
\newcommand{\SpeculativeContentWarning}{%
\noindent\textit{\textbf{Note:} This section contains speculative content that extends beyond experimentally validated physics. See repository documentation for theory status and limitations.}
\medskip
}

% GR Compatibility statement (positive statement about what IS established)
\newcommand{\GRCompatibilityNote}{%
\noindent\textbf{Note on General Relativity Compatibility:} The Unified Biquaternion Theory (UBT) \textbf{generalizes Einstein's General Relativity} by embedding it within a biquaternionic field defined over complex time $\tau = t + i\psi$. In the real-valued limit (where imaginary components vanish), UBT \textbf{exactly reproduces Einstein's field equations} for all curvature regimes. All experimental confirmations of General Relativity are therefore automatically compatible with UBT, as they probe the real sector where the theories are identical. UBT extends (not replaces) GR through additional degrees of freedom that may be relevant for dark sector physics and quantum corrections.
}
 or % THEORY_STATUS_DISCLAIMER.tex
% This file contains standard disclaimers to be included in UBT LaTeX documents
% to ensure proper scientific transparency about the theory's current status.
%
% Usage: \input{THEORY_STATUS_DISCLAIMER} or \input{../THEORY_STATUS_DISCLAIMER}

% Main theory status disclaimer (for general use)
\newcommand{\UBTStatusDisclaimer}{%
\begin{center}
\fbox{\begin{minipage}{0.95\textwidth}
\textbf{WARNING: RESEARCH FRAMEWORK IN DEVELOPMENT}

\medskip
\noindent The Unified Biquaternion Theory (UBT) is currently a \textbf{research framework in early development} (Year 5), not a validated scientific theory. Recent progress (November 2025) includes substantial mathematical formalization, but significant challenges remain:

\begin{itemize}
\item \textbf{Limited peer-review} (not yet externally validated, submission in progress)
\item \textbf{Mathematical foundations}: substantially complete but not yet peer-reviewed
\item \textbf{Testable predictions}: CMB analysis feasible (1-2 years), but most predictions unobservable
\item \textbf{SM gauge group}: now rigorously derived from geometry (Nov 2025)
\item \textbf{Fermion masses}: not yet calculated from first principles
\item \textbf{Complex time}: causality/unitarity partially addressed, active research ongoing
\item \textbf{Consciousness claims}: highly speculative, lack neuroscientific grounding
\end{itemize}

\noindent UBT generalizes Einstein's General Relativity (recovering GR equations in the real limit) but extends beyond validated physics. Treat as \textbf{exploratory research}, not established science.

\medskip
\noindent For detailed assessment and November 2025 updates, see: \texttt{UBT\_UPDATED\_SCIENTIFIC\_RATING\_2025.md}, \texttt{CHALLENGES\_STATUS\_UPDATE\_NOV\_2025.md}, and \texttt{REMAINING\_CHALLENGES\_DETAILED\_STATUS.md}
\end{minipage}}
\end{center}
}

% Consciousness-specific disclaimer
\newcommand{\ConsciousnessDisclaimer}{%
\begin{center}
\fbox{\begin{minipage}{0.95\textwidth}
\textbf{WARNING: SPECULATIVE HYPOTHESIS - CONSCIOUSNESS CLAIMS}

\medskip
\noindent The following content presents \textbf{speculative philosophical ideas} about consciousness that are \textbf{NOT currently supported} by neuroscience or experimental evidence. These ideas represent long-term research directions.

\medskip
\noindent \textbf{Critical Issues:}
\begin{itemize}
\item No operational definition of consciousness in physical terms
\item No connection to established neuroscience findings
\item No testable predictions for brain function or behavior
\item Parameters (psychon mass, coupling constants) completely unspecified
\item Hard problem of consciousness not solved
\end{itemize}

\medskip
\noindent \textbf{Readers should:}
\begin{itemize}
\item Consult established neuroscience for scientific understanding of consciousness
\item NOT make medical, therapeutic, or life decisions based on these speculations
\item Recognize this as exploratory theoretical work requiring decades of validation
\end{itemize}

\medskip
\noindent See \texttt{CONSCIOUSNESS\_CLAIMS\_ETHICS.md} for ethical guidelines and detailed discussion.
\end{minipage}}
\end{center}
}

% Fine-structure constant disclaimer (Updated November 2025)
\newcommand{\AlphaDerivationDisclaimer}{%
\begin{center}
\fbox{\begin{minipage}{0.95\textwidth}
\textbf{IMPORTANT: FINE-STRUCTURE CONSTANT STATUS (Nov 2025)}

\medskip
\noindent This document discusses the fine-structure constant $\alpha$ within UBT. \textbf{Updated status (November 2025):}

\begin{itemize}
\item \textbf{Dimensional consistency}: Now proven - all quantities have correct dimensions
\item \textbf{Emergent geometric normalization}: $\alpha$ arises from $\Theta$-field self-interaction
\item \textbf{Ratio B/A $\approx$ 20.3}: Determines $n_{opt} = 137$ with energy scale factoring out
\item \textbf{Framework where $\alpha$ might emerge}: Not ab initio parameter-free prediction
\item \textbf{Still contains one adjustable parameter}: B/A ratio not yet uniquely derived
\item \textbf{Honest classification}: Emergent normalization with phenomenological matching
\end{itemize}

\medskip
\noindent \textbf{What would constitute complete derivation:}
\begin{enumerate}
\item Calculate B/A ratio from first principles (without adjustment)
\item Derive discrete parameter N from symmetry/topology alone
\item Show why $\alpha^{-1} = 137.036$ (not just 137) emerges uniquely
\item Account for quantum corrections without additional assumptions
\end{enumerate}

\medskip
\noindent \textbf{Progress made}: Dimensional analysis complete, geometric origin clarified, honest about limitations. \textbf{Remaining challenge}: Derive B/A from first principles or list as input parameter. See \texttt{CHALLENGES\_STATUS\_UPDATE\_NOV\_2025.md} for details.
\end{minipage}}
\end{center}
}

% Short-form disclaimer for appendices
\newcommand{\SpeculativeContentWarning}{%
\noindent\textit{\textbf{Note:} This section contains speculative content that extends beyond experimentally validated physics. See repository documentation for theory status and limitations.}
\medskip
}

% GR Compatibility statement (positive statement about what IS established)
\newcommand{\GRCompatibilityNote}{%
\noindent\textbf{Note on General Relativity Compatibility:} The Unified Biquaternion Theory (UBT) \textbf{generalizes Einstein's General Relativity} by embedding it within a biquaternionic field defined over complex time $\tau = t + i\psi$. In the real-valued limit (where imaginary components vanish), UBT \textbf{exactly reproduces Einstein's field equations} for all curvature regimes. All experimental confirmations of General Relativity are therefore automatically compatible with UBT, as they probe the real sector where the theories are identical. UBT extends (not replaces) GR through additional degrees of freedom that may be relevant for dark sector physics and quantum corrections.
}


% Main theory status disclaimer (for general use)
\newcommand{\UBTStatusDisclaimer}{%
\begin{center}
\fbox{\begin{minipage}{0.95\textwidth}
\textbf{WARNING: RESEARCH FRAMEWORK IN DEVELOPMENT}

\medskip
\noindent The Unified Biquaternion Theory (UBT) is currently a \textbf{research framework in early development} (Year 5), not a validated scientific theory. Recent progress (November 2025) includes substantial mathematical formalization, but significant challenges remain:

\begin{itemize}
\item \textbf{Limited peer-review} (not yet externally validated, submission in progress)
\item \textbf{Mathematical foundations}: substantially complete but not yet peer-reviewed
\item \textbf{Testable predictions}: CMB analysis feasible (1-2 years), but most predictions unobservable
\item \textbf{SM gauge group}: now rigorously derived from geometry (Nov 2025)
\item \textbf{Fermion masses}: not yet calculated from first principles
\item \textbf{Complex time}: causality/unitarity partially addressed, active research ongoing
\item \textbf{Consciousness claims}: highly speculative, lack neuroscientific grounding
\end{itemize}

\noindent UBT generalizes Einstein's General Relativity (recovering GR equations in the real limit) but extends beyond validated physics. Treat as \textbf{exploratory research}, not established science.

\medskip
\noindent For detailed assessment and November 2025 updates, see: \texttt{UBT\_UPDATED\_SCIENTIFIC\_RATING\_2025.md}, \texttt{CHALLENGES\_STATUS\_UPDATE\_NOV\_2025.md}, and \texttt{REMAINING\_CHALLENGES\_DETAILED\_STATUS.md}
\end{minipage}}
\end{center}
}

% Consciousness-specific disclaimer
\newcommand{\ConsciousnessDisclaimer}{%
\begin{center}
\fbox{\begin{minipage}{0.95\textwidth}
\textbf{WARNING: SPECULATIVE HYPOTHESIS - CONSCIOUSNESS CLAIMS}

\medskip
\noindent The following content presents \textbf{speculative philosophical ideas} about consciousness that are \textbf{NOT currently supported} by neuroscience or experimental evidence. These ideas represent long-term research directions.

\medskip
\noindent \textbf{Critical Issues:}
\begin{itemize}
\item No operational definition of consciousness in physical terms
\item No connection to established neuroscience findings
\item No testable predictions for brain function or behavior
\item Parameters (psychon mass, coupling constants) completely unspecified
\item Hard problem of consciousness not solved
\end{itemize}

\medskip
\noindent \textbf{Readers should:}
\begin{itemize}
\item Consult established neuroscience for scientific understanding of consciousness
\item NOT make medical, therapeutic, or life decisions based on these speculations
\item Recognize this as exploratory theoretical work requiring decades of validation
\end{itemize}

\medskip
\noindent See \texttt{CONSCIOUSNESS\_CLAIMS\_ETHICS.md} for ethical guidelines and detailed discussion.
\end{minipage}}
\end{center}
}

% Fine-structure constant disclaimer (Updated November 2025)
\newcommand{\AlphaDerivationDisclaimer}{%
\begin{center}
\fbox{\begin{minipage}{0.95\textwidth}
\textbf{IMPORTANT: FINE-STRUCTURE CONSTANT STATUS (Nov 2025)}

\medskip
\noindent This document discusses the fine-structure constant $\alpha$ within UBT. \textbf{Updated status (November 2025):}

\begin{itemize}
\item \textbf{Dimensional consistency}: Now proven - all quantities have correct dimensions
\item \textbf{Emergent geometric normalization}: $\alpha$ arises from $\Theta$-field self-interaction
\item \textbf{Ratio B/A $\approx$ 20.3}: Determines $n_{opt} = 137$ with energy scale factoring out
\item \textbf{Framework where $\alpha$ might emerge}: Not ab initio parameter-free prediction
\item \textbf{Still contains one adjustable parameter}: B/A ratio not yet uniquely derived
\item \textbf{Honest classification}: Emergent normalization with phenomenological matching
\end{itemize}

\medskip
\noindent \textbf{What would constitute complete derivation:}
\begin{enumerate}
\item Calculate B/A ratio from first principles (without adjustment)
\item Derive discrete parameter N from symmetry/topology alone
\item Show why $\alpha^{-1} = 137.036$ (not just 137) emerges uniquely
\item Account for quantum corrections without additional assumptions
\end{enumerate}

\medskip
\noindent \textbf{Progress made}: Dimensional analysis complete, geometric origin clarified, honest about limitations. \textbf{Remaining challenge}: Derive B/A from first principles or list as input parameter. See \texttt{CHALLENGES\_STATUS\_UPDATE\_NOV\_2025.md} for details.
\end{minipage}}
\end{center}
}

% Short-form disclaimer for appendices
\newcommand{\SpeculativeContentWarning}{%
\noindent\textit{\textbf{Note:} This section contains speculative content that extends beyond experimentally validated physics. See repository documentation for theory status and limitations.}
\medskip
}

% GR Compatibility statement (positive statement about what IS established)
\newcommand{\GRCompatibilityNote}{%
\noindent\textbf{Note on General Relativity Compatibility:} The Unified Biquaternion Theory (UBT) \textbf{generalizes Einstein's General Relativity} by embedding it within a biquaternionic field defined over complex time $\tau = t + i\psi$. In the real-valued limit (where imaginary components vanish), UBT \textbf{exactly reproduces Einstein's field equations} for all curvature regimes. All experimental confirmations of General Relativity are therefore automatically compatible with UBT, as they probe the real sector where the theories are identical. UBT extends (not replaces) GR through additional degrees of freedom that may be relevant for dark sector physics and quantum corrections.
}


% Main theory status disclaimer (for general use)
\newcommand{\UBTStatusDisclaimer}{%
\begin{center}
\fbox{\begin{minipage}{0.95\textwidth}
\textbf{WARNING: RESEARCH FRAMEWORK IN DEVELOPMENT}

\medskip
\noindent The Unified Biquaternion Theory (UBT) is currently a \textbf{research framework in early development} (Year 5), not a validated scientific theory. Recent progress (November 2025) includes substantial mathematical formalization, but significant challenges remain:

\begin{itemize}
\item \textbf{Limited peer-review} (not yet externally validated, submission in progress)
\item \textbf{Mathematical foundations}: substantially complete but not yet peer-reviewed
\item \textbf{Testable predictions}: CMB analysis feasible (1-2 years), but most predictions unobservable
\item \textbf{SM gauge group}: now rigorously derived from geometry (Nov 2025)
\item \textbf{Fermion masses}: not yet calculated from first principles
\item \textbf{Complex time}: causality/unitarity partially addressed, active research ongoing
\item \textbf{Consciousness claims}: highly speculative, lack neuroscientific grounding
\end{itemize}

\noindent UBT generalizes Einstein's General Relativity (recovering GR equations in the real limit) but extends beyond validated physics. Treat as \textbf{exploratory research}, not established science.

\medskip
\noindent For detailed assessment and November 2025 updates, see: \texttt{UBT\_UPDATED\_SCIENTIFIC\_RATING\_2025.md}, \texttt{CHALLENGES\_STATUS\_UPDATE\_NOV\_2025.md}, and \texttt{REMAINING\_CHALLENGES\_DETAILED\_STATUS.md}
\end{minipage}}
\end{center}
}

% Consciousness-specific disclaimer
\newcommand{\ConsciousnessDisclaimer}{%
\begin{center}
\fbox{\begin{minipage}{0.95\textwidth}
\textbf{WARNING: SPECULATIVE HYPOTHESIS - CONSCIOUSNESS CLAIMS}

\medskip
\noindent The following content presents \textbf{speculative philosophical ideas} about consciousness that are \textbf{NOT currently supported} by neuroscience or experimental evidence. These ideas represent long-term research directions.

\medskip
\noindent \textbf{Critical Issues:}
\begin{itemize}
\item No operational definition of consciousness in physical terms
\item No connection to established neuroscience findings
\item No testable predictions for brain function or behavior
\item Parameters (psychon mass, coupling constants) completely unspecified
\item Hard problem of consciousness not solved
\end{itemize}

\medskip
\noindent \textbf{Readers should:}
\begin{itemize}
\item Consult established neuroscience for scientific understanding of consciousness
\item NOT make medical, therapeutic, or life decisions based on these speculations
\item Recognize this as exploratory theoretical work requiring decades of validation
\end{itemize}

\medskip
\noindent See \texttt{CONSCIOUSNESS\_CLAIMS\_ETHICS.md} for ethical guidelines and detailed discussion.
\end{minipage}}
\end{center}
}

% Fine-structure constant disclaimer (Updated November 2025)
\newcommand{\AlphaDerivationDisclaimer}{%
\begin{center}
\fbox{\begin{minipage}{0.95\textwidth}
\textbf{IMPORTANT: FINE-STRUCTURE CONSTANT STATUS (Nov 2025)}

\medskip
\noindent This document discusses the fine-structure constant $\alpha$ within UBT. \textbf{Updated status (November 2025):}

\begin{itemize}
\item \textbf{Dimensional consistency}: Now proven - all quantities have correct dimensions
\item \textbf{Emergent geometric normalization}: $\alpha$ arises from $\Theta$-field self-interaction
\item \textbf{Ratio B/A $\approx$ 20.3}: Determines $n_{opt} = 137$ with energy scale factoring out
\item \textbf{Framework where $\alpha$ might emerge}: Not ab initio parameter-free prediction
\item \textbf{Still contains one adjustable parameter}: B/A ratio not yet uniquely derived
\item \textbf{Honest classification}: Emergent normalization with phenomenological matching
\end{itemize}

\medskip
\noindent \textbf{What would constitute complete derivation:}
\begin{enumerate}
\item Calculate B/A ratio from first principles (without adjustment)
\item Derive discrete parameter N from symmetry/topology alone
\item Show why $\alpha^{-1} = 137.036$ (not just 137) emerges uniquely
\item Account for quantum corrections without additional assumptions
\end{enumerate}

\medskip
\noindent \textbf{Progress made}: Dimensional analysis complete, geometric origin clarified, honest about limitations. \textbf{Remaining challenge}: Derive B/A from first principles or list as input parameter. See \texttt{CHALLENGES\_STATUS\_UPDATE\_NOV\_2025.md} for details.
\end{minipage}}
\end{center}
}

% Short-form disclaimer for appendices
\newcommand{\SpeculativeContentWarning}{%
\noindent\textit{\textbf{Note:} This section contains speculative content that extends beyond experimentally validated physics. See repository documentation for theory status and limitations.}
\medskip
}

% GR Compatibility statement (positive statement about what IS established)
\newcommand{\GRCompatibilityNote}{%
\noindent\textbf{Note on General Relativity Compatibility:} The Unified Biquaternion Theory (UBT) \textbf{generalizes Einstein's General Relativity} by embedding it within a biquaternionic field defined over complex time $\tau = t + i\psi$. In the real-valued limit (where imaginary components vanish), UBT \textbf{exactly reproduces Einstein's field equations} for all curvature regimes. All experimental confirmations of General Relativity are therefore automatically compatible with UBT, as they probe the real sector where the theories are identical. UBT extends (not replaces) GR through additional degrees of freedom that may be relevant for dark sector physics and quantum corrections.
}

\SpeculativeContentWarning

\begin{abstract}
We demonstrate how the Unified Biquaternion Theory (UBT), with its fundamental field $\Theta(q, \tau)$ defined over a biquaternionic spacetime $q \in \mathbb{B}$ and complex time $\tau = t + i\psi$, naturally leads to a theory of quantum gravity. Unlike conventional approaches that attempt to quantize spacetime or the metric tensor $g_{\mu\nu}$ directly, UBT derives spacetime geometry and gravitational interaction as emergent phenomena from the quantum properties of the $\Theta$ field.
\end{abstract}

\section{Introduction}

A consistent theory of quantum gravity remains one of the major unsolved problems in physics. Traditional approaches like string theory or loop quantum gravity attempt to quantize the metric structure of spacetime, often leading to formidable mathematical difficulties. UBT provides a fundamentally different and conceptually elegant route.

\section{Field $\Theta(q, \tau)$ as Quantum Substrate}

The field $\Theta$ is a quantum field defined over a non-commutative, algebraically rich space---the space of biquaternions $\mathbb{B}$---and a complexified time variable $\tau$. It obeys a generalized Fokker–Planck-like equation and supports excitations that correspond to physical observables.

\subsection{No Need to Quantize the Metric}

In UBT, the classical spacetime metric $g_{\mu\nu}$ is not fundamental. It arises from derivatives and bilinear forms of the field $\Theta$, via constructs like:
\[
g_{\mu\nu} = \Re\left(\partial_\mu \Theta^\dagger \cdot \partial_\nu \Theta \right)
\]
Thus, quantizing $\Theta$ is sufficient to produce quantum properties of geometry.

\section{Emergent Geometry and Gravitation}

Spacetime and gravity appear as macroscopic manifestations of the underlying dynamics of $\Theta$. This mirrors how fluid dynamics emerges from molecular motion.

\subsection{Analogy}

\begin{itemize}
  \item \textbf{Molecular Level:} Discrete quantum excitations of $\Theta$
  \item \textbf{Macroscopic Level:} Smooth geometry with classical gravitational field
\end{itemize}

\section{Implications for Planck Scale Physics}

At small scales, fluctuations of $\Theta$ generate metric fluctuations---providing a mechanism for quantum spacetime foam and a UV-complete description of gravity.

\section{Conclusion}

UBT avoids the pitfalls of direct metric quantization by focusing on the quantization of a fundamental, algebraic field $\Theta$. This naturally yields a viable framework for quantum gravity.

\section*{License}
This work is licensed under a Creative Commons Attribution 4.0 International License.

\end{document}
