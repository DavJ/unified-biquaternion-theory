% © 2025 Ing. David Jaroš — CC BY-NC-ND 4.0
%
% This work is licensed under a Creative Commons Attribution-NonCommercial-NoDerivatives
% 4.0 International License (CC BY-NC-ND 4.0).
%
% arXiv version of the SU(3) triplet paper.
% This file is a self-contained arXiv submission package.
% Status: Ready for arXiv submission (math-ph)
% Version: 1.0
% Date: 2026-03-01

\documentclass[12pt,a4paper]{article}

\usepackage{amsmath,amssymb,amsthm}
\usepackage{hyperref}
\usepackage{geometry}
\geometry{margin=2.5cm}

% Theorem environments
\newtheorem{theorem}{Theorem}[section]
\newtheorem{lemma}[theorem]{Lemma}
\newtheorem{proposition}[theorem]{Proposition}
\newtheorem{corollary}[theorem]{Corollary}
\newtheorem{definition}[theorem]{Definition}
\theoremstyle{remark}
\newtheorem{remark}[theorem]{Remark}

% Notation shorthands
\newcommand{\B}{\mathcal{B}}
\newcommand{\C}{\mathbb{C}}
\newcommand{\R}{\mathbb{R}}
\newcommand{\H}{\mathbb{H}}
\newcommand{\Vc}{V_c}

\title{%
  A Natural $\mathrm{SU}(3)$ Action on a Canonical Subspace of\\
  $\mathbb{C}\otimes\mathbb{H}$ via $\mathbb{Z}_2^3$ Involutions
}

\author{David Jaro\v{s}}

\date{2026}

\begin{document}

\maketitle

% -------------------------------------------------------
% arXiv abstract — conservative, no physics claims
% -------------------------------------------------------
\begin{abstract}
We identify a canonical three-dimensional complex subspace $V_c$ inside the
biquaternion algebra $\mathcal{B} = \mathbb{C}\otimes_\mathbb{R}\mathbb{H}$,
constructed as the $P_2=-1$ eigenspace of the quaternion-conjugation involution.
The subspace $V_c = \mathrm{span}_\mathbb{C}\{\mathbf{I},\mathbf{J},\mathbf{K}\}$
carries a natural Hermitian inner product, and we prove that the group of
$\mathbb{C}$-linear isometries of $V_c$ with determinant one is isomorphic to
$\mathrm{SU}(3)$.  The construction uses two additional commuting real-linear
involutions (complex conjugation $P_1$ and an axis-flip $P_3$), whose joint
spectrum decomposes $\mathcal{B}$ into a $\mathbb{Z}_2^3$-graded direct sum.
The entire argument is algebraic and self-contained, requires no octonions, and
uses only the associative algebra $\mathcal{B}\cong M_2(\mathbb{C})$.
We compare with classical results ($\mathrm{Aut}(\mathbb{H})\cong\mathrm{SO}(3)$,
the $\mathrm{GL}(2,\mathbb{C})$ automorphism obstruction, Dixon and Furey) and
confirm that our construction is complementary and non-contradictory.
Open problems are stated, including whether the $\mathrm{SU}(3)$ action
extends to a dynamical symmetry of natural kinetic operators.
\end{abstract}

% MSC codes for arXiv math-ph
% Primary: 17A35 (nonassociative rings and algebras, general)
% Secondary: 22E70 (applications of Lie groups to physics), 81R05
\noindent\textbf{MSC 2020:} 17A35, 22E70, 81R05.\\
\noindent\textbf{Keywords:} biquaternions, involutions, SU(3), Hermitian form, associative algebra.

\tableofcontents

% ============================================================
\section{Introduction}
\label{sec:intro}
% ============================================================

The biquaternion algebra $\mathcal{B} = \mathbb{C}\otimes_\mathbb{R}\mathbb{H}$
is an 8-dimensional real associative algebra isomorphic to $M_2(\mathbb{C})$.
It has long appeared in mathematical physics, from Hamilton's biquaternions
\cite{Hamilton1866} to modern treatments in Clifford algebras \cite{Porteous1995}
and algebraic approaches to fundamental physics \cite{Dixon1994,Furey2016}.

This paper proves a purely algebraic result: starting from a system of three
commuting real-linear involutions on $\mathcal{B}$, one can extract a canonical
$\mathbb{C}^3$-subspace carrying a natural $\mathrm{SU}(3)$ symmetry.

\medskip
\noindent\textbf{Main theorems} (proved in Sections~\ref{sec:involutions}--\ref{sec:su3}):
\begin{enumerate}
  \item \textbf{Involution decomposition} (Theorem~\ref{thm:involution_decomp}):
    Three commuting involutions $P_1,P_2,P_3$ produce a $\mathbb{Z}_2^3$-graded
    decomposition of $\mathcal{B}$.
  \item \textbf{Canonical triplet space} (Theorem~\ref{thm:triplet_space}):
    The $P_2=-1$ eigenspace is $V_c\cong\mathbb{C}^3$ with a canonical
    Hermitian form.
  \item \textbf{SU(3) action preserves Hermitian form}
    (Theorem~\ref{thm:su3_action}):
    $\mathrm{SU}(3)_{V_c}\cong\mathrm{SU}(3)$ acts faithfully on $V_c$
    preserving the inner product.
\end{enumerate}

\medskip
\noindent\textbf{Explicit non-claims.}  We do not claim:
\begin{itemize}
  \item that $\mathrm{SU}(3)_{V_c}$ is the colour gauge group of QCD;
  \item that our construction proves any Standard Model gauge group derivation;
  \item that the result is related to masses, coupling constants, or gravitational physics.
\end{itemize}
These remain open problems (Section~\ref{sec:open}).

% ============================================================
\section{Algebraic Preliminaries}
\label{sec:prelim}
% ============================================================

Let $\mathbb{H}$ be the quaternion $\mathbb{R}$-algebra with units
$\mathbf{I},\mathbf{J},\mathbf{K}$ satisfying $\mathbf{I}^2=\mathbf{J}^2
=\mathbf{K}^2=-1$ and $\mathbf{I}\mathbf{J}=\mathbf{K}$ (cyclic).  Set
$\mathcal{B}:=\mathbb{C}\otimes_\mathbb{R}\mathbb{H}$ with the ordered
$\mathbb{R}$-basis
\[
  \mathcal{B}_\mathrm{basis} = \{1,\mathbf{I},\mathbf{J},\mathbf{K},
    i,i\mathbf{I},i\mathbf{J},i\mathbf{K}\}.
\]
The complex unit $i$ is central: $ih=hi$ for all $h\in\mathbb{H}$
(Lemma~2.1 of the full paper \cite{mainpaper}).

% ============================================================
\section{Three Commuting Involutions}
\label{sec:involutions}
% ============================================================

\begin{definition}
  \label{def:involutions}
  Define $\mathbb{R}$-linear maps $P_k:\mathcal{B}\to\mathcal{B}$:
  $P_1$ acts by $i\mapsto -i$ fixing $\mathbf{I},\mathbf{J},\mathbf{K}$;
  $P_2$ acts by $\mathbf{X}\mapsto -\mathbf{X}$ for $\mathbf{X}\in\{
  \mathbf{I},\mathbf{J},\mathbf{K}\}$ fixing $i$;
  $P_3(x)=\mathbf{I}x\mathbf{I}^{-1}$.
\end{definition}

\begin{theorem}[Involution decomposition]
  \label{thm:involution_decomp}
  Each $P_k^2=\mathrm{id}$; the $P_k$ commute pairwise; they generate a
  $\mathbb{Z}_2^3$ action decomposing $\mathcal{B}$ into simultaneous
  eigenspaces as in Theorem~1 of \cite{mainpaper}.
  The sectors $(+,+,-)$ and $(-,+,-)$ are provably empty.
\end{theorem}

% ============================================================
\section{The Canonical Triplet Space}
\label{sec:triplet}
% ============================================================

\begin{theorem}[Canonical triplet space]
  \label{thm:triplet_space}
  $V_c := \ker(P_2+\mathrm{id})
  = \mathrm{span}_\mathbb{C}\{\mathbf{I},\mathbf{J},\mathbf{K}\}$
  satisfies $\dim_\mathbb{C}V_c=3$ and $V_c\cong\mathbb{C}^3$.
  The Hermitian form $\langle X,Y\rangle = \frac{1}{4}\mathrm{Tr}(X^\dagger Y)$
  is the standard inner product on $\mathbb{C}^3$ under this identification.
\end{theorem}

% ============================================================
\section{SU(3) Action}
\label{sec:su3}
% ============================================================

\begin{theorem}[$\mathrm{SU}(3)$ preserves the Hermitian form]
  \label{thm:su3_action}
  The group $\mathrm{SU}(3)_{V_c}$ of $\mathbb{C}$-linear unitary
  det-1 maps on $(V_c,\langle\cdot,\cdot\rangle)$ is isomorphic to
  $\mathrm{SU}(3)$, acts faithfully on $V_c$, and preserves the Hermitian
  form.  It is \emph{not} a subgroup of $\mathrm{Aut}_\mathbb{R}(\mathcal{B})$.
\end{theorem}

\begin{remark}[Non-contradiction with classical results]
  $\mathrm{Aut}_\mathbb{R}(\mathcal{B})\cong[\mathrm{GL}(2,\mathbb{C})^2]/\mathbb{Z}_2$
  does not contain $\mathrm{SU}(3)$ \cite{Porteous1995}, consistent with
  Remark~\ref{rem:not_aut} and the GL(2,ℂ) rank obstruction.
  Our result does not contradict this: $\mathrm{SU}(3)_{V_c}$ is a symmetry
  of the \emph{subspace} $V_c$, not of the algebra.
\end{remark}

% ============================================================
\section{Open Problems}
\label{sec:open}
% ============================================================

\begin{enumerate}
  \item Does $\mathrm{SU}(3)_{V_c}$ extend to a symmetry of a natural
    kinetic operator on $V_c$-valued fields?
  \item Can a local (gauge) $\mathrm{SU}(3)$ be constructed consistently
    within the associative framework?
  \item Is the construction of $V_c$ canonical up to
    $\mathrm{Aut}_\mathbb{R}(\mathcal{B})$?
  \item Does any physical interpretation of $\mathrm{SU}(3)_{V_c}$ follow
    without additional input?
\end{enumerate}

% ============================================================
\section{Conclusion}
\label{sec:conclusion}
% ============================================================

We have proved three theorems establishing a natural, associative, octonion-free
$\mathrm{SU}(3)$ action on a canonical subspace of $\mathcal{B}=\mathbb{C}\otimes\mathbb{H}$.
The result is self-contained and complementary to, rather than contradicting,
classical algebra theorems.  Physical interpretation and dynamical extension
remain open problems.

\begin{thebibliography}{99}

\bibitem{mainpaper}
D.~Jaro\v{s},
\emph{A Natural SU(3) Action on a Canonical Subspace of
$\mathbb{C}\otimes\mathbb{H}$ via $\mathbb{Z}_2^3$ Involutions},
full version, \texttt{papers/su3\_triplet/main.tex} (2026).

\bibitem{Hamilton1866}
W.~R. Hamilton,
\emph{Elements of Quaternions}, Longmans, Green, 1866.

\bibitem{Clifford1878}
W.~K. Clifford,
Applications of Grassmann's Extensive Algebra,
\emph{Am.\ J.\ Math.} \textbf{1} (1878) 350--358.

\bibitem{Dixon1994}
G.~M. Dixon,
\emph{Division Algebras: Octonions, Quaternions, Complex Numbers and the
Algebraic Design of Physics}, Kluwer, 1994.

\bibitem{Furey2016}
C.~Furey,
Standard Model Physics from an Algebra?,
PhD Thesis, U.~Waterloo, 2015; \texttt{arXiv:1611.09182}.

\bibitem{Connes1994}
A.~Connes,
\emph{Noncommutative Geometry}, Academic Press, 1994.

\bibitem{Porteous1995}
I.~R. Porteous,
\emph{Clifford Algebras and the Classical Groups}, Cambridge U.P., 1995.

\end{thebibliography}

\end{document}
