\documentclass[11pt]{article}

\usepackage{amsmath, amssymb, amsthm}
\usepackage{geometry}
\usepackage{hyperref}
\usepackage{bm}

\geometry{margin=1in}

\title{Spectral Stability of Twin-Prime Modes in a Toroidal Unified Biquaternion Framework}
\author{David Jaroš}
\date{\today}

\newtheorem{theorem}{Theorem}
\newtheorem{lemma}{Lemma}
\newtheorem{definition}{Definition}
\newtheorem{corollary}{Corollary}

\begin{document}
\maketitle

\begin{abstract}
We present a spectral analysis of the Unified Biquaternion Theory (UBT), modeling the pre-geometric substrate of the universe as an $N$-dimensional torus. We demonstrate that under projection to observable four-dimensional spacetime, only a restricted class of eigenmodes of the toroidal Laplacian survives. In particular, we prove that stable, non-degenerate, information-carrying modes necessarily occur in twin-prime pairs. This provides a structural explanation for the persistent appearance of the number $137$ in fundamental physics and predicts the unavoidable presence of its twin $139$ as a complementary carrier mode.
\end{abstract}

\section{Motivation}

Empirical physics repeatedly encounters the fine-structure constant $\alpha \approx 1/137.036$, yet no consensus exists on why this number appears to be privileged. Rather than treating $137$ as a numerological artifact or isolated constant, we investigate whether it arises as a \emph{spectrally stable mode} of an underlying geometric structure.

In the Unified Biquaternion Theory (UBT), spacetime is not fundamental but emerges as a projection of a higher-dimensional toroidal manifold. The goal of this work is to identify which spectral modes of this torus remain stable under projection and therefore can manifest as observable physical constants.

\section{Geometric Framework}

\subsection{Toroidal Pre-Geometry}

We assume the fundamental configuration space is an $N$-dimensional torus
\[
\mathbb{T}^N = \mathbb{R}^N / (2\pi \mathbb{Z})^N,
\]
equipped with a positive-definite metric tensor $G$.

Coordinates on the torus are angular variables
\[
\bm{\theta} = (\theta_1, \dots, \theta_N), \quad \theta_i \sim \theta_i + 2\pi.
\]

\subsection{Laplace Operator on the Torus}

The Laplace--Beltrami operator on $\mathbb{T}^N$ is
\[
\Delta_{\mathbb{T}^N} = \sum_{i,j} (G^{-1})_{ij} \frac{\partial^2}{\partial \theta_i \partial \theta_j}.
\]

Its eigenfunctions are plane waves
\[
\psi_{\bm{n}}(\bm{\theta}) = e^{i \bm{n} \cdot \bm{\theta}}, \quad \bm{n} \in \mathbb{Z}^N,
\]
with eigenvalues
\[
\lambda(\bm{n}) = \bm{n}^T G^{-1} \bm{n}.
\]

These integer vectors $\bm{n}$ replace the familiar spherical harmonic quantum numbers $(\ell,m)$.

\section{Projection to Four Dimensions}

\subsection{Projection Operator}

Observable spacetime corresponds to a projection
\[
\Pi: \mathbb{T}^N \to \mathbb{R}^{1,3}.
\]

This projection induces a linear operator on eigenmodes. Modes lying in the kernel
\[
\ker(\Pi) = \{ \psi_{\bm{n}} \mid \Pi(\psi_{\bm{n}}) = 0 \}
\]
do not survive as observable degrees of freedom.

\subsection{Degeneracy-Induced Collapse}

If two distinct integer vectors $\bm{n}, \bm{m} \in \mathbb{Z}^N$ satisfy
\[
\lambda(\bm{n}) = \lambda(\bm{m}),
\]
their corresponding eigenfunctions mix under projection and interfere destructively. Such modes generically lie in $\ker(\Pi)$ and are eliminated.

\section{Stability Criterion}

\begin{definition}[Spectrally Stable Mode]
A mode $\bm{n}$ is called \emph{spectrally stable} if:
\begin{enumerate}
\item Its eigenvalue $\lambda(\bm{n})$ is non-degenerate.
\item It admits no alternative integer decomposition with equal or nearby eigenvalue.
\item It survives projection: $\psi_{\bm{n}} \notin \ker(\Pi)$.
\end{enumerate}
\end{definition}

\section{Twin-Prime Stability Theorem}

We now restrict to an effective two-dimensional toroidal subspace corresponding to the two dominant cycles of the UBT torus.

\subsection{Integer Mode Pairs}

Let $(n_1,n_2) \in \mathbb{Z}^2$ denote the mode numbers. The eigenvalue is
\[
\lambda(n_1,n_2) = a n_1^2 + b n_2^2,
\]
with $a,b>0$ determined by $G$.

\begin{theorem}[Twin-Prime Stability]
Let $p$ and $p+2$ be twin primes. Then the pair $(p, p+2)$ defines a spectrally stable mode pair. Conversely, no pair of composite integers $(n,n+2)$ yields a spectrally stable mode.
\end{theorem}

\begin{proof}
Consider the eigenvalue difference:
\[
\lambda(p+2) - \lambda(p) = a[(p+2)^2 - p^2] = a(4p+4).
\]

This is the minimal positive separation between distinct eigenvalues for integer modes.

Now assume $p$ is prime.

\begin{enumerate}
\item \textbf{Uniqueness:}  
Because $p$ is prime, there exists no nontrivial integer factorization producing the same quadratic form value. Hence no alternative integer vector $\bm{m}$ satisfies $\lambda(\bm{m}) = \lambda(p)$.

\item \textbf{Non-degeneracy:}  
Twin primes guarantee that $p$ and $p+2$ are isolated from other primes by at least $2$, preventing near-degenerate eigenvalues.

\item \textbf{Minimal coupling:}  
The pair $(p,p+2)$ forms the minimal-energy doublet that cannot be decomposed into smaller interacting modes.

\item \textbf{Composite failure:}  
If $n$ is composite, then $n = ab$ admits multiple representations in $\mathbb{Z}^2$, leading to degeneracy:
\[
n^2 = (ab)^2 = a^2 b^2,
\]
which allows alternative eigenmodes with identical or near-identical eigenvalues. These modes mix and fall into $\ker(\Pi)$.
\end{enumerate}

Thus, twin primes uniquely satisfy the stability criteria.
\end{proof}

\begin{corollary}
All isolated primes and all composite integers are eliminated under projection. Only twin-prime mode pairs survive as stable carriers of information.
\end{corollary}

\section{Information-Theoretic Interpretation}

The surviving twin-prime pair constitutes a dual-carrier system analogous to Orthogonal Frequency-Division Multiplexing (OFDM). Information is encoded not in absolute frequencies but in:
\begin{itemize}
\item relative phase,
\item relative amplitude,
\item cross-channel coherence.
\end{itemize}

This structure is robust against noise and projection-induced distortion.

\section{Implications}

\begin{enumerate}
\item The fine-structure constant $\alpha^{-1} \approx 137$ arises as a spectrally stable toroidal mode.
\item Its twin $139$ is not optional but mandatory.
\item Observable physics accesses only the interference of these twin modes, never individual carriers.
\item This explains why attempts to detect a single isolated frequency fail.
\end{enumerate}

\section{Conclusion}

We have shown that twin primes emerge as the unique spectrally stable modes of a toroidal pre-geometric substrate under projection to four-dimensional spacetime. This result is independent of numerology and follows from the spectral theory of the Laplacian and the arithmetic structure of the integers. The ubiquitous appearance of $137$ in physics is therefore not accidental but a structural necessity.

\section*{Acknowledgments}

The author thanks iterative computational experiments and critical analysis for eliminating false positives and isolating the correct geometric mechanism.

\end{document}

