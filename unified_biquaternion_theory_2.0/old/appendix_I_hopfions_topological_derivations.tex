
\appendix
\section*{Appendix I: Hopfions and Topological Field Configurations in the Unified Biquaternion Theory}
\addcontentsline{toc}{section}{Appendix I: Hopfions and Topological Field Configurations in the Unified Biquaternion Theory}

\subsection*{I.1 Introduction}
In the Unified Biquaternion Theory (UBT), \emph{hopfions} are finite-energy knotted field configurations characterized by a nontrivial Hopf invariant $Q_H\in\mathbb{Z}$. 
They emerge as topological solitons within the master field $\Theta(q,\tau)$, where $\tau=t+i\psi$. 
This appendix provides \emph{derivations} underlying the qualitative predictions: field equations, scaling laws, stability bounds, and leading-order laboratory/astrophysical signatures.

\subsection*{I.2 Hopf Invariant and Gauge Construction}
Let $z(\mathbf{x})\in\mathbb{C}^2$ with $z^\dagger z=1$ parametrize $S^3$ and define $n^a=z^\dagger\sigma^a z\in S^2$.
The $U(1)$ connection $A_i=-i\,z^\dagger\partial_i z$ has curvature $F_{ij}=\partial_i A_j-\partial_j A_i$. 
The Hopf invariant is
\begin{equation}
Q_H=\frac{1}{4\pi^2}\int_{\mathbb{R}^3} d^3x\; \epsilon^{ijk} A_i F_{jk}\in\mathbb{Z}.
\label{eq:HopfQ}
\end{equation}
Finite energy enforces $n(\mathbf{x})\to n_\infty$ at $|\mathbf{x}|\to\infty$, compactifying space to $S^3$ and ensuring $Q_H$ is topological. 
(For an explicit $Q_H{=}1$ map and estimates, see Appendix G.)

\subsection*{I.3 Energy Functional from UBT Lagrangian}
Projecting the $\Theta$-sector to a unit vector $n(\mathbf{x})\in S^2$ yields the (generalized) Faddeev--Skyrme energy
\begin{equation}
E[n]=\int d^3x\;\Big\{\alpha\,(\partial_i n)^2+\beta\, \big(\epsilon^{ijk}\,n\cdot\partial_i n\times\partial_j n\big)^2 \Big\},
\label{eq:FSenergy}
\end{equation}
with $\alpha,\beta>0$ functions of UBT couplings (Appendix H). 
The Euler--Lagrange equation reads
\begin{align}
0&=\frac{\delta E}{\delta n}-\lambda(\mathbf{x})\,n
= -2\alpha\,\Delta n -4\beta\,\partial_i\Big[ \big(n\cdot\partial_i n\times\partial_j n\big)\, \partial_j n\times n \Big]-\lambda n,\quad n\cdot n=1,
\label{eq:EL}
\end{align}
where $\lambda$ enforces the $|n|=1$ constraint. 
Stable hopfions solve \eqref{eq:EL} in a fixed $Q_H$ sector.

\subsection*{I.4 Derrick Scaling and Stability Bound}
Under spatial scaling $n(\mathbf{x})\mapsto n(\lambda\mathbf{x})$, one finds
\begin{equation}
E(\lambda)=\alpha\,\lambda\,I_2+\beta\,\lambda^{-1} I_4,\qquad
I_2=\int (\partial_i n)^2,\ \ I_4=\int \big(\epsilon^{ijk}n\cdot\partial_i n\times\partial_j n\big)^2.
\end{equation}
Stationarity $\partial_\lambda E|_{\lambda_\star}=0$ gives $\lambda_\star=\sqrt{\beta I_4/(\alpha I_2)}$ and
\begin{equation}
E_{\min}=2\sqrt{\alpha\beta\,I_2 I_4}.
\end{equation}
Using geometric inequalities $I_2 I_4 \ge C\,|Q_H|^{3/2}$ yields the Vakulenko--Kapitanski bound
\begin{equation}
E \;\ge\; c\, |Q_H|^{3/4},\qquad c=2\sqrt{\alpha\beta\,C},
\label{eq:VK}
\end{equation}
ensuring stability against collapse for fixed $Q_H$.

\subsection*{I.5 Complex-Time Corrections and Small $\psi$-Sector Effects}
Allow a slow $\psi$-dependence of coefficients: $\alpha\to\alpha(\psi)$, $\beta\to\beta(\psi)$, and include a weak kinetic term $\frac12\kappa_\psi(\partial_\psi n)^2$ in the effective Lagrangian. 
Linearizing around a static hopfion solution $n_0$ gives the fluctuation operator
\begin{equation}
\mathcal{H}_{\mathrm{fluct}} \sim -2\alpha\,\Delta + \text{Skyrme terms} + \kappa_\psi\,\partial_\psi^2 + \cdots,
\end{equation}
whose positive spectrum (modulo zero modes) ensures linear stability. 
The $\psi$-sector shifts $E$ and the size $R_H\sim\sqrt{\beta/\alpha}$ by $\mathcal{O}(\partial_\psi^2)$ terms but preserves the topological bound \eqref{eq:VK}.

\subsection*{I.6 Couplings to Electromagnetism and Suppression of Cross Sections}
At leading order, gauge-invariant low-energy couplings read
\begin{equation}
\mathcal{L}_{\mathrm{int}}=- g_{nA}\,(\partial_i n\cdot\partial_j n) F^{ij} - g_{nF}\, (F_{ij})^2 (\partial_k n)^2 + \cdots.
\label{eq:Lint}
\end{equation}
For an incoming photon of momentum $k$ scattering off a static hopfion of size $R_H$, the dominant amplitude scales as
\begin{equation}
\mathcal{M} \sim g_{nA}\, (k R_H)^2\quad\Rightarrow\quad \sigma_{\gamma H}\sim |\mathcal{M}|^2 \sim g_{nA}^2\,(k R_H)^4,
\end{equation}
implying strong suppression at long wavelengths ($kR_H\ll 1$). 
This underlies the \emph{dark} character (Appendix G) while allowing tiny polarization-dependent effects in extreme fields (Appendix D).

\subsection*{I.7 Cavity Frequency Shift (Laboratory Signature)}
In a toroidal superconducting cavity (Appendix E), the hopfion-induced renormalization of the EM energy density modifies eigenfrequencies. 
To first order in \eqref{eq:Lint},
\begin{equation}
\frac{\Delta f}{f} \;=\; -\frac{1}{2}\frac{\int d^3x\ \delta\epsilon_{\mathrm{eff}}(\mathbf{x})\, |E(\mathbf{x})|^2 + \delta\mu_{\mathrm{eff}}(\mathbf{x})\, |B(\mathbf{x})|^2}{\int d^3x\ \epsilon_0 |E|^2 + \mu_0^{-1}|B|^2}\,,
\end{equation}
with effective perturbations $\delta\epsilon_{\mathrm{eff}},\delta\mu_{\mathrm{eff}}\propto g_{nA}\,(\partial n)^2 + g_{nF}\,(\partial n)^4$. 
For a localized hopfion of size $R_H$ at a field antinode,
\begin{equation}
\left|\frac{\Delta f}{f}\right| \sim \mathcal{O}\!\big( g_{nA}\, \langle(\partial n)^2\rangle\, V_H/V_{\mathrm{mode}} \big)\ \ll\ 1,
\end{equation}
where $V_H\sim R_H^3$ and $V_{\mathrm{mode}}$ is the mode volume. This provides a quantitative target for high-$Q$ measurements.

\subsection*{I.8 Astrophysical Scaling (Lensing / Rotation Curves)}
An ensemble of hopfions with number density $n_H$ and mean energy $E_H$ contributes $\rho_{\mathrm{DM}}=n_H E_H$. 
In spherical halos, the circular velocity obeys $v_c^2(r)=G M(<r)/r$; fitting $M(<r)=\int_0^r 4\pi r'^2 \rho_{\mathrm{DM}}(r')dr'$ with $E_H(Q_H)$ from \eqref{eq:VK} yields constraints on $n_H(r)$ and $R_H(r)$. 
Weak-lensing convergence $\kappa(\theta)\propto\Sigma(\theta)$ is likewise sensitive to the hopfion density profile, enabling consistency checks with observed NFW-like profiles.

\subsection*{I.9 Summary of Derived Predictions}
\begin{itemize}
\item \textbf{Stability \& size:} $R_H\sim \sqrt{\beta/\alpha}$, $E\ge c|Q_H|^{3/4}$; small $\psi$-sector corrections do not spoil the bound.
\item \textbf{EM scattering:} $\sigma_{\gamma H}\propto (k R_H)^4$ (long-wavelength suppression), enabling \emph{dark} behavior.
\item \textbf{Cavity shift:} $\Delta f/f$ proportional to $g_{nA}\langle(\partial n)^2\rangle$ times volume fraction $V_H/V_{\mathrm{mode}}$ (target for Appendix E setups).
\item \textbf{Astrophysics:} halo mass profiles and lensing consistent with ensembles of topologically stable knots parameterized by $(\alpha,\beta,Q_H)$.
\end{itemize}

\subsection*{I.10 Notes and Cross-References}
Detailed constructions, $Q_H{=}1$ examples, and order-of-magnitude numerics are provided in Appendix G. 
Couplings to QED and constraints from precision experiments are discussed in Appendix D; laboratory protocols in Appendix E; $\psi$-sector dynamics in Appendix F; relations to constants in Appendix H.
