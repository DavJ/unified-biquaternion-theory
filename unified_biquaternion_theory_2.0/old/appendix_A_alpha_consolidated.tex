
\section{Appendix A: Fine-Structure Constant in UBT — Topology, Precise Running, and Prime/$p$-Adic Multiplexing}
\label{app:alpha-consolidated}

\subsection{Topological origin: \(\alpha_0^{-1}=N\)}
We adopt the Chern quantization of the \(U(1)\) bundle (Appendix~\ref{app:adelic-holonomy}):
\begin{equation}
\frac{1}{2\pi}\int_{\Sigma} F = N \in \mathbb{Z}\,, \qquad \Rightarrow \qquad \alpha_0^{-1}=N\,.
\end{equation}
For the physical branch we take a prime, indecomposable class \(N=137\).

\subsection{Precise low-energy value via QED running}
Quantum vacuum polarization shifts \(\alpha\) according to one-loop QED
\begin{equation}
\alpha^{-1}(\mu) = N + \frac{1}{3\pi}\ln\!\frac{m_e^2}{\mu^2} \;(+\;\text{thresholds and higher orders})\,.
\end{equation}
With \(N=137\) this is satisfied by \(\mu/m_e \approx 0.84397\), consistent with the internal-mode scale of the UBT electron solution.
Higher-order and threshold terms can be included in the standard way without altering the topological identification of \(N\).

\subsection{Prime/$p$-adic multiplexing (optional hypothesis)}
As a conservative extension (see Appendix~\ref{app:padic-prelim}), one can regard prime \(N\) as a discrete ``channel'' labeling independent
theta-modes of the complex-time torus. This suggests a family of prime branches with
\begin{equation}
\alpha_0^{-1}=N\,, \quad N\in\mathbb{P}\,,
\end{equation}
where nearby primes (e.g. \(131, 139\)) give universes similar to ours after the same QED running. The $p$-adic layer
provides a natural arithmetic underpinning of the prime preference; we keep it as an interpretive add-on without changing the
core derivation above.
