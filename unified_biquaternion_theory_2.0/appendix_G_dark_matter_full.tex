
\appendix{G}{Dark Matter as Topological Excitations in $\Theta(q,\tau)$ Field}

\section{Introduction}
In the Unified Biquaternion Theory (UBT), dark matter (DM) emerges naturally as a set of topologically protected excitations in the fundamental $\Theta(q,\tau)$ field, where $q$ are the biquaternionic coordinates and $\tau = t + i\psi$ is the complex time. The theory identifies these excitations with \emph{hopfions}, knotted field configurations classified by the Hopf index $Q_H \in \mathbb{Z}$, which are stable due to topological constraints.

\section{Relation to Baryonic Matter and Pseudospin States}
While baryonic matter corresponds to localized, symmetry-broken excitations with strong coupling to the $SU(3) \times SU(2) \times U(1)$ gauge fields, dark matter modes occupy orthogonal pseudospin sectors of $\Theta(q,\tau)$.
The $\psi$ (imaginary time) component effectively decouples them from direct gauge interactions in our visible sector. This explains why dark matter interacts only gravitationally and via higher-order topological couplings.

Let $\Theta = (\theta_1, \theta_2, ..., \theta_N)$ denote the components of the field in an $SU(2)$ pseudospin basis. Baryonic matter resides in subspaces with $\psi \approx 0$, whereas DM modes have $\psi \neq 0$, leading to an effective suppression factor
\begin{equation}
g_{\mathrm{eff}} \sim e^{-\frac{\psi^2}{\psi_0^2}},
\end{equation}
where $\psi_0$ is a characteristic phase scale.

\section{Weak Interaction Mechanism}
The suppression factor above follows from integrating out the orthogonal modes in the path integral formulation of UBT. The leading-order interaction term between DM hopfions and baryons has the schematic form:
\begin{equation}
\mathcal{L}_{\mathrm{int}} \sim \frac{\lambda_{\mathrm{top}}}{M_{\mathrm{Pl}}^2} \, J_{\mathrm{baryon}}^\mu J_{\mathrm{DM},\mu},
\end{equation}
where $\lambda_{\mathrm{top}}$ is a dimensionless topological coupling constant, $J_{\mathrm{baryon}}^\mu$ is the baryonic current, and $J_{\mathrm{DM}}^\mu$ is the dark matter topological current. The Planck-scale suppression explains the extreme weakness of the interaction.

\section{Energy Density Calculation}
Let the spectrum of DM topological modes be indexed by $n$, with each mode having an energy
\begin{equation}
E_n = \hbar \omega_0 \sqrt{n^2 + \alpha Q_H^2},
\end{equation}
where $\alpha$ is a coupling constant between the mode number and the Hopf index.

Assuming a thermal-like distribution in the early universe with an effective temperature $T_{\mathrm{DM}}$, the energy density becomes
\begin{equation}
\rho_{\mathrm{DM}} = \frac{1}{V} \sum_{n,Q_H} g_{n,Q_H} \, E_n \, e^{-E_n/k_B T_{\mathrm{DM}}}.
\end{equation}
For a continuous approximation and $T_{\mathrm{DM}} \ll \hbar\omega_0$, the integral yields
\begin{equation}
\rho_{\mathrm{DM}} \approx \frac{g_{\mathrm{eff}} \, (\hbar\omega_0)^{5/2}}{(2\pi)^{3/2}} \, e^{-\hbar\omega_0/k_B T_{\mathrm{DM}}}.
\end{equation}
Matching to Planck satellite cosmological measurements $\rho_{\mathrm{DM}}/\rho_{\mathrm{crit}} \approx 0.265$ allows extraction of $\omega_0$ or $g_{\mathrm{eff}}$ in terms of fundamental UBT parameters.

\section{Experimental Detection in UBT Framework}
A possible detection method involves modulating the phase $\psi$ in a toroidal $\Theta$-resonator, tuned to frequencies corresponding to $\omega_0$ of the DM spectrum. The coupling is expected to produce minute phase shifts or sideband structures in the resonator's electromagnetic response.

\subsection{Dark Photons and $p$-adic Extension}
In the $p$-adic extension of UBT, each prime $p$ defines an independent $\Theta_p(q,\tau)$ sector. In some universes, the dominant DM component may correspond to $p$-adic photons (``dark photons''), which are massless gauge bosons in the $\Theta_p$ sector, invisible to our $p= \infty$ (real-number) sector except via gravity or suppressed topological mixing.

The mixing Lagrangian can be schematically written as:
\begin{equation}
\mathcal{L}_{\mathrm{mix}} \sim \epsilon_{p} F^{\mu\nu}(\Theta_{\infty}) F_{\mu\nu}(\Theta_p),
\end{equation}
where $\epsilon_{p} \ll 1$ is the $p$-dependent kinetic mixing parameter.

\section{Conclusion}
The UBT description of dark matter as hopfions and other topological modes in $\Theta(q,\tau)$ provides a unified explanation for its stability, weak interactions, and observed abundance. The inclusion of $p$-adic sectors opens a pathway for describing alternate-reality DM components such as dark photons, potentially accessible through precision topological resonance experiments.
