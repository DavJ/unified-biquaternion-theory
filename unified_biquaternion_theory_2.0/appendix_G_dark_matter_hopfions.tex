
\appendix
\section*{Appendix G: Dark Matter, Hopfions, and Topological Excitations in the $\Theta$-Field}
\addcontentsline{toc}{section}{Appendix G: Dark Matter, Hopfions, and Topological Excitations in the $\Theta$-Field}

\subsection*{G.1 Motivation}
Within the Unified Biquaternion Theory (UBT), the master field $\Theta(q,\tau)$ allows for nontrivial topological configurations. 
A particularly relevant class are \emph{Hopfions}---localized, finite-energy field configurations characterized by a nonzero Hopf invariant. 
We outline their mathematical structure, dynamics in complex time $\tau=t+i\psi$, energetic properties, and gravitational phenomenology as candidates for (a component of) dark matter.

\subsection*{G.2 Hopf Map and Topological Index}
The canonical Hopf fibration maps $S^3 \to S^2$. 
Introduce a complex doublet $z=(z_1,z_2)^T\in\mathbb{C}^2$ with $z^\dagger z = 1$ (so that $z$ parametrizes $S^3$). 
The unit vector $n:\mathbb{R}^3\cup\{\infty\}\simeq S^3 \to S^2$ is
\begin{equation}
n^a = z^\dagger \sigma^a z,\qquad a=1,2,3,
\end{equation}
where $\sigma^a$ are the Pauli matrices. The $U(1)$ connection and curvature are
\begin{equation}
A_i = -i\, z^\dagger \partial_i z,\qquad F_{ij} = \partial_i A_j - \partial_j A_i.
\end{equation}
The \emph{Hopf invariant} is
\begin{equation}
Q_H \;=\; \frac{1}{4\pi^2}\int_{\mathbb{R}^3} \mathrm{d}^3x\; \epsilon^{ijk}\, A_i F_{jk} \;\in\; \mathbb{Z},
\end{equation}
which counts the linking number of preimages in $n(\mathbf{x})$. Finite-energy boundary conditions compactify space to $S^3$, ensuring $Q_H$ is topological.

\subsection*{G.3 Biquaternionic Representation}
In UBT, the $\Theta$-field takes values in the biquaternion algebra $\mathbb{B}$. 
A minimal embedding of a Hopfion uses a normalized doublet $z(q,\tau)$ extracted from $\Theta(q,\tau)$ through a projection $\Pi:\mathbb{B}\to\mathbb{C}^2$:
\begin{equation}
z(q,\tau) \;=\; \Pi\!\left[\Theta(q,\tau)\right],\qquad z^\dagger z = 1.
\end{equation}
The physical $n(q,\tau)$ and the $U(1)$ potential $A_i(q,\tau)$ then follow from the above constructions, with possible $\psi$-dependence via $\tau=t+i\psi$.

\subsection*{G.4 Energetics and the Faddeev--Skyrme Functional}
A standard energy functional supporting stable Hopfions is the Faddeev--Skyrme model:
\begin{equation}
E[n] \;=\; \int\! \mathrm{d}^3x \;\Big\{ \alpha\, (\partial_i n)^2 \;+\; \beta\, (F_{ij})^2 \Big\},\qquad n\cdot n=1.
\end{equation}
In our setting, $\alpha,\beta$ can acquire a weak $\psi$-dependence from the complex-time sector,
\begin{equation}
\alpha \to \alpha(\psi),\qquad \beta \to \beta(\psi),
\end{equation}
and additional small corrections may arise from couplings to electromagnetic and matter fields (Appendix D, F). 
The topological lower bound scales as $E \gtrsim c\, |Q_H|^{3/4}$ for some constant $c>0$, consistent with known Hopfion energy--charge relations.

\subsection*{G.5 Dynamics in Complex Time}
Allowing slow $\psi$-evolution, the effective Lagrangian density reads
\begin{equation}
\mathcal{L}_{\mathrm{Hopf}} \;=\; \frac{1}{2} \kappa_\psi \, (\partial_\psi n)^2 \;+\; \frac{1}{2}\rho \, (\partial_t n)^2 \;-\; \alpha(\psi)\, (\partial_i n)^2 \;-\; \beta(\psi)\, (F_{ij})^2 \;-\; V(n;\psi),
\end{equation}
with $V$ encoding weak explicit symmetry breaking or environmental pinning. 
Variations give a generalized Landau--Lifshitz-type equation with a Skyrme term and small $\psi$-derivative corrections, stabilizing knotted textures under perturbations:
\begin{equation}
\rho\, \partial_t^2 n + \kappa_\psi\, \partial_\psi^2 n \;-\; 2\,\alpha \,\Delta n \;+\; \cdots \;=\; \lambda(x,\psi,t)\, n,\qquad n\cdot n=1.
\end{equation}

\subsection*{G.6 Gravitational Coupling and Dark-Matter Phenomenology}
Coupling to gravity proceeds via the stress--energy tensor derived from $\mathcal{L}_{\mathrm{Hopf}}$:
\begin{equation}
T_{\mu\nu} \;=\; \frac{\partial \mathcal{L}}{\partial(\partial^\mu n)}\cdot \partial_\nu n \;-\; g_{\mu\nu}\,\mathcal{L},
\end{equation}
supplemented by contributions from the $\psi$-sector. 
In galactic halos, an ensemble of dilute, long-lived Hopfions with typical size $R_H$ and charge $Q_H=\mathcal{O}(1\text{--}10)$ yields an effective mass density
\begin{equation}
\rho_{\mathrm{DM}} \;\approx\; \frac{N_H\, E(Q_H)}{V_{\mathrm{halo}}} \;\sim\; \frac{N_H}{V_{\mathrm{halo}}}\, c\, |Q_H|^{3/4},
\end{equation}
where $N_H$ is the number of Hopfions in volume $V_{\mathrm{halo}}$. 
Since Hopfions are neutral in the $U(1)$ electromagnetic sector (projection to $A_\mu$ vanishes at infinity) and interact mainly through gravity (and possibly very weak $\psi$-mediated channels), they are natural dark-matter candidates in UBT.

\subsection*{G.7 Couplings to Electromagnetism and Matter}
At low energies, allowed gauge-invariant couplings are suppressed by small parameters:
\begin{equation}
\mathcal{L}_{\mathrm{int}} \;=\; - g_{nA}\, ( \partial_i n \cdot \partial_j n )\, F^{ij} \;-\; g_{nF}\, (F_{ij})^2\, ( \partial_k n )^2 \;-\; g_{n\psi}\, (\partial_\psi n)^2\, \bar{\Psi}\Psi \;+\; \cdots,
\end{equation}
leading to minute birefringence-like effects or dispersion shifts (Appendix D) only in extreme conditions. 
Psychon-induced $\psi$-fluctuations (Appendix F) may modulate Hopfion stability via $\kappa_\psi$ and $\beta(\psi)$, offering a route to small, structured signatures without spoiling standard-model tests.

\subsection*{G.8 3D Schematic: Linked Flux Surfaces (Hopf Link)}
\begin{figure}[h!]
\centering
\tdplotsetmaincoords{70}{120}
\begin{tikzpicture}[tdplot_main_coords,scale=1.0, line cap=round,line join=round]
  % Parameters for two linked circles (Hopf link)
  \def\R{2.2}
  \def\r{0.9}

  % Circle 1 in x-y plane
  \draw[very thick, blue!70!black]
    plot[domain=0:360, samples=200, variable=\t]
    ({\R*cos(\t)}, {\R*sin(\t)}, {0});

  % Circle 2 in x-z plane, shifted to link with circle 1
  \draw[very thick, red!70!black]
    plot[domain=0:360, samples=200, variable=\u]
    ({0}, {\r*cos(\u)}, {\r*sin(\u)});

  % Soft field-line hints
  \foreach \k in {0,30,...,330}{
    \draw[blue!40!black, opacity=0.35]
      plot[domain=0:360, samples=80, variable=\t]
      ({(\R+0.3*cos(\k))*cos(\t)}, {(\R+0.3*cos(\k))*sin(\t)}, {0.2*sin(\k)});
  }

  \node[anchor=south] at (0,-2.9,0) {\small Schematic Hopf link representing linked preimages of $n(\mathbf{x})$ (topological charge $Q_H=1$)};
\end{tikzpicture}
\caption{Conceptual 3D schematic of a Hopfion as a linked flux structure (Hopf link).}
\label{fig:hopf_link}
\end{figure}

\subsection*{G.9 Observational and Experimental Windows}
\begin{itemize}
\item \textbf{Astrophysics:} core/halo profiles modified by a population of stable knotted excitations; lensing and rotation curve fits with effective pressure terms from Skyrme corrections.
\item \textbf{Cosmology:} early-universe production of Hopfions during symmetry-breaking transitions; relic abundance depending on $\alpha(\psi),\beta(\psi)$.
\item \textbf{Laboratory analogues:} knotted light and superfluid vortices as table-top proxies to test relaxation/stability scaling.
\end{itemize}

\subsection*{G.10 Summary and Open Questions}
Hopfions provide a topologically protected, electromagnetically neutral sector consistent with UBT and compatible with precision QED. 
Key open directions include: quantitative halo modeling with $\psi$-dependent parameters, production mechanisms in cosmology, and refined bounds on weak couplings to standard fields.
