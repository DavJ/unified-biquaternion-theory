% Biquaternionic Tetrad Formalism
% Version: 1.0
% Date: 2026-01-07
% Status: Canonical - FUNDAMENTAL GEOMETRY

\section{The Biquaternionic Tetrad Field $E_\mu$}
\label{sec:canonical:biquaternion_tetrad}

\subsection{Fundamental Postulate}

\textbf{The tetrad (vierbein) field is fundamentally a biquaternionic object.}

The most fundamental geometric object in UBT is the \textbf{biquaternionic tetrad field}:

\begin{equation}
\label{eq:canonical:biq_tetrad_fundamental}
\boxed{E_\mu(x) \in \mathbb{B} = \mathbb{H} \otimes \mathbb{C}}
\end{equation}

\noindent where:
\begin{itemize}
    \item $\mu = 0, 1, 2, 3$ is the spacetime index
    \item $\mathbb{B}$ denotes the algebra of biquaternions
    \item $x = (x^0, x^1, x^2, x^3)$ are spacetime coordinates
\end{itemize}

\subsection{Tetrad Structure}

The tetrad field provides a local frame at each point in spacetime. It can be decomposed as:

\begin{equation}
\label{eq:canonical:biq_tetrad_decomposition}
E_\mu = e_\mu + \mathbf{I} f_\mu + \mathbf{J} \cdot \mathbf{v}_\mu
\end{equation}

\noindent where:
\begin{itemize}
    \item $e_\mu \in \mathbb{R}$ is the \textbf{real tetrad component} (classical vierbein)
    \item $f_\mu \in \mathbb{R}$ is the \textbf{imaginary scalar component} (phase frame)
    \item $\mathbf{v}_\mu = (v^1_\mu, v^2_\mu, v^3_\mu) \in \mathbb{R}^3$ is the \textbf{quaternionic vector component} (inertial frame)
\end{itemize}

\subsection{Metric from Tetrad}

The biquaternionic metric is \textbf{derived exclusively} from the tetrad field:

\begin{equation}
\label{eq:canonical:metric_from_tetrad}
\boxed{\mathcal{G}_{\mu\nu}(x) = \text{Sc}(E_\mu E_\nu^\dagger)}
\end{equation}

\noindent where:
\begin{itemize}
    \item $E_\nu^\dagger$ is the Hermitian conjugate (biquaternionic conjugation)
    \item $\text{Sc}(\cdot)$ extracts the scalar part of the biquaternion
    \item The product $E_\mu E_\nu^\dagger$ is the full biquaternionic multiplication (non-commutative)
\end{itemize}

\begin{tcolorbox}[colback=red!5!white,colframe=red!75!black,title=Prohibition]
\textbf{It is FORBIDDEN to introduce the metric $g_{\mu\nu}$ or $\mathcal{G}_{\mu\nu}$ directly without deriving it from the tetrad field via Eq.~\ref{eq:canonical:metric_from_tetrad}.}
\end{tcolorbox}

\subsection{Hermitian Conjugate}

The Hermitian conjugate of a biquaternion $E_\mu = e_\mu + \mathbf{I} f_\mu + \sum_k j_k v^k_\mu$ is:

\begin{equation}
\label{eq:canonical:biq_tetrad_conjugate}
E_\mu^\dagger = e_\mu - \mathbf{I} f_\mu - \sum_k j_k v^k_\mu
\end{equation}

This combines complex conjugation (for $\mathbf{I}$) with quaternionic conjugation (for $j_k$).

\subsection{Biquaternionic Product}

The product $E_\mu E_\nu^\dagger$ uses full biquaternionic multiplication:

\begin{equation}
\label{eq:canonical:biq_product}
E_\mu E_\nu^\dagger = (e_\mu + \mathbf{I} f_\mu + \mathbf{J} \cdot \mathbf{v}_\mu)(e_\nu - \mathbf{I} f_\nu - \mathbf{J} \cdot \mathbf{v}_\nu)
\end{equation}

The scalar part extraction gives:

\begin{equation}
\text{Sc}(E_\mu E_\nu^\dagger) = e_\mu e_\nu + f_\mu f_\nu + \mathbf{v}_\mu \cdot \mathbf{v}_\nu
\end{equation}

This ensures the metric has the correct Hermitian structure.

\subsection{Real Tetrad Projection}

The classical tetrad field is obtained by projection:

\begin{equation}
\label{eq:canonical:real_tetrad}
e_\mu^a := \text{Re}(E_\mu^a)
\end{equation}

where $a = 0, 1, 2, 3$ is a local Lorentz index.

The classical metric is then:

\begin{equation}
g_{\mu\nu} = \eta_{ab} e_\mu^a e_\nu^b
\end{equation}

where $\eta_{ab} = \text{diag}(+1, -1, -1, -1)$ is the Minkowski metric in the local frame.

\subsection{Tetrad Postulates}

The tetrad field must satisfy:

\subsubsection{Non-degeneracy}

\begin{equation}
\det(E_\mu) \neq 0
\end{equation}

This ensures the tetrad provides a valid frame at each point.

\subsubsection{Hermiticity of Metric}

\begin{equation}
\mathcal{G}_{\mu\nu}^\dagger = \mathcal{G}_{\nu\mu}
\end{equation}

This follows automatically from the tetrad construction:
\begin{align}
\mathcal{G}_{\mu\nu}^\dagger &= [\text{Sc}(E_\mu E_\nu^\dagger)]^\dagger \\
&= \text{Sc}(E_\nu E_\mu^\dagger) \\
&= \mathcal{G}_{\nu\mu}
\end{align}

\subsubsection{Signature Condition}

In the real limit, the tetrad must produce Lorentzian signature:

\begin{equation}
\text{signature}(\mathcal{G}_{\mu\nu}|_{\text{real}}) = (+, -, -, -)
\end{equation}

\subsection{Inverse Tetrad}

The inverse tetrad $E^\mu$ satisfies:

\begin{equation}
\label{eq:canonical:inverse_tetrad}
E^\mu \star E_\nu = \delta^\mu_\nu
\end{equation}

where $\star$ denotes the biquaternionic product.

This allows raising and lowering of indices:

\begin{align}
V_\mu &= \mathcal{G}_{\mu\nu} V^\nu \\
V^\mu &= \mathcal{G}^{\mu\nu} V_\nu
\end{align}

\subsection{Tetrad Transformations}

\subsubsection{Local Lorentz Transformations}

Under local Lorentz transformations $\Lambda^a_b(x) \in \text{SO}(1,3)$:

\begin{equation}
E_\mu^a \to \Lambda^a_b(x) E_\mu^b
\end{equation}

The metric is invariant:
\begin{equation}
\mathcal{G}_{\mu\nu} \to \mathcal{G}_{\mu\nu}
\end{equation}

\subsubsection{Diffeomorphisms}

Under coordinate transformations $x^\mu \to x'^\mu$:

\begin{equation}
E'_\alpha(x') = \frac{\partial x^\mu}{\partial x'^\alpha} E_\mu(x)
\end{equation}

\subsubsection{Gauge Transformations}

Under internal biquaternionic gauge transformations $U \in \text{SU(2)} \times \text{U(1)}$:

\begin{equation}
E_\mu \to U E_\mu U^\dagger
\end{equation}

The metric transforms covariantly:
\begin{equation}
\mathcal{G}_{\mu\nu} \to U \mathcal{G}_{\mu\nu} U^\dagger
\end{equation}

\subsection{Connection to $\Theta$ Field}

The tetrad field is related to the fundamental biquaternionic field $\Theta(q,\tau)$ via:

\begin{equation}
\label{eq:canonical:tetrad_from_theta}
E_\mu(x) = \partial_\mu \Theta(x) \cdot \mathcal{N}
\end{equation}

where $\mathcal{N}$ is a normalization operator ensuring proper tetrad structure.

Alternatively:
\begin{equation}
E_\mu = \frac{\partial \Theta}{\partial x^\mu} \Big/ \sqrt{\text{Sc}(\partial_\mu \Theta \partial^\mu \Theta^\dagger)}
\end{equation}

This connects the tetrad formalism to the fundamental $\Theta$ field dynamics.

\subsection{Spin Connection}

The tetrad formalism naturally introduces the spin connection (see Section~\ref{sec:canonical:biquaternion_connection}):

\begin{equation}
\Omega_\mu^{ab} = E^\nu_a \nabla_\mu E_\nu^b + E^\nu_a \Gamma^\lambda_{\mu\nu} E_\lambda^b
\end{equation}

where $\Gamma^\lambda_{\mu\nu}$ are the Christoffel symbols (derived, not fundamental).

In the biquaternionic formalism, this becomes:

\begin{equation}
\Omega_\mu = E^\nu \nabla_\mu E_\nu + \text{(biquaternionic correction terms)}
\end{equation}

\subsection{Torsion}

Unlike classical GR, the biquaternionic tetrad allows for non-vanishing torsion:

\begin{equation}
T^\lambda_{\mu\nu} = \Omega^\lambda_{\mu\nu} - \Omega^\lambda_{\nu\mu}
\end{equation}

Torsion arises from the imaginary components:

\begin{equation}
T^\lambda_{\mu\nu} \propto \text{Im}(E^\lambda [\partial_\mu E_\nu - \partial_\nu E_\mu])
\end{equation}

This is zero in the real limit but can be non-zero when $f_\mu \neq 0$ or $\mathbf{v}_\mu \neq 0$.

\subsection{Physical Interpretation}

\subsubsection{Real Component: $e_\mu$}

\begin{itemize}
    \item Classical vierbein field
    \item Defines local Lorentz frames
    \item Couples to ordinary matter
    \item Observable in GR experiments
\end{itemize}

\subsubsection{Phase Component: $f_\mu$}

\begin{itemize}
    \item Imaginary time frame orientation
    \item Couples to consciousness fields (psychons)
    \item Invisible to classical observations
    \item Responsible for phase-locked coherent states
\end{itemize}

\subsubsection{Inertial Component: $\mathbf{v}_\mu$}

\begin{itemize}
    \item Quaternionic frame directions
    \item Encodes spin-gravity coupling
    \item Responsible for dark matter effects (via p-adic extensions)
    \item Produces torsion in strong gravity regimes
\end{itemize}

\subsection{Field Equations}

The tetrad satisfies the biquaternionic field equation derived from the $\Theta$ field:

\begin{equation}
\nabla^\dagger \nabla E_\mu = \kappa \mathcal{J}_\mu
\end{equation}

where $\mathcal{J}_\mu$ is the biquaternionic current source.

\subsection{Compatibility Condition}

The tetrad and connection must satisfy metric compatibility:

\begin{equation}
\label{eq:canonical:tetrad_compatibility}
\boxed{\nabla_\mu E_\nu = \partial_\mu E_\nu + \Omega_\mu \circ E_\nu - \Gamma^\lambda_{\mu\nu} E_\lambda = 0}
\end{equation}

\noindent where:
\begin{itemize}
    \item $\Omega_\mu$ is the biquaternionic connection (Section~\ref{sec:canonical:biquaternion_connection})
    \item $\circ$ denotes the biquaternionic product (associative, non-commutative)
    \item $\Gamma^\lambda_{\mu\nu}$ are the Christoffel symbols (derived from the metric)
\end{itemize}

\textbf{Do NOT simplify} the commutators; ordering of factors matters because the product is non-commutative. Associativity $(AB)C = A(BC)$ holds in $\mathbb{B}$ and may be used freely.

\subsection{Explicit Construction}

For practical calculations, the tetrad can be constructed from the $\Theta$ field:

\textbf{Step 1:} Compute derivatives of $\Theta$:
\begin{equation}
\partial_\mu \Theta(x)
\end{equation}

\textbf{Step 2:} Normalize to obtain tetrad:
\begin{equation}
E_\mu = \frac{\partial_\mu \Theta}{\|\partial_\mu \Theta\|_\mathbb{B}}
\end{equation}

\textbf{Step 3:} Construct metric via:
\begin{equation}
\mathcal{G}_{\mu\nu} = \text{Sc}(E_\mu E_\nu^\dagger)
\end{equation}

\textbf{Step 4:} Extract classical metric:
\begin{equation}
g_{\mu\nu} = \text{Re}(\mathcal{G}_{\mu\nu})
\end{equation}

\subsection{Conflict Resolution}

This canonical definition supersedes all previous formulations that:

\begin{enumerate}
    \item ❌ Introduce the metric directly without tetrad
    \item ❌ Use real-valued tetrads as fundamental
    \item ❌ Postulate Christoffel symbols independently
\end{enumerate}

\textbf{Canonical resolution}: The tetrad field $E_\mu(x) \in \mathbb{B}$ is the most fundamental geometric object. All other geometric quantities (metric, connection, curvature) are derived from it.

\subsection{Historical Note}

The biquaternionic tetrad formalism provides the deepest level of geometric description in UBT. By making the tetrad fundamental rather than the metric, we allow for richer geometric structures including torsion, non-metricity, and phase curvature that are invisible in classical GR.

% End of biquaternionic tetrad formalism
