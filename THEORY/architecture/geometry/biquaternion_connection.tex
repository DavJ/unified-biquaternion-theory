% Biquaternionic Connection
% Version: 1.0
% Date: 2026-01-07
% Status: Canonical - FUNDAMENTAL GEOMETRY

\section{The Biquaternionic Connection $\Omega_\mu$}
\label{sec:canonical:biquaternion_connection}

\subsection{Fundamental Postulate}

\textbf{The gravitational connection is fundamentally a biquaternionic object, not a set of real Christoffel symbols.}

The fundamental connection in UBT is the \textbf{biquaternionic connection}:

\begin{equation}
\label{eq:canonical:biq_connection_fundamental}
\boxed{\Omega_\mu(x) \in \mathbb{B} = \mathbb{H} \otimes \mathbb{C}}
\end{equation}

\noindent where:
\begin{itemize}
    \item $\mu = 0, 1, 2, 3$ is the spacetime index
    \item $\mathbb{B}$ denotes the algebra of biquaternions
    \item $\Omega_\mu$ encodes all geometric connection information
\end{itemize}

\subsection{Christoffel Symbols as Derived Quantities}

\begin{tcolorbox}[colback=red!5!white,colframe=red!75!black,title=Important]
\textbf{The Christoffel symbols $\Gamma^\lambda_{\mu\nu}$ are NOT fundamental.}

They are derived quantities obtained from the biquaternionic connection in the real limit:

\begin{equation}
\label{eq:canonical:christoffel_derived}
\Gamma^\lambda_{\mu\nu} = \text{Re}(\Omega_\mu)^\lambda{}_{\nu} \quad \text{(real projection)}
\end{equation}

\textbf{It is FORBIDDEN to postulate Christoffel symbols independently of the biquaternionic connection.}
\end{tcolorbox}

\subsection{Biquaternionic Decomposition}

The biquaternionic connection decomposes as:

\begin{equation}
\label{eq:canonical:biq_connection_decomposition}
\Omega_\mu = \omega_\mu + \mathbf{I} \phi_\mu + \mathbf{J} \cdot \boldsymbol{\psi}_\mu
\end{equation}

\noindent where:
\begin{itemize}
    \item $\omega_\mu$ is the \textbf{real connection component} (classical spin connection / Christoffel symbols)
    \item $\phi_\mu$ is the \textbf{phase connection} (imaginary time parallel transport)
    \item $\boldsymbol{\psi}_\mu = (\psi^1_\mu, \psi^2_\mu, \psi^3_\mu)$ is the \textbf{quaternionic connection} (inertial/torsion components)
\end{itemize}

\subsection{Metric Compatibility Condition}

The connection is uniquely determined by the compatibility condition with the tetrad field:

\begin{equation}
\label{eq:canonical:connection_compatibility}
\boxed{\nabla_\mu E_\nu = \partial_\mu E_\nu + \Omega_\mu \circ E_\nu - \Gamma^\lambda_{\mu\nu} E_\lambda = 0}
\end{equation}

\noindent where:
\begin{itemize}
    \item $E_\nu$ is the biquaternionic tetrad (Section~\ref{sec:canonical:biquaternion_tetrad})
    \item $\circ$ denotes the biquaternionic product (non-commutative)
    \item $\Gamma^\lambda_{\mu\nu}$ are the derived Christoffel symbols
\end{itemize}

\begin{tcolorbox}[colback=yellow!5!white,colframe=orange!75!black,title=Non-Commutativity (ordering matters)]
\textbf{Do NOT assume commutativity of $\Omega_\mu$ and $E_\nu$.}

Biquaternionic multiplication ($\mathbb{C}\otimes\mathbb{H}$) is:
\begin{itemize}
    \item \textbf{Non-commutative}: $AB \neq BA$ in general
    \item \textbf{Associative}: $(AB)C = A(BC)$ always (inherited from the matrix/tensor-product representation)
\end{itemize}

Therefore the associator $[\Omega_\mu, \Omega_\nu, E_\rho] := (\Omega_\mu \Omega_\nu)E_\rho - \Omega_\mu(\Omega_\nu E_\rho) = 0$ identically; only ordering of factors matters.
Simplifications valid only in the real/Hermitian limit must be labeled ``GR limit''.
\end{tcolorbox}

\subsection{Derivation of $\Omega_\mu$ from the Tetrad Postulate}
\label{subsec:connection_from_tetrad}

\textbf{The biquaternionic connection is defined and derived from the tetrad postulate, not by substituting $\mathcal{G}$ into the classical Christoffel formula.}

\begin{tcolorbox}[colback=blue!5!white,colframe=blue!75!black,title=Assumption: Existence of inverse tetrad]
We assume the biquaternionic tetrad $E_\mu$ has a left-inverse $\tilde{E}^\mu$ satisfying
$\tilde{E}^\mu \circ E_\mu = \mathbf{1}$ (summation over $\mu$),
so that we can solve the tetrad postulate for $\Omega_\mu$.
\end{tcolorbox}

\begin{lemma}[Derivation of $\Omega_\mu$ from tetrad postulate]
\label{lem:omega_from_tetrad}
Given biquaternionic tetrad fields $E_\nu \in \mathbb{B}$ and their inverse $\tilde{E}^\nu$, the
connection $\Omega_\mu$ satisfying the tetrad postulate
\begin{equation}
\nabla_\mu E_\nu = \partial_\mu E_\nu + \Omega_\mu \circ E_\nu - \Gamma^\lambda_{\mu\nu} E_\lambda = 0
\end{equation}
is given by
\begin{equation}
\label{eq:canonical:omega_from_tetrad}
\boxed{\Omega_\mu = \bigl(-\partial_\mu E_\nu + \Gamma^\lambda_{\mu\nu} E_\lambda\bigr) \circ \tilde{E}^\nu}
\end{equation}
where the affine part $\Gamma^\lambda_{\mu\nu}$ is the (real) Levi-Civita connection of $g_{\mu\nu} = \mathrm{Re}(\mathcal{G}_{\mu\nu})$, and right-multiplication by $\tilde{E}^\nu$ is explicit to respect non-commutativity.
\end{lemma}

\begin{proof}
Starting from the tetrad postulate $\partial_\mu E_\nu + \Omega_\mu \circ E_\nu = \Gamma^\lambda_{\mu\nu} E_\lambda$, right-multiply both sides by $\tilde{E}^\nu$ and sum over $\nu$:
\begin{align}
\partial_\mu E_\nu \circ \tilde{E}^\nu + \Omega_\mu \circ (E_\nu \circ \tilde{E}^\nu) &= \Gamma^\lambda_{\mu\nu} E_\lambda \circ \tilde{E}^\nu.
\end{align}
By the inverse tetrad assumption, $E_\nu \circ \tilde{E}^\nu = \mathbf{1}$, hence $\Omega_\mu \circ \mathbf{1} = \Omega_\mu$.  Rearranging gives Eq.~\eqref{eq:canonical:omega_from_tetrad}.
The derivation uses only associativity $(AB)C = A(BC)$ of $\mathbb{B}$, which holds since $\mathbb{B} = \mathbb{C}\otimes\mathbb{H}$ is a tensor product of associative algebras.
\end{proof}

\begin{remark}
The classical Christoffel formula $\Omega^\lambda_{\mu\nu} = \tfrac{1}{2}\mathcal{G}^{\lambda\rho}(\partial_\mu \mathcal{G}_{\nu\rho}+\cdots)$
is \emph{not} valid as a definition of $\Omega_\mu$ in the biquaternionic setting, because
$\mathcal{G}_{\mu\nu}$ is non-commutative and ``$\mathcal{G}^{-1}$'' does not commute with the
metric components.  The tetrad postulate derivation above is the correct first-principles approach.
In the commutative real limit $\mathcal{G}_{\mu\nu} \to g_{\mu\nu} \in \mathbb{R}$, both approaches
agree and yield the Levi-Civita connection (see Lemma~\ref{lem:re_omega_levi_civita} in Appendix~R).
\end{remark}

\subsection{Covariant Derivative}

The covariant derivative of a biquaternionic field $\Psi$ is defined as:

\begin{equation}
\label{eq:canonical:covariant_derivative}
\nabla_\mu \Psi = \partial_\mu \Psi + \Omega_\mu \circ \Psi
\end{equation}

For the fundamental $\Theta$ field:

\begin{equation}
\nabla_\mu \Theta = \partial_\mu \Theta + \Omega_\mu \circ \Theta
\end{equation}

This generalizes to:

\begin{equation}
\nabla_\mu = \partial_\mu + \Omega_\mu^{\text{grav}} + A_\mu^{\text{SM}}
\end{equation}

where $A_\mu^{\text{SM}}$ includes Standard Model gauge connections (see Section~\ref{sec:sm_gauge}).

\subsection{Properties of the Connection}

\subsubsection{Hermiticity}

\begin{equation}
\Omega_\mu^\dagger = -\Omega_\mu + \text{(metric terms)}
\end{equation}

This ensures the covariant derivative preserves Hermiticity.

\subsubsection{Torsion}

The torsion tensor is:

\begin{equation}
\label{eq:canonical:torsion}
\mathcal{T}^\lambda_{\mu\nu} = \Omega^\lambda_{\mu\nu} - \Omega^\lambda_{\nu\mu}
\end{equation}

In classical GR with the Levi-Civita connection (real projection of $\Omega_\mu$), torsion vanishes:
\begin{equation}
\Gamma^\lambda_{\mu\nu} = \Gamma^\lambda_{\nu\mu} \quad \Rightarrow \quad T^\lambda_{\mu\nu} = 0
\end{equation}

In UBT, torsion can be non-zero due to imaginary components:

\begin{equation}
\mathcal{T}^\lambda_{\mu\nu} = \text{Im}(\Omega^\lambda_{\mu\nu} - \Omega^\lambda_{\nu\mu}) \neq 0
\end{equation}

This arises from quaternionic components $\boldsymbol{\psi}_\mu$.

\subsubsection{Non-metricity}

The non-metricity tensor is:

\begin{equation}
Q_{\lambda\mu\nu} = \nabla_\lambda \mathcal{G}_{\mu\nu}
\end{equation}

In classical GR with metric compatibility, $Q_{\lambda\mu\nu} = 0$.

In UBT, non-metricity can arise from phase components:

\begin{equation}
Q_{\lambda\mu\nu} \propto \text{Im}(\phi_\lambda) \mathcal{G}_{\mu\nu}
\end{equation}

\subsection{Connection in Different Representations}

\subsubsection{Coordinate Representation}

In coordinate basis:

\begin{equation}
\Omega_\mu = \Omega^\lambda_{\mu\nu} \partial_\lambda \otimes dx^\nu
\end{equation}

where $\Omega^\lambda_{\mu\nu} \in \mathbb{B}$ are the connection coefficients.

\subsubsection{Tetrad Representation}

In tetrad basis:

\begin{equation}
\Omega_\mu = \Omega_\mu^{ab} \Sigma_{ab}
\end{equation}

where:
\begin{itemize}
    \item $\Sigma_{ab} = \frac{1}{4}[\gamma_a, \gamma_b]$ are Lorentz generators
    \item $\gamma_a$ are Dirac gamma matrices (in spinor formulation)
\end{itemize}

\subsubsection{Biquaternionic Representation}

Directly as a biquaternion:

\begin{equation}
\Omega_\mu = \omega_\mu^0 + \omega_\mu^1 j_1 + \omega_\mu^2 j_2 + \omega_\mu^3 j_3 + i(\phi_\mu^0 + \phi_\mu^1 j_1 + \cdots)
\end{equation}

\subsection{Transformation Laws}

\subsubsection{Gauge Transformations}

Under biquaternionic gauge transformation $U(x) \in \mathbb{B}$:

\begin{equation}
\Omega_\mu \to U \Omega_\mu U^{-1} - (\partial_\mu U) U^{-1}
\end{equation}

This is the non-Abelian gauge transformation law.

\subsubsection{Coordinate Transformations}

Under diffeomorphisms $x^\mu \to x'^\mu$:

\begin{equation}
\Omega'_\alpha = \frac{\partial x^\mu}{\partial x'^\alpha} \Omega_\mu + \text{(inhomogeneous terms)}
\end{equation}

\subsection{Curvature from Connection}

The biquaternionic curvature (field strength) is defined as:

\begin{equation}
\label{eq:canonical:curvature_from_connection}
\boxed{\mathcal{R}_{\mu\nu} = \partial_\mu \Omega_\nu - \partial_\nu \Omega_\mu + [\Omega_\mu, \Omega_\nu]_\star}
\end{equation}

\noindent where $[\cdot, \cdot]_\star$ denotes the biquaternionic commutator:

\begin{equation}
[\Omega_\mu, \Omega_\nu]_\star = \Omega_\mu \star \Omega_\nu - \Omega_\nu \star \Omega_\mu
\end{equation}

with $\star$ being the full biquaternionic product (non-commutative).

See Section~\ref{sec:canonical:biquaternion_curvature} for details.

\subsection{Physical Interpretation}

\subsubsection{Real Component: $\omega_\mu$}

\begin{itemize}
    \item Classical spin connection
    \item Equivalent to Christoffel symbols in coordinate basis
    \item Describes parallel transport in classical spacetime
    \item Observable via gravitational effects
\end{itemize}

\subsubsection{Phase Component: $\phi_\mu$}

\begin{itemize}
    \item Imaginary time parallel transport
    \item Couples to internal auxiliary sector (speculative extensions are in \texttt{speculative\_extensions/})
    \item Produces Berry phase-like effects (potential experimental signature)
    \item Invisible to classical observations
\end{itemize}

\subsubsection{Quaternionic Component: $\boldsymbol{\psi}_\mu$}

\begin{itemize}
    \item Torsional connection
    \item Couples to spin and angular momentum
    \item Responsible for:
    \begin{itemize}
        \item Dark matter halo structure (via p-adic extensions)
        \item Spin-orbit coupling in strong gravity
        \item Modified frame-dragging effects
    \end{itemize}
\end{itemize}

\subsection{Relation to Standard Model Gauge Fields}

The total connection in UBT includes both gravitational and gauge components:

\begin{equation}
\label{eq:canonical:total_connection}
\nabla_\mu = \partial_\mu + \Omega_\mu^{\text{grav}} + A_\mu^{\text{SM}}
\end{equation}

where:

\begin{equation}
A_\mu^{\text{SM}} = i g_1 B_\mu Y + i g_2 W_\mu^a T^a + i g_3 G_\mu^A \Lambda^A
\end{equation}

See Section~\ref{sec:canonical:sm_gauge} for details on gauge group emergence.

\subsection{Field Equations for the Connection}

The connection satisfies the biquaternionic Yang-Mills-like equation:

\begin{equation}
\nabla^\mu \mathcal{R}_{\mu\nu} = \kappa \mathcal{J}_\nu
\end{equation}

where $\mathcal{J}_\nu$ is the biquaternionic current.

In the real limit, this reduces to the Einstein equation constraint:

\begin{equation}
\nabla^\mu G_{\mu\nu} = 0 \quad \text{(Bianchi identity)}
\end{equation}

\subsection{Parallel Transport}

A vector $V^\mu$ is parallel-transported along a curve if:

\begin{equation}
\frac{DV^\mu}{d\lambda} = \frac{dV^\mu}{d\lambda} + \Omega^\mu_{\rho\sigma} \frac{dx^\rho}{d\lambda} V^\sigma = 0
\end{equation}

For biquaternionic vectors, this becomes:

\begin{equation}
\frac{\mathcal{D}\mathcal{V}^\mu}{d\lambda} = \frac{d\mathcal{V}^\mu}{d\lambda} + \Omega^\mu_{\rho} \circ \mathcal{V} \frac{dx^\rho}{d\lambda} = 0
\end{equation}

The non-commutativity of $\Omega_\mu$ leads to path-dependent parallel transport even in flat space.

\subsection{Holonomy}

The holonomy around a closed loop $\mathcal{C}$ is:

\begin{equation}
\mathcal{H} = \mathcal{P} \exp\left(\oint_\mathcal{C} \Omega_\mu dx^\mu\right)
\end{equation}

where $\mathcal{P}$ denotes path ordering (essential for non-commutative connections).

This generalizes Wilson loops to biquaternionic geometry.

\subsection{Explicit Computation}

For practical calculations:

\textbf{Step 1:} Compute the biquaternionic metric $\mathcal{G}_{\mu\nu}$ from the tetrad.

\textbf{Step 2:} Derive connection via tetrad postulate (Lemma~\ref{lem:omega_from_tetrad}):
\begin{equation}
\Omega_\mu = \bigl(-\partial_\mu E_\nu + \Gamma^\lambda_{\mu\nu} E_\lambda\bigr) \circ \tilde{E}^\nu
\end{equation}
where $\tilde{E}^\nu$ is the inverse tetrad and $\Gamma^\lambda_{\mu\nu}$ is the Levi-Civita connection of $g_{\mu\nu} = \mathrm{Re}(\mathcal{G}_{\mu\nu})$.

\textbf{Step 3:} Extract real Christoffel symbols:
\begin{equation}
\Gamma^\lambda_{\mu\nu} = \text{Re}(\Omega^\lambda_{\mu\nu})
\end{equation}

\textbf{Step 4:} Compute curvature via:
\begin{equation}
\mathcal{R}_{\mu\nu\rho\sigma} = \partial_\rho \Omega_{\sigma\mu\nu} - \partial_\sigma \Omega_{\rho\mu\nu} + \Omega_{\rho\mu}^\lambda \Omega_{\sigma\lambda\nu} - \Omega_{\sigma\mu}^\lambda \Omega_{\rho\lambda\nu}
\end{equation}

\subsection{Consistency with General Relativity}

In the limit where all imaginary components vanish:

\begin{equation}
\phi_\mu \to 0, \quad \boldsymbol{\psi}_\mu \to 0 \quad \Rightarrow \quad \Omega_\mu \to \omega_\mu
\end{equation}

The connection reduces to the classical Levi-Civita connection (real projection / GR limit):

\begin{equation}
\omega^\lambda_{\mu\nu} = \Gamma^\lambda_{\mu\nu} = \frac{1}{2} g^{\lambda\rho} \left( \partial_\mu g_{\nu\rho} + \partial_\nu g_{\mu\rho} - \partial_\rho g_{\mu\nu} \right)
\end{equation}

\textbf{UBT exactly reproduces GR in the real limit.}

\subsection{Conflict Resolution}

This canonical definition supersedes:

\begin{enumerate}
    \item ❌ Direct postulation of Christoffel symbols as fundamental
    \item ❌ Levi-Civita connection without biquaternionic structure
    \item ❌ Torsion-free assumption as an axiom
\end{enumerate}

\textbf{Canonical resolution}: The biquaternionic connection $\Omega_\mu(x) \in \mathbb{B}$ is fundamental. Christoffel symbols are its real projection.

\subsection{Historical Note}

The biquaternionic connection provides the deepest level of connection geometry in UBT, allowing for torsion, non-metricity, and phase-dependent parallel transport that are absent in classical GR.

% End of biquaternionic connection definition
