% © 2025 Ing. David Jaroš — CC BY-NC-ND 4.0
%
% This work is licensed under a Creative Commons Attribution-NonCommercial-NoDerivatives
% 4.0 International License (CC BY-NC-ND 4.0).
%
% Track A math-ph paper: SU(3) candidate via Z2^3 involutions on B = C⊗H
% Status: Draft for arXiv submission — math-ph
% Version: 1.0
% Date: 2026-03-01

\documentclass[12pt,a4paper]{article}

\usepackage{amsmath,amssymb,amsthm}
\usepackage{hyperref}
\usepackage{geometry}
\geometry{margin=2.5cm}

% Theorem environments
\newtheorem{theorem}{Theorem}[section]
\newtheorem{lemma}[theorem]{Lemma}
\newtheorem{proposition}[theorem]{Proposition}
\newtheorem{corollary}[theorem]{Corollary}
\newtheorem{definition}[theorem]{Definition}
\theoremstyle{remark}
\newtheorem{remark}[theorem]{Remark}

% Notation
\newcommand{\B}{\mathcal{B}}          % biquaternion algebra
\newcommand{\C}{\mathbb{C}}
\newcommand{\R}{\mathbb{R}}
\newcommand{\H}{\mathbb{H}}
\newcommand{\Vc}{V_c}                 % carrier space
\newcommand{\su}{\mathfrak{su}}

\title{%
  A Natural $\mathrm{SU}(3)$ Action on a Canonical Subspace of\\
  $\mathbb{C}\otimes\mathbb{H}$ via $\mathbb{Z}_2^3$ Involutions
}

\author{David Jaro\v{s}}

\date{2026}

\begin{document}

\maketitle

\begin{abstract}
We construct a canonical three-dimensional complex subspace $V_c$ of the
biquaternion algebra $\B = \C\otimes\H$, extracted via a system of three
commuting $\R$-linear involutions $P_1, P_2, P_3$ on $\B$.  The subspace
$V_c = \mathrm{span}_\C\{\mathbf{I},\mathbf{J},\mathbf{K}\}$ (the $P_2 = -1$
eigenspace under quaternion conjugation) carries a natural Hermitian inner
product, and the group of $\C$-linear isometries of determinant one acting on
$V_c$ is isomorphic to $\mathrm{SU}(3)$.  We prove three theorems: (1) the
involution decomposition of $\B$; (2) the canonical triplet structure of
$V_c \cong \C^3$; and (3) that $\mathrm{SU}(3)$ preserves the Hermitian form
on $V_c$.  The entire construction is associative ($\B \cong M_2(\C)$); no
octonions or non-associative structures appear.  We note that
$\mathrm{SU}(3)_{V_c}$ is not a subgroup of $\mathrm{Aut}(\B)$, and we
explicitly compare our result with classical theorems on $\mathrm{Aut}(\H)
\cong \mathrm{SO}(3)$ and the $\mathrm{GL}(2,\C)$ obstruction.  Open problems
are stated.
\end{abstract}

\tableofcontents

% ============================================================
\section{Introduction}
% ============================================================

The biquaternion algebra $\B = \C\otimes_\R\H$ is an associative
$8$-dimensional real algebra isomorphic to $M_2(\C)$, the algebra of
$2\times 2$ complex matrices.  It has appeared in special relativity, quantum
mechanics, and mathematical physics in various guises since Hamilton and
Clifford \cite{Hamilton1866,Clifford1878}.

In this paper we study the algebraic decomposition of $\B$ induced by a
system of three commuting real-linear involutions and identify a canonical
three-complex-dimensional subspace on which a natural $\mathrm{SU}(3)$
symmetry acts.

Our motivation comes from the observation that biquaternionic field theories
\cite{Dixon1994,Furey2016} seek to understand Standard Model gauge symmetries
as arising from the algebraic structure of $\B$ or related algebras.  Our
construction isolates a clean, proof-ready $\mathrm{SU}(3)$ action that does
not require octonions and that can be stated and proved entirely within
classical algebra.

\medskip
\noindent\textbf{Main results.}
\begin{itemize}
\item \textbf{Theorem~\ref{thm:involution_decomp}}: Three commuting
  $\mathbb{Z}_2$-involutions $P_1,P_2,P_3$ decompose $\B$ into eight
  simultaneous eigenspaces.  The $\mathbb{Z}_2^3$ action is explicit and
  canonical.
\item \textbf{Theorem~\ref{thm:triplet_space}}: The $P_2=-1$ eigenspace is a
  three-dimensional complex vector space $V_c \cong \C^3$ with a canonical
  Hermitian inner product.
\item \textbf{Theorem~\ref{thm:su3_action}}: The group $\mathrm{SU}(3)_{V_c}$
  of $\C$-linear isometries of $(V_c,\langle\cdot,\cdot\rangle)$ with
  determinant one is isomorphic to $\mathrm{SU}(3)$.  It acts naturally on
  $V_c$ and preserves the Hermitian form.
\end{itemize}

\medskip
\noindent\textbf{What we do not claim.}
We do not claim that $\mathrm{SU}(3)_{V_c}$ is the colour gauge group of
Quantum Chromodynamics, nor that it is a subgroup of $\mathrm{Aut}(\B)$.
These are open problems stated in Section~\ref{sec:open_problems}.  The paper
is written as a pure mathematics result.

% ============================================================
\section{Algebraic Background}
\label{sec:background}
% ============================================================

\subsection{The biquaternion algebra}

Let $\H = \R\langle 1,\mathbf{I},\mathbf{J},\mathbf{K}\rangle$ denote the
quaternion algebra with $\mathbf{I}^2=\mathbf{J}^2=\mathbf{K}^2=-1$,
$\mathbf{I}\mathbf{J}=\mathbf{K}$, and cyclic permutations.  The biquaternion
algebra is
\[
  \B \;:=\; \C \otimes_\R \H,
\]
an $8$-dimensional real (or $4$-dimensional complex) associative algebra.
The complex unit $i\in\C$ lies in the \emph{centre} of $\B$ (it commutes with
all quaternion units).

\begin{lemma}[$\C$ is central]
  \label{lem:C_central}
  For all $h\in\H$ and $z\in\C$, $z\cdot h = h\cdot z$ in $\B$.
\end{lemma}
\begin{proof}
  In $\C\otimes_\R\H$, $(z\otimes 1)(1\otimes h) = z\otimes h
  = (1\otimes h)(z\otimes 1)$.
\end{proof}

\subsection{Ordered basis}

We fix the ordered $\R$-basis
\begin{equation}
  \label{eq:basis}
  \mathcal{B}_\mathrm{basis} = \{1,\,\mathbf{I},\,\mathbf{J},\,\mathbf{K},\,
    i,\,i\mathbf{I},\,i\mathbf{J},\,i\mathbf{K}\}.
\end{equation}

% ============================================================
\section{Three Commuting Involutions and Their Decomposition}
\label{sec:involutions}
% ============================================================

\begin{definition}[Three involutions]
  \label{def:involutions}
  Define three $\R$-linear maps $P_k:\B\to\B$ by their action on the basis
  \eqref{eq:basis}:
  \begin{align*}
    P_1 &:\; i\mapsto -i,\quad \mathbf{I}\mapsto\mathbf{I},\quad
      \mathbf{J}\mapsto\mathbf{J},\quad \mathbf{K}\mapsto\mathbf{K}.
      \quad\text{(complex conjugation)}\\
    P_2 &:\; i\mapsto i,\quad \mathbf{I}\mapsto-\mathbf{I},\quad
      \mathbf{J}\mapsto-\mathbf{J},\quad \mathbf{K}\mapsto-\mathbf{K}.
      \quad\text{(quaternion conjugation)}\\
    P_3 &:\; P_3(x) = \mathbf{I}x\mathbf{I}^{-1},\; x\in\B.
      \quad\text{(axis-flip, inner automorphism by }\mathbf{I})
  \end{align*}
\end{definition}

\begin{theorem}[Involution decomposition]
  \label{thm:involution_decomp}
  (a) Each $P_k$ is an $\R$-linear involution: $P_k^2 = \mathrm{id}$.
  (b) The $P_k$ commute pairwise: $[P_i,P_j]=0$.
  (c) They generate a $\mathbb{Z}_2^3$ action on $\B$.
  (d) The simultaneous eigenspace decomposition
  $\B = \bigoplus_{s\in\{+1,-1\}^3} \B_s$, where $\B_s = \{b\in\B:P_k(b)=s_k b\}$,
  has the following non-zero components:
  \[
    \B_{(+,+,+)} = \R\cdot 1,\quad
    \B_{(+,-,+)} = \R\cdot\mathbf{I},\quad
    \B_{(+,-,-)} = \R\cdot\mathbf{J}\oplus\R\cdot\mathbf{K},
  \]
  \[
    \B_{(-,+,+)} = \R\cdot i,\quad
    \B_{(-,-,+)} = \R\cdot i\mathbf{I},\quad
    \B_{(-,-,-)} = \R\cdot i\mathbf{J}\oplus\R\cdot i\mathbf{K}.
  \]
  The sectors $(+,+,-)$ and $(-,+,-)$ are empty.
\end{theorem}

\begin{proof}
  (a) Every basis vector $b\in\mathcal{B}_\mathrm{basis}$ satisfies
  $P_k(b) = s_k b$ for some $s_k\in\{+1,-1\}$ (see the signature table).
  Then $P_k^2(b) = s_k^2 b = b$; by $\R$-linearity, $P_k^2=\mathrm{id}$.

  (b) For any basis element, $P_i(P_j(b)) = P_i(s_j b) = s_j s_i b
  = s_i s_j b = P_j(P_i(b))$; by linearity $[P_i,P_j]=0$.

  (c) $\{P_1,P_2,P_3\}$ generates a group under composition; since each
  generator has order $2$ and they commute, this group is $\Z_2^3$.

  (d) Direct computation from Definition~\ref{def:involutions}: the
  signature of each basis element is read from the table.  For the emptiness
  of $(+,+,-)$: any element with $P_1=+1$ and $P_2=+1$ is a real scalar
  multiple of $1$; since $P_3(1)=\mathbf{I}\cdot 1\cdot\mathbf{I}^{-1}=1$,
  it lies in $(+,+,+)$ not $(+,+,-)$.
\end{proof}

% ============================================================
\section{The Canonical Triplet Space}
\label{sec:triplet}
% ============================================================

\begin{theorem}[Canonical triplet space]
  \label{thm:triplet_space}
  The $P_2=-1$ eigenspace
  \begin{equation}
    \label{eq:Vc}
    V_c \;:=\; \ker(P_2+\mathrm{id})
       \;=\; \mathrm{span}_\C\{\mathbf{I},\mathbf{J},\mathbf{K}\}
  \end{equation}
  is a three-dimensional complex vector space ($\dim_\C V_c = 3$) and is
  naturally isomorphic to $\C^3$ via
  $\phi: V_c\to\C^3$, $\phi(a\mathbf{I}+b\mathbf{J}+c\mathbf{K}) = (a,b,c)$.
\end{theorem}

\begin{proof}
  The six $\R$-vectors $\{\mathbf{I},i\mathbf{I},\mathbf{J},i\mathbf{J},
  \mathbf{K},i\mathbf{K}\}$ are distinct elements of the basis
  \eqref{eq:basis} and hence $\R$-linearly independent.  Every element of
  $V_c$ is $a\mathbf{I}+b\mathbf{J}+c\mathbf{K}$ with $a,b,c\in\C$;
  so $\{\mathbf{I},\mathbf{J},\mathbf{K}\}$ is a $\C$-basis and
  $\dim_\C V_c = 3$.  The map $\phi$ is manifestly an isomorphism of
  complex vector spaces.
\end{proof}

\begin{definition}[Hermitian form on $V_c$]
  \label{def:inner_product}
  For $X,Y\in V_c$, define
  \begin{equation}
    \label{eq:inner_product}
    \langle X,Y\rangle \;:=\;
    \tfrac{1}{4}\,\mathrm{Tr}(X^\dagger Y),
  \end{equation}
  where $X^\dagger = P_1(P_2(X)^*)\in V_c$ and $\mathrm{Tr}$ is the reduced
  trace of $\B$.
\end{definition}

Under $\phi$, this reduces to the standard Hermitian inner product on $\C^3$:
$\langle X,Y\rangle = \overline{a_1}a_2+\overline{b_1}b_2+\overline{c_1}c_2$
for $X=a_1\mathbf{I}+b_1\mathbf{J}+c_1\mathbf{K}$ and
$Y=a_2\mathbf{I}+b_2\mathbf{J}+c_2\mathbf{K}$.

% ============================================================
\section{SU(3) Action on $V_c$}
\label{sec:su3}
% ============================================================

\begin{theorem}[$\mathrm{SU}(3)$ preserves the Hermitian form]
  \label{thm:su3_action}
  Define
  \begin{equation}
    \label{eq:SU3_Vc}
    \mathrm{SU}(3)_{V_c} \;:=\;
    \bigl\{U\in\mathrm{GL}_\C(V_c) \;\big|\;
    \langle Ux,Uy\rangle = \langle x,y\rangle\;\forall\,x,y\in V_c,\;
    \det_\C(U)=1\bigr\}.
  \end{equation}
  Then:
  \begin{enumerate}
    \item[(a)] $\mathrm{SU}(3)_{V_c}$ is isomorphic to the standard
      $\mathrm{SU}(3)$.
    \item[(b)] The action $U\cdot x = Ux$ preserves the inner product
      \eqref{eq:inner_product} on $V_c$.
    \item[(c)] The action is faithful.
  \end{enumerate}
\end{theorem}

\begin{proof}
  (a) Via the isomorphism $\phi:V_c\to\C^3$, the inner product
  \eqref{eq:inner_product} becomes the standard Hermitian inner product on
  $\C^3$, so $\mathrm{SU}(3)_{V_c}\cong\mathrm{SU}(3)$ by definition.

  (b) By the defining property of $\mathrm{SU}(3)_{V_c}$.

  (c) If $Ux=x$ for all $x\in V_c$, then $U=\mathrm{id}_{V_c}$.
\end{proof}

\begin{remark}[$\mathrm{SU}(3)_{V_c}$ is not in $\mathrm{Aut}(\B)$]
  \label{rem:not_aut}
  The group $\mathrm{SU}(3)_{V_c}$ acts on the \emph{subspace} $V_c\subset\B$;
  it is not a subgroup of the algebra automorphism group
  $\mathrm{Aut}_\R(\B)\cong[\mathrm{GL}(2,\C)\times\mathrm{GL}(2,\C)]/\mathbb{Z}_2$.
  This is consistent with the classical observation that
  $\mathrm{Aut}(\H)\cong\mathrm{SO}(3)$ and the $\mathrm{GL}(2,\C)$ rank
  obstruction; see Section~\ref{sec:related} for details.
\end{remark}

% ============================================================
\section{Related Work and Literature Comparison}
\label{sec:related}
% ============================================================

% © 2025 Ing. David Jaroš — CC BY-NC-ND 4.0
%
% This work is licensed under a Creative Commons Attribution-NonCommercial-NoDerivatives
% 4.0 International License (CC BY-NC-ND 4.0).
%
% Related Work: Literature positioning for the SU(3) triplet paper.
% Included by papers/su3_triplet/main.tex via % © 2025 Ing. David Jaroš — CC BY-NC-ND 4.0
%
% This work is licensed under a Creative Commons Attribution-NonCommercial-NoDerivatives
% 4.0 International License (CC BY-NC-ND 4.0).
%
% Related Work: Literature positioning for the SU(3) triplet paper.
% Included by papers/su3_triplet/main.tex via % © 2025 Ing. David Jaroš — CC BY-NC-ND 4.0
%
% This work is licensed under a Creative Commons Attribution-NonCommercial-NoDerivatives
% 4.0 International License (CC BY-NC-ND 4.0).
%
% Related Work: Literature positioning for the SU(3) triplet paper.
% Included by papers/su3_triplet/main.tex via \input{related_work}.
% Status: Draft
% Version: 1.0
% Date: 2026-03-01

\subsection{Classical result: $\mathrm{Aut}(\mathbb{H}) \cong \mathrm{SO}(3)$}

It is well known \cite{Porteous1995} that the automorphism group of the
quaternion algebra $\mathbb{H}$ over $\mathbb{R}$ is the rotation group:
\[
  \mathrm{Aut}_\mathbb{R}(\mathbb{H}) \cong \mathrm{SO}(3).
\]
Every automorphism of $\mathbb{H}$ is an inner automorphism of the form
$h\mapsto qhq^{-1}$ for some unit quaternion $q\in\mathbb{H}$, $|q|=1$.

Our construction is \textbf{not} an automorphism of $\mathbb{H}$ or of
$\mathcal{B}$.  The group $\mathrm{SU}(3)_{V_c}$ acts on the three-dimensional
complex subspace $V_c=\mathrm{span}_\mathbb{C}\{\mathbf{I},\mathbf{J},\mathbf{K}\}$,
not on the full algebra.  This is therefore \emph{complementary to}, not a
contradiction of, the classical $\mathrm{SO}(3)$ result.

\subsection{$\mathrm{GL}(2,\mathbb{C})$ rank obstruction}

Because $\mathcal{B}\cong M_2(\mathbb{C})$ as complex algebras, the
$\mathbb{R}$-algebra automorphism group is
\[
  \mathrm{Aut}_\mathbb{R}(\mathcal{B})
  \cong \bigl[\mathrm{GL}(2,\mathbb{C})\times\mathrm{GL}(2,\mathbb{C})\bigr]/\mathbb{Z}_2.
\]
The maximal compact subgroup of $\mathrm{Aut}_\mathbb{R}(\mathcal{B})$ is
$[\mathrm{U}(2)\times\mathrm{U}(2)]/\mathbb{Z}_2$, which does not contain
$\mathrm{SU}(3)$ as a subgroup (the rank of $\mathrm{SU}(3)$ is $2$, while
the rank of $\mathrm{U}(2)$ is $2$, but their embedding structure is
incompatible).  This obstruction is verified computationally in
\texttt{reports/associative\_su3\_scan.md}.

Our $\mathrm{SU}(3)_{V_c}$ avoids this obstruction by \emph{not} being an
automorphism of $\mathcal{B}$; it is only a symmetry group of the subspace
$V_c$ equipped with its Hermitian form.

\subsection{Comparison with Dixon and Furey}

Dixon \cite{Dixon1994} constructs Standard Model gauge groups from the tensor
product of division algebras $\mathbb{R}\otimes\mathbb{C}\otimes\mathbb{H}
\otimes\mathbb{O}$, where $\mathbb{O}$ denotes the octonions.  The
$\mathrm{SU}(3)$ factor in Dixon's construction requires the non-associative
octonion algebra.

Furey \cite{Furey2016} similarly uses the complex octonions $\mathbb{C}
\otimes\mathbb{O}$ to generate Standard Model multiplets, with $\mathrm{SU}(3)$
arising from $G_2$-invariant structure of $\mathbb{O}$.

\textbf{Our construction differs in a fundamental way}: we work entirely within
the \emph{associative} algebra $\mathcal{B}=\mathbb{C}\otimes\mathbb{H}$, with
no octonions.  The $\mathrm{SU}(3)$ action we identify is on the $P_2=-1$
eigenspace of a quaternion conjugation involution, not on a minimal left ideal
of an octonion algebra.

\medskip
\begin{center}
\begin{tabular}{lccc}
  \hline
  Reference & Algebra & Octonion-free & SU(3) as Aut? \\
  \hline
  Dixon \cite{Dixon1994} & $\mathbb{R}\otimes\mathbb{C}\otimes\mathbb{H}\otimes\mathbb{O}$ & No & Partially \\
  Furey \cite{Furey2016} & $\mathbb{C}\otimes\mathbb{O}$ & No & Via $G_2$ \\
  Connes \cite{Connes1994} & Spectral triple & No & Via standard model input \\
  \textbf{This work} & $\mathbb{C}\otimes\mathbb{H}$ & \textbf{Yes} & No (acts on $V_c$) \\
  \hline
\end{tabular}
\end{center}

\subsection{Non-contradiction with classical results}

Our Theorem~\ref{thm:su3_action} does not contradict any classical theorem
because:
\begin{enumerate}
  \item $\mathrm{SU}(3)_{V_c}$ is not claimed to be a subgroup of
    $\mathrm{Aut}_\mathbb{R}(\mathcal{B})$.
  \item The classical $\mathrm{SO}(3)\cong\mathrm{Aut}(\mathbb{H})$ result
    concerns the quaternion algebra alone; our $V_c$ is a complex subspace of
    $\mathcal{B}$, not a sub-algebra of $\mathbb{H}$.
  \item The $\mathrm{GL}(2,\mathbb{C})$ rank obstruction applies to algebra
    automorphisms; our symmetry is a \emph{unitary group acting on a module},
    not an algebra automorphism.
\end{enumerate}

The result is therefore a new structural observation about $\mathcal{B}$, not
in tension with any known classical theorem.
.
% Status: Draft
% Version: 1.0
% Date: 2026-03-01

\subsection{Classical result: $\mathrm{Aut}(\mathbb{H}) \cong \mathrm{SO}(3)$}

It is well known \cite{Porteous1995} that the automorphism group of the
quaternion algebra $\mathbb{H}$ over $\mathbb{R}$ is the rotation group:
\[
  \mathrm{Aut}_\mathbb{R}(\mathbb{H}) \cong \mathrm{SO}(3).
\]
Every automorphism of $\mathbb{H}$ is an inner automorphism of the form
$h\mapsto qhq^{-1}$ for some unit quaternion $q\in\mathbb{H}$, $|q|=1$.

Our construction is \textbf{not} an automorphism of $\mathbb{H}$ or of
$\mathcal{B}$.  The group $\mathrm{SU}(3)_{V_c}$ acts on the three-dimensional
complex subspace $V_c=\mathrm{span}_\mathbb{C}\{\mathbf{I},\mathbf{J},\mathbf{K}\}$,
not on the full algebra.  This is therefore \emph{complementary to}, not a
contradiction of, the classical $\mathrm{SO}(3)$ result.

\subsection{$\mathrm{GL}(2,\mathbb{C})$ rank obstruction}

Because $\mathcal{B}\cong M_2(\mathbb{C})$ as complex algebras, the
$\mathbb{R}$-algebra automorphism group is
\[
  \mathrm{Aut}_\mathbb{R}(\mathcal{B})
  \cong \bigl[\mathrm{GL}(2,\mathbb{C})\times\mathrm{GL}(2,\mathbb{C})\bigr]/\mathbb{Z}_2.
\]
The maximal compact subgroup of $\mathrm{Aut}_\mathbb{R}(\mathcal{B})$ is
$[\mathrm{U}(2)\times\mathrm{U}(2)]/\mathbb{Z}_2$, which does not contain
$\mathrm{SU}(3)$ as a subgroup (the rank of $\mathrm{SU}(3)$ is $2$, while
the rank of $\mathrm{U}(2)$ is $2$, but their embedding structure is
incompatible).  This obstruction is verified computationally in
\texttt{reports/associative\_su3\_scan.md}.

Our $\mathrm{SU}(3)_{V_c}$ avoids this obstruction by \emph{not} being an
automorphism of $\mathcal{B}$; it is only a symmetry group of the subspace
$V_c$ equipped with its Hermitian form.

\subsection{Comparison with Dixon and Furey}

Dixon \cite{Dixon1994} constructs Standard Model gauge groups from the tensor
product of division algebras $\mathbb{R}\otimes\mathbb{C}\otimes\mathbb{H}
\otimes\mathbb{O}$, where $\mathbb{O}$ denotes the octonions.  The
$\mathrm{SU}(3)$ factor in Dixon's construction requires the non-associative
octonion algebra.

Furey \cite{Furey2016} similarly uses the complex octonions $\mathbb{C}
\otimes\mathbb{O}$ to generate Standard Model multiplets, with $\mathrm{SU}(3)$
arising from $G_2$-invariant structure of $\mathbb{O}$.

\textbf{Our construction differs in a fundamental way}: we work entirely within
the \emph{associative} algebra $\mathcal{B}=\mathbb{C}\otimes\mathbb{H}$, with
no octonions.  The $\mathrm{SU}(3)$ action we identify is on the $P_2=-1$
eigenspace of a quaternion conjugation involution, not on a minimal left ideal
of an octonion algebra.

\medskip
\begin{center}
\begin{tabular}{lccc}
  \hline
  Reference & Algebra & Octonion-free & SU(3) as Aut? \\
  \hline
  Dixon \cite{Dixon1994} & $\mathbb{R}\otimes\mathbb{C}\otimes\mathbb{H}\otimes\mathbb{O}$ & No & Partially \\
  Furey \cite{Furey2016} & $\mathbb{C}\otimes\mathbb{O}$ & No & Via $G_2$ \\
  Connes \cite{Connes1994} & Spectral triple & No & Via standard model input \\
  \textbf{This work} & $\mathbb{C}\otimes\mathbb{H}$ & \textbf{Yes} & No (acts on $V_c$) \\
  \hline
\end{tabular}
\end{center}

\subsection{Non-contradiction with classical results}

Our Theorem~\ref{thm:su3_action} does not contradict any classical theorem
because:
\begin{enumerate}
  \item $\mathrm{SU}(3)_{V_c}$ is not claimed to be a subgroup of
    $\mathrm{Aut}_\mathbb{R}(\mathcal{B})$.
  \item The classical $\mathrm{SO}(3)\cong\mathrm{Aut}(\mathbb{H})$ result
    concerns the quaternion algebra alone; our $V_c$ is a complex subspace of
    $\mathcal{B}$, not a sub-algebra of $\mathbb{H}$.
  \item The $\mathrm{GL}(2,\mathbb{C})$ rank obstruction applies to algebra
    automorphisms; our symmetry is a \emph{unitary group acting on a module},
    not an algebra automorphism.
\end{enumerate}

The result is therefore a new structural observation about $\mathcal{B}$, not
in tension with any known classical theorem.
.
% Status: Draft
% Version: 1.0
% Date: 2026-03-01

\subsection{Classical result: $\mathrm{Aut}(\mathbb{H}) \cong \mathrm{SO}(3)$}

It is well known \cite{Porteous1995} that the automorphism group of the
quaternion algebra $\mathbb{H}$ over $\mathbb{R}$ is the rotation group:
\[
  \mathrm{Aut}_\mathbb{R}(\mathbb{H}) \cong \mathrm{SO}(3).
\]
Every automorphism of $\mathbb{H}$ is an inner automorphism of the form
$h\mapsto qhq^{-1}$ for some unit quaternion $q\in\mathbb{H}$, $|q|=1$.

Our construction is \textbf{not} an automorphism of $\mathbb{H}$ or of
$\mathcal{B}$.  The group $\mathrm{SU}(3)_{V_c}$ acts on the three-dimensional
complex subspace $V_c=\mathrm{span}_\mathbb{C}\{\mathbf{I},\mathbf{J},\mathbf{K}\}$,
not on the full algebra.  This is therefore \emph{complementary to}, not a
contradiction of, the classical $\mathrm{SO}(3)$ result.

\subsection{$\mathrm{GL}(2,\mathbb{C})$ rank obstruction}

Because $\mathcal{B}\cong M_2(\mathbb{C})$ as complex algebras, the
$\mathbb{R}$-algebra automorphism group is
\[
  \mathrm{Aut}_\mathbb{R}(\mathcal{B})
  \cong \bigl[\mathrm{GL}(2,\mathbb{C})\times\mathrm{GL}(2,\mathbb{C})\bigr]/\mathbb{Z}_2.
\]
The maximal compact subgroup of $\mathrm{Aut}_\mathbb{R}(\mathcal{B})$ is
$[\mathrm{U}(2)\times\mathrm{U}(2)]/\mathbb{Z}_2$, which does not contain
$\mathrm{SU}(3)$ as a subgroup (the rank of $\mathrm{SU}(3)$ is $2$, while
the rank of $\mathrm{U}(2)$ is $2$, but their embedding structure is
incompatible).  This obstruction is verified computationally in
\texttt{reports/associative\_su3\_scan.md}.

Our $\mathrm{SU}(3)_{V_c}$ avoids this obstruction by \emph{not} being an
automorphism of $\mathcal{B}$; it is only a symmetry group of the subspace
$V_c$ equipped with its Hermitian form.

\subsection{Comparison with Dixon and Furey}

Dixon \cite{Dixon1994} constructs Standard Model gauge groups from the tensor
product of division algebras $\mathbb{R}\otimes\mathbb{C}\otimes\mathbb{H}
\otimes\mathbb{O}$, where $\mathbb{O}$ denotes the octonions.  The
$\mathrm{SU}(3)$ factor in Dixon's construction requires the non-associative
octonion algebra.

Furey \cite{Furey2016} similarly uses the complex octonions $\mathbb{C}
\otimes\mathbb{O}$ to generate Standard Model multiplets, with $\mathrm{SU}(3)$
arising from $G_2$-invariant structure of $\mathbb{O}$.

\textbf{Our construction differs in a fundamental way}: we work entirely within
the \emph{associative} algebra $\mathcal{B}=\mathbb{C}\otimes\mathbb{H}$, with
no octonions.  The $\mathrm{SU}(3)$ action we identify is on the $P_2=-1$
eigenspace of a quaternion conjugation involution, not on a minimal left ideal
of an octonion algebra.

\medskip
\begin{center}
\begin{tabular}{lccc}
  \hline
  Reference & Algebra & Octonion-free & SU(3) as Aut? \\
  \hline
  Dixon \cite{Dixon1994} & $\mathbb{R}\otimes\mathbb{C}\otimes\mathbb{H}\otimes\mathbb{O}$ & No & Partially \\
  Furey \cite{Furey2016} & $\mathbb{C}\otimes\mathbb{O}$ & No & Via $G_2$ \\
  Connes \cite{Connes1994} & Spectral triple & No & Via standard model input \\
  \textbf{This work} & $\mathbb{C}\otimes\mathbb{H}$ & \textbf{Yes} & No (acts on $V_c$) \\
  \hline
\end{tabular}
\end{center}

\subsection{Non-contradiction with classical results}

Our Theorem~\ref{thm:su3_action} does not contradict any classical theorem
because:
\begin{enumerate}
  \item $\mathrm{SU}(3)_{V_c}$ is not claimed to be a subgroup of
    $\mathrm{Aut}_\mathbb{R}(\mathcal{B})$.
  \item The classical $\mathrm{SO}(3)\cong\mathrm{Aut}(\mathbb{H})$ result
    concerns the quaternion algebra alone; our $V_c$ is a complex subspace of
    $\mathcal{B}$, not a sub-algebra of $\mathbb{H}$.
  \item The $\mathrm{GL}(2,\mathbb{C})$ rank obstruction applies to algebra
    automorphisms; our symmetry is a \emph{unitary group acting on a module},
    not an algebra automorphism.
\end{enumerate}

The result is therefore a new structural observation about $\mathcal{B}$, not
in tension with any known classical theorem.


% ============================================================
\section{Open Problems}
\label{sec:open_problems}
% ============================================================

\begin{enumerate}
  \item \textbf{Dynamics.}  Does the $\mathrm{SU}(3)_{V_c}$ symmetry extend
    to a symmetry of a natural kinetic operator on sections of a bundle with
    fibre $V_c$?  Specifically, is the operator
    $\Box_\Theta = \nabla^\dagger\nabla$ acting on $\Theta$-triplet fields
    $(\Theta^1,\Theta^2,\Theta^3)\in V_c$ invariant under constant
    $U\in\mathrm{SU}(3)_{V_c}$?  (This is the global invariance question;
    see \texttt{research\_tracks/associative\_su3/global\_invariance.tex}.)

  \item \textbf{Gauge extension.}  Can a local $\mathrm{SU}(3)$ gauge
    extension be consistently formulated, and does it remain associative?

  \item \textbf{Physical interpretation.}  Is $\mathrm{SU}(3)_{V_c}$ related
    to any observed symmetry of particle physics?  This is an open and
    speculative question, not addressed in this paper.

  \item \textbf{Alternative involutions.}  Can a modified $P_3'$ be defined
    that fills the empty $(+,+,-)$ sector while maintaining associativity?

  \item \textbf{Uniqueness.}  Is the construction of $V_c$ canonical up to
    automorphisms of $\B$, or are there other natural three-dimensional
    complex subspaces?
\end{enumerate}

% ============================================================
\section{Conclusion}
\label{sec:conclusion}
% ============================================================

We have proved three theorems establishing a natural $\mathrm{SU}(3)$
symmetry acting on a canonical three-dimensional complex subspace $V_c$ of
the biquaternion algebra $\B = \C\otimes\H$.  The construction is entirely
within the associative algebra $\B\cong M_2(\C)$; no octonions or
non-associative structures are needed.

The group $\mathrm{SU}(3)_{V_c}$ is not a subgroup of $\mathrm{Aut}(\B)$, in
agreement with classical results.  Its potential role as a symmetry of
dynamical equations on biquaternionic fields remains an open problem.

\begin{thebibliography}{99}

\bibitem{Hamilton1866}
W.~R. Hamilton,
\emph{Elements of Quaternions},
Longmans, Green, 1866.

\bibitem{Clifford1878}
W.~K. Clifford,
Applications of Grassmann's Extensive Algebra,
\emph{Am.\ J.\ Math.} \textbf{1} (1878) 350--358.

\bibitem{Dixon1994}
G.~M. Dixon,
\emph{Division Algebras: Octonions, Quaternions, Complex Numbers and the
Algebraic Design of Physics},
Kluwer Academic Publishers, 1994.

\bibitem{Furey2016}
C.~Furey,
Standard Model Physics from an Algebra?,
Ph.D.\ Thesis, University of Waterloo, 2015; \texttt{arXiv:1611.09182}.

\bibitem{Connes1994}
A.~Connes,
\emph{Noncommutative Geometry},
Academic Press, 1994.

\bibitem{Porteous1995}
I.~R. Porteous,
\emph{Clifford Algebras and the Classical Groups},
Cambridge University Press, 1995.

\bibitem{FitzGerald2009}
R.~W. FitzGerald,
\emph{Biquaternions and the Clifford Algebra $\mathrm{Cl}_{3,0}$},
Advances in Applied Clifford Algebras, 2009.

\end{thebibliography}

\end{document}
