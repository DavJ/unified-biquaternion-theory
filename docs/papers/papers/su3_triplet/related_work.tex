% © 2025 Ing. David Jaroš — CC BY-NC-ND 4.0
%
% This work is licensed under a Creative Commons Attribution-NonCommercial-NoDerivatives
% 4.0 International License (CC BY-NC-ND 4.0).
%
% Related Work: Literature positioning for the SU(3) triplet paper.
% Included by papers/su3_triplet/main.tex via % © 2025 Ing. David Jaroš — CC BY-NC-ND 4.0
%
% This work is licensed under a Creative Commons Attribution-NonCommercial-NoDerivatives
% 4.0 International License (CC BY-NC-ND 4.0).
%
% Related Work: Literature positioning for the SU(3) triplet paper.
% Included by papers/su3_triplet/main.tex via % © 2025 Ing. David Jaroš — CC BY-NC-ND 4.0
%
% This work is licensed under a Creative Commons Attribution-NonCommercial-NoDerivatives
% 4.0 International License (CC BY-NC-ND 4.0).
%
% Related Work: Literature positioning for the SU(3) triplet paper.
% Included by papers/su3_triplet/main.tex via % © 2025 Ing. David Jaroš — CC BY-NC-ND 4.0
%
% This work is licensed under a Creative Commons Attribution-NonCommercial-NoDerivatives
% 4.0 International License (CC BY-NC-ND 4.0).
%
% Related Work: Literature positioning for the SU(3) triplet paper.
% Included by papers/su3_triplet/main.tex via \input{related_work}.
% Status: Draft
% Version: 1.0
% Date: 2026-03-01

\subsection{Classical result: $\mathrm{Aut}(\mathbb{H}) \cong \mathrm{SO}(3)$}

It is well known \cite{Porteous1995} that the automorphism group of the
quaternion algebra $\mathbb{H}$ over $\mathbb{R}$ is the rotation group:
\[
  \mathrm{Aut}_\mathbb{R}(\mathbb{H}) \cong \mathrm{SO}(3).
\]
Every automorphism of $\mathbb{H}$ is an inner automorphism of the form
$h\mapsto qhq^{-1}$ for some unit quaternion $q\in\mathbb{H}$, $|q|=1$.

Our construction is \textbf{not} an automorphism of $\mathbb{H}$ or of
$\mathcal{B}$.  The group $\mathrm{SU}(3)_{V_c}$ acts on the three-dimensional
complex subspace $V_c=\mathrm{span}_\mathbb{C}\{\mathbf{I},\mathbf{J},\mathbf{K}\}$,
not on the full algebra.  This is therefore \emph{complementary to}, not a
contradiction of, the classical $\mathrm{SO}(3)$ result.

\subsection{$\mathrm{GL}(2,\mathbb{C})$ rank obstruction}

Because $\mathcal{B}\cong M_2(\mathbb{C})$ as complex algebras, the
$\mathbb{R}$-algebra automorphism group is
\[
  \mathrm{Aut}_\mathbb{R}(\mathcal{B})
  \cong \bigl[\mathrm{GL}(2,\mathbb{C})\times\mathrm{GL}(2,\mathbb{C})\bigr]/\mathbb{Z}_2.
\]
The maximal compact subgroup of $\mathrm{Aut}_\mathbb{R}(\mathcal{B})$ is
$[\mathrm{U}(2)\times\mathrm{U}(2)]/\mathbb{Z}_2$, which does not contain
$\mathrm{SU}(3)$ as a subgroup (the rank of $\mathrm{SU}(3)$ is $2$, while
the rank of $\mathrm{U}(2)$ is $2$, but their embedding structure is
incompatible).  This obstruction is verified computationally in
\texttt{reports/associative\_su3\_scan.md}.

Our $\mathrm{SU}(3)_{V_c}$ avoids this obstruction by \emph{not} being an
automorphism of $\mathcal{B}$; it is only a symmetry group of the subspace
$V_c$ equipped with its Hermitian form.

\subsection{Comparison with Dixon and Furey}

Dixon \cite{Dixon1994} constructs Standard Model gauge groups from the tensor
product of division algebras $\mathbb{R}\otimes\mathbb{C}\otimes\mathbb{H}
\otimes\mathbb{O}$, where $\mathbb{O}$ denotes the octonions.  The
$\mathrm{SU}(3)$ factor in Dixon's construction requires the non-associative
octonion algebra.

Furey \cite{Furey2016} similarly uses the complex octonions $\mathbb{C}
\otimes\mathbb{O}$ to generate Standard Model multiplets, with $\mathrm{SU}(3)$
arising from $G_2$-invariant structure of $\mathbb{O}$.

\textbf{Our construction differs in a fundamental way}: we work entirely within
the \emph{associative} algebra $\mathcal{B}=\mathbb{C}\otimes\mathbb{H}$, with
no octonions.  The $\mathrm{SU}(3)$ action we identify is on the $P_2=-1$
eigenspace of a quaternion conjugation involution, not on a minimal left ideal
of an octonion algebra.

\medskip
\begin{center}
\begin{tabular}{lccc}
  \hline
  Reference & Algebra & Octonion-free & SU(3) as Aut? \\
  \hline
  Dixon \cite{Dixon1994} & $\mathbb{R}\otimes\mathbb{C}\otimes\mathbb{H}\otimes\mathbb{O}$ & No & Partially \\
  Furey \cite{Furey2016} & $\mathbb{C}\otimes\mathbb{O}$ & No & Via $G_2$ \\
  Connes \cite{Connes1994} & Spectral triple & No & Via standard model input \\
  \textbf{This work} & $\mathbb{C}\otimes\mathbb{H}$ & \textbf{Yes} & No (acts on $V_c$) \\
  \hline
\end{tabular}
\end{center}

\subsection{Non-contradiction with classical results}

Our Theorem~\ref{thm:su3_action} does not contradict any classical theorem
because:
\begin{enumerate}
  \item $\mathrm{SU}(3)_{V_c}$ is not claimed to be a subgroup of
    $\mathrm{Aut}_\mathbb{R}(\mathcal{B})$.
  \item The classical $\mathrm{SO}(3)\cong\mathrm{Aut}(\mathbb{H})$ result
    concerns the quaternion algebra alone; our $V_c$ is a complex subspace of
    $\mathcal{B}$, not a sub-algebra of $\mathbb{H}$.
  \item The $\mathrm{GL}(2,\mathbb{C})$ rank obstruction applies to algebra
    automorphisms; our symmetry is a \emph{unitary group acting on a module},
    not an algebra automorphism.
\end{enumerate}

The result is therefore a new structural observation about $\mathcal{B}$, not
in tension with any known classical theorem.
.
% Status: Draft
% Version: 1.0
% Date: 2026-03-01

\subsection{Classical result: $\mathrm{Aut}(\mathbb{H}) \cong \mathrm{SO}(3)$}

It is well known \cite{Porteous1995} that the automorphism group of the
quaternion algebra $\mathbb{H}$ over $\mathbb{R}$ is the rotation group:
\[
  \mathrm{Aut}_\mathbb{R}(\mathbb{H}) \cong \mathrm{SO}(3).
\]
Every automorphism of $\mathbb{H}$ is an inner automorphism of the form
$h\mapsto qhq^{-1}$ for some unit quaternion $q\in\mathbb{H}$, $|q|=1$.

Our construction is \textbf{not} an automorphism of $\mathbb{H}$ or of
$\mathcal{B}$.  The group $\mathrm{SU}(3)_{V_c}$ acts on the three-dimensional
complex subspace $V_c=\mathrm{span}_\mathbb{C}\{\mathbf{I},\mathbf{J},\mathbf{K}\}$,
not on the full algebra.  This is therefore \emph{complementary to}, not a
contradiction of, the classical $\mathrm{SO}(3)$ result.

\subsection{$\mathrm{GL}(2,\mathbb{C})$ rank obstruction}

Because $\mathcal{B}\cong M_2(\mathbb{C})$ as complex algebras, the
$\mathbb{R}$-algebra automorphism group is
\[
  \mathrm{Aut}_\mathbb{R}(\mathcal{B})
  \cong \bigl[\mathrm{GL}(2,\mathbb{C})\times\mathrm{GL}(2,\mathbb{C})\bigr]/\mathbb{Z}_2.
\]
The maximal compact subgroup of $\mathrm{Aut}_\mathbb{R}(\mathcal{B})$ is
$[\mathrm{U}(2)\times\mathrm{U}(2)]/\mathbb{Z}_2$, which does not contain
$\mathrm{SU}(3)$ as a subgroup (the rank of $\mathrm{SU}(3)$ is $2$, while
the rank of $\mathrm{U}(2)$ is $2$, but their embedding structure is
incompatible).  This obstruction is verified computationally in
\texttt{reports/associative\_su3\_scan.md}.

Our $\mathrm{SU}(3)_{V_c}$ avoids this obstruction by \emph{not} being an
automorphism of $\mathcal{B}$; it is only a symmetry group of the subspace
$V_c$ equipped with its Hermitian form.

\subsection{Comparison with Dixon and Furey}

Dixon \cite{Dixon1994} constructs Standard Model gauge groups from the tensor
product of division algebras $\mathbb{R}\otimes\mathbb{C}\otimes\mathbb{H}
\otimes\mathbb{O}$, where $\mathbb{O}$ denotes the octonions.  The
$\mathrm{SU}(3)$ factor in Dixon's construction requires the non-associative
octonion algebra.

Furey \cite{Furey2016} similarly uses the complex octonions $\mathbb{C}
\otimes\mathbb{O}$ to generate Standard Model multiplets, with $\mathrm{SU}(3)$
arising from $G_2$-invariant structure of $\mathbb{O}$.

\textbf{Our construction differs in a fundamental way}: we work entirely within
the \emph{associative} algebra $\mathcal{B}=\mathbb{C}\otimes\mathbb{H}$, with
no octonions.  The $\mathrm{SU}(3)$ action we identify is on the $P_2=-1$
eigenspace of a quaternion conjugation involution, not on a minimal left ideal
of an octonion algebra.

\medskip
\begin{center}
\begin{tabular}{lccc}
  \hline
  Reference & Algebra & Octonion-free & SU(3) as Aut? \\
  \hline
  Dixon \cite{Dixon1994} & $\mathbb{R}\otimes\mathbb{C}\otimes\mathbb{H}\otimes\mathbb{O}$ & No & Partially \\
  Furey \cite{Furey2016} & $\mathbb{C}\otimes\mathbb{O}$ & No & Via $G_2$ \\
  Connes \cite{Connes1994} & Spectral triple & No & Via standard model input \\
  \textbf{This work} & $\mathbb{C}\otimes\mathbb{H}$ & \textbf{Yes} & No (acts on $V_c$) \\
  \hline
\end{tabular}
\end{center}

\subsection{Non-contradiction with classical results}

Our Theorem~\ref{thm:su3_action} does not contradict any classical theorem
because:
\begin{enumerate}
  \item $\mathrm{SU}(3)_{V_c}$ is not claimed to be a subgroup of
    $\mathrm{Aut}_\mathbb{R}(\mathcal{B})$.
  \item The classical $\mathrm{SO}(3)\cong\mathrm{Aut}(\mathbb{H})$ result
    concerns the quaternion algebra alone; our $V_c$ is a complex subspace of
    $\mathcal{B}$, not a sub-algebra of $\mathbb{H}$.
  \item The $\mathrm{GL}(2,\mathbb{C})$ rank obstruction applies to algebra
    automorphisms; our symmetry is a \emph{unitary group acting on a module},
    not an algebra automorphism.
\end{enumerate}

The result is therefore a new structural observation about $\mathcal{B}$, not
in tension with any known classical theorem.
.
% Status: Draft
% Version: 1.0
% Date: 2026-03-01

\subsection{Classical result: $\mathrm{Aut}(\mathbb{H}) \cong \mathrm{SO}(3)$}

It is well known \cite{Porteous1995} that the automorphism group of the
quaternion algebra $\mathbb{H}$ over $\mathbb{R}$ is the rotation group:
\[
  \mathrm{Aut}_\mathbb{R}(\mathbb{H}) \cong \mathrm{SO}(3).
\]
Every automorphism of $\mathbb{H}$ is an inner automorphism of the form
$h\mapsto qhq^{-1}$ for some unit quaternion $q\in\mathbb{H}$, $|q|=1$.

Our construction is \textbf{not} an automorphism of $\mathbb{H}$ or of
$\mathcal{B}$.  The group $\mathrm{SU}(3)_{V_c}$ acts on the three-dimensional
complex subspace $V_c=\mathrm{span}_\mathbb{C}\{\mathbf{I},\mathbf{J},\mathbf{K}\}$,
not on the full algebra.  This is therefore \emph{complementary to}, not a
contradiction of, the classical $\mathrm{SO}(3)$ result.

\subsection{$\mathrm{GL}(2,\mathbb{C})$ rank obstruction}

Because $\mathcal{B}\cong M_2(\mathbb{C})$ as complex algebras, the
$\mathbb{R}$-algebra automorphism group is
\[
  \mathrm{Aut}_\mathbb{R}(\mathcal{B})
  \cong \bigl[\mathrm{GL}(2,\mathbb{C})\times\mathrm{GL}(2,\mathbb{C})\bigr]/\mathbb{Z}_2.
\]
The maximal compact subgroup of $\mathrm{Aut}_\mathbb{R}(\mathcal{B})$ is
$[\mathrm{U}(2)\times\mathrm{U}(2)]/\mathbb{Z}_2$, which does not contain
$\mathrm{SU}(3)$ as a subgroup (the rank of $\mathrm{SU}(3)$ is $2$, while
the rank of $\mathrm{U}(2)$ is $2$, but their embedding structure is
incompatible).  This obstruction is verified computationally in
\texttt{reports/associative\_su3\_scan.md}.

Our $\mathrm{SU}(3)_{V_c}$ avoids this obstruction by \emph{not} being an
automorphism of $\mathcal{B}$; it is only a symmetry group of the subspace
$V_c$ equipped with its Hermitian form.

\subsection{Comparison with Dixon and Furey}

Dixon \cite{Dixon1994} constructs Standard Model gauge groups from the tensor
product of division algebras $\mathbb{R}\otimes\mathbb{C}\otimes\mathbb{H}
\otimes\mathbb{O}$, where $\mathbb{O}$ denotes the octonions.  The
$\mathrm{SU}(3)$ factor in Dixon's construction requires the non-associative
octonion algebra.

Furey \cite{Furey2016} similarly uses the complex octonions $\mathbb{C}
\otimes\mathbb{O}$ to generate Standard Model multiplets, with $\mathrm{SU}(3)$
arising from $G_2$-invariant structure of $\mathbb{O}$.

\textbf{Our construction differs in a fundamental way}: we work entirely within
the \emph{associative} algebra $\mathcal{B}=\mathbb{C}\otimes\mathbb{H}$, with
no octonions.  The $\mathrm{SU}(3)$ action we identify is on the $P_2=-1$
eigenspace of a quaternion conjugation involution, not on a minimal left ideal
of an octonion algebra.

\medskip
\begin{center}
\begin{tabular}{lccc}
  \hline
  Reference & Algebra & Octonion-free & SU(3) as Aut? \\
  \hline
  Dixon \cite{Dixon1994} & $\mathbb{R}\otimes\mathbb{C}\otimes\mathbb{H}\otimes\mathbb{O}$ & No & Partially \\
  Furey \cite{Furey2016} & $\mathbb{C}\otimes\mathbb{O}$ & No & Via $G_2$ \\
  Connes \cite{Connes1994} & Spectral triple & No & Via standard model input \\
  \textbf{This work} & $\mathbb{C}\otimes\mathbb{H}$ & \textbf{Yes} & No (acts on $V_c$) \\
  \hline
\end{tabular}
\end{center}

\subsection{Non-contradiction with classical results}

Our Theorem~\ref{thm:su3_action} does not contradict any classical theorem
because:
\begin{enumerate}
  \item $\mathrm{SU}(3)_{V_c}$ is not claimed to be a subgroup of
    $\mathrm{Aut}_\mathbb{R}(\mathcal{B})$.
  \item The classical $\mathrm{SO}(3)\cong\mathrm{Aut}(\mathbb{H})$ result
    concerns the quaternion algebra alone; our $V_c$ is a complex subspace of
    $\mathcal{B}$, not a sub-algebra of $\mathbb{H}$.
  \item The $\mathrm{GL}(2,\mathbb{C})$ rank obstruction applies to algebra
    automorphisms; our symmetry is a \emph{unitary group acting on a module},
    not an algebra automorphism.
\end{enumerate}

The result is therefore a new structural observation about $\mathcal{B}$, not
in tension with any known classical theorem.
.
% Status: Draft
% Version: 1.0
% Date: 2026-03-01

\subsection{Classical result: $\mathrm{Aut}(\mathbb{H}) \cong \mathrm{SO}(3)$}

It is well known \cite{Porteous1995} that the automorphism group of the
quaternion algebra $\mathbb{H}$ over $\mathbb{R}$ is the rotation group:
\[
  \mathrm{Aut}_\mathbb{R}(\mathbb{H}) \cong \mathrm{SO}(3).
\]
Every automorphism of $\mathbb{H}$ is an inner automorphism of the form
$h\mapsto qhq^{-1}$ for some unit quaternion $q\in\mathbb{H}$, $|q|=1$.

Our construction is \textbf{not} an automorphism of $\mathbb{H}$ or of
$\mathcal{B}$.  The group $\mathrm{SU}(3)_{V_c}$ acts on the three-dimensional
complex subspace $V_c=\mathrm{span}_\mathbb{C}\{\mathbf{I},\mathbf{J},\mathbf{K}\}$,
not on the full algebra.  This is therefore \emph{complementary to}, not a
contradiction of, the classical $\mathrm{SO}(3)$ result.

\subsection{$\mathrm{GL}(2,\mathbb{C})$ rank obstruction}

Because $\mathcal{B}\cong M_2(\mathbb{C})$ as complex algebras, the
$\mathbb{R}$-algebra automorphism group is
\[
  \mathrm{Aut}_\mathbb{R}(\mathcal{B})
  \cong \bigl[\mathrm{GL}(2,\mathbb{C})\times\mathrm{GL}(2,\mathbb{C})\bigr]/\mathbb{Z}_2.
\]
The maximal compact subgroup of $\mathrm{Aut}_\mathbb{R}(\mathcal{B})$ is
$[\mathrm{U}(2)\times\mathrm{U}(2)]/\mathbb{Z}_2$, which does not contain
$\mathrm{SU}(3)$ as a subgroup (the rank of $\mathrm{SU}(3)$ is $2$, while
the rank of $\mathrm{U}(2)$ is $2$, but their embedding structure is
incompatible).  This obstruction is verified computationally in
\texttt{reports/associative\_su3\_scan.md}.

Our $\mathrm{SU}(3)_{V_c}$ avoids this obstruction by \emph{not} being an
automorphism of $\mathcal{B}$; it is only a symmetry group of the subspace
$V_c$ equipped with its Hermitian form.

\subsection{Comparison with Dixon and Furey}

Dixon \cite{Dixon1994} constructs Standard Model gauge groups from the tensor
product of division algebras $\mathbb{R}\otimes\mathbb{C}\otimes\mathbb{H}
\otimes\mathbb{O}$, where $\mathbb{O}$ denotes the octonions.  The
$\mathrm{SU}(3)$ factor in Dixon's construction requires the non-associative
octonion algebra.

Furey \cite{Furey2016} similarly uses the complex octonions $\mathbb{C}
\otimes\mathbb{O}$ to generate Standard Model multiplets, with $\mathrm{SU}(3)$
arising from $G_2$-invariant structure of $\mathbb{O}$.

\textbf{Our construction differs in a fundamental way}: we work entirely within
the \emph{associative} algebra $\mathcal{B}=\mathbb{C}\otimes\mathbb{H}$, with
no octonions.  The $\mathrm{SU}(3)$ action we identify is on the $P_2=-1$
eigenspace of a quaternion conjugation involution, not on a minimal left ideal
of an octonion algebra.

\medskip
\begin{center}
\begin{tabular}{lccc}
  \hline
  Reference & Algebra & Octonion-free & SU(3) as Aut? \\
  \hline
  Dixon \cite{Dixon1994} & $\mathbb{R}\otimes\mathbb{C}\otimes\mathbb{H}\otimes\mathbb{O}$ & No & Partially \\
  Furey \cite{Furey2016} & $\mathbb{C}\otimes\mathbb{O}$ & No & Via $G_2$ \\
  Connes \cite{Connes1994} & Spectral triple & No & Via standard model input \\
  \textbf{This work} & $\mathbb{C}\otimes\mathbb{H}$ & \textbf{Yes} & No (acts on $V_c$) \\
  \hline
\end{tabular}
\end{center}

\subsection{Non-contradiction with classical results}

Our Theorem~\ref{thm:su3_action} does not contradict any classical theorem
because:
\begin{enumerate}
  \item $\mathrm{SU}(3)_{V_c}$ is not claimed to be a subgroup of
    $\mathrm{Aut}_\mathbb{R}(\mathcal{B})$.
  \item The classical $\mathrm{SO}(3)\cong\mathrm{Aut}(\mathbb{H})$ result
    concerns the quaternion algebra alone; our $V_c$ is a complex subspace of
    $\mathcal{B}$, not a sub-algebra of $\mathbb{H}$.
  \item The $\mathrm{GL}(2,\mathbb{C})$ rank obstruction applies to algebra
    automorphisms; our symmetry is a \emph{unitary group acting on a module},
    not an algebra automorphism.
\end{enumerate}

The result is therefore a new structural observation about $\mathcal{B}$, not
in tension with any known classical theorem.
