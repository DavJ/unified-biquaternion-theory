
\section{Biquaternion Spectral Operator}

\begin{definition}[Biquaternion Time]
Let
\begin{equation}
\tau_{\mathrm{BQ}} =
(t_0 + t_1\mathbf{i} + t_2\mathbf{j} + t_3\mathbf{k})
+ i(u_0 + u_1\mathbf{i} + u_2\mathbf{j} + u_3\mathbf{k})
\in \mathbb{C}\otimes\mathbb{H}.
\end{equation}
We define its quaternionic and complex conjugations as
\begin{align}
\overline{\tau}_{\mathrm{BQ}} &=
(t_0 - t_1\mathbf{i} - t_2\mathbf{j} - t_3\mathbf{k})
+ i(u_0 - u_1\mathbf{i} - u_2\mathbf{j} - u_3\mathbf{k}), \\
\tau_{\mathrm{BQ}}^{*} &=
(t_0 + t_1\mathbf{i} + t_2\mathbf{j} + t_3\mathbf{k})
- i(u_0 + u_1\mathbf{i} + u_2\mathbf{j} + u_3\mathbf{k}), \\
\tau_{\mathrm{BQ}}^{\dagger} &= (\overline{\tau}_{\mathrm{BQ}})^{*}.
\end{align}
\end{definition}

\begin{definition}[Biquaternion Spectral Operator]
Let $\tau_{\mathrm{BQ}} = (t_0 + t_1\mathbf{i} + t_2\mathbf{j} + t_3\mathbf{k})
+ i(u_0 + u_1\mathbf{i} + u_2\mathbf{j} + u_3\mathbf{k}) \in \mathbb{C}\otimes\mathbb{H}$.
Define
\begin{equation}
M_{\mathrm{BQ}} f(\tau_{\mathrm{BQ}}) =
-\sum_{\mu=0}^{3} e_\mu \frac{\partial f}{\partial t_\mu}
- i \sum_{\mu=0}^{3} e_\mu \frac{\partial f}{\partial u_\mu}
+ V(\tau_{\mathrm{BQ}}) f(\tau_{\mathrm{BQ}}),
\end{equation}
where $e_0=1$, $e_1=\mathbf{i}$, $e_2=\mathbf{j}$, $e_3=\mathbf{k}$,
and $V(\tau_{\mathrm{BQ}})^\dagger = V(\tau_{\mathrm{BQ}})$.
\end{definition}

\begin{proposition}[Hermiticity Condition]
If $V(\tau_{\mathrm{BQ}})$ is Hermitian and the boundary terms vanish at infinity,
then $M_{\mathrm{BQ}}$ is self-adjoint in $\mathcal{H}(\mathbb{C}\otimes\mathbb{H})$:
\begin{equation}
\langle f, M_{\mathrm{BQ}} g \rangle = \langle M_{\mathrm{BQ}} f, g \rangle.
\end{equation}
\end{proposition}

\begin{remark}
The classical complex-time operator $M_{\mathbb{C}}$ corresponds
to the projection $\Pi(\tau_{\mathrm{BQ}})=t_0+i u_0$.
Thus the complex zeta spectrum is a slice projection of the full
biquaternionic spectrum.
\end{remark}
