
\section{Biquaternion Spectral Operator}

\begin{definition}[Biquaternion Time]
Let
\begin{equation}
\tau_{\mathrm{BQ}} =
(t_0 + t_1\mathbf{i} + t_2\mathbf{j} + t_3\mathbf{k})
+ i(u_0 + u_1\mathbf{i} + u_2\mathbf{j} + u_3\mathbf{k})
\in \mathbb{C}\otimes\mathbb{H}.
\end{equation}
We define its quaternionic and complex conjugations as
\begin{align}
\overline{\tau}_{\mathrm{BQ}} &=
(t_0 - t_1\mathbf{i} - t_2\mathbf{j} - t_3\mathbf{k})
+ i(u_0 - u_1\mathbf{i} - u_2\mathbf{j} - u_3\mathbf{k}), \\
\tau_{\mathrm{BQ}}^{*} &=
(t_0 + t_1\mathbf{i} + t_2\mathbf{j} + t_3\mathbf{k})
- i(u_0 + u_1\mathbf{i} + u_2\mathbf{j} + u_3\mathbf{k}), \\
\tau_{\mathrm{BQ}}^{\dagger} &= (\overline{\tau}_{\mathrm{BQ}})^{*}.
\end{align}
\end{definition}

\begin{definition}[Biquaternion Spectral Operator]
Let $\tau_{\mathrm{BQ}} = (t_0 + t_1\mathbf{i} + t_2\mathbf{j} + t_3\mathbf{k})
+ i(u_0 + u_1\mathbf{i} + u_2\mathbf{j} + u_3\mathbf{k}) \in \mathbb{C}\otimes\mathbb{H}$.
Define
\begin{equation}
M_{\mathrm{BQ}} f(\tau_{\mathrm{BQ}}) =
-\sum_{\mu=0}^{3} e_\mu \frac{\partial f}{\partial t_\mu}
- i \sum_{\mu=0}^{3} e_\mu \frac{\partial f}{\partial u_\mu}
+ V(\tau_{\mathrm{BQ}}) f(\tau_{\mathrm{BQ}}),
\end{equation}
where $e_0=1$, $e_1=\mathbf{i}$, $e_2=\mathbf{j}$, $e_3=\mathbf{k}$,
and $V(\tau_{\mathrm{BQ}})^\dagger = V(\tau_{\mathrm{BQ}})$.
\end{definition}

\begin{proposition}[Hermiticity Condition]
If $V(\tau_{\mathrm{BQ}})$ is Hermitian and the boundary terms vanish at infinity,
then $M_{\mathrm{BQ}}$ is self-adjoint in $\mathcal{H}(\mathbb{C}\otimes\mathbb{H})$:
\begin{equation}
\langle f, M_{\mathrm{BQ}} g \rangle = \langle M_{\mathrm{BQ}} f, g \rangle.
\end{equation}
\end{proposition}

\begin{remark}
The classical complex-time operator $M_{\mathbb{C}}$ corresponds
to the projection $\Pi(\tau_{\mathrm{BQ}})=t_0+i u_0$.
Thus the complex zeta spectrum is a slice projection of the full
biquaternionic spectrum.
\end{remark}

\begin{remark}[Relation to Riemann Hypothesis]
The Riemann zeta spectrum corresponds to the complex projection of the
real spectrum of the self-adjoint operator $M_{\mathrm{BQ}}$ in $\mathbb{C}\otimes\mathbb{H}$.
This connection is \emph{structural} and represents a natural mathematical analogy,
not a proof of the Riemann Hypothesis. The spectral properties of $M_{\mathrm{BQ}}$
are defined independently of number-theoretic conjectures.

\textbf{Important:} The spectral framework is just a mathematical tool. It is not clear
whether it can help prove RH, and we should not attempt to prove RH within UBT.
While it is useful to show the connection of UBT with RH, zeta numbers, and theta,
we must be very careful that we or somebody else does not claim to have proven anything about RH.
\end{remark}
