% © 2025 Ing. David Jaroš — CC BY-NC-ND 4.0
% Complex Chronofactor τ=t+iψ: Spectral Consequences and Links to S_Θ and Σ_Θ
% LaTeX Version - Suitable for Paper Appendix or Standalone Document

\section{Complex Chronofactor $\tau=t+i\psi$: Spectral Consequences}
\label{sec:complex-chronofactor-spectrum}

\subsection{Abstract}

This section provides a rigorous symbolic derivation of how a complex chronofactor $\tau = t + i\psi$ manifests in spectral properties of the biquaternionic field $\Theta(\tau)$. We establish the formal mathematical connection between complex-time evolution and the UBT invariants $S_\Theta = k_B \log \det(\Theta^\dagger\Theta)$ (entropy channel) and $\Sigma_\Theta = k_B \arg \det \Theta$ (phase channel). The analysis focuses on linear complex-time flow, eigenmode decomposition, and the emergence of testable discriminators that distinguish real-time evolution from complex-time dynamics.

\subsection{Setup and Assumptions}
\label{sec:setup-assumptions}

\subsubsection{The Biquaternionic Field}

We work with a complex matrix field $\Theta(\tau)$ where:
\begin{itemize}
\item $\Theta$: $n\times n$ complex matrix field
\item $\tau = t + i\psi$: Complex chronofactor ($t$ = real time, $\psi$ = imaginary/phase component)
\item $q$: Spatial coordinates (suppressed when focusing on temporal evolution)
\end{itemize}

\subsubsection{Fundamental Assumptions}

\textbf{Assumption 1.1} (Regularity): $\Theta(\tau)$ is holomorphic in a neighborhood of the real axis $\mathrm{Im}(\tau) = 0$ for physical time evolution.

\textbf{Assumption 1.2} (Positive-Definiteness on Real Slice): For real $t$ with fixed $\psi$, the Hermitian product $O(t) := \Theta^\dagger(t+i\psi) \Theta(t+i\psi)$ is positive semidefinite.

\textbf{Assumption 1.3} (Linear Flow): The evolution is governed by a time-independent generator $G$:
\begin{equation}
\partial_\tau \Theta = G \Theta
\label{eq:linear-flow}
\end{equation}
where $G$ is a fixed $n\times n$ complex matrix (generally non-Hermitian).

\subsubsection{UBT Invariants}

The theory defines two primary channels:

\textbf{Entropy Channel}:
\begin{equation}
S_\Theta(\tau) = k_B \log \det(\Theta^\dagger(\tau) \Theta(\tau)) = 2k_B \mathrm{Re}[\log \det \Theta(\tau)]
\label{eq:entropy-channel}
\end{equation}

\textbf{Phase Channel}:
\begin{equation}
\Sigma_\Theta(\tau) = k_B \arg \det \Theta(\tau) = k_B \mathrm{Im}[\log \det \Theta(\tau)]
\label{eq:phase-channel}
\end{equation}
where $k_B$ is Boltzmann's constant (set to 1 in natural units for this derivation).

\subsection{Linear Complex-$\tau$ Flow}
\label{sec:linear-flow}

\subsubsection{Formal Solution}

The linear flow equation \eqref{eq:linear-flow} has the formal solution:

\begin{proposition}[Exponential Flow]
\label{prop:exponential-flow}
\begin{equation}
\Theta(\tau) = \exp(\tau G) \Theta_0
\end{equation}
where $\Theta_0 = \Theta(0)$ is the initial condition and $\exp(\tau G)$ is the matrix exponential.
\end{proposition}

\subsubsection{Decomposition of Generator}

Write the generator $G$ in terms of its real and imaginary parts:
\begin{equation}
G = A + iB
\end{equation}
where $A = \mathrm{Re}(G)$ and $B = \mathrm{Im}(G)$ are real $n\times n$ matrices.

\begin{lemma}[Complex-Time Substitution]
\label{lem:complex-time-sub}
Substituting $\tau = t + i\psi$:
\begin{equation}
\tau G = (t + i\psi)(A + iB) = (tA - \psi B) + i(tB + \psi A)
\end{equation}
\end{lemma}

This shows that complex time $\tau$ couples the real and imaginary parts of $G$ in a non-trivial way:
\begin{itemize}
\item Real part of $\tau G$: $tA - \psi B$ (combines real-time evolution with imaginary generator)
\item Imaginary part of $\tau G$: $tB + \psi A$ (combines imaginary-time evolution with real generator)
\end{itemize}

\subsubsection{Commutator Analysis}

\begin{theorem}[Decomposition Conditions]
\label{thm:decomposition}
The matrix exponential simplifies if and only if $[A, B] = 0$:

\textbf{Case 1} (Commuting): If $[A, B] = 0$, then:
\begin{equation}
\exp(\tau G) = \exp(tA - \psi B) \exp(i(tB + \psi A))
\end{equation}

\textbf{Case 2} (Non-Commuting): If $[A, B] \neq 0$, use the Baker-Campbell-Hausdorff (BCH) formula:
\begin{equation}
\exp(\tau G) = \exp(tA - \psi B) \exp(i(tB + \psi A)) \exp(C_{\text{BCH}})
\end{equation}
where to first order:
\begin{equation}
C_{\text{BCH}} \approx \frac{1}{2}[tA - \psi B, i(tB + \psi A)] + O([[[A,B],A],B])
\end{equation}
\end{theorem}

Expanding the commutator explicitly:
\begin{align}
[tA - \psi B, i(tB + \psi A)] &= i[tA - \psi B, tB + \psi A] \nonumber \\
&= i(t^2[A,B] + t\psi[A,A] - t\psi[B,B] - \psi^2[B,A])
\end{align}

Using the identities $[A,A] = 0$, $[B,B] = 0$, and $[B,A] = -[A,B]$:
\begin{align}
[tA - \psi B, i(tB + \psi A)] &= i(t^2[A,B] + 0 - 0 - \psi^2(-[A,B])) \nonumber \\
&= i(t^2[A,B] + \psi^2[A,B]) \nonumber \\
&= i(t^2 + \psi^2)[A,B]
\end{align}

Therefore:
\begin{equation}
C_{\text{BCH}} \approx \frac{i}{2}(t^2 + \psi^2)[A,B] + O([[[A,B],A],B])
\end{equation}

\textbf{Physical Interpretation}: Non-commutativity introduces additional phase factors that depend on both $t$ and $\psi$ quadratically, with the phase correction growing as $(t^2 + \psi^2)$. For small $\psi$, the correction is order $O(\psi^2)$. The positive sign indicates that both real and imaginary time components contribute additively to the non-commutative phase.

\textbf{Remark on Higher-Order Terms}: The BCH formula contains higher-order nested commutators such as $[[A,B],A]$ and $[[A,B],B]$. The leading first-order approximation is valid when $||A||, ||B||$ are small or over short complex time intervals where $||\tau G|| \ll 1$.

\subsection{Spectral Decomposition and Eigenmode Evolution}
\label{sec:spectral-decomposition}

\subsubsection{Eigendecomposition}

Assume $G$ is diagonalizable:
\begin{equation}
G v_k = \mu_k v_k, \quad k = 1, \ldots, n
\end{equation}
where:
\begin{itemize}
\item $\mu_k = \alpha_k + i\omega_k$: Complex eigenvalues ($\alpha_k, \omega_k \in \mathbb{R}$)
\item $v_k$: Right eigenvectors
\item $w_k$: Left eigenvectors (bi-orthogonal: $w_k^\dagger v_j = \delta_{kj}$)
\end{itemize}

\begin{proposition}[Spectral Expansion]
\label{prop:spectral-expansion}
\begin{equation}
\Theta(\tau) = \exp(\tau G) \Theta_0 = \sum_k \exp(\tau \mu_k) v_k w_k^\dagger \Theta_0
\end{equation}
\end{proposition}

Each eigenmode $k$ evolves independently with amplitude factor $\exp(\tau \mu_k)$.

\subsection{Complex $\tau$ Effects: Growth/Decay + Oscillation}
\label{sec:complex-tau-effects}

\subsubsection{Single Eigenmode Evolution}

Consider a single eigenmode with eigenvalue $\mu = \alpha + i\omega$:
\begin{align}
\exp(\tau\mu) &= \exp((t + i\psi)(\alpha + i\omega)) \nonumber \\
              &= \exp((t\alpha - \psi\omega) + i(t\omega + \psi\alpha)) \nonumber \\
              &= \exp(t\alpha - \psi\omega) \cdot \exp(i(t\omega + \psi\alpha))
\end{align}

\begin{theorem}[Mode Amplitude Factor]
\label{thm:mode-amplitude}
The mode amplitude factor decomposes as:
\begin{align}
|\exp(\tau\mu)| &= \exp(t\alpha - \psi\omega) \\
\arg(\exp(\tau\mu)) &= t\omega + \psi\alpha
\end{align}
\end{theorem}

\textbf{Physical Interpretation}:
\begin{enumerate}
\item \textbf{Growth/Decay} (Amplitude): $|\exp(\tau\mu)| = \exp(t\alpha - \psi\omega)$
   \begin{itemize}
   \item Real time $t$ couples to $\mathrm{Re}(\mu) = \alpha$ (standard exponential growth/decay)
   \item Imaginary time $\psi$ couples to $\mathrm{Im}(\mu) = \omega$ (modulates amplitude!)
   \item A non-zero $\psi$ can turn decay into growth (if $\omega < 0$) or vice versa
   \end{itemize}

\item \textbf{Oscillation} (Phase): $\arg(\exp(\tau\mu)) = t\omega + \psi\alpha$
   \begin{itemize}
   \item Real time $t$ couples to $\mathrm{Im}(\mu) = \omega$ (standard oscillation)
   \item Imaginary time $\psi$ couples to $\mathrm{Re}(\mu) = \alpha$ (phase shift!)
   \item Growth rate $\alpha$ now contributes to phase accumulation
   \end{itemize}
\end{enumerate}

\begin{corollary}[Cross-Coupling Principle]
\label{cor:cross-coupling}
Complex time $\tau$ introduces a fundamental cross-coupling:
\begin{itemize}
\item What is ``growth'' in real time becomes ``oscillation'' in imaginary time
\item What is ``oscillation'' in real time becomes ``growth'' in imaginary time
\end{itemize}
This is the spectral signature of complex chronofactor evolution.
\end{corollary}

\subsection{Evolution of Observable Matrix $O(\tau) = \Theta^\dagger\Theta$}
\label{sec:observable-evolution}

Define the observable matrix:
\begin{equation}
O(t, \psi) := \Theta^\dagger(t + i\psi) \Theta(t + i\psi)
\end{equation}

\begin{theorem}[Real-Time Evolution of $O$]
\label{thm:O-evolution}
Taking $\partial_t$ derivative with $\psi$ fixed:
\begin{equation}
\partial_t O = \Theta^\dagger(G^\dagger + G)\Theta = 2\Theta^\dagger\mathrm{Herm}(G)\Theta
\end{equation}
where $\mathrm{Herm}(G) := (G + G^\dagger)/2$ is the Hermitian part of $G$.
\end{theorem}

\begin{corollary}[Conservation Condition]
\label{cor:conservation}
If $\mathrm{Herm}(G) = 0$ (i.e., $G$ is anti-Hermitian), then $\partial_t O = 0$ and $O$ is conserved on the real-time slice.
\end{corollary}

\textbf{Physical Interpretation}: Anti-Hermitian generators preserve the norm. This is the spectral manifestation of unitary evolution.

\subsubsection{Imaginary-Time Dependence}

If $\psi$ varies with $t$, i.e., $\psi = \psi(t)$, then:
\begin{equation}
\frac{dO}{dt} = \partial_t O + (\partial_\psi O)\frac{d\psi}{dt}
\end{equation}

Computing $\partial_\psi O$:
\begin{equation}
\partial_\psi O = 2i\Theta^\dagger\mathrm{AntiHerm}(G)\Theta
\end{equation}
where $\mathrm{AntiHerm}(G) := (G - G^\dagger)/(2i)$ is the anti-Hermitian part.

\begin{theorem}[Full Time Derivative with Variable $\psi$]
\label{thm:full-derivative}
\begin{equation}
\frac{dO}{dt} = 2\Theta^\dagger\left[\mathrm{Herm}(G) + i\frac{d\psi}{dt}\mathrm{AntiHerm}(G)\right]\Theta
\end{equation}
\end{theorem}

\textbf{Physical Consequence}: If $\psi(t)$ varies, the effective generator becomes:
\begin{equation}
G_{\text{eff}} = \mathrm{Herm}(G) + i\frac{d\psi}{dt}\mathrm{AntiHerm}(G)
\end{equation}
This can cause spectral broadening even if $G$ itself is anti-Hermitian!

\subsection{Determinant Channel and Connection to $S_\Theta$, $\Sigma_\Theta$}
\label{sec:determinant-channel}

\begin{theorem}[Determinant Identity for Linear Flow]
\label{thm:det-identity}
For linear flow $\partial_\tau \Theta = G\Theta$:
\begin{equation}
\det \Theta(\tau) = \det(\exp(\tau G)) \det \Theta_0 = \exp(\tau \mathrm{Tr}\, G) \det \Theta_0
\end{equation}
Taking logarithm:
\begin{equation}
\log \det \Theta(\tau) = \tau \mathrm{Tr}\, G + \log \det \Theta_0
\label{eq:log-det-identity}
\end{equation}
\end{theorem}

\subsubsection{Real and Imaginary Parts}

Substituting $\tau = t + i\psi$ and $\mathrm{Tr}\, G = \mathrm{Tr}\, A + i \mathrm{Tr}\, B$:
\begin{align}
\log \det \Theta(\tau) &= (t + i\psi)(\mathrm{Tr}\, A + i \mathrm{Tr}\, B) + \log \det \Theta_0 \nonumber \\
&= (t \mathrm{Tr}\, A - \psi \mathrm{Tr}\, B) + i(t \mathrm{Tr}\, B + \psi \mathrm{Tr}\, A) + \log \det \Theta_0
\end{align}

\begin{proposition}[Re/Im Decomposition]
\label{prop:re-im-decomposition}
\begin{align}
\mathrm{Re}[\log \det \Theta(\tau)] &= t \mathrm{Tr}\, A - \psi \mathrm{Tr}\, B + \mathrm{Re}[\log \det \Theta_0] \\
\mathrm{Im}[\log \det \Theta(\tau)] &= t \mathrm{Tr}\, B + \psi \mathrm{Tr}\, A + \mathrm{Im}[\log \det \Theta_0]
\end{align}
\end{proposition}

\begin{theorem}[UBT Invariant Expressions]
\label{thm:ubt-invariants}
\textbf{Entropy Channel}:
\begin{equation}
S_\Theta(t, \psi) = 2k_B(t \mathrm{Tr}\, A - \psi \mathrm{Tr}\, B + \mathrm{Re}[\log \det \Theta_0])
\label{eq:entropy-explicit}
\end{equation}

\textbf{Phase Channel}:
\begin{equation}
\Sigma_\Theta(t, \psi) = k_B(t \mathrm{Tr}\, B + \psi \mathrm{Tr}\, A + \mathrm{Im}[\log \det \Theta_0])
\label{eq:phase-explicit}
\end{equation}
\end{theorem}

\textbf{Key Observation}: The cross-coupling appears explicitly:
\begin{itemize}
\item $S_\Theta$: Real time couples to $\mathrm{Tr}\, A$, imaginary time couples to $\mathrm{Tr}\, B$
\item $\Sigma_\Theta$: Real time couples to $\mathrm{Tr}\, B$, imaginary time couples to $\mathrm{Tr}\, A$
\end{itemize}

\begin{corollary}[Evolution Rates]
\label{cor:evolution-rates}
\begin{align}
\partial_t S_\Theta &= 2k_B \mathrm{Tr}\, A, \quad \partial_\psi S_\Theta = -2k_B \mathrm{Tr}\, B \\
\partial_t \Sigma_\Theta &= k_B \mathrm{Tr}\, B, \quad \partial_\psi \Sigma_\Theta = k_B \mathrm{Tr}\, A
\end{align}
\end{corollary}

\subsection{Phase Channel as Holonomy}
\label{sec:phase-holonomy}

The phase channel $\Sigma_\Theta = k_B \arg \det \Theta$ can be interpreted as a holonomy or winding number in the complex-$\tau$ plane.

\begin{definition}[Phase Holonomy]
\label{def:phase-holonomy}
For a closed loop $\gamma$ in the complex-$\tau$ plane:
\begin{equation}
\Delta\Sigma_\Theta[\gamma] = k_B \oint_\gamma d\tau\, \mathrm{Tr}(\Theta^{-1} \partial_\tau \Theta) = k_B \left(\mathrm{Tr}\, G\right) \oint_\gamma d\tau
\end{equation}
\end{definition}

\begin{proposition}[Imaginary-Time Winding]
\label{prop:im-time-winding}
Consider a loop in the $\psi$-direction at fixed $t$: $\psi \in [0, \Delta\psi]$:
\begin{equation}
\Delta\Sigma_\Theta = k_B (i\Delta\psi) (\mathrm{Tr}\, A + i \mathrm{Tr}\, B) = -k_B \Delta\psi\, \mathrm{Tr}\, B + ik_B \Delta\psi\, \mathrm{Tr}\, A
\end{equation}
The real winding is:
\begin{equation}
\mathrm{Re}[\Delta\Sigma_\Theta] = -k_B \Delta\psi\, \mathrm{Tr}\, B
\end{equation}
\end{proposition}

\textbf{Physical Meaning}: The phase accumulation under imaginary-time translation is controlled by $\mathrm{Tr}\, B$, the imaginary part of the generator trace. This structure is analogous to Berry phase in quantum mechanics.

\subsection{Non-Hermitian Case and Pseudospectrum}
\label{sec:non-hermitian}

\subsubsection{Non-Normal Matrices}

A matrix $G$ is \textbf{normal} if $[G, G^\dagger] = 0$. For non-normal $G$:
\begin{itemize}
\item Eigenvalues can be highly sensitive to perturbations
\item Eigenvectors are not orthogonal
\item Transient growth can occur even if all $\mathrm{Re}(\mu_k) < 0$
\end{itemize}

\subsubsection{Pseudospectrum}

\begin{definition}[$\epsilon$-Pseudospectrum]
\label{def:pseudospectrum}
The $\epsilon$-pseudospectrum of $G$ is:
\begin{equation}
\sigma_\epsilon(G) = \{z \in \mathbb{C} : ||(zI - G)^{-1}|| \geq 1/\epsilon\}
\end{equation}
\end{definition}

\textbf{Physical Significance}: Even if eigenvalues suggest stability ($\mathrm{Re}(\mu_k) < 0$), the pseudospectrum can extend into $\mathrm{Re}(z) > 0$, indicating potential for transient growth.

\begin{theorem}[Transient Amplification]
\label{thm:transient-amplification}
For non-normal $G$, there exists $t_{\max}$ such that:
\begin{equation}
||\Theta(t_{\max})|| > ||\Theta_0|| \exp\left(\max_k \mathrm{Re}(\mu_k) \cdot t_{\max}\right)
\end{equation}
\end{theorem}

This is transient growth: temporary amplification beyond what eigenvalues predict.

\subsection{Diagnostic Invariants and Discriminators}
\label{sec:discriminators}

We define three testable discriminators for distinguishing real-$\tau$ from complex-$\tau$ evolution.

\subsubsection{Discriminator D1: Phase-Entropy Coupling Coefficient}

\begin{definition}[Coupling Coefficient]
\label{def:coupling-coeff}
\begin{equation}
C_{\Sigma S}(t) := \frac{\partial_t \Sigma_\Theta}{\partial_t S_\Theta} = \frac{k_B \mathrm{Tr}\, B}{2k_B \mathrm{Tr}\, A} = \frac{\mathrm{Tr}\, B}{2 \mathrm{Tr}\, A}
\end{equation}
\end{definition}

\textbf{Test}:
\begin{itemize}
\item \textbf{Real $\tau$ ($\psi = 0$)}: $C_{\Sigma S}$ is constant, determined solely by $G$
\item \textbf{Complex $\tau$ ($\psi \neq 0$)}: $C_{\Sigma S}$ can vary if $\psi(t)$ changes or if $G$ depends on $\psi$
\end{itemize}

\textbf{Prediction}: Observe time-varying $C_{\Sigma S}(t)$ as a signature of complex-time evolution.

\subsubsection{Discriminator D2: Conservation Test}

\begin{definition}[Spectral Drift]
\label{def:spectral-drift}
Define the eigenvalue variance of $O(t)$:
\begin{equation}
\mathrm{Var}_O(t) := \frac{\mathrm{Tr}(O^2(t))}{(\mathrm{Tr}\, O(t))^2} - \frac{1}{n}
\end{equation}
\end{definition}

\textbf{Test}:
\begin{itemize}
\item \textbf{Anti-Hermitian $G$ with real $\tau$}: $\mathrm{Var}_O(t) = $ constant
\item \textbf{Complex $\tau$ with $\mathrm{Herm}(G) \neq 0$}: $\mathrm{Var}_O(t)$ drifts
\end{itemize}

\textbf{Prediction}: Monitor $\mathrm{Var}_O(t)$ for unexpected drift:
\begin{equation}
\frac{d(\mathrm{Var}_O)}{dt} \neq 0 \implies \mathrm{Herm}(G) \neq 0 \text{ or complex-}\tau \text{ effects}
\end{equation}

\subsubsection{Discriminator D3: Mode Pairing and Oscillatory Signatures}

\begin{definition}[Conjugate Mode Correlation]
\label{def:mode-correlation}
For complex eigenvalue pair $\mu = \alpha \pm i\omega$, the phase channel exhibits:
\begin{equation}
\Sigma_\Theta(t) \propto t\cdot\omega \quad \text{(oscillatory component)}
\end{equation}
while the entropy channel varies as:
\begin{equation}
S_\Theta(t) \propto t\cdot\alpha \quad \text{(monotonic growth/decay)}
\end{equation}
\end{definition}

\textbf{Test}:
\begin{itemize}
\item \textbf{Real $\tau$}: $\Sigma_\Theta$ and $S_\Theta$ are uncorrelated
\item \textbf{Complex $\tau$}: Oscillations in $\Sigma_\Theta$ correlate with features in $S_\Theta$
\end{itemize}

\textbf{Measurement Protocol}:
\begin{equation}
\text{FFT}[\Sigma_\Theta(t)] \to \text{extract } \omega_{\text{obs}}, \quad \text{FFT}[S_\Theta(t)] \to \text{check for features at } \omega_{\text{obs}}
\end{equation}

\subsubsection{Summary of Discriminators}

\begin{table}[h]
\centering
\begin{tabular}{|l|l|l|l|}
\hline
\textbf{Discriminator} & \textbf{Observable} & \textbf{Real $\tau$ Behavior} & \textbf{Complex $\tau$ Signature} \\
\hline
D1: $C_{\Sigma S}$ & $\partial_t \Sigma_\Theta / \partial_t S_\Theta$ & Constant & Time-varying \\
\hline
D2: $\mathrm{Var}_O$ & $\mathrm{Tr}(O^2)/(\mathrm{Tr}\, O)^2$ & Conserved (if $G$ anti-Herm) & Drifts \\
\hline
D3: Mode Pairing & $\text{FFT}[\Sigma_\Theta]$ vs $\text{FFT}[S_\Theta]$ & Uncorrelated & Cross-correlation at $\omega_k$ \\
\hline
\end{tabular}
\caption{Summary of three discriminators for real vs. complex chronofactor evolution}
\label{tab:discriminators}
\end{table}

\subsection{Appendix A: $2\times 2$ Worked Example}
\label{sec:appendix-2x2}

Consider a $2\times 2$ generator:
\begin{equation}
G = \begin{pmatrix} \alpha_1 & i\omega \\ -i\omega & \alpha_2 \end{pmatrix}
\end{equation}
where $\alpha_1, \alpha_2, \omega \in \mathbb{R}$.

The characteristic polynomial is:
\begin{equation}
\det(G - \lambda I) = (\alpha_1 - \lambda)(\alpha_2 - \lambda) + \omega^2 = 0
\end{equation}

Eigenvalues:
\begin{equation}
\lambda_\pm = \frac{\alpha_1 + \alpha_2}{2} \pm \sqrt{\frac{(\alpha_1 - \alpha_2)^2}{4} - \omega^2}
\end{equation}

For complex-time evolution $\tau = t + i\psi$:
\begin{equation}
\det \Theta(\tau) = \exp(\tau(\alpha_1 + \alpha_2)) \det \Theta_0
\end{equation}

Thus:
\begin{align}
S_\Theta(\tau) &= 2k_B t(\alpha_1 + \alpha_2) \\
\Sigma_\Theta(\tau) &= k_B \psi(\alpha_1 + \alpha_2)
\end{align}

For an eigenvector $v_+$ with eigenvalue $\lambda_+ = \alpha_+ + i\omega_+$:
\begin{align}
|\exp(\tau\lambda_+)| &= \exp(t\alpha_+ - \psi\omega_+) \\
\arg(\exp(\tau\lambda_+)) &= t\omega_+ + \psi\alpha_+
\end{align}

This explicitly shows the cross-coupling: $\psi$ modulates the amplitude through $\omega_+$.

\subsection{Appendix B: Heat-Kernel Representation}
\label{sec:appendix-heat-kernel}

The heat kernel representation relates the determinant to the trace of the evolution operator:
\begin{equation}
\log \det \Theta = \mathrm{Tr}\, \log \Theta
\end{equation}

For the flow $\Theta(\tau) = \exp(\tau G)\Theta_0$:
\begin{equation}
\log \det \Theta(\tau) = \mathrm{Tr}[\log(\exp(\tau G)\Theta_0)] = \mathrm{Tr}[\tau G + \log \Theta_0] = \tau\, \mathrm{Tr}\, G + \mathrm{Tr}\, \log \Theta_0
\end{equation}

This reproduces Theorem~\ref{thm:det-identity}. The entropy channel connects to the trace logarithm:
\begin{equation}
S_\Theta = k_B \log \det(\Theta^\dagger\Theta) = k_B \mathrm{Tr}\, \log(\Theta^\dagger\Theta) = 2k_B \mathrm{Re}[\mathrm{Tr}\, \log \Theta]
\end{equation}

This is a standard quantity in quantum field theory.

\subsection{Appendix C: Regularization Near $\det \to 0$}
\label{sec:appendix-regularization}

When $\det \Theta \to 0$, the matrix becomes singular. As $\det \Theta \to 0$:
\begin{equation}
\log \det \Theta \to -\infty
\end{equation}

This divergence represents:
\begin{enumerate}
\item \textbf{Entropy collapse}: $S_\Theta \to -\infty$ (infinite negative entropy)
\item \textbf{Information loss}: The system approaches a zero-volume configuration
\item \textbf{Phase singularity}: $\arg \det \Theta$ becomes ill-defined
\end{enumerate}

\textbf{Regularization Strategy}: Add a small regularization:
\begin{equation}
\Theta_{\text{reg}} = \Theta + \epsilon I
\end{equation}
Then:
\begin{equation}
\log \det \Theta_{\text{reg}} = \log \det(\Theta + \epsilon I) \approx \log \det \Theta + \epsilon\, \mathrm{Tr}(\Theta^{-1}) + O(\epsilon^2)
\end{equation}

The divergence $\log \det \Theta \to -\infty$ is not a pathology but a physical statement encoding critical information about singularities in the field configuration.

\subsection{Summary}
\label{sec:summary}

\textbf{Main Results}:
\begin{enumerate}
\item Complex-$\tau$ couples growth and oscillation at the eigenmode level: $\exp(\tau\mu_k) = \exp(t\alpha_k - \psi\omega_k) \exp(i(t\omega_k + \psi\alpha_k))$
\item Determinant identity: $\log \det \Theta(\tau) = \tau \mathrm{Tr}\, G + \log \det \Theta_0$ provides exact expressions for $S_\Theta$ and $\Sigma_\Theta$
\item Three discriminators (D1-D3) provide testable signatures for real vs. complex-time evolution
\item Non-Hermitian pseudospectrum explains sudden spectral broadening
\end{enumerate}

% © 2025 Ing. David Jaroš — CC BY-NC-ND 4.0
