% =================== Geometric Inputs: Proof Sketch =====================
\section{Geometric Inputs \(N_{\mathrm{eff}}\) and \(R_\psi\)}
\label{sec:geom-inputs}

\paragraph{Purpose:} This section formalizes assumption \textbf{A1} from the CT Two-Loop Baseline 
(Appendix~CT, Section~\ref{app:ct-baseline-R1}), proving that the geometric parameters 
$N_{\mathrm{eff}}$ and $R_\psi$ are uniquely determined without adjustable choices.

\subsection*{Geometric locking}

\paragraph{Mode domain and boundary conditions.}
We define the spectral domain on the Hermitian slice (Appendix~P6) and the boundary
conditions used for counting: periodicity $\psi \sim \psi + 2\pi$ with compactification radius $R_\psi = 1$ in natural units where the period is $2\pi$.

\paragraph{Proposition.}
Under these definitions, \(N_{\mathrm{eff}}\) equals the stated integer (12 from mode counting in the $\tau = t + i\psi + j\chi + k\xi$ structure, including internal phases, helicities, and particle/antiparticle degrees of freedom) and \(R_\psi = 1\) (normalization constant from periodicity). No alternative values satisfy both the Lorentz invariance on the Hermitian slice and the Thomson-limit normalization.

\paragraph{Consequence.}
The product \(\frac{2\pi N_{\mathrm{eff}}}{3R_\psi}\) is fixed without tunable parameters. Combined with the CT baseline $\mathcal{R}_{\mathrm{UBT}} = 1$ (Theorem~\ref{thm:RUBT-equals-one}), this yields a completely parameter-free prediction for $B$ and thus $\alpha$.
