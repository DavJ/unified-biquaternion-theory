% === AUTO-INSERT V PARAM FIX BEGIN ===
\subsection*{V.x Parameter fixing for the internal torus}

\paragraph{Discrete choices (no tuning).}
To make all fermion-sector predictions testable and free of continuous tuning, we \emph{fix} the torus data by discrete assumptions:
\begin{itemize}
  \item \textbf{Modulus branch:} we restrict to modular fixed points (or their $\mathrm{SL}_2(\mathbb{Z})$ images). The minimal choices are the square and hexagonal branches,
  \[ \tau \in \Big\{\, i,\ e^{i\pi/3} \,\Big\}, \]
  with the working baseline $\tau=i$ unless stated otherwise.\footnote{This is consistent with the lepton-spectrum derivation in App.~W and keeps the ansatz strictly discrete.}
  \item \textbf{Spin structure / Wilson branch:} we choose the Hosotani/Wilson background on one fundamental cycle
  \[ \theta_{H}=\pi, \qquad \text{NS shift } \delta=\tfrac{1}{2} \text{ on the Wilson cycle}, \]
  and the remaining cycle periodic. Equivalently, holonomy representatives $a_i \in \{0,\tfrac12\}$.
  \item \textbf{Scale:} the overall radius is left to App.~K.5 where $R$ is fixed \emph{once} by the two-loop $\Lambda_{\mathrm{QCD}}$ relation with a discrete threshold factor $\Theta_{\rm th}(\tau,h)$.
\end{itemize}

\paragraph{Operator.} Internal eigenvalues are computed from the Dirac operator on $\mathbb{T}^2(\tau)$ with the above discrete spin/holonomy data (Appendix~W), not from a Euclidean placeholder. This convention is used throughout QA (quarks \& CKM) and K.5 ($\Lambda_{\mathrm{QCD}}$).

\paragraph{Rationale.} These choices pin the geometry to finitely many branches; all remaining freedom is integer mode data and discrete holonomies. No continuous parameters are introduced.
% === AUTO-INSERT V PARAM FIX END ===
