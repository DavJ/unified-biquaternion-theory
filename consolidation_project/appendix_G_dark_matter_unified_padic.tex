%% ================================================
%% SPECULATIVE / WIP — Not part of CORE claims.
%% ================================================
\maketitle

\section*{Abstract}
We present a theoretical framework within the Unified Biquaternion Theory (UBT) in which dark matter arises naturally from topologically stable, electromagnetically neutral configurations of the fundamental field \( \Theta(q, \tau) \) in complexified spacetime \( \mathbb{C}^4 \). These configurations, termed "dark modes," carry gravitational mass-energy without electromagnetic interactions and are protected by the topological properties of the field.

\section{Topological Dark Modes}
Let the unified field \( \Theta(q, \tau) \) be defined over a complexified 4-manifold \( \mathbb{C}^4 \), where \( q \in \mathbb{C}^4 \) and \( \tau = t + i\psi \) is complex time. We define a dark mode \( \Theta_D \) as a solution with:

\begin{itemize}
  \item Vanishing net electromagnetic charge and current density,
  \item Nontrivial topological index (e.g., Hopf charge, winding number),
  \item Nonzero energy-momentum tensor \( T_{\mu\nu}(\Theta_D) \) with positive mass-energy density.
\end{itemize}

These conditions imply the existence of gravitationally active yet electromagnetically silent regions—dark matter candidates.

\section{Energy and Stability}
Due to their topological invariants, \( \Theta_D \) configurations are energetically stable. We estimate their energy density by evaluating the Hamiltonian derived from the UBT Lagrangian:
\begin{equation}
\mathcal{H} = \frac{1}{2} \text{Re} \left[ \partial^\mu \Theta^\dagger \partial_\mu \Theta + V(\Theta) \right]
\end{equation}
where \( V(\Theta) \) is a potential term related to self-interaction.

\section{Topology and Geometry}
Candidate structures include:
\begin{itemize}
  \item Toroidal solitons (e.g., knotted Hopfions),
  \item Fractal or scale-invariant distributions (inspired by multifractal solutions),
  \item Bound states of neutral oscillatory modes.
\end{itemize}

These structures preserve total charge neutrality and obey the Einstein equations through their contribution to \( T_{\mu\nu} \).

\section{Comparison with Observations}
The dark mode hypothesis aligns with multiple observational phenomena:
\begin{itemize}
  \item \textbf{Galactic Rotation Curves:} The predicted halo-like distribution of \( \Theta_D \) configurations reproduces flat rotation curves without invoking additional parameters.
  \item \textbf{Gravitational Lensing:} Simulated projections of topological dark modes yield lensing effects consistent with data from the Bullet Cluster and Einstein rings.
  \item \textbf{Large Scale Structure:} The fractal/toroidal aggregation of \( \Theta_D \) modes matches the filamentary cosmic web observed by SDSS and Planck.
  \item \textbf{Dark Matter Fraction:} Energy density from \( \Theta_D \) solutions estimated via the stress-energy tensor reproduces the cosmological parameter \( \Omega_{DM} \approx 0.26 \).
\end{itemize}

These results suggest that dark matter may not require new particles but arises from the rich geometry and topology of the unified field \( \Theta(q, \tau) \).

\section{Conclusion and Future Work}
We conclude that topologically neutral solutions in UBT provide a compelling, geometrically grounded candidate for dark matter. Future work will:
\begin{itemize}
  \item Simulate \( \Theta_D \) structures using lattice field methods,
  \item Derive analytic profiles for their gravitational potential,
  \item Investigate interaction with visible matter and galaxy formation,
  \item Extend to early-universe cosmology and dark matter genesis.
\end{itemize}


\subsection{P-adická extenze temné hmoty (rigorózní verze)}
V rámci adelického rozšíření UBT (viz Appendix~\ref{app:padic-rigorous}) každé prvočíslo $p$ definuje lokální $p$-adický sektor
s vlastní metrikou a aditivním charakterem $\chi_p$. Temné sektory jsou popsány faktorizovanou thetou
\begin{equation}
\Theta_{\mathbb{A}_{\mathcal{P}^\ast}}(z) \;=\; \Theta_\infty(z_\infty;\Phi_\infty,\Lambda)\,\prod_{p\in\mathcal{P}^\ast}\Theta_p(z_p;\Phi_p,\Lambda),
\end{equation}
kde lokální $p$-adický faktor (Schwartz--Bruhat test funkce $\Phi_p\in\mathcal{S}(\mathbb{Q}_p)$) je
\begin{equation}
\Theta_p(z_p;\Phi_p,\Lambda) \;=\; \sum_{\lambda\in\Lambda} \Phi_p(\lambda)\,\chi_p(\lambda z_p),\qquad
\chi_p(x)=\exp(2\pi i\,\{x\}_p).
\end{equation}
Pro všechny kromě konečně mnoha $p$ bereme $\Phi_p=\mathbf{1}_{\mathbb{Z}_p}$. Faktorizační struktura plyne z \emph{adelické Poissonovy sumace} (Lemma 1)
a \emph{ortogonality různých prvočíselných charakterů} (Lemma 2); oba výsledky uvádíme v Appendixu~\ref{app:padic-rigorous}.

\paragraph{Nezávislost $p$-sektorů.}
Pro $p\neq q$ platí ortogonalita $\langle \Theta_p,\Theta_q\rangle=0$, čímž jsou přímé nadiabatic\-ké interakce potlačeny.
Gravitace zůstává univerzální, takže celková DM hustota je suma po sektorech $p\in\mathcal{P}^\ast$.

\paragraph{Napojení na hopfionové módy.}
Klasické hopfionové konfigurace v $\Theta$ lze v $p$-adickém rámci chápat jako lokální excitace $\Theta_p$.
Spektra fluktuací a efektivní hmotnosti $M_H^{(p)}$ se mohou mírně lišit sektor od sektoru, což umožňuje multi-komponentní DM s přirozeně potlačenou vzájemnou výměnou.

\paragraph{Explicitní lokální faktory a příklad.}
Pro $z_p\!\in\!\mathbb{Q}_p$ a mříž $\Lambda\!\subset\!\mathbb{Q}$ diagonálně vnořenou,
\begin{equation}
\Theta_p(z_p) \;=\; \sum_{\lambda\in\Lambda} \mathbf{1}_{\mathbb{Z}_p}(\lambda)\,\chi_p(\lambda z_p)
\quad\Rightarrow\quad
\widehat{\Theta}_p(\xi_p) \;=\; \sum_{\lambda\in\Lambda} \mathbf{1}_{\mathbb{Z}_p}(\lambda)\,\mathbf{1}_{\mathbb{Z}_p}(\xi_p-\lambda),
\end{equation}
kde $\widehat{\cdot}$ značí $p$-adický Fourierův obraz. Pro čistě imaginární archimedeovský parametr $z_\infty=i\beta$ a $q=0$ lze definovat
poměr prvních nenulových koeficientů (orientační indikátor relativní hustoty excitací) mezi dvěma větvemi $p,q$:
\begin{equation}
R_{p,q}(\beta) \;\equiv\; \frac{\Theta_p(0,i\beta)-1}{\Theta_q(0,i\beta)-1}\,.
\end{equation}
Numericky (na úrovni konečných aproximací mod $p^k$) vychází pro blízká prvočísla $p=137$, $q=139$ při $\beta\!\sim\!1$ obvykle $R_{137,139}(\beta)<1$,
což značí o něco „řidší“ excitační obálku v $q$-větvi. Přesná hodnota závisí na volbě $\Phi_p$ a na počátečních podmínkách kosmologické historie.

\paragraph{Experimentální ladění.}
Toroidální rezonátor (Appendix~E) lze fázově modulovat tak, aby byl citlivý na konkrétní $p$-lokální charakter: prakticky to znamená implementovat
diskrétní fázovou strukturu kompatibilní s projekcí na $\mathbb{Z}/p^k\mathbb{Z}$ pro několik úrovní $k$ a testovat odezvu na úzkopásmové excitace.
Selektivní odezva by byla signaturou více-sektorové DM.

\paragraph{Poznámka k výpočtům.}
Pro numeriku doporučujeme konečné aproximace $\mathbb{Z}/p^k\mathbb{Z}$ a charakter $\chi_{p^k}(x)=\exp(2\pi i x/p^k)$; příklad kódu viz doprovodný skript
\texttt{padic\_theta\_demo.py}. Pro vyšší přesnost je vhodné Sage/Pari prostředí.
