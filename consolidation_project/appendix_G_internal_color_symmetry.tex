% =====================================================================
% Appendix G: Internal Color Symmetry as a Modular Subgroup of \Theta
% Status: theoretical derivation (core-compatible), speculative notes marked
% =====================================================================

\appendix
\section*{Appendix G \\ Internal Color Symmetry as a Modular Subgroup of \texorpdfstring{$\Theta$}{Theta}}
\addcontentsline{toc}{section}{Appendix G: Internal Color Symmetry as a Modular Subgroup of $\Theta$}

\subsection*{G.0 Overview (Core-Compatible, Non-Disruptive)}
This appendix derives QCD color symmetry $\mathrm{SU}(3)_{\mathrm{color}}$ as an \emph{internal modular automorphism} of the $\Theta$-field phase manifold, without introducing an external gauge stack. The construction preserves UBT core principles:
(i) biquaternionic base for spacetime/kinematics, 
(ii) complex time $\tau=t+i\psi$, 
(iii) metric from $\mathrm{Re}(\Theta^\dagger \Theta)$, 
(iv) gauge/phase data encoded in the holomorphic structure of $\Theta$.
Color interactions arise from \emph{multi-dimensional phase degrees of freedom} of $\Theta$; Yang–Mills variables appear as phase connections on a rank-3 internal bundle. GR limit and QED/weak structure remain unchanged.

\subsection*{G.1 Theta Field with Multi-Phase Structure}
Let $\Theta:\; \mathcal{M}\times \mathbb{T}_\psi \to \mathbb{B}\otimes \mathbb{C}$ be the biquaternionic field on spacetime $\mathcal{M}$ with complex time $\tau=t+i\psi$, 
and let $\mathcal{F}$ denote its internal phase manifold. 
We promote the scalar phase to a \emph{matrix phase} by writing
\begin{equation}
\label{eq:G1_theta_factorization}
\Theta(x,\tau)\;=\; \Xi(x,\tau)\,\mathcal{U}(x,\tau),
\end{equation}
where $\Xi$ carries the biquaternionic kinematics and real metric content, while 
$\mathcal{U}(x,\tau)$ is a unitary phase factor acting on a complex rank-$3$ internal fiber:
\begin{equation}
\mathcal{U}(x,\tau)\;\in\;\mathrm{U}(3), 
\qquad \mathcal{U}^\dagger\mathcal{U}=\mathbf{1}_3.
\end{equation}
The \emph{color subgroup} is identified with the traceless part:
\begin{equation}
\label{eq:G1_su3_subgroup}
\mathrm{SU}(3)_{\text{color}}\;\subset\;\mathrm{U}(3), 
\qquad \mathcal{U}=\exp\big(i\,\Phi\big),\quad \Phi\in \mathfrak{u}(3), \quad \mathrm{tr}\,\Phi=0 \;\Rightarrow\; \Phi\in \mathfrak{su}(3).
\end{equation}
Thus, color rotations are \emph{internal automorphisms} of the phase of $\Theta$, not external fields.

\paragraph{Remark (Compatibility).}
The factorization \eqref{eq:G1_theta_factorization} leaves 
$g_{\mu\nu}=\mathrm{Re}\big(\Theta^\dagger \Theta\big)$ unchanged under $\mathcal{U}$ because $\mathcal{U}$ is unitary on the internal fiber. Hence GR-limit and metric sector are preserved.

\subsection*{G.2 Modular (Theta-Function) Realization}
A concrete realization uses a multi-variable theta function:
\begin{equation}
\label{eq:G2_multi_theta}
\Theta(x,\tau)\;=\;\sum_{n\in \mathbb{Z}^3}\exp\Big(i\pi\, n^{\!\top}\,\Omega(x,\tau)\,n \;+\; 2\pi i\, n^{\!\top} z(x,\tau)\Big)\, \Xi(x,\tau),
\end{equation}
where $\Omega\in \mathrm{Mat}_{3\times 3}(\mathbb{C})$ is a symmetric period matrix with 
$\mathrm{Im}\,\Omega>0$, and $z\in \mathbb{C}^3$ is the internal phase coordinate. 
Modular transformations $(\Omega,z)\mapsto (\tilde\Omega,\tilde z)$ act on $\Theta$ via automorphisms. 
We identify the $\mathrm{SU}(3)$ color \emph{subgroup} as a subgroup of these internal phase automorphisms that preserve the traceless condition on the effective phase generator $\Phi$ in \eqref{eq:G1_su3_subgroup}. 
At fixed $(\Omega,z)$, local phase variations define a unitary frame $\mathcal{U}(x,\tau)$.

\paragraph{Interpretation.} 
The eight color degrees of freedom correspond to the traceless part of phase deformations in the 3D internal phase torus characterized by $(\Omega,z)$. The “$9\to 8$” reduction is the removal of the overall $\mathrm{U}(1)$ trace.

\subsection*{G.3 Color Connection as Phase Maurer–Cartan Form}
Define the internal color connection by the (right-invariant) Maurer–Cartan form on the phase frame:
\begin{equation}
\label{eq:G3_MC}
\mathcal{A}_\mu\;\equiv\; \mathcal{U}^\dagger \partial_\mu \mathcal{U}\;\in\;\mathfrak{u}(3), 
\qquad 
A_\mu\;\equiv\;\mathcal{A}_\mu\;-\;\tfrac{1}{3}\mathrm{tr}(\mathcal{A}_\mu)\,\mathbf{1}_3\;\in\;\mathfrak{su}(3),
\end{equation}
and similarly for the complex-time direction $\partial_\tau$. 
This is \emph{not} an externally postulated gauge potential: it is the intrinsic phase connection of $\Theta$’s internal fiber.

The corresponding field strength is the curvature of the phase connection:
\begin{equation}
\label{eq:G3_curvature}
F_{\mu\nu}\;=\;\partial_\mu A_\nu-\partial_\nu A_\mu+[A_\mu,A_\nu]\;\in\;\mathfrak{su}(3),
\end{equation}
with the standard Bianchi identity $D_{[\mu}F_{\nu\rho]}=0$.

\paragraph{Walk-back to Yang–Mills.}
The phase curvature \eqref{eq:G3_curvature} is \emph{identical in form} to Yang–Mills field strength once $A_\mu$ is identified with the traceless part of $\mathcal{U}^\dagger \partial_\mu \mathcal{U}$. No external gauge structure was added: YM is the geometry of the internal $\Theta$-phase bundle.

\subsection*{G.4 Covariant Derivative on \texorpdfstring{$\Theta$}{Theta} with Color Phase}
Let $\Theta$ transform under the internal phase $\mathcal{U}$ on the right:
$\Theta \mapsto \Theta\,\mathcal{U}$. 
Then the color-covariant derivative on $\Theta$ is
\begin{equation}
\label{eq:G4_covD}
D_\mu \Theta \;\equiv\; \partial_\mu \Theta \;+\; \Theta\, A_\mu,
\qquad A_\mu \in \mathfrak{su}(3),
\end{equation}
which ensures $D_\mu \Theta \mapsto (D_\mu \Theta)\,\mathcal{U}$ under $\mathcal{U}$.
(Left actions carry the usual biquaternionic/spinorial covariances already present in the core UBT; right action hosts color.)

\paragraph{Kinetic and interaction terms.}
A minimal Lagrangian density for the color sector, invariant under internal phase rotations, reads
\begin{equation}
\label{eq:G4_lagrangian}
\mathcal{L}_{\mathrm{color}}\;=\; -\frac{1}{4}\,\mathrm{tr}(F_{\mu\nu}F^{\mu\nu})
\;+\; \mathrm{Re}\,\Big\langle D_\mu\Theta,\, D^\mu\Theta\Big\rangle_{\! \mathbb{B}\otimes \mathbb{C}}
\end{equation}
with the trace over color indices and the biquaternionic–complex hermitian pairing as in the core appendices. 
Gauge coupling $g_s$ is absorbed into the normalization of $A_\mu$ (or equivalently, into the phase metric on the internal fiber).

\subsection*{G.5 Algebraic Checks (SU(3) Structure)}
Let $\{T^a\}_{a=1}^8$ be generators of $\mathfrak{su}(3)$ with 
$[T^a,T^b]= i f^{abc} T^c$ and $\mathrm{tr}(T^a T^b)=\frac{1}{2}\delta^{ab}$.
Writing $A_\mu = A_\mu^a T^a$, eqs.~\eqref{eq:G3_curvature}–\eqref{eq:G4_lagrangian} reproduce
\begin{equation}
F_{\mu\nu}^a \;=\;\partial_\mu A_\nu^a-\partial_\nu A_\mu^a + f^{abc} A_\mu^b A_\nu^c,
\qquad 
\mathcal{L}_{\mathrm{YM}} = -\frac{1}{4} F_{\mu\nu}^a F^{a\mu\nu}.
\end{equation}
Since $A_\mu$ is the traceless part of $\mathcal{U}^\dagger \partial_\mu \mathcal{U}$, 
the $9\to 8$ reduction emerges from projecting out the $\mathrm{U}(1)$ trace, consistent with 
$\mathrm{SU}(3)=\{U\in \mathrm{U}(3)\,|\,\det U=1\}$.

\subsection*{G.6 Embedding in Core UBT and GR Limit}
\paragraph{Metric sector.} 
Because $\mathcal{U}$ is unitary on the internal fiber, 
$g_{\mu\nu}=\mathrm{Re}(\Theta^\dagger \Theta)$ is invariant under color rotations. 
Thus the Einstein limit and gravitational sector of UBT remain unchanged.

\paragraph{Electromagnetic and weak sectors.} 
The complex-$\mathrm{U}(1)$ phase and quaternionic commutators (yielding $\mathrm{SU}(2)$) remain as in the core theory. 
The color sector is orthogonal to these phases (traceless part of $\mathrm{U}(3)$).

\subsection*{G.7 Running, Anomalies, and Consistency (Sketch)}
\paragraph{Beta function (qualitative).}
Identifying $A_\mu$ as a phase connection allows standard perturbative renormalization with the same one-loop $\beta$-function sign as QCD (asymptotic freedom) provided the internal phase metric is positive and the color matter content (effective $\Theta$-components charged under right action) matches the SM representations. A full computation requires fixing the phase-fiber metric and matter embedding (left vs. right action), left here as future work.

\paragraph{Anomalies.}
Since color acts vectorially on the right and unitary, pure $\mathrm{SU}(3)$ anomalies cancel as in SM. Mixed anomalies with the left biquaternionic sector are absent if left–right actions are in orthogonal bundles (as constructed here). A thorough anomaly analysis will be provided in a dedicated appendix.

\subsection*{G.8 Relation to Multi-Theta (Modular) Data}
The theta realization \eqref{eq:G2_multi_theta} ties color to modular deformations $(\Omega,z)$: infinitesimal traceless deformations of $(\Omega,z)$ generate $\Phi\in\mathfrak{su}(3)$ and hence $A_\mu$. This identifies gluon dynamics with curvature of the internal modular torus over spacetime, i.e. a geometric (not ad hoc) origin for the color connection.

\subsection*{G.9 Phenomenology and Tests (Program)}
\begin{enumerate}
\item \textbf{No change in GR tests.} Solar-system, binary pulsars, GW waveforms unaffected by color phases.
\item \textbf{Low-energy QCD.} At hadronic scales, confinement emerges from non-abelian curvature of $A_\mu$; lattice-inspired effective actions can be mapped to internal phase curvature energy.
\item \textbf{Running couplings.} The internal phase metric provides a calculable geometric origin of $g_s(\mu)$ (future work).
\end{enumerate}

\subsection*{G.10 Speculative Notes (Non-Core, Clearly Marked)}
\emph{Speculative.} If the internal modular space couples weakly to complex-time phase $\psi$, topological defects (domain walls in $(\Omega,z)$) could imprint tiny, energy-dependent modulations in color sector at ultra-high energies. No experimental attempt is known; this is \textbf{not} part of the core claims.

\subsection*{G.11 Summary}
We have shown that $\mathrm{SU}(3)_{\mathrm{color}}$ arises naturally as the traceless unitary automorphism subgroup of the multi-dimensional phase of $\Theta$. The non-abelian connection $A_\mu$ and curvature $F_{\mu\nu}$ are the Maurer–Cartan data of the internal phase frame $\mathcal{U}$, yielding the standard Yang–Mills structure without grafting an external gauge sector. GR limit and the core UBT claims remain intact.

\subsection*{G.12 One-Loop Running of \texorpdfstring{$g_s(\mu)$}{g\_s(mu)} in Emergent Formulation}
\label{sec:G12_running}

\paragraph{Phase fiber metric and coupling definition.}
The strong coupling $g_s$ emerges geometrically from the normalization of the internal phase connection. Let $h_{ab}$ denote the metric on the internal modular fiber (parametrized by $(\Omega,z)$), with indices $a,b=1,\ldots,8$ running over the traceless $\mathfrak{su}(3)$ directions. The Yang–Mills kinetic term in \eqref{eq:G4_lagrangian} can be rewritten as
\begin{equation}
\mathcal{L}_{\mathrm{YM}} = -\frac{1}{4g_s^2}\,\mathrm{tr}(F_{\mu\nu}F^{\mu\nu})
= -\frac{1}{4}\, h^{ab} F_{\mu\nu}^a F^{b\mu\nu},
\end{equation}
identifying $g_s^{-2} = \mathrm{tr}(h^{ab}T^a T^b)/2$ for the standard normalization $\mathrm{tr}(T^a T^b)=\frac{1}{2}\delta^{ab}$. Thus, $g_s^2(\mu)$ is directly tied to the effective volume element of the internal phase torus at scale $\mu$.

\paragraph{Beta function from geometric flow.}
In standard QCD with $n_f$ quark flavors, the one-loop $\beta$-function is
\begin{equation}
\label{eq:G12_beta}
\mu\frac{\mathrm{d}g_s}{\mathrm{d}\mu} = \beta_0 g_s^3 + \mathcal{O}(g_s^5),
\qquad 
\beta_0 = -\frac{1}{(4\pi)^2}\Big(11 - \frac{2n_f}{3}\Big).
\end{equation}
For $n_f=6$ (SM quarks), $\beta_0 = -7/(4\pi)^2 < 0$, yielding asymptotic freedom.

In the UBT emergent picture, this beta function arises from the renormalization-group flow of the internal fiber metric $h_{ab}(\mu)$. Quantum fluctuations of $\Theta$ induce a scale-dependent deformation of $(\Omega,z)$, modifying the effective phase volume. The one-loop contribution from gluon self-interactions (proportional to the $\mathfrak{su}(3)$ Casimir $C_2(\mathrm{adj})=3$) and quark loops (proportional to $C_2(\mathbf{3})=4/3$ per flavor) combine to give \eqref{eq:G12_beta}.

\paragraph{Explicit geometric realization (sketch).}
Let the modular period matrix $\Omega(x,\mu)$ depend on the RG scale $\mu$ via
\begin{equation}
\mu\frac{\partial \Omega_{ij}}{\partial \mu} = \gamma_{ij}[\Omega,g_s(\mu)],
\end{equation}
where $\gamma_{ij}$ is the anomalous dimension matrix for the internal phase modes. The traceless constraint $\mathrm{tr}\,\Omega=0$ (mod integer shifts) ensures $\mathrm{SU}(3)$ rather than $\mathrm{U}(3)$. The induced flow of $g_s^2 \propto (\det\,\mathrm{Im}\,\Omega)^{-1/3}$ then matches \eqref{eq:G12_beta} at one-loop order, provided the matter content (effective right-action charges of $\Theta$ components) corresponds to $n_f=6$ fundamental representations. A complete two-loop analysis (analogous to appendix K.5 for $\Lambda_{\mathrm{QCD}}$) is left for future work.

\paragraph{Running coupling solution.}
Integrating \eqref{eq:G12_beta} from a reference scale $\mu_0$ to $\mu$ gives
\begin{equation}
\alpha_s(\mu) = \frac{g_s^2(\mu)}{4\pi} = \frac{\alpha_s(\mu_0)}{1 + \alpha_s(\mu_0)\beta_0(4\pi)\ln(\mu/\mu_0)},
\end{equation}
reproducing the standard QCD running. For $\mu_0=M_Z$ with $\alpha_s(M_Z)\approx 0.118$, this yields $\alpha_s(1\,\mathrm{GeV})\approx 0.5$ and $\Lambda_{\overline{\mathrm{MS}}}\approx 200$–$300$ MeV in the $\overline{\mathrm{MS}}$ scheme, consistent with lattice QCD and experimental data.

\subsection*{G.13 Detailed Anomaly Analysis: Left-Right Factorization}
\label{sec:G13_anomalies}

\paragraph{Separation of left and right actions on \texorpdfstring{$\Theta$}{Theta}.}
The biquaternionic structure of $\Theta$ naturally factorizes its symmetry actions:
\begin{itemize}
\item \textbf{Left action:} Spacetime/spinorial symmetries (Lorentz group, biquaternionic rotations) and electroweak gauge transformations act on the left: $\Theta \mapsto L\,\Theta$, where $L$ encodes $\mathrm{SU}(2)_L \times \mathrm{U}(1)_Y$ and spacetime covariances.
\item \textbf{Right action:} Internal color phase rotations act on the right via the unitary frame $\mathcal{U}$: $\Theta \mapsto \Theta\,\mathcal{U}$, with $\mathcal{U}\in\mathrm{SU}(3)_{\mathrm{color}}$ as in \eqref{eq:G1_su3_subgroup}.
\end{itemize}
This orthogonality is encoded in the tensor product structure $\Theta \in \mathbb{B}\otimes \mathbb{C}^{N_L} \otimes \mathbb{C}^3$, where $\mathbb{C}^{N_L}$ carries the left electroweak multiplet (e.g., $N_L=2$ for doublets, $N_L=1$ for singlets) and $\mathbb{C}^3$ is the color triplet for quarks (or singlet for leptons).

\paragraph{Pure \texorpdfstring{$\mathrm{SU}(3)$}{SU(3)} anomalies.}
The pure color anomaly for three $\mathrm{SU}(3)$ currents vanishes identically because $\mathrm{SU}(3)$ is vectorial (fermions in complex representations plus their conjugates). Explicitly, the triangle diagram with three gluon vertices gives
\begin{equation}
\mathcal{A}_{\mathrm{color}}^3 \propto \sum_f \mathrm{tr}(\{T^a,T^b\}T^c) = 0
\end{equation}
by tracelessness and antisymmetry of the structure constants. This remains true in UBT because the right-action color charges are vector-like: each quark flavor $q$ in $\mathbf{3}$ has a corresponding antiquark $\bar{q}$ in $\bar{\mathbf{3}}$.

\paragraph{Mixed anomalies with electroweak sector.}
The potential mixed anomaly $\mathrm{SU}(3)^2 \times \mathrm{U}(1)_Y$ or $\mathrm{SU}(3)^2 \times \mathrm{SU}(2)_L$ must also vanish for consistency. In the Standard Model, this cancellation is automatic because:
\begin{enumerate}
\item Color acts the same on left-handed and right-handed quarks (vector coupling).
\item The sum over quark doublets and singlets with their hypercharges cancels the $\mathrm{U}(1)_Y$ contribution.
\end{enumerate}
In UBT, the left-right factorization ensures that color (right action) and electroweak (left action) reside in orthogonal bundles. The covariant derivative factorizes as
\begin{equation}
D_\mu \Theta = (\partial_\mu + \Omega_\mu^L)\,\Theta + \Theta\,A_\mu^R,
\end{equation}
where $\Omega_\mu^L$ contains $\mathrm{SU}(2)_L \times \mathrm{U}(1)_Y$ connections and $A_\mu^R \in \mathfrak{su}(3)$ is the color connection. The mixed anomaly diagram then factors into separate left and right loops, and the trace over color is independent of the electroweak charges:
\begin{equation}
\mathcal{A}_{\mathrm{mixed}} \propto \mathrm{tr}_{\mathrm{color}}(T^a T^b) \cdot \mathrm{tr}_{\mathrm{EW}}(Y) = C_2(\mathbf{3})\,\delta^{ab} \cdot \sum_f Y_f.
\end{equation}
The SM quark content ensures $\sum_f Y_f = 0$ (each generation contributes zero), so $\mathcal{A}_{\mathrm{mixed}}=0$.

\paragraph{Representation content and consistency.}
For UBT to reproduce SM phenomenology, $\Theta$ must contain components transforming as:
\begin{itemize}
\item Quarks: $(\mathbf{2},\mathbf{3})_{+1/6}$ (left doublet) and $(\mathbf{1},\mathbf{3})_{+2/3},\,(\mathbf{1},\mathbf{3})_{-1/3}$ (right singlets) under $\mathrm{SU}(2)_L \times \mathrm{SU}(3)_{\mathrm{color}} \times \mathrm{U}(1)_Y$.
\item Leptons: $(\mathbf{2},\mathbf{1})_{-1/2}$ and $(\mathbf{1},\mathbf{1})_{-1}$ (colorless).
\end{itemize}
The biquaternionic tensor structure $\mathbb{B}\otimes \mathbb{C}^{N_L}\otimes \mathbb{C}^{N_R}$ with appropriate Clifford algebra embeddings naturally accommodates these representations. The key point is that \emph{all anomaly cancellations of the SM are inherited} because the effective low-energy spectrum matches the SM fermion content.

\paragraph{Gravitational anomalies.}
The gravitational anomaly (relevant for chiral theories) involves the trace anomaly in curved spacetime. In UBT, the real metric $g_{\mu\nu}=\mathrm{Re}(\Theta^\dagger\Theta)$ is invariant under color rotations \eqref{eq:G1_su3_subgroup}, so color does not contribute to the gravitational anomaly. The chiral electroweak sector's gravitational anomaly cancels via the standard SM mechanism (equal numbers of left-handed and right-handed Weyl fermions modulo Higgs couplings).

\paragraph{Summary of anomaly checks.}
\begin{equation}
\begin{aligned}
\mathcal{A}[\mathrm{SU}(3)^3] &= 0 \quad \text{(vectorial color)}, \\
\mathcal{A}[\mathrm{SU}(3)^2 \times \mathrm{U}(1)_Y] &= 0 \quad \text{(left-right orthogonality + SM charge assignment)}, \\
\mathcal{A}[\mathrm{SU}(3)^2 \times \text{gravity}] &= 0 \quad \text{(color decoupled from metric)}.
\end{aligned}
\end{equation}
All standard SM anomaly cancellations are preserved in the UBT framework.

\subsection*{G.14 Explicit Mapping: \texorpdfstring{$(\Omega,z)$}{(Omega,z)}-Deformations to Gell-Mann Generators}
\label{sec:G14_mapping}

We now provide an explicit dictionary between infinitesimal deformations of the modular data $(\Omega,z)$ and the eight traceless $\mathfrak{su}(3)$ generators $T^a$ (Gell-Mann matrices). This establishes a concrete realization of the abstract phase connection \eqref{eq:G3_MC}.

\paragraph{Setup.}
Let $\Omega \in \mathrm{Mat}_{3\times 3}(\mathbb{C})$ be a symmetric matrix with $\mathrm{Im}\,\Omega > 0$ (positive definite imaginary part), parametrizing the modular structure of the internal phase torus $\mathbb{T}^3$. Let $z=(z_1,z_2,z_3)^{\top}\in \mathbb{C}^3$ be the phase coordinates. The multi-variable theta function \eqref{eq:G2_multi_theta} is invariant under modular transformations and shifts $z \mapsto z + \Omega\,m + n$ for $m,n\in\mathbb{Z}^3$. Infinitesimal traceless deformations $(\delta\Omega,\delta z)$ generate phase variations that, when projected to the traceless subspace, yield $\mathfrak{su}(3)$.

\paragraph{Lemma G.1 (Traceless deformation basis).}
Consider infinitesimal variations $\Omega \mapsto \Omega + \epsilon\,\delta\Omega$, $z \mapsto z + \epsilon\,\delta z$ with $\epsilon \ll 1$. Imposing the traceless constraint $\mathrm{tr}(\delta\Omega)=0$ (to preserve $\det\,\mathcal{U}=1$), the space of such deformations is 8-dimensional (real degrees of freedom: $3\times 3$ symmetric traceless matrix has 8 real parameters; $z$ contributes phases but couples to $\Omega$). Explicitly, we parametrize
\begin{equation}
\delta\Omega = i\sum_{a=1}^8 \theta^a\,\Lambda^a, 
\qquad 
\Lambda^a \in \mathrm{Mat}_{3\times 3}(\mathbb{R}), \quad \mathrm{tr}\,\Lambda^a=0,
\end{equation}
where $\Lambda^a$ are symmetric traceless $3\times 3$ matrices (8 independent). These correspond to the 8 directions in $\mathfrak{su}(3)$.

\paragraph{Proposition G.2 (Explicit $T^a$ correspondence).}
The standard Gell-Mann matrices $T^a$ ($a=1,\ldots,8$) acting on $\mathbb{C}^3$ can be mapped to the modular deformation generators $\Lambda^a$ via the isomorphism $\mathfrak{su}(3) \cong \mathbb{R}^8$ (as Lie algebras). Concretely:
\begin{enumerate}
\item \textbf{Diagonal generators} ($T^3,T^8$):
\begin{equation}
T^3 = \frac{1}{2}\begin{pmatrix}1&0&0\\0&-1&0\\0&0&0\end{pmatrix}, 
\quad 
T^8 = \frac{1}{2\sqrt{3}}\begin{pmatrix}1&0&0\\0&1&0\\0&0&-2\end{pmatrix}
\end{equation}
correspond to $\Lambda^3,\Lambda^8$ as diagonal deformations of $\Omega$:
\begin{equation}
\Lambda^3 = \mathrm{diag}(\tfrac{1}{2},-\tfrac{1}{2},0), 
\quad 
\Lambda^8 = \mathrm{diag}(\tfrac{1}{2\sqrt{3}},\tfrac{1}{2\sqrt{3}},-\tfrac{1}{\sqrt{3}}).
\end{equation}
These shift the relative phases $\mathrm{Im}(\Omega_{11})$ vs. $\mathrm{Im}(\Omega_{22})$ vs. $\mathrm{Im}(\Omega_{33})$, corresponding to color rotations in the Cartan subalgebra.

\item \textbf{Off-diagonal generators} ($T^{1,2},\ldots,T^{6,7}$):
For $T^{1,2}$ (acting on the $1$–$2$ subspace), $T^{4,5}$ ($1$–$3$), $T^{6,7}$ ($2$–$3$), we have
\begin{equation}
T^{1,2} = \frac{1}{2}\begin{pmatrix}0&1\\\pm i&0\end{pmatrix} \oplus 0, 
\quad \text{etc.}
\end{equation}
These correspond to off-diagonal deformations of $\Omega$:
\begin{equation}
\Lambda^{1} = \begin{pmatrix}0&1&0\\1&0&0\\0&0&0\end{pmatrix}, 
\quad 
\Lambda^{2} = \begin{pmatrix}0&-i&0\\i&0&0\\0&0&0\end{pmatrix}, 
\quad \text{etc.}
\end{equation}
Infinitesimal shifts $\delta\Omega_{ij} = i\epsilon\,\Lambda^a_{ij}$ induce phase twists between color indices, generating non-abelian rotations.
\end{enumerate}

\paragraph{Corollary G.3 (Connection components).}
The color connection $A_\mu = A_\mu^a T^a$ at a point $x$ is obtained by pulling back the modular deformation:
\begin{equation}
A_\mu^a(x) = \frac{1}{2}\,\mathrm{tr}(T^a\,\mathcal{U}^\dagger \partial_\mu \mathcal{U}),
\end{equation}
where $\mathcal{U}(x) = \exp(i\Phi)$ with $\Phi = \sum_a \phi^a(x)\,T^a$. The phases $\phi^a(x)$ are determined by the local values of $(\Omega(x),z(x))$. Explicitly:
\begin{equation}
\phi^a(x) \approx \theta^a(x) + \mathcal{O}(\theta^2),
\end{equation}
where $\theta^a(x)$ parametrize the traceless part of $\Omega(x)$ as in Lemma G.1. Thus, spacetime variations $\partial_\mu \theta^a$ directly yield the gluon potentials $A_\mu^a$.

\paragraph{Representation table.}
For reference, we summarize the 8 generators and their modular realizations:

\begin{center}
\begin{tabular}{c|l|l}
\hline
$a$ & Gell-Mann $T^a$ & Modular $\Lambda^a$ (traceless symmetric $3\times 3$) \\
\hline
1 & $\frac{1}{2}(|1\rangle\langle 2| + |2\rangle\langle 1|)$ & $\Lambda^1_{12}=\Lambda^1_{21}=\frac{1}{2}$, rest 0 \\
2 & $\frac{1}{2}(-i|1\rangle\langle 2| + i|2\rangle\langle 1|)$ & $\Lambda^2_{12}=-i/2$, $\Lambda^2_{21}=+i/2$, rest 0 \\
3 & $\frac{1}{2}(|1\rangle\langle 1| - |2\rangle\langle 2|)$ & $\Lambda^3 = \mathrm{diag}(1/2,-1/2,0)$ \\
4 & $\frac{1}{2}(|1\rangle\langle 3| + |3\rangle\langle 1|)$ & $\Lambda^4_{13}=\Lambda^4_{31}=\frac{1}{2}$ \\
5 & $\frac{1}{2}(-i|1\rangle\langle 3| + i|3\rangle\langle 1|)$ & $\Lambda^5_{13}=-i/2$, $\Lambda^5_{31}=+i/2$ \\
6 & $\frac{1}{2}(|2\rangle\langle 3| + |3\rangle\langle 2|)$ & $\Lambda^6_{23}=\Lambda^6_{32}=\frac{1}{2}$ \\
7 & $\frac{1}{2}(-i|2\rangle\langle 3| + i|3\rangle\langle 2|)$ & $\Lambda^7_{23}=-i/2$, $\Lambda^7_{32}=+i/2$ \\
8 & $\frac{1}{2\sqrt{3}}(|1\rangle\langle 1| + |2\rangle\langle 2| - 2|3\rangle\langle 3|)$ & $\Lambda^8 = \mathrm{diag}(1/(2\sqrt{3}),1/(2\sqrt{3}),-1/\sqrt{3})$ \\
\hline
\end{tabular}
\end{center}

This table provides the explicit dictionary for translating geometric phase deformations on the internal modular torus into standard QCD gauge field components.

\paragraph{Remark (Higher-order terms).}
The above mapping is linearized (valid for small $\theta^a$). For finite transformations, $\mathcal{U} = \exp(i\sum_a \theta^a T^a)$ involves the BCH formula and generates the full $\mathrm{SU}(3)$ group manifold. The modular realization respects the Lie bracket $[T^a,T^b]=if^{abc}T^c$, ensuring consistency with the structure constants of $\mathfrak{su}(3)$.

% =====================================================================
% End of Appendix G
% =====================================================================

