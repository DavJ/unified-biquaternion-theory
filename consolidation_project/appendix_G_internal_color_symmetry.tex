% =====================================================================
% Appendix G: Internal Color Symmetry as a Modular Subgroup of Θ
% Status: theoretical derivation (core-compatible), speculative notes marked
% =====================================================================

\appendix
\section*{Appendix G \\ Internal Color Symmetry as a Modular Subgroup of \texorpdfstring{$\Theta$}{Theta}}
\addcontentsline{toc}{section}{Appendix G: Internal Color Symmetry as a Modular Subgroup of $\Theta$}

\subsection*{G.0 Overview (Core-Compatible, Non-Disruptive)}
This appendix derives QCD color symmetry $\mathrm{SU}(3)_{\mathrm{color}}$ as an \emph{internal modular automorphism} of the $\Theta$-field phase manifold, without introducing an external gauge stack. The construction preserves UBT core principles:
(i) biquaternionic base for spacetime/kinematics, 
(ii) complex time $\tau=t+i\psi$, 
(iii) metric from $\mathrm{Re}(\Theta^\dagger \Theta)$, 
(iv) gauge/phase data encoded in the holomorphic structure of $\Theta$.
Color interactions arise from \emph{multi-dimensional phase degrees of freedom} of $\Theta$; Yang–Mills variables appear as phase connections on a rank-3 internal bundle. GR limit and QED/weak structure remain unchanged.

\subsection*{G.1 Theta Field with Multi-Phase Structure}
Let $\Theta:\; \mathcal{M}\times \mathbb{T}_\psi \to \mathbb{B}\otimes \mathbb{C}$ be the biquaternionic field on spacetime $\mathcal{M}$ with complex time $\tau=t+i\psi$, 
and let $\mathcal{F}$ denote its internal phase manifold. 
We promote the scalar phase to a \emph{matrix phase} by writing
\begin{equation}
\label{eq:G1_theta_factorization}
\Theta(x,\tau)\;=\; \Xi(x,\tau)\,\mathcal{U}(x,\tau),
\end{equation}
where $\Xi$ carries the biquaternionic kinematics and real metric content, while 
$\mathcal{U}(x,\tau)$ is a unitary phase factor acting on a complex rank-$3$ internal fiber:
\begin{equation}
\mathcal{U}(x,\tau)\;\in\;\mathrm{U}(3), 
\qquad \mathcal{U}^\dagger\mathcal{U}=\mathbf{1}_3.
\end{equation}
The \emph{color subgroup} is identified with the traceless part:
\begin{equation}
\label{eq:G1_su3_subgroup}
\mathrm{SU}(3)_{\text{color}}\;\subset\;\mathrm{U}(3), 
\qquad \mathcal{U}=\exp\big(i\,\Phi\big),\quad \Phi\in \mathfrak{u}(3), \quad \mathrm{tr}\,\Phi=0 \;\Rightarrow\; \Phi\in \mathfrak{su}(3).
\end{equation}
Thus, color rotations are \emph{internal automorphisms} of the phase of $\Theta$, not external fields.

\paragraph{Remark (Compatibility).}
The factorization \eqref{eq:G1_theta_factorization} leaves 
$g_{\mu\nu}=\mathrm{Re}\big(\Theta^\dagger \Theta\big)$ unchanged under $\mathcal{U}$ because $\mathcal{U}$ is unitary on the internal fiber. Hence GR-limit and metric sector are preserved.

\subsection*{G.2 Modular (Theta-Function) Realization}
A concrete realization uses a multi-variable theta function:
\begin{equation}
\label{eq:G2_multi_theta}
\Theta(x,\tau)\;=\;\sum_{n\in \mathbb{Z}^3}\exp\Big(i\pi\, n^{\!\top}\,\Omega(x,\tau)\,n \;+\; 2\pi i\, n^{\!\top} z(x,\tau)\Big)\, \Xi(x,\tau),
\end{equation}
where $\Omega\in \mathrm{Mat}_{3\times 3}(\mathbb{C})$ is a symmetric period matrix with 
$\mathrm{Im}\,\Omega>0$, and $z\in \mathbb{C}^3$ is the internal phase coordinate. 
Modular transformations $(\Omega,z)\mapsto (\tilde\Omega,\tilde z)$ act on $\Theta$ via automorphisms. 
We identify the $\mathrm{SU}(3)$ color \emph{subgroup} as a subgroup of these internal phase automorphisms that preserve the traceless condition on the effective phase generator $\Phi$ in \eqref{eq:G1_su3_subgroup}. 
At fixed $(\Omega,z)$, local phase variations define a unitary frame $\mathcal{U}(x,\tau)$.

\paragraph{Interpretation.} 
The eight color degrees of freedom correspond to the traceless part of phase deformations in the 3D internal phase torus characterized by $(\Omega,z)$. The “$9\to 8$” reduction is the removal of the overall $\mathrm{U}(1)$ trace.

\subsection*{G.3 Color Connection as Phase Maurer–Cartan Form}
Define the internal color connection by the (right-invariant) Maurer–Cartan form on the phase frame:
\begin{equation}
\label{eq:G3_MC}
\mathcal{A}_\mu\;\equiv\; \mathcal{U}^\dagger \partial_\mu \mathcal{U}\;\in\;\mathfrak{u}(3), 
\qquad 
A_\mu\;\equiv\;\mathcal{A}_\mu\;-\;\tfrac{1}{3}\mathrm{tr}(\mathcal{A}_\mu)\,\mathbf{1}_3\;\in\;\mathfrak{su}(3),
\end{equation}
and similarly for the complex-time direction $\partial_\tau$. 
This is \emph{not} an externally postulated gauge potential: it is the intrinsic phase connection of $\Theta$’s internal fiber.

The corresponding field strength is the curvature of the phase connection:
\begin{equation}
\label{eq:G3_curvature}
F_{\mu\nu}\;=\;\partial_\mu A_\nu-\partial_\nu A_\mu+[A_\mu,A_\nu]\;\in\;\mathfrak{su}(3),
\end{equation}
with the standard Bianchi identity $D_{[\mu}F_{\nu\rho]}=0$.

\paragraph{Walk-back to Yang–Mills.}
The phase curvature \eqref{eq:G3_curvature} is \emph{identical in form} to Yang–Mills field strength once $A_\mu$ is identified with the traceless part of $\mathcal{U}^\dagger \partial_\mu \mathcal{U}$. No external gauge structure was added: YM is the geometry of the internal $\Theta$-phase bundle.

\subsection*{G.4 Covariant Derivative on \texorpdfstring{$\Theta$}{Theta} with Color Phase}
Let $\Theta$ transform under the internal phase $\mathcal{U}$ on the right:
$\Theta \mapsto \Theta\,\mathcal{U}$. 
Then the color-covariant derivative on $\Theta$ is
\begin{equation}
\label{eq:G4_covD}
D_\mu \Theta \;\equiv\; \partial_\mu \Theta \;+\; \Theta\, A_\mu,
\qquad A_\mu \in \mathfrak{su}(3),
\end{equation}
which ensures $D_\mu \Theta \mapsto (D_\mu \Theta)\,\mathcal{U}$ under $\mathcal{U}$.
(Left actions carry the usual biquaternionic/spinorial covariances already present in the core UBT; right action hosts color.)

\paragraph{Kinetic and interaction terms.}
A minimal Lagrangian density for the color sector, invariant under internal phase rotations, reads
\begin{equation}
\label{eq:G4_lagrangian}
\mathcal{L}_{\mathrm{color}}\;=\; -\frac{1}{4}\,\mathrm{tr}(F_{\mu\nu}F^{\mu\nu})
\;+\; \mathrm{Re}\,\Big\langle D_\mu\Theta,\, D^\mu\Theta\Big\rangle_{\! \mathbb{B}\otimes \mathbb{C}}
\end{equation}
with the trace over color indices and the biquaternionic–complex hermitian pairing as in the core appendices. 
Gauge coupling $g_s$ is absorbed into the normalization of $A_\mu$ (or equivalently, into the phase metric on the internal fiber).

\subsection*{G.5 Algebraic Checks (SU(3) Structure)}
Let $\{T^a\}_{a=1}^8$ be generators of $\mathfrak{su}(3)$ with 
$[T^a,T^b]= i f^{abc} T^c$ and $\mathrm{tr}(T^a T^b)=\frac{1}{2}\delta^{ab}$.
Writing $A_\mu = A_\mu^a T^a$, eqs.~\eqref{eq:G3_curvature}–\eqref{eq:G4_lagrangian} reproduce
\begin{equation}
F_{\mu\nu}^a \;=\;\partial_\mu A_\nu^a-\partial_\nu A_\mu^a + f^{abc} A_\mu^b A_\nu^c,
\qquad 
\mathcal{L}_{\mathrm{YM}} = -\frac{1}{4} F_{\mu\nu}^a F^{a\mu\nu}.
\end{equation}
Since $A_\mu$ is the traceless part of $\mathcal{U}^\dagger \partial_\mu \mathcal{U}$, 
the $9\to 8$ reduction emerges from projecting out the $\mathrm{U}(1)$ trace, consistent with 
$\mathrm{SU}(3)=\{U\in \mathrm{U}(3)\,|\,\det U=1\}$.

\subsection*{G.6 Embedding in Core UBT and GR Limit}
\paragraph{Metric sector.} 
Because $\mathcal{U}$ is unitary on the internal fiber, 
$g_{\mu\nu}=\mathrm{Re}(\Theta^\dagger \Theta)$ is invariant under color rotations. 
Thus the Einstein limit and gravitational sector of UBT remain unchanged.

\paragraph{Electromagnetic and weak sectors.} 
The complex-$\mathrm{U}(1)$ phase and quaternionic commutators (yielding $\mathrm{SU}(2)$) remain as in the core theory. 
The color sector is orthogonal to these phases (traceless part of $\mathrm{U}(3)$).

\subsection*{G.7 Running, Anomalies, and Consistency (Sketch)}
\paragraph{Beta function (qualitative).}
Identifying $A_\mu$ as a phase connection allows standard perturbative renormalization with the same one-loop $\beta$-function sign as QCD (asymptotic freedom) provided the internal phase metric is positive and the color matter content (effective $\Theta$-components charged under right action) matches the SM representations. A full computation requires fixing the phase-fiber metric and matter embedding (left vs. right action), left here as future work.

\paragraph{Anomalies.}
Since color acts vectorially on the right and unitary, pure $\mathrm{SU}(3)$ anomalies cancel as in SM. Mixed anomalies with the left biquaternionic sector are absent if left–right actions are in orthogonal bundles (as constructed here). A thorough anomaly analysis will be provided in a dedicated appendix.

\subsection*{G.8 Relation to Multi-Theta (Modular) Data}
The theta realization \eqref{eq:G2_multi_theta} ties color to modular deformations $(\Omega,z)$: infinitesimal traceless deformations of $(\Omega,z)$ generate $\Phi\in\mathfrak{su}(3)$ and hence $A_\mu$. This identifies gluon dynamics with curvature of the internal modular torus over spacetime, i.e. a geometric (not ad hoc) origin for the color connection.

\subsection*{G.9 Phenomenology and Tests (Program)}
\begin{enumerate}
\item \textbf{No change in GR tests.} Solar-system, binary pulsars, GW waveforms unaffected by color phases.
\item \textbf{Low-energy QCD.} At hadronic scales, confinement emerges from non-abelian curvature of $A_\mu$; lattice-inspired effective actions can be mapped to internal phase curvature energy.
\item \textbf{Running couplings.} The internal phase metric provides a calculable geometric origin of $g_s(\mu)$ (future work).
\end{enumerate}

\subsection*{G.10 Speculative Notes (Non-Core, Clearly Marked)}
\emph{Speculative.} If the internal modular space couples weakly to complex-time phase $\psi$, topological defects (domain walls in $(\Omega,z)$) could imprint tiny, energy-dependent modulations in color sector at ultra-high energies. No experimental attempt is known; this is \textbf{not} part of the core claims.

\subsection*{G.11 Summary}
We have shown that $\mathrm{SU}(3)_{\mathrm{color}}$ arises naturally as the traceless unitary automorphism subgroup of the multi-dimensional phase of $\Theta$. The non-abelian connection $A_\mu$ and curvature $F_{\mu\nu}$ are the Maurer–Cartan data of the internal phase frame $\mathcal{U}$, yielding the standard Yang–Mills structure without grafting an external gauge sector. GR limit and the core UBT claims remain intact.

% =====================================================================
% End of Appendix G
% =====================================================================

