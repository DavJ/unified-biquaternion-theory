\documentclass[11pt,a4paper]{article}
\usepackage{slashed}
\usepackage{amsmath,amssymb,amsfonts}
\usepackage{geometry}
\usepackage{graphicx}
\usepackage{hyperref}
\usepackage{xcolor}
\usepackage{tikz}
\usetikzlibrary{calc,decorations.pathmorphing,decorations.markings,positioning,arrows.meta,plotmarks}
\usepackage{pgfplots}
\pgfplotsset{compat=1.18}
\geometry{margin=1in}

\title{Unified Biquaternion Theory (Consolidation) --- CORE Manuscript}
\author{UBT Team}
\date{\today}

\begin{document}
\maketitle

\appendix
\tableofcontents

\section*{CORE Scope and Claims}
This CORE manuscript presents a biquaternion formulation that \textbf{generalizes and embeds Einstein's General Relativity} while recovering the Einstein--Maxwell--Dirac system under a standard variational action. The theory demonstrates the correct limits (Minkowski and weak-field), and \textbf{in the real-valued limit, exactly reproduces Einstein's field equations} for all curvature regimes, including cases where the Ricci scalar $R \neq 0$.

UBT extends GR by introducing biquaternionic degrees of freedom that represent phase-like and nonlocal components of spacetime. These additional components remain invisible to classical observations but may be relevant for dark sector physics and quantum gravitational corrections. \textbf{We do not claim an ab-initio derivation of the fine-structure constant}~$\alpha$; in the CORE track $\alpha(\mu)$ is treated as an empirical input consistent with QED running. Any links between the auxiliary phase coordinate~$\psi$ and consciousness are considered \emph{interpretive and speculative} and are not part of the CORE results. Quantitative, testable predictions are stated with their assumptions and orders of magnitude.


% ---- CORE APPENDICES (auto-included) ----
% NOTE: do not include speculative/WIP files here.


\section{Biquaternion Gravity}

This appendix presents the gravitational sector of the Unified Biquaternion Theory (UBT).
It combines the core theoretical framework from the original Biquaternion Gravity appendix
with the detailed analysis from the Quantum Gravity solution files, reformulated for
clarity and coherence.

\subsection{Introduction}

The gravitational field in UBT emerges naturally from the covariant formulation of the
biquaternionic tensor-spinor field equations. UBT generalizes Einstein's General Relativity (GR) 
by embedding the metric tensor in a biquaternionic field $\Theta(q,\tau)$ defined over 
complex time $\tau = t + i\psi$. The real part of this field corresponds to the classical 
metric tensor, ensuring that \textbf{GR is fully contained within UBT as a special case}.

In the real-valued limit (or equivalently, when the imaginary time component $\psi \to 0$), 
the biquaternionic field equations reduce exactly to Einstein's field equations, including 
all cases where the scalar curvature $R \neq 0$. The extended biquaternionic structure 
introduces additional degrees of freedom that represent phase-like and nonlocal components 
of spacetime, which have no signature in classical observations but may be relevant for 
dark sector physics and quantum gravitational effects.

In this formulation, spacetime is represented by a projection from a higher-dimensional
complex manifold, and curvature is encoded in the covariant derivatives of the
biquaternion field $\Theta(q,\tau)$. The gravitational interaction is therefore not an
independent postulate, but a manifestation of the underlying field geometry.

\subsection{Core Equations}

The line element is expressed as:
\begin{equation}
  ds^2 = g_{\mu\nu} \, dx^\mu dx^\nu ,
\end{equation}
where the metric tensor $g_{\mu\nu}$ is obtained from the biquaternion field via:
\begin{equation}
  g_{\mu\nu} = \Re\left[ \frac{\partial_\mu \Theta \cdot \partial_\nu \Theta^\dagger}{\mathcal{N}} \right],
\end{equation}
with $\mathcal{N}$ a normalisation factor ensuring the correct signature.

The Einstein tensor in this framework takes the standard form:
\begin{equation}
  G_{\mu\nu} = R_{\mu\nu} - \frac{1}{2} g_{\mu\nu} R ,
\end{equation}
but with curvature tensors $R_{\mu\nu}$ and $R$ propose from the biquaternionic connection
coefficients $\Gamma^\rho_{\mu\nu}$ obtained from the extended algebra.

The gravitational field equations couple $G_{\mu\nu}$ to the stress-energy tensor
$T_{\mu\nu}$ constructed from the biquaternion field invariants:
\begin{equation}
  G_{\mu\nu} = 8\pi G \, T_{\mu\nu}[\Theta] .
\end{equation}

\subsection{analysis Summary}

The original analysis proceeds by first defining the biquaternionic connection
compatible with the metric propose from $\Theta(q,\tau)$. The connection coefficients
are computed as:
\begin{equation}
  \Gamma^\rho_{\mu\nu} =
  \frac{1}{2} g^{\rho\sigma} \left( \partial_\mu g_{\nu\sigma}
  + \partial_\nu g_{\mu\sigma}
  - \partial_\sigma g_{\mu\nu} \right),
\end{equation}
where $g_{\mu\nu}$ is substituted from the field definition above.

The curvature tensor $R^\rho_{\ \sigma\mu\nu}$ is then obtained from:
\begin{equation}
  R^\rho_{\ \sigma\mu\nu} =
  \partial_\mu \Gamma^\rho_{\nu\sigma} -
  \partial_\nu \Gamma^\rho_{\mu\sigma} +
  \Gamma^\rho_{\mu\lambda} \Gamma^\lambda_{\nu\sigma} -
  \Gamma^\rho_{\nu\lambda} \Gamma^\lambda_{\mu\sigma} .
\end{equation}

Contracting appropriately yields $R_{\mu\nu}$ and the scalar curvature $R$.

\paragraph{Phase Curvature and Invisibility.}
The biquaternionic field $\Theta(q,\tau)$ has both real and imaginary components. While the 
real part $g_{\mu\nu} = \Re[\Theta_{\mu\nu}]$ corresponds to the classical metric tensor that 
couples to ordinary matter and radiation, the imaginary components $\Im[\Theta_{\mu\nu}]$ represent 
what we call \textbf{phase curvature}.

These phase curvature components satisfy their own field equations within the biquaternionic 
algebra and can carry energy-momentum in configurations where:
\begin{equation}
\Re[G_{\mu\nu}] = 0, \quad \text{but} \quad \Im[G_{\mu\nu}] \neq 0.
\end{equation}

Such configurations are mathematically consistent solutions but remain \textbf{invisible} to 
classical observations because ordinary matter and electromagnetic radiation couple only to the 
real-valued metric $g_{\mu\nu}$. They have no real-valued curvature signature ($\Re[R_{\mu\nu}] = 0$) 
yet possess nonlocal energy structure in the imaginary components. This invisibility is a 
mathematical property arising from the complex structure of the biquaternionic algebra, not a 
contradiction with General Relativity.

In the quantum gravity extension, fluctuations of the field $\Theta$ are quantised,
leading to corrections to the classical curvature in the form of effective stress-energy
terms arising from vacuum polarisation effects. The semiclassical approximation shows that
these corrections become significant near the Planck scale, modifying the black hole
horizon structure and potentially allowing for stable micro-horizon configurations.

\subsection{Summary}

Biquaternion Gravity generalizes Einstein's General Relativity by embedding it within a
richer biquaternionic algebraic framework. In the real-valued limit, UBT reproduces 
Einstein's field equations exactly, ensuring full compatibility with all experimental 
confirmations of GR—from the perihelion precession of Mercury to gravitational wave 
detection. The extended structure unifies gravity with other interactions at the
geometric level and introduces additional degrees of freedom corresponding to phase 
curvature and nonlocal energy configurations.

These imaginary components of the biquaternionic metric may represent dark sector 
phenomena or quantum gravitational corrections, but they remain invisible to classical 
matter and electromagnetic radiation that couple only to the real metric $g_{\mu\nu}$. 
The connection to Quantum Gravity arises from treating $\Theta$ as a quantum field, 
where gravitational effects emerge from its covariant structure. 

This approach may suggest possible deviations from classical GR at very small scales 
(near the Planck scale), while reducing exactly to Einstein's equations in all 
macroscopic regimes where GR has been tested. UBT does not contradict or replace 
General Relativity; it extends and embeds it within a broader mathematical framework. 
For detailed derivations of the GR recovery, see Appendix~R.


\section{Recovery of General Relativity from Biquaternionic Field Equations}

\subsection{Introduction}

The Unified Biquaternion Theory (UBT) is formulated as a mathematical generalization of Einstein's General Relativity (GR). This appendix demonstrates rigorously that GR is fully contained within UBT as a special case—specifically, as the real-valued projection of the biquaternionic field equations. UBT does not contradict or replace General Relativity; rather, it extends and embeds it within a richer algebraic structure that includes additional degrees of freedom corresponding to phase-like and nonlocal components of spacetime.

The core claim is:
\begin{quote}
\textbf{In the real-valued limit, the biquaternionic field equations reduce exactly to Einstein's field equations, including all cases where the Ricci scalar $R \neq 0$.}
\end{quote}

This compatibility holds regardless of the curvature magnitude, as UBT's extended structure naturally accommodates both flat and curved spacetime geometries.

\subsection{The Biquaternionic Field Equation}

The fundamental field equation in UBT is:
\begin{equation}
\nabla^\dagger \nabla \Theta(q, \tau) = \kappa \, \mathcal{T}(q, \tau),
\label{eq:ubt_field_eq}
\end{equation}
where:
\begin{itemize}
  \item $\Theta(q, \tau)$ is the biquaternionic metric-like field defined over the complex time coordinate $\tau = t + i\psi$,
  \item $\mathcal{T}(q, \tau)$ is the biquaternionic stress-energy tensor,
  \item $\nabla^\dagger$ denotes the adjoint covariant derivative in the biquaternionic algebra $\mathbb{H} \otimes \mathbb{C}$,
  \item $\kappa = 8\pi G$ is the gravitational coupling constant.
\end{itemize}

The field $\Theta(q,\tau)$ has the general biquaternionic decomposition:
\begin{equation}
\Theta(q, \tau) = g_{\mu\nu}(x) + i\psi_{\mu\nu}(x) + \mathbf{j}\,\xi_{\mu\nu}(x) + \mathbf{k}\,\chi_{\mu\nu}(x),
\label{eq:theta_decomposition}
\end{equation}
where $g_{\mu\nu}(x)$ is the real part corresponding to the classical metric tensor, and $\psi_{\mu\nu}$, $\xi_{\mu\nu}$, $\chi_{\mu\nu}$ are the imaginary components representing phase curvature and nonlocal energy configurations.

\subsection{Real-Valued Projection and the Einstein Tensor}

To recover General Relativity, we take the real part of the biquaternionic field equation. The operator $\nabla^\dagger \nabla$ acting on $\Theta$ produces a tensor that, when decomposed, has both real and imaginary components.

In the limit where the imaginary time component $\psi \to 0$ and we project onto the real spacetime manifold $\mathbb{R}^{1,3}$, the field equation reduces to:
\begin{equation}
\Re\big(\nabla^\dagger \nabla \Theta\big) = \kappa \, \Re(\mathcal{T}).
\label{eq:real_projection}
\end{equation}

The left-hand side can be shown to yield the Einstein tensor. Specifically, the biquaternionic covariant derivative structure, when restricted to real coordinates and real metric components, reproduces the standard Levi-Civita connection:
\begin{equation}
\Gamma^\rho_{\mu\nu} = \frac{1}{2} g^{\rho\sigma} \left( \partial_\mu g_{\nu\sigma} + \partial_\nu g_{\mu\sigma} - \partial_\sigma g_{\mu\nu} \right),
\end{equation}
where $g_{\mu\nu} = \Re[\Theta_{\mu\nu}]$.

The Riemann curvature tensor is then:
\begin{equation}
R^\rho_{\ \sigma\mu\nu} = \partial_\mu \Gamma^\rho_{\nu\sigma} - \partial_\nu \Gamma^\rho_{\mu\sigma} + \Gamma^\rho_{\mu\lambda} \Gamma^\lambda_{\nu\sigma} - \Gamma^\rho_{\nu\lambda} \Gamma^\lambda_{\mu\sigma},
\end{equation}
from which we obtain the Ricci tensor $R_{\mu\nu} = R^\lambda_{\ \mu\lambda\nu}$ and scalar curvature $R = g^{\mu\nu} R_{\mu\nu}$.

Therefore:
\begin{equation}
\Re\big(\nabla^\dagger \nabla \Theta\big) = R_{\mu\nu} - \tfrac{1}{2} g_{\mu\nu} R = G_{\mu\nu},
\end{equation}
which is precisely the Einstein tensor.

\subsection{The Einstein Field Equations}

Combining equation~\eqref{eq:real_projection} with the identification of the Einstein tensor, and noting that $\Re(\mathcal{T}) = T_{\mu\nu}$ (the physical stress-energy tensor), we obtain:
\begin{equation}
R_{\mu\nu} - \tfrac{1}{2} g_{\mu\nu} R = 8 \pi G \, T_{\mu\nu}.
\label{eq:einstein_equations}
\end{equation}

This is exactly Einstein's field equation for General Relativity. The derivation holds for arbitrary spacetime curvature, including:
\begin{itemize}
  \item Flat spacetime (Minkowski): $R_{\mu\nu} = 0$, $R = 0$
  \item Weak-field limit: linearized gravity
  \item Strong-field regimes: black holes, neutron stars, gravitational waves
  \item Cosmological solutions: FLRW metrics with $R \neq 0$
  \item Any solution of Einstein's equations with nonzero curvature
\end{itemize}

\subsection{Extended Curvature Structure}

While GR operates entirely within the real-valued metric sector, UBT introduces additional degrees of freedom through the imaginary components of $\Theta(q,\tau)$. These components satisfy their own field equations and can carry curvature and energy that do not contribute to the real Einstein tensor:
\begin{equation}
\Re[G_{\mu\nu}] = 0, \quad \text{but} \quad \Im[G_{\mu\nu}] \neq 0.
\end{equation}

Such configurations represent \textbf{phase curvature} or \textbf{nonlocal energy}, which are mathematically consistent solutions of the biquaternionic field equations but remain invisible to classical matter and electromagnetic radiation that couple only to the real metric $g_{\mu\nu}$.

These extended degrees of freedom may be relevant for:
\begin{itemize}
  \item Dark matter and dark energy phenomena
  \item Quantum gravitational corrections
  \item Phase-space structure of consciousness models (in speculative extensions)
  \item Topological defects and nonperturbative configurations
\end{itemize}

However, in all physical regimes where General Relativity has been tested and confirmed, the imaginary components are either absent or negligible, and UBT reduces exactly to GR.

\subsection{Summary and Theoretical Position}

The Unified Biquaternion Theory (UBT) recovers General Relativity as its real-valued limit and extends it through the inclusion of biquaternionic curvature components. The field equations remain covariant and yield the Einstein tensor $G_{\mu\nu}$ when projected into real spacetime, confirming full compatibility with GR while generalizing its domain.

Key points:
\begin{enumerate}
  \item UBT \textbf{generalizes} GR by embedding the metric tensor in a biquaternionic field.
  \item In the real-valued limit, UBT \textbf{reproduces} Einstein's equations exactly.
  \item Additional degrees of freedom correspond to phase or nonlocal curvature components that have no classical observational signature but may explain phenomena beyond the Standard Model.
  \item UBT does not contradict GR; it extends it to a richer mathematical structure.
\end{enumerate}

Therefore, all experimental confirmations of General Relativity—from perihelion precession of Mercury to gravitational wave detection—are automatically compatible with UBT, as they probe the real-valued sector where the theories are identical.


\input{appendix_C_electromagnetism_gauge_consolidated}
\input{appendix_D_qed_consolidated}
\input{appendix_E_SM_QCD_embedding}
\input{appendix_K_maxwell_curved_space}
\input{appendix_Z_bibliography}

\input{appendix_alpha_statement}

\section*{License}
This work is licensed under a Creative Commons Attribution 4.0 International License (CC BY 4.0).

\end{document}
