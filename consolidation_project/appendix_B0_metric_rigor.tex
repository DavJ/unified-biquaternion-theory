% VERSION: v17 Stable Release
% © 2025 Ing. David Jaroš — CC BY-NC-ND 4.0

\section{Rigorous Metric Tensor Properties}

This appendix provides a rigorous, self-contained proof sketch that the biquaternionic metric 
$\mathcal{G}_{\mu\nu}$ and its real projection $g_{\mu\nu}$ possess the required transformation 
properties and gauge invariance expected from a physical metric tensor.

\subsection{Assumptions and Definitions}

Throughout this appendix, we make the following assumptions:

\begin{enumerate}
\item \textbf{Smoothness}: The biquaternionic field $\Theta(q,\tau)$ is smooth (at least $C^2$) on the spacetime manifold $\mathcal{M}$.
\item \textbf{Covariant derivative}: The gauge-covariant derivative is defined as $D_\mu = \partial_\mu + A_\mu$, where $A_\mu$ is the gauge connection (with appropriate gauge group representation).
\item \textbf{Unitary gauge group}: Gauge transformations are generated by unitary operators $U \in G$ satisfying $U^\dagger U = \mathbf{1}$, where the adjoint $(\cdot)^\dagger$ is the biquaternionic adjoint (complex conjugate transpose for matrix-valued fields).
\item \textbf{Gauge transformation law}: Under $\Theta \to U\Theta$, the connection transforms as $A_\mu \to U A_\mu U^\dagger + U \partial_\mu U^\dagger$ to preserve covariance.
\item \textbf{Normalization}: The metric is properly normalized to ensure physically meaningful signature and units.
\end{enumerate}

\subsection{Lemma 1: Tensor Transformation of $\mathcal{G}_{\mu\nu}$}

\begin{tcolorbox}[colback=green!5!white,colframe=green!75!black,title=Lemma 1]
\textbf{Statement}: The biquaternionic metric $\mathcal{G}_{\mu\nu} := (D_\mu\Theta)^\dagger D_\nu\Theta$ transforms as a $(0,2)$ tensor under coordinate transformations $x^\mu \to x'^{\mu}(x)$.
\end{tcolorbox}

\textbf{Proof sketch}:

Under a coordinate transformation $x^\mu \to x'^{\mu}(x)$, partial derivatives transform as:
\begin{equation}
\partial_\mu \to \partial'_{\mu} = \frac{\partial x^\rho}{\partial x'^{\mu}} \partial_\rho .
\end{equation}

For the gauge connection, the transformation law is:
\begin{equation}
A_\mu \to A'_{\mu} = \frac{\partial x^\rho}{\partial x'^{\mu}} A_\rho ,
\end{equation}
ensuring that the covariant derivative transforms covariantly:
\begin{equation}
D_\mu = \partial_\mu + A_\mu \to D'_{\mu} = \frac{\partial x^\rho}{\partial x'^{\mu}} D_\rho .
\end{equation}

Applying this to the biquaternionic field $\Theta$, which transforms as a scalar under coordinate transformations (though it may transform non-trivially under internal gauge transformations), we have:
\begin{equation}
D_\mu\Theta \to D'_{\mu}\Theta = \frac{\partial x^\rho}{\partial x'^{\mu}} D_\rho\Theta .
\end{equation}

The adjoint operation commutes with the coordinate transformation (it acts on internal indices only), so:
\begin{equation}
(D_\mu\Theta)^\dagger \to (D'_{\mu}\Theta)^\dagger = \frac{\partial x^\rho}{\partial x'^{\mu}} (D_\rho\Theta)^\dagger .
\end{equation}

Therefore, the metric transforms as:
\begin{equation}
\mathcal{G}_{\mu\nu} = (D_\mu\Theta)^\dagger D_\nu\Theta \to \mathcal{G}'_{\mu\nu} = \frac{\partial x^\rho}{\partial x'^{\mu}} \frac{\partial x^\sigma}{\partial x'^{\nu}} (D_\rho\Theta)^\dagger D_\sigma\Theta = \frac{\partial x^\rho}{\partial x'^{\mu}} \frac{\partial x^\sigma}{\partial x'^{\nu}} \mathcal{G}_{\rho\sigma} .
\end{equation}

This is precisely the transformation law for a $(0,2)$ tensor. \qed

\subsection{Proposition 2: Real Tensor Property of $g_{\mu\nu}$}

\begin{tcolorbox}[colback=green!5!white,colframe=green!75!black,title=Proposition 2]
\textbf{Statement}: The real projection $g_{\mu\nu} := \text{Re}(\mathcal{G}_{\mu\nu})$ is a real-valued $(0,2)$ tensor covariant under diffeomorphisms.
\end{tcolorbox}

\textbf{Proof sketch}:

From Lemma 1, we have established that $\mathcal{G}_{\mu\nu}$ transforms as:
\begin{equation}
\mathcal{G}_{\mu\nu} \to \mathcal{G}'_{\mu\nu} = \frac{\partial x^\rho}{\partial x'^{\mu}} \frac{\partial x^\sigma}{\partial x'^{\nu}} \mathcal{G}_{\rho\sigma} .
\end{equation}

The real part operator $\text{Re}(\cdot)$ is linear and commutes with the coordinate transformation (which acts only on spacetime indices, not on the biquaternionic structure):
\begin{equation}
g'_{\mu\nu} = \text{Re}(\mathcal{G}'_{\mu\nu}) = \text{Re}\left( \frac{\partial x^\rho}{\partial x'^{\mu}} \frac{\partial x^\sigma}{\partial x'^{\nu}} \mathcal{G}_{\rho\sigma} \right) = \frac{\partial x^\rho}{\partial x'^{\mu}} \frac{\partial x^\sigma}{\partial x'^{\nu}} \text{Re}(\mathcal{G}_{\rho\sigma}) = \frac{\partial x^\rho}{\partial x'^{\mu}} \frac{\partial x^\sigma}{\partial x'^{\nu}} g_{\rho\sigma} .
\end{equation}

This confirms that $g_{\mu\nu}$ transforms as a $(0,2)$ tensor.

To verify that $g_{\mu\nu}$ is real-valued, we note that:
\begin{equation}
g_{\mu\nu} = \text{Re}(\mathcal{G}_{\mu\nu}) = \text{Re}\left( (D_\mu\Theta)^\dagger D_\nu\Theta \right) = \frac{1}{2}\left[ (D_\mu\Theta)^\dagger D_\nu\Theta + \left((D_\mu\Theta)^\dagger D_\nu\Theta\right)^\dagger \right] .
\end{equation}

Using the property that $(AB)^\dagger = B^\dagger A^\dagger$:
\begin{equation}
\left((D_\mu\Theta)^\dagger D_\nu\Theta\right)^\dagger = (D_\nu\Theta)^\dagger (D_\mu\Theta)^{\dagger\dagger} = (D_\nu\Theta)^\dagger D_\mu\Theta .
\end{equation}

If the metric is symmetric (which can be imposed or derived from the action principle), then $g_{\mu\nu} = g_{\nu\mu}$, and the above expression simplifies to show that $g_{\mu\nu}$ is manifestly real and symmetric. \qed

\subsection{Proposition 3: Gauge Invariance}

\begin{tcolorbox}[colback=green!5!white,colframe=green!75!black,title=Proposition 3]
\textbf{Statement}: Both $\mathcal{G}_{\mu\nu}$ and $g_{\mu\nu}$ are invariant under gauge transformations $\Theta \to U\Theta$ where $U \in G$ satisfies $U^\dagger U = \mathbf{1}$.
\end{tcolorbox}

\textbf{Proof sketch}:

Under the gauge transformation $\Theta \to \Theta' = U\Theta$, the covariant derivative transforms as:
\begin{equation}
D_\mu\Theta \to D_\mu\Theta' = D_\mu(U\Theta) = (\partial_\mu U)\Theta + U(\partial_\mu\Theta) + A'_\mu U\Theta ,
\end{equation}
where $A'_\mu = U A_\mu U^\dagger + U \partial_\mu U^\dagger$ is the transformed connection.

Substituting and simplifying:
\begin{equation}
D_\mu\Theta' = (\partial_\mu U)\Theta + U\partial_\mu\Theta + (U A_\mu U^\dagger + U \partial_\mu U^\dagger) U\Theta .
\end{equation}

Using $U^\dagger U = \mathbf{1}$, we have $\partial_\mu U^\dagger U + U^\dagger \partial_\mu U = 0$, so $U \partial_\mu U^\dagger = -(\partial_\mu U) U^\dagger$. Therefore:
\begin{equation}
D_\mu\Theta' = (\partial_\mu U)\Theta + U\partial_\mu\Theta + U A_\mu\Theta - (\partial_\mu U)\Theta = U(\partial_\mu\Theta + A_\mu\Theta) = U D_\mu\Theta .
\end{equation}

This is the standard gauge-covariant transformation law. Now, computing the transformed metric:
\begin{equation}
\mathcal{G}'_{\mu\nu} = (D_\mu\Theta')^\dagger D_\nu\Theta' = (U D_\mu\Theta)^\dagger (U D_\nu\Theta) = (D_\mu\Theta)^\dagger U^\dagger U D_\nu\Theta .
\end{equation}

Since $U^\dagger U = \mathbf{1}$, we obtain:
\begin{equation}
\mathcal{G}'_{\mu\nu} = (D_\mu\Theta)^\dagger D_\nu\Theta = \mathcal{G}_{\mu\nu} .
\end{equation}

Thus, $\mathcal{G}_{\mu\nu}$ is gauge-invariant. Taking the real part:
\begin{equation}
g'_{\mu\nu} = \text{Re}(\mathcal{G}'_{\mu\nu}) = \text{Re}(\mathcal{G}_{\mu\nu}) = g_{\mu\nu} ,
\end{equation}
confirming that $g_{\mu\nu}$ is also gauge-invariant. \qed

\subsection{Summary}

We have rigorously established:
\begin{enumerate}
\item The biquaternionic metric $\mathcal{G}_{\mu\nu} := (D_\mu\Theta)^\dagger D_\nu\Theta$ transforms as a $(0,2)$ tensor under coordinate transformations.
\item Its real projection $g_{\mu\nu} := \text{Re}(\mathcal{G}_{\mu\nu})$ is a real-valued, symmetric $(0,2)$ tensor.
\item Both $\mathcal{G}_{\mu\nu}$ and $g_{\mu\nu}$ are invariant under unitary gauge transformations $\Theta \to U\Theta$ with $U^\dagger U = \mathbf{1}$.
\end{enumerate}

These properties ensure that the metric derived from the biquaternionic field $\Theta$ behaves correctly under all relevant symmetry transformations, providing a consistent geometric foundation for the Unified Biquaternion Theory.
