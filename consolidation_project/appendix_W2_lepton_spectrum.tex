
\appendix
\section*{Appendix W: Lepton Mass Ratios from Toroidal Eigenmodes}

\begin{center}
\fbox{\parbox{0.9\linewidth}{%
\textbf{Reproducibility audit notice (2026-02-28).}
A forensic audit (\texttt{reports/lepton\_audit/}) shows that eq.~(W.2) as written
yields $E_{0,2}/E_{0,1}\approx 1.844$ and $E_{1,0}/E_{0,2}\approx 0.728$ (wrong
direction), not $207.3$ and $16.9$.  The values 207.3 and 16.9 are
\emph{experimental reference values}, not predictions derived from eq.~(W.2).
Sections W.4--W.5 and table W.T have been updated accordingly.
See \texttt{tools/reproduce\_lepton\_ratios.py} and
\texttt{reports/lepton\_audit/reproduction.md} for details.
}}
\end{center}

\subsection*{W.1 Motivation}
In Appendix~K we argued that the electron mass arises as the first non-trivial eigenmode of the internal Dirac operator on the compact torus $T^2(\tau)$ with Hosotani background. 
It is then natural to conjecture that higher charged leptons ($\mu,\tau$) correspond to higher eigenmodes of the same operator. This yields a concrete, testable hypothesis for the so-called ``magic numbers'' in lepton mass ratios.

\subsection*{W.2 Dirac Operator on $T^2(\tau)$}
The internal Dirac operator on the torus reads
\begin{equation}
D \;=\; i\gamma^\psi \!\left(\partial_\psi + i Q \theta_H/L_\psi\right) \;+\; i\gamma^\phi \partial_\phi ,
\end{equation}
with eigenfunctions labelled by integers $(n,m)\in\mathbb{Z}^2$.
The Hosotani background induces shifts $(\delta,\delta')$, leading to the spectrum
\begin{equation}
\label{eq:W2_spectrum}
E_{n,m} \;=\; \frac{1}{R}\,\sqrt{(n+\delta)^2+(m+\delta')^2}.
\end{equation}

\subsection*{W.3 Electron as $(0,1)$}
For $Q=-1$ and $\theta_H=\pi$ we obtain $\delta=\tfrac{1}{2}$ along the Wilson cycle, so the lowest non-trivial mode is $(n,m)=(0,1)$ with
\begin{equation}
m_e \;=\; E_{0,1} \;\simeq\; \frac{1}{R}.
\end{equation}

\subsection*{W.4 Higher Modes --- Conjecture (Open)}
We conjecture that heavier leptons correspond to higher eigenmodes.
Natural mode candidates are:
\begin{align}
m_\mu &\;\sim\; E_{0,2}, \\
m_\tau &\;\sim\; E_{0,3} \;\text{ or higher}.
\end{align}
\textbf{Important caveat.}
Evaluating eq.~\eqref{eq:W2_spectrum} with $\delta=\tfrac{1}{2}$, $\delta'=0$ gives
\[
\frac{E_{0,2}}{E_{0,1}} = \sqrt{\frac{0.25+4}{0.25+1}} \approx 1.844,
\]
which is \emph{two orders of magnitude smaller} than the experimental ratio $m_\mu/m_e\approx 206.8$.
Furthermore, $E_{1,0}<E_{0,2}$ under eq.~\eqref{eq:W2_spectrum}, so
mode $(1,0)$ cannot represent the $\tau$ without violating the mass hierarchy.
The mode assignment in this section remains a qualitative conjecture;
no closed derivation from eq.~\eqref{eq:W2_spectrum} currently yields the experimental
ratios.  See \texttt{reports/lepton\_audit/missing\_step.md} for a systematic
analysis of why all straightforward generalisations (rectangular modulus $y_*$,
power-law mapping, 3-torus mode counting) also fail.

\subsection*{W.5 Experimental Reference Values (not derived from eq.~W.2)}
The experimentally observed ratios (PDG 2022) are listed below for reference.
These values are \textbf{not predictions of eq.~\eqref{eq:W2_spectrum}}.
An earlier draft of this appendix incorrectly presented them as derived results;
that error is documented in \texttt{reports/lepton\_audit/reproduction.md}.
\begin{align}
\left(\frac{m_\mu}{m_e}\right)_{\rm exp} &= 206.768\,283 \quad (\text{OPEN: not derived from eq.~W.2}), \\
\left(\frac{m_\tau}{m_\mu}\right)_{\rm exp} &= 16.817\,16 \quad (\text{OPEN: not derived from eq.~W.2}).
\end{align}

\subsection*{W.6 Discussion and Open Problems}
A complete derivation of the lepton mass hierarchy within the toroidal eigenmode framework
requires at minimum one of:
\begin{itemize}
\item A corrected eigenvalue formula that explicitly incorporates the torus modulus
      $\tau_*$ from Appendix~V and produces ratios $\approx 207$ and $\approx 16.8$
      with at most one calibration parameter.
\item An explicit nonlinear mass mapping $M(E)$ with a physical justification that
      simultaneously gives the right ratios for \emph{both} $\mu/e$ and $\tau/\mu$.
\end{itemize}
Neither ingredient is currently present.  The ``magic numbers'' $207$ and $3477$ are
\emph{experimental observations}, not predictions of the theory as presently formulated.
Future work should close this gap.

\subsection*{W.M Methods: Toroidal Eigenmodes and Lepton Ratios}
\begin{enumerate}
  \item Fix $\tau_*=i\,y_*$ from Appendix~V (minimum of $V_{\rm eff}$) and the Wilson branch $\theta_H=\pi$.
  \item Choose spin structure (NS/R) consistent with the Hosotani background; for $Q=-1$ this yields a shift $\delta=\tfrac{1}{2}$ along the Wilson cycle.
  \item Solve the internal Dirac spectrum on $T^2(\tau_*)$: 
  \[
  E_{n,m}=\frac{1}{R}\sqrt{(n+\delta)^2+(m+\delta')^2}\,,
  \]
  with $(n,m)\in\mathbb{Z}^2$ and the second shift $\delta'$ fixed by the spin structure.
  \item Identify $m_e=E_{0,1}$ as the first non-trivial mode.
        \textbf{Note:} the naive ratios $E_{0,2}/E_{0,1}\approx 1.844$ and
        $E_{1,0}/E_{0,2}\approx 0.728$ do \emph{not} match experiment;
        the derivation of $m_\mu/m_e$ from this formula is currently open.
  \item Provide the table of eigenvalues $E_{n,m}/E_{0,1}$ and document any gap
        relative to experimental values. 
\end{enumerate}

\subsection*{W.T First Eigenmodes (formula values vs.\ experimental reference)}
\begin{center}
\begin{tabular}{c c c c c}
\toprule
Mode $(n,m)$ & $E_{n,m}\cdot R$ & Formula ratio to $E_{0,1}$ & Exp.\ ratio (ref.) & Candidate \\
\midrule
$(0,1)$ & $\sqrt{0.25+1}=1.118$ & $1.000$ & $1.000$ & $e$ \\
$(0,2)$ & $\sqrt{0.25+4}=2.062$ & $\mathbf{1.844}$ & $206.77\;(\text{exp})$ & $\mu$ (conjecture) \\
$(1,0)$ & $\sqrt{2.25+0}=1.500$ & $\mathbf{1.342}$ & $3477\;(\text{exp})$ & \emph{lighter than $\mu$; cannot be $\tau$} \\
$(0,3)$ & $\sqrt{0.25+9}=3.041$ & $\mathbf{2.720}$ & --- & higher mode \\
\bottomrule
\end{tabular}
\end{center}
\noindent\textbf{The ``Formula ratio'' column is computed from eq.~\eqref{eq:W2_spectrum}.
The ``Exp.\ ratio'' column lists experimental PDG values.
These two columns are \emph{not equal}; the gap is the open problem of this appendix.}
