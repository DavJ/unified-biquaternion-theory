% ==================================================================
% Appendix QA — Quarks and CKM in Unified Biquaternion Theory (UBT)
% ==================================================================
\appendix
\section*{Appendix QA: Quarks and CKM in Unified Biquaternion Theory}

\subsection*{Scope and Relationship to Core}
This appendix extends the geometric/toroidal mode analysis developed for leptons in Appendix~W
and the emergent fine-structure framework in Appendix~V to the quark sector.
It provides (i) an assignment of quark flavors $(u,d,s,c,b,t)$ to discrete internal modes
on the same torus geometry, (ii) mass ratios derived purely from mode integers and spin-structure/holonomy choices,
(iii) an origin for CKM mixing angles/phases as overlap integrals of nearby modes,
and (iv) a derivation of $\Lambda_{\mathrm{QCD}}$ from the same scale (cross-referencing Appendix~K.5).

\subsection*{Design Principles (Roadmap Alignment)}
\begin{enumerate}
  \item \textbf{No continuous fits:} Only integer mode labels and discrete spin-structure/holonomy choices are allowed.
  \item \textbf{Scheme independence:} Results are shown both in a physical scheme and in $\overline{\mathrm{MS}}$ (cf. Appendix~K).
  \item \textbf{Shared geometry:} The internal torus and boundary conditions are the same as used for $\alpha$ (Appendix~V) and leptons (Appendix~W).
  \item \textbf{Consistency:} Embedding respects the SM gauge structure as consolidated in Appendix~E.
\end{enumerate}

\section{Mode Assignment for Quark Flavors}
\label{sec:QA_modes}
We postulate that each quark flavor corresponds to a normal mode on the internal torus $\mathbb{T}^2$ with
integer labels $\mathbf{n}=(n_1,n_2)\in\mathbb{Z}^2$, spin structure $\sigma\in\{\pm\}$,
and gauge/holonomy class $h\in \mathcal{H}$ (as in Appendix~V).
The (tree-level) spectral relation generalizes the lepton case (Appendix~W) as
\begin{equation}
  m_q^{(0)}(\mathbf{n},\sigma,h) = \frac{1}{R}\,\mathcal{F}_q\!\left(\mathbf{n},\sigma,h;\tau\right),
  \label{eq:QA_masses_tree}
\end{equation}
where $R$ is the same geometric scale entering $V_{\rm eff}$ (Appendix~V)
and $\tau$ encodes complex structure/holonomies.
Radiative corrections and renormalization-group (RG) flow are treated analogously to the lepton analysis.

\subsection{Generational Structure and Hierarchies}
A minimal consistent assignment uses adjacent mode shells for generations:
\begin{align}
  (u,d) &: \ \mathbf{n}\in \mathcal{S}_1, \nonumber\\
  (s,c) &: \ \mathbf{n}\in \mathcal{S}_2, \nonumber\\
  (b,t) &: \ \mathbf{n}\in \mathcal{S}_3, \label{eq:QA_shells}
\end{align}
with $\mathcal{S}_k$ denoting shells defined by $|\mathbf{n}|^2$ and spin structure $\sigma$.
Top-quark enhancement arises for specific $(\sigma,h)$ that lift the degeneracy and raise the mode energy
without introducing continuous tuning.

\section{Quark Mass Ratios Without Tuning}
\label{sec:QA_mass_ratios}
The predictive content lies in \emph{ratios} that cancel $R$ and common factors:
\begin{equation}
  \frac{m_{q_i}}{m_{q_j}} \approx
  \frac{\mathcal{F}_{q_i}(\mathbf{n}_i,\sigma_i,h_i;\tau)}{\mathcal{F}_{q_j}(\mathbf{n}_j,\sigma_j,h_j;\tau)}
  \times \left[1+\Delta^{\rm RG}_{ij} + \Delta^{\rm th}_{ij}\right],
  \label{eq:QA_ratios}
\end{equation}
where $\Delta^{\rm RG}_{ij}$ accounts for scheme-dependent running between common scales
and $\Delta^{\rm th}_{ij}$ collects higher-order threshold effects (see Appendix~K for scheme discussion).
\paragraph{Checkpoint (Roadmap):} Provide integer assignments $(\mathbf{n},\sigma,h)$ such that
$(u{:}d{:}s{:}c{:}b{:}t)$ match experimental ratios at the sub-percent level, using only discrete choices.

\section{CKM Mixing from Mode Overlaps}
\label{sec:QA_ckm}
Let $\{\psi_{q}(\mathbf{y})\}$ denote normalized internal wavefunctions on $\mathbb{T}^2$
for the assigned modes in Sec.~\ref{sec:QA_modes}.
Yukawa-like effective couplings inherit overlap integrals of the form
\begin{equation}
  Y_{ij} \propto \int_{\mathbb{T}^2}\! d^2y \ \psi^{(u)}_i(\mathbf{y})\,\Phi(\mathbf{y})\,\psi^{(d)}_j(\mathbf{y}),
  \label{eq:QA_overlap}
\end{equation}
with a background profile $\Phi$ fixed by the same holonomies that minimize $V_{\rm eff}$ (Appendix~V).
Diagonalization of $Y_u$ and $Y_d$ yields the CKM matrix $V_{\rm CKM}=U_u^\dagger U_d$.
\paragraph{Phases and CP violation.}
Complex holonomies/spin structures generate physical phases in $Y_{ij}$, producing a CKM phase $\delta$ without ad hoc parameters.

\section{The QCD Scale \texorpdfstring{$\Lambda_{\mathrm{QCD}}$}{Lambda\_QCD} from Geometry}
\label{sec:QA_lambda_qcd}
We link $\Lambda_{\mathrm{QCD}}$ to the same geometric scale $R$ and complex structure $\tau$ via
\begin{equation}
  \Lambda_{\mathrm{QCD}} \sim \frac{1}{R}\,\exp\!\Big(-\frac{8\pi^2}{b_0\,g_3^2(R^{-1})}\Big)\,\Xi(\tau,\text{holonomy}) ,
  \label{eq:QA_lambdaQCD}
\end{equation}
with $b_0$ the one-loop QCD beta-function coefficient, $g_3$ evaluated at $\mu\!=\!R^{-1}$,
and $\Xi$ a discrete factor capturing holonomy-induced threshold shifts (to be cross-checked in Appendix~K.5).
\paragraph{Checkpoint (Roadmap):} Provide a quantitative estimate of $\Lambda_{\mathrm{QCD}}$ consistent with hadronic phenomenology
and verify stability under scheme changes (physical vs.\ $\overline{\mathrm{MS}}$).

\section{Predictions, Constraints, and Tests}
\label{sec:QA_tests}
\begin{itemize}
  \item \textbf{No continuous fits:} integer $(\mathbf{n},\sigma,h)$ fully determine mass ratios and CKM parameters.
  \item \textbf{Precision tests:} small, correlated shifts in CKM angles and rare-decay amplitudes arising from overlap structure.
  \item \textbf{Robustness:} vary spin structure/holonomies over the discrete set; require stability of ratios and CKM entries.
  \item \textbf{Cross-references:} consistency with Appendices E (SM/QCD embedding), K (schemes/constants), V (holonomy minimum), W (mode spectroscopy).
\end{itemize}

\section*{Summary}
Quark masses and CKM mixing follow from the same geometric mechanism that produced $\alpha$ and lepton masses.
The framework is predictive (integer data only), scheme-aware (Appendix K), and SM-compatible (Appendix E).
Filling in the explicit integer assignments, overlap integrals, and the $\Lambda_{\mathrm{QCD}}$ estimate completes the Quarks item of the roadmap.
