% ================== Yukawa Couplings in H_C ==================
% VERSION: v1.0 - Program Sketch
% AUTHOR: UBT Team
% PURPOSE: Outline the algebraic form of Yukawa couplings on the Hermitian slice
%
% DEPENDENCIES:
% Requires: \usepackage{amsmath,amssymb,amsthm}
% References: Appendix P6 (Lorentz in H_C), Appendix E (SM geometry)

\section{Yukawa Couplings in \texorpdfstring{$\mathbb H_{\mathbb C}$}{H\_C}}
\label{app:yukawa-in-HC}

\subsection{Overview}

In the Standard Model, fermion masses arise from Yukawa couplings to the Higgs field. For UBT to predict fermion masses from first principles, we must:
\begin{enumerate}
  \item Derive the algebraic form of Yukawa couplings on the Hermitian slice of \(\mathbb H_{\mathbb C}\).
  \item Show how left/right actions and involutions constrain the coupling structure.
  \item Connect to the discrete symmetries (P, T, C) via the three involutions.
\end{enumerate}

This appendix provides a program sketch. Full derivations are in progress.

\subsection{Algebraic Structure}

The Yukawa interaction in the Standard Model has the form:
\[
\mathcal L_{\text{Yukawa}} = -Y_{ij} \bar{\psi}_L^i \phi \psi_R^j + \text{h.c.}
\]
where \(Y_{ij}\) are the Yukawa coupling matrices (one for each fermion sector).

In the biquaternionic framework:
\begin{itemize}
  \item Fermions are represented as left ideals in \(\mathbb H_{\mathbb C}\).
  \item The Higgs field \(\phi\) emerges from the Hermitian slice structure.
  \item Left/right actions distinguish chiral components.
\end{itemize}

\paragraph{Constraint from involutions.}
The three involutions (complex conjugation, quaternionic conjugation, Hermitian adjoint) impose restrictions on the allowed Yukawa structures. A key question is whether these constraints are sufficient to determine the coupling ratios \(Y_{ij}/Y_{kl}\) without empirical input.

\subsection{Connection to Geometric Invariants}

The goal is to express Yukawa couplings in terms of geometric invariants on the Hermitian slice:
\begin{itemize}
  \item Overlap integrals of mode functions.
  \item Volume measures from the determinant.
  \item Topological winding numbers.
\end{itemize}

If successful, this would provide a ``locking'' mechanism analogous to the geometric locking for \(\alpha\).

\subsection{Roadmap}

\begin{enumerate}
  \item \textbf{Formalize fermion representations}: Define left/right ideals explicitly.
  \item \textbf{Derive coupling structure}: Show how involutions constrain \(Y_{ij}\).
  \item \textbf{Connect to topology}: Express couplings via winding numbers or other invariants.
  \item \textbf{Falsifiable predictions}: Derive mass ratios \(m_e/m_\mu\), \(m_\mu/m_\tau\), etc.
\end{enumerate}

\textbf{Status}: Framework outlined; detailed calculations in progress.

% ================== END ==================
