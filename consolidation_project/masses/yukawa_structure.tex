% ================== Yukawa Structure in UBT ==================
% AUTHOR: UBT Team
% PURPOSE: Algebraic form of Yukawa couplings in the biquaternionic framework

\section{Yukawa Couplings in \texorpdfstring{$\mathbb{H}_{\mathbb{C}}$}{H_C}}
\label{sec:yukawa-structure}

\subsection{General Framework}

In the Unified Biquaternion Theory, fermion masses arise from Yukawa couplings between
the Higgs field \(\Phi\) and fermion fields \(\Psi_L, \Psi_R\) within the biquaternionic
framework. The general form of the Yukawa Lagrangian is:
\begin{equation}
\label{eq:yukawa-general}
\mathcal{L}_{\text{Yukawa}} = -y_{ij} \overline{\Psi}_{L,i} \Phi \Psi_{R,j} + \text{h.c.}
\end{equation}
where \(y_{ij}\) are the Yukawa coupling matrices for different fermion sectors (quarks, leptons).

\subsection{Biquaternionic Constraints}

The biquaternionic structure \(\mathbb{H}_{\mathbb{C}}\) imposes geometric constraints on
the Yukawa matrices. Specifically:

\paragraph{Hermiticity conditions.} On the Hermitian slice used to realize Lorentzian signature
(Appendix~P6), the Yukawa couplings must satisfy reality conditions that relate left- and
right-handed sectors:
\begin{equation}
y_{ij}^* = \mathcal{C}_{ik} y_{kl} \mathcal{C}_{lj}^{-1}
\end{equation}
where \(\mathcal{C}\) is the charge conjugation operator in the biquaternionic representation.

\paragraph{Flavor structure.} The three-generation structure of the Standard Model emerges
from the dimensional reduction of the biquaternionic field onto the Hermitian slice. The
Yukawa matrices naturally decompose as:
\begin{equation}
y = y_0 \mathbb{1}_3 + y_a T^a + y_{ab} [T^a, T^b] + \cdots
\end{equation}
where \(T^a\) are generators of the flavor symmetry group.

\subsection{Hierarchy Problem and Geometric Scales}

The observed fermion mass hierarchy (top quark: 173 GeV, electron: 0.511 MeV) suggests
geometric origin rather than fine-tuning. Within UBT, we conjecture that:

\begin{itemize}
  \item The top quark mass sets the electroweak scale and couples directly to the vacuum
  expectation value of \(\Phi\).
  \item Lighter fermions couple through suppressed channels involving higher-order
  terms in the biquaternionic expansion.
  \item The mass ratios may be expressible in terms of geometric invariants of the
  Hermitian slice.
\end{itemize}

Further development of this program requires explicit computation of the mode structure
and holonomy constraints (see Section~\ref{sec:rg-flows} for renormalization-group analysis).

% =======================================================================
