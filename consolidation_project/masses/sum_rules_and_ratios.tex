% ================== Sum Rules and Ratios for Fermion Masses ==================
% VERSION: v1.0 - Program Sketch
% AUTHOR: UBT Team
% PURPOSE: Outline candidate mass sum rules as falsifiable predictions
%
% DEPENDENCIES:
% Requires: \usepackage{amsmath,amssymb,amsthm}
% References: yukawa_in_HC.tex, ct_renorm_strategy.tex

\section{Sum Rules and Mass Ratios}
\label{app:mass-sum-rules}

\subsection{Overview}

A key testable prediction from UBT would be relations among fermion masses that are **independent of the overall mass scale**. Such sum rules would:
\begin{itemize}
  \item Be falsifiable by experiment (measure mass ratios to high precision).
  \item Not require knowledge of the absolute scale (which may depend on Higgs vev or other inputs).
  \item Provide strong constraints on the Yukawa structure.
\end{itemize}

\subsection{Example: Lepton Mass Ratios}

Consider the charged lepton masses \(m_e, m_\mu, m_\tau\). In the Standard Model, these are independent parameters (three Yukawa eigenvalues). UBT might predict relations such as:
\[
\frac{m_\mu}{m_e} = f_1(N_{\mathrm{eff}}, R_\psi, \ldots),
\quad
\frac{m_\tau}{m_\mu} = f_2(N_{\mathrm{eff}}, R_\psi, \ldots),
\]
where \(f_1, f_2\) are functions derived from the biquaternionic geometry.

\paragraph{Experimental values (for comparison):}
\begin{align*}
\frac{m_\mu}{m_e} &\approx 206.77, \\
\frac{m_\tau}{m_\mu} &\approx 16.82.
\end{align*}

If UBT predicts different values, the theory is **falsified**. If it matches, we have a nontrivial confirmation.

\subsection{Candidate Sum Rules}

Here are some hypothetical sum rules that could emerge from the biquaternionic structure:

\begin{enumerate}
  \item \textbf{Ratio of ratios}: 
  \[
  \frac{m_\tau/m_\mu}{m_\mu/m_e} = \text{(simple rational number or function of } N_{\mathrm{eff}}\text{)}.
  \]
  
  \item \textbf{Winding number constraint}:
  If fermion masses are related to topological winding, one might have:
  \[
  \frac{m_i}{m_j} = \left(\frac{n_i}{n_j}\right)^p,
  \]
  where \(n_i, n_j\) are integer winding numbers and \(p\) is a power determined by the geometry.
  
  \item \textbf{Logarithmic spacing}:
  Analogous to the Koide formula (which relates \(e, \mu, \tau\) masses), UBT might predict:
  \[
  \frac{m_e + m_\mu + m_\tau}{\sqrt{m_e^2 + m_\mu^2 + m_\tau^2}} = \text{(geometric constant)}.
  \]
\end{enumerate}

\textbf{Important}: These are **placeholders**. The actual sum rules must be **derived** from first principles, not fitted.

\subsection{Quark Sector}

Similar sum rules could apply to quark masses. The quark sector is more complex due to:
\begin{itemize}
  \item QCD running (masses are scale-dependent).
  \item CKM mixing (flavor structure).
  \item Three generations × two flavors = six parameters (up, down, charm, strange, top, bottom).
\end{itemize}

Deriving quark mass ratios from UBT would be a major achievement.

\subsection{Roadmap and Falsification}

\begin{enumerate}
  \item \textbf{Derive sum rules}: Use Yukawa structure from \(\mathbb H_{\mathbb C}\) to predict mass ratios.
  \item \textbf{Compare to data}: Check against PDG values.
  \item \textbf{Publish predictions}: Before measuring (to avoid hindsight bias).
  \item \textbf{Accept falsification}: If ratios disagree, revise or abandon the mass derivation program.
\end{enumerate}

\textbf{Status}: No specific predictions yet; program in exploratory phase.

\subsection{No Numerology}

We emphasize: **No fitting allowed**. Any sum rule must emerge from the biquaternionic structure and CT renormalization logic. Post-hoc fits to match experimental values would undermine the scientific integrity of UBT.

% ================== END ==================
