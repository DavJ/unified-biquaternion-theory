
\documentclass[12pt]{article}
\usepackage{amsmath,amssymb,amsfonts}
\usepackage{bm}
\usepackage{physics}

\title{Noether $\to$ $\alpha$ v0.5: Current Matching and What Fixes $g_5$}
\author{Draft for UBT Project}
\date{\today}

\begin{document}
\maketitle

\section{Goal}
We derived
\begin{equation}
\alpha(\mu_0) \;=\; \frac{g_5^2}{4\pi\, Z^\star}, 
\qquad 
Z^\star = \int_0^{L_\psi^\star}\! d\psi\, e^{B-2A}\,|\xi_0(\psi)|^2
= L_\psi^\star f^\star(\tau,\mathrm{BC}).
\end{equation}
This section shows how the Noether charge and canonical normalization fix the \emph{field scale}, and how $g_5$ is or is not fixed, depending on the UV principle chosen in UBT.

\section{Zero-Mode Decomposition and Canonical Normalization}
Expand the unified field in $\psi$-modes (for simplicity: complex scalar prototype for the charged excitation)
\begin{equation}
\Theta(x,\psi) = \chi_0(\psi)\,\phi(x) + \cdots, 
\qquad 
\int_0^{L_\psi}\! d\psi\; e^{B+2A}\, |\chi_0(\psi)|^2 = 1.
\end{equation}
Similarly, the photon zero-mode (canonically normalized in 4D) is
\begin{equation}
A_\mu(x,\psi) = \xi_0(\psi)\, A_\mu^{(0)}(x) + \cdots, 
\qquad 
\int_0^{L_\psi}\! d\psi\; e^{B-2A}\, |\xi_0(\psi)|^2 = Z.
\end{equation}
With these conventions, the 4D kinetic term of $\phi$ is canonical,
\begin{equation}
\mathcal L_{\rm 4D} \supset |\partial_\mu \phi|^2 - m_0^2 |\phi|^2,
\end{equation}
and the 4D Noether current takes the standard form
\begin{equation}
j^\mu = i \left(\phi^\dagger \partial^\mu \phi - (\partial^\mu \phi^\dagger)\phi\right).
\end{equation}

\section{Current Matching: $g_4$ from the 5D Coupling}
The 5D interaction is $\int d^5x\,\sqrt{|g_5|}\; g_5\, J^\mu A_\mu$, which reduces to
\begin{equation}
S_{\rm int}^{(0)} = \int d^4x\, g_5 
\Big[\int_0^{L_\psi}\! d\psi\; e^{B}\, |\chi_0|^2\, \xi_0 \Big]\; j^\mu A_\mu^{(0)}
= \int d^4x\, \frac{g_5}{\sqrt{Z}}\; j^\mu A_\mu^{(0)}.
\end{equation}
Hence
\begin{equation}
g_4 \equiv \frac{g_5}{\sqrt{Z}}, \qquad \alpha = \frac{g_4^2}{4\pi}.
\end{equation}
The result is robust (spinor/vector generalizations give the same overlap structure).

\section{Noether Charge and Field Normalization}
The 5D Noether charge
\begin{equation}
Q = \int d^3x \int_0^{L_\psi}\! d\psi \,\sqrt{|g_5|}\, J^0
\end{equation}
reduces, for the normalized zero-mode, to the 4D number operator. Choosing the fundamental excitation to have $Q=\pm1$ \emph{fixes the normalization of $\phi$} and removes any free rescaling of $\Theta$. This does \emph{not} fix the numerical value of $g_5$ by itself.

\section{What Can Fix $g_5$ (and Thus $\alpha$) Ab-Initio?}
Noether symmetry and compactification give $\alpha = g_5^2/(4\pi Z^\star)$ but leave $g_5$ as a \emph{UV parameter}. To arrive at a number, UBT needs one of the following principles (each consistent with the framework):

\subsection*{(A) Bare Gauge Term From UBT Lagrangian + Extra Constraint}
Assume a tree-level gauge kinetic term $-\tfrac14 F^2$. Then $g_5$ is a fundamental 5D coupling of mass dimension $-1/2$ and is not fixed by Noether alone. It can be related to other UBT scales if UBT supplies a constraint, e.g.
\begin{itemize}
\item tying $g_5$ to the $\Theta$ sector (coupling unification in the biquaternionic algebra),
\item matching to gravity via a universal scale (Sakharov-like, but determined by the CCT/UBT cutoff),
\item or quantization coming from a topological term (e.g.\ a 5D abelian Chern--Simons coefficient that is quantized and links to minimal charge units).
\end{itemize}

\subsection*{(B) \emph{Induced} Photon Kinetic Term (Emergent Gauge Field)}
Set the bare coefficient to zero and generate $F^2$ entirely from loops of $\Theta$-sector fields. Compactification to 4D yields
\begin{equation}
\frac{1}{g_4^2(\mu)} \;=\; \sum_{j,\,\text{KK}} \frac{b_j}{8\pi^2}\, \ln\!\frac{\Lambda}{m_{j,{\rm KK}}}
\;+\; \cdots ,
\end{equation}
with $b_j$ the standard 4D beta-function weights for each KK mode and $\Lambda$ the physical UV scale of UBT (finite in CCT/UBT due to the compact $\psi$ and possible warping). In this scenario,
\begin{equation}
\alpha(\mu_0) \;\text{is fixed by field content, BC, and}\; (L_\psi^\star,\theta_H^\star,A,B)
\quad \text{(no $g_5$ input)}.
\end{equation}
This produces a numerical $\alpha$ once the spectrum is specified; the calculation is technically involved but conceptually parameter-free.

\subsection*{(C) Holonomy Quantization With Dynamical $A_\psi$}
If the vacuum fixes a nontrivial $\langle A_\psi\rangle$ through the Hosotani mechanism and UBT relates $\langle A_\psi\rangle$ to a geometric invariant, then
\begin{equation}
\theta_H^\star = g_5 \int_0^{L_\psi^\star}\!\! A_\psi\, d\psi = 2\pi n^\star
\end{equation}
imposes a relation among $g_5$ and $L_\psi^\star$ that \emph{can} remove the $g_5$ freedom. Whether this happens depends on the detailed UBT background equations; in many simple models $\theta_H^\star$ is $0$ or $\pi$ independently of $g_5$, leaving $g_5$ unfixed.

\section{Recommendation}
For a \textbf{fully ab-initio} numeric value of $\alpha$ within the present UBT framework, the cleanest route is (B): treat the photon as emergent and compute the induced $1/g_4^2$ from KK towers of the $\Theta$-sector at the vacuum point $(L_\psi^\star,\theta_H^\star,A,B)$. This uses only symmetries, geometry and field content, with no free gauge coupling left.

\section{Next Steps (Minimal Viable Calculation)}
\begin{enumerate}
\item Choose a concrete $\Theta$-sector (spins, charges $q_j$, masses $m_j$) and BC.
\item Determine $(L_\psi^\star,\theta_H^\star)$ by minimizing the one-loop $V_{\rm eff}$ (Sec.~v0.4).
\item Compute the KK spectrum $m_{j,{\rm KK}}(n)$ on that background.
\item Evaluate the induced gauge kinetic term in 4D, e.g.\ summing the standard one-loop polarizations over KK levels with a regulator consistent with UBT (compact $\psi$ and warp imply a physical cutoff).
\item Read off $\alpha(\mu_0)$, then run to $M_Z$ for comparison.
\end{enumerate}

\paragraph{Consistency with $m_e$.}
The same background must yield the lightest charged eigenmode mass $m_e$; matching both $\alpha$ and $m_e$ with a \emph{single} background is the decisive non-numerological test.


\section*{License}
This work is licensed under a Creative Commons Attribution 4.0 International License (CC BY 4.0).

\end{document}