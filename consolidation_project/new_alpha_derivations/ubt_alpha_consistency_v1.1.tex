
\documentclass[12pt]{article}
\usepackage{amsmath,amssymb,amsthm,amsfonts}

\title{UBT $\to$ $\alpha$: Self-consistency equation v1.1}
\author{Draft for UBT Project}
\date{\today}

\begin{document}
\maketitle

\section*{Alpha-fixing consistency equation}

From routes (A) and (B) and the internal scale condition, UBT yields the \emph{self-consistency equation} for the fine-structure constant:
\begin{equation}\label{eq:alpha_consistency}
1 \;=\;
\frac{\theta_H^{\star 2}}{4\pi\,(\mathcal I_\psi^\star)^2}\;
\Bigg[
\underbrace{\Phi_{\rm UBT}\ \sum_B C_5^{(B)}\,\Tr_{\mathcal F_B}(Q^2)}_{\text{algebraic/topological invariant}}
\;+\;
\underbrace{\sum_B \frac{b_B}{4\pi}\!\int_0^{\Phi_{\rm UBT}/Z^\*}\!\!\!\rho_B(m)\,\ln\!\frac{\Phi_{\rm UBT}/Z^\*}{m}\,dm}_{\text{spectral log part}}
\Bigg].
\end{equation}

Every symbol is fixed by the UBT vacuum and its algebra:
\begin{itemize}
\item $\theta_H^\star$: discrete holonomy angle (Hosotani mechanism).  
\item $\mathcal I_\psi^\star$: background integral of $A_\psi$.  
\item $Z^\star$: photon zero-mode norm.  
\item $\Tr_{\mathcal F_B}(Q^2)$: invariant trace of the $U(1)$ generator on the $\Theta$-fiber block $B$.  
\item $\Phi_{\rm UBT}$: internal scale number, relating $\Lambda Z^\star=\Phi_{\rm UBT}$.  
\item $\rho_B(m)$: KK spectral density of block $B$ on the same vacuum.  
\end{itemize}

Equation \eqref{eq:alpha_consistency} fixes $\alpha$ without any external field input.

\section*{Lemmas}

\paragraph{Lemma 1 (Discrete holonomy).}
Large gauge transformations on $S^1_\psi$ enforce
\begin{equation}
\theta_H^\star = 2\pi n,\qquad n\in\mathbb Z\setminus\{0\}.
\end{equation}
Thus $\theta_H^\star$ is not a free parameter but quantized by topology.

\paragraph{Lemma 2 (Invariant trace).}
The quadratic Casimir of the $\Theta$-fiber representation yields
\begin{equation}
\sum_j d_j q_j^2 \;=\; \Tr_{\mathcal F}(Q^2).
\end{equation}
This number is fixed once the algebra of $\Theta$ is specified; no external ``field list'' enters.

\paragraph{Lemma 3 (Internal scale).}
Regularization in $D=5$ with preserved UBT symmetry identifies
\begin{equation}
\Lambda Z^\star = \Phi_{\rm UBT},
\end{equation}
where $\Phi_{\rm UBT}$ is a pure number determined by topology/warping/heat-kernel index. This eliminates the last dimensionful freedom.

\paragraph{Lemma 4 (Spectral logs).}
The KK tower contribution can be expressed as a spectral integral
\begin{equation}
\sum_B \frac{b_B}{4\pi}\!\int_0^{\Lambda}\rho_B(m)\ln\!\frac{\Lambda}{m}\,dm.
\end{equation}
In many vacua, symmetry enforces cancellations among blocks, or the integral is fully computable from the same $\Delta_\Theta$. Either way, no external input is required.

\section*{Theorem (UBT $\to$ $\alpha$).}
Given Lemmas 1--4, the self-consistency equation \eqref{eq:alpha_consistency} determines $\alpha$ uniquely as a pure number.
No external parameters or arbitrary field content are introduced.

\paragraph{Interpretation.}
UBT thereby predicts the fine-structure constant as an intrinsic invariant of its vacuum structure and algebraic fiber. 
Agreement with experiment (e.g.\ $\alpha^{-1}\approx137$ at low energy) would serve as a decisive confirmation.


\section*{License}
This work is licensed under a Creative Commons Attribution 4.0 International License (CC BY 4.0).

\end{document}