
\documentclass[12pt]{article}
\usepackage{amsmath,amssymb,amsfonts}
\usepackage{bm}
\usepackage{physics}

\title{Noether $\to$ $\alpha$ v0.4: Ab-initio fixing of $L_\psi$, $Z$, $g_5$ \\ \large with derivation of the resummed $V_{\rm eff}$ and massless limit}
\author{Draft for UBT Project}
\date{\today}

\begin{document}
\maketitle

\section{Setup: 5D Action, Geometry, BC}
We consider $M^4 \times S^1_\psi$ with coordinates $x^\mu$ and $\psi \sim \psi+L_\psi$.
Let $\Theta(x,\psi)$ denote the unified (biquaternionic) field and $A_M(x,\psi)$ a $U(1)$ gauge field.
The 5D action (natural units $c=\hbar=1$) is
\begin{equation}
S = \int d^4x \int_0^{L_\psi} d\psi\, \sqrt{|g_5|}\left[
\, g^{MN} (D_M\Theta)^\dagger (D_N\Theta)
 - V(\Theta) - \frac{1}{4} g^{MR} g^{NS} F_{MN}F_{RS}\,\right],
\label{eq:5Daction}
\end{equation}
with $D_M=\partial_M + i g_5 A_M$, $F_{MN}=\partial_M A_N-\partial_N A_M$.
We allow a warped background
\begin{equation}
ds^2 = e^{2A(\psi)} \eta_{\mu\nu}dx^\mu dx^\nu + e^{2B(\psi)} d\psi^2,
\qquad \sqrt{|g_5|}=e^{4A(\psi)+B(\psi)}.
\end{equation}
Boundary conditions (BC) along $S^1_\psi$ may be periodic or twisted (phase $\delta$).

\paragraph{Holonomy (Wilson line).}
The gauge-invariant holonomy (Hosotani parameter) is
\begin{equation}
\theta_H \equiv g_5 \oint_{S^1_\psi} \! A_\psi \, d\psi = g_5 \int_0^{L_\psi} d\psi\, A_\psi(x,\psi).
\label{eq:thetaH}
\end{equation}
Large gauge transformations shift $\theta_H \to \theta_H + 2\pi n$, $n\in\mathbb{Z}$.

\section{Reduction and Canonical Normalization}
Assume the photon zero-mode $A_\mu^{(0)}(x)$ is independent of $\psi$ and normalized canonically in 4D.
Reducing the gauge kinetic term in \eqref{eq:5Daction} gives
\begin{equation}
S_{\rm gauge} \supset -\frac{1}{4}\int d^4x\, Z\, F_{\mu\nu}^{(0)}F^{\mu\nu}_{(0)},
\qquad
Z \equiv \int_0^{L_\psi} d\psi\, e^{B(\psi)-2A(\psi)},
\label{eq:Zfactor}
\end{equation}
so the canonically normalized 4D photon is $A_\mu^{(0)} \to A_\mu^{(0)}/\sqrt{Z}$.
The covariant derivative contributes the interaction
\begin{equation}
\int d^4x \int_0^{L_\psi}\! d\psi \,\sqrt{|g_5|}\; J^\mu A_\mu
\;\;\longrightarrow\;\;
\int d^4x \; g_4\, J^\mu_{(0)} A^{(0)}_\mu,
\qquad
g_4 = \frac{g_5}{\sqrt{Z}},
\label{eq:g4}
\end{equation}
where $J^\mu$ is the Noether current density and $J^\mu_{(0)}$ its overlap with the photon zero-mode.
Therefore the fine-structure constant at a reference scale $\mu_0$ is
\begin{equation}
\boxed{ \;\alpha(\mu_0) \;=\; \frac{g_4^2}{4\pi} \;=\; \frac{g_5^2}{4\pi\, Z}\; }.
\label{eq:alpha-master}
\end{equation}

\section{Noether Charge and Current Matching}
Global $U(1)$: $\Theta \to e^{i\lambda} \Theta$ yields the 5D Noether current
\begin{equation}
J^M = i \left[ \Theta^\dagger (D^M \Theta) - (D^M \Theta)^\dagger \Theta \right] \Big|_{A=0}.
\end{equation}
We fix the normalization of $\Theta$ such that the fundamental charged excitation has unit Noether charge
\begin{equation}
Q \equiv \int d^3x \int_0^{L_\psi} d\psi \, \sqrt{|g_5|}\, J^0 = \pm 1.
\label{eq:Q1}
\end{equation}
This fixes the overall scale entering the coupling \eqref{eq:g4}; no additional free normalization survives.

\section{Wilson Line and Quantization}
On the compact $\psi$-cycle the holonomy \eqref{eq:thetaH} is physical.
For a field with $U(1)$ charge $q$ and BC phase $\delta$ the KK momenta are shifted by
\begin{equation}
p_\psi^{(n)} = \frac{2\pi}{L_\psi}\Big(n + a\Big),\qquad a \equiv \frac{q\,\theta_H}{2\pi} + \delta,\qquad n\in\mathbb{Z}.
\label{eq:shift}
\end{equation}
Stationary vacua (Hosotani mechanism) are determined dynamically and may select nontrivial $\theta_H^\star$.
Large gauge invariance enforces periodicity $\theta_H \sim \theta_H + 2\pi$; stable vacua satisfy $\partial V_{\rm eff}/\partial \theta_H=0$.

\section{One-Loop Effective Potential $V_{\rm eff}(L_\psi,\theta_H)$}
For each field $j$ with spin-statistics sign $\sigma_j=\pm 1$ (boson $+1$, fermion $-1$), degeneracy $d_j$, mass $m_j$,
charge $q_j$, and twist $\delta_j$, the one-loop contribution is
\begin{align}
V_j(L_\psi,\theta_H)
&=\frac{\sigma_j d_j}{2}\int \!\frac{d^4p}{(2\pi)^4}\sum_{n\in\mathbb{Z}}
\ln\!\Big[p^2 + m_j^2 + \big(\tfrac{2\pi}{L_\psi}\big)^2\big(n+a_j\big)^2\Big],
\label{eq:Vsum}
\\
a_j &\equiv \frac{q_j\,\theta_H}{2\pi} + \delta_j.
\end{align}
Using standard contour/Matsubara techniques (equivalently Poisson resummation) one obtains an exact resummed form
\begin{equation}
\boxed{
V_j(L_\psi,\theta_H) \;=\; \sigma_j d_j \int \!\frac{d^4p}{(2\pi)^4}\; \frac{1}{L_\psi}\;
\ln\!\Big(1 - 2 e^{-L_\psi \omega_j(p)} \cos 2\pi a_j + e^{-2 L_\psi \omega_j(p)}\Big) ,
}
\label{eq:Veff-resummed}
\end{equation}
where $\omega_j(p)=\sqrt{p^2+m_j^2}$. The total potential is
\begin{equation}
V_{\rm eff}(L_\psi,\theta_H) \;=\; \sum_j V_j(L_\psi,\theta_H).
\end{equation}

\paragraph{Massless scaling (correction).}
In the massless limit $m_j\to 0$, $V_j$ scales as $1/L_\psi^{5}$ (not $1/L_\psi^4$). This is consistent with dimensional analysis in 5D.

\section{Determining $Z$ (Warped Case)}
If the vacuum back-reacts and generates a nontrivial warp, the 4D gauge kinetic factor is
\begin{equation}
Z \;=\; \int_0^{L_\psi}\! d\psi\; e^{B(\psi)-2A(\psi)}\, \abs{\xi_0(\psi)}^2,
\label{eq:Zwarp}
\end{equation}
where $\xi_0(\psi)$ is the photon zero-mode profile (constant in flat space).
The warp factors $A(\psi),B(\psi)$ and $\xi_0(\psi)$ follow from the coupled background equations of motion
derived from \eqref{eq:5Daction} (e.g.\ via a first-order BPS system if available).
In flat space, $\xi_0(\psi)\equiv 1$ and $Z=L_\psi$. In general we write
\begin{equation}
Z \;=\; L_\psi \, f(\tau,\mathrm{BC}) ,
\end{equation}
where $f$ encodes the modular parameter $\tau$ (complex-time sector) and boundary data.

\section{Putting It Together: $\alpha$ from UBT}
Combining \eqref{eq:alpha-master}, the stationarity conditions and \eqref{eq:Zwarp},
\begin{equation}
\boxed{
\alpha(\mu_0) \;=\; \frac{g_5^2}{4\pi \, Z^\star}
\;=\; \frac{g_5^2}{4\pi \, L_\psi^\star f^\star(\tau,\mathrm{BC})},
\qquad
(L_\psi^\star,\theta_H^\star)\; \text{solve}\; \partial V_{\rm eff}=0.}
\end{equation}
Here $g_5$ is fixed by Noether normalization \eqref{eq:Q1} (unit charge for the fundamental excitation) and current matching \eqref{eq:g4}.
Thus, given the field content, BC, and UBT potential $V(\Theta)$, the pair $(L_\psi^\star,\theta_H^\star)$ and the warp $A,B$ are determined
and \emph{no free fit survives} in $\alpha$. Standard QED running then connects $\alpha(\mu_0)$ to experimental scales.

\section{Electron Mass Consistency (Sketch)}
The same background must produce the lightest charged KK eigenvalue
\begin{equation}
m_e^2 \;=\; \lambda_{\rm min}\Big[ -e^{-B}\partial_\psi\!\big( e^{B-2A}\partial_\psi \cdot \big) + m_\Theta^2 e^{-2A} + \cdots \Big],
\end{equation}
for the appropriate (spinor) sector and BC, including the holonomy shift \eqref{eq:shift}.
This provides a nontrivial cross-check: the \emph{same} $(L_\psi^\star,\theta_H^\star, A,B)$ that gives $\alpha$ must also yield $m_e$.

\appendix

\section*{Appendix A: Derivation of the Resummed Form \eqref{eq:Veff-resummed}}
Starting from \eqref{eq:Vsum}, define $\omega=\sqrt{p^2+m^2}$ and write
\begin{equation}
\sum_{n\in\mathbb{Z}}\ln\!\Big[ \big(n+a\big)^2 + \big(\tfrac{L_\psi \omega}{2\pi}\big)^2 \Big]
= \sum_{n\in\mathbb{Z}} \Big\{ \ln\!\big( n+a + i \tfrac{L_\psi\omega}{2\pi} \big) + \ln\!\big( n+a - i \tfrac{L_\psi\omega}{2\pi} \big) \Big\}.
\end{equation}
Using the product representation for $\sin(\pi z)$
and standard finite-temperature/Matsubara algebra, one obtains (up to $p$- and $m$-independent constants that drop out)
\begin{equation}
\sum_{n\in\mathbb{Z}}\ln\!\Big[ \big(n+a\big)^2 + b^2 \Big]
= \ln\!\Big(1 - 2 e^{-2\pi b} \cos 2\pi a + e^{-4\pi b}\Big) + \text{const},
\end{equation}
where $b=L_\psi \omega/(2\pi)$.
Restoring factors and integrating over $p$, we get \eqref{eq:Veff-resummed}:
\begin{equation}
V(L_\psi,\theta_H) \;=\; \sigma d \int \!\frac{d^4p}{(2\pi)^4}\; \frac{1}{L_\psi}\;
\ln\!\Big(1 - 2 e^{-L_\psi \omega(p)} \cos 2\pi a + e^{-2 L_\psi \omega(p)}\Big).
\end{equation}
An equivalent route is to differentiate with respect to $m^2$, perform the geometric series or Poisson resummation, and integrate back.

\section*{Appendix B: Massless Limit and Polylogarithms}
For $m\to 0$, set $\omega(p)=|p|$ and use the factorization
\begin{equation}
\ln\!\big(1 - 2 c\, t + t^2\big) = \ln(1-e^{i\theta} t)+\ln(1-e^{-i\theta} t)
= - \sum_{k=1}^\infty \frac{2 \cos(k\theta)}{k}\, t^k,
\qquad c=\cos\theta,\; t=e^{-L_\psi |p|}.
\end{equation}
Then
\begin{align}
V_{\rm massless}(L_\psi,\theta_H)
&= - \frac{2\sigma d}{L_\psi} \sum_{k=1}^\infty \frac{\cos(k\theta)}{k}\;
\int \!\frac{d^4p}{(2\pi)^4}\, e^{-k L_\psi |p|}
\\[4pt]
&= - \frac{2\sigma d}{L_\psi} \sum_{k=1}^\infty \frac{\cos(k\theta)}{k}\;
\frac{2\pi^2}{(2\pi)^4} \int_0^\infty\! dp\, p^3\, e^{-k L_\psi p}
\\[4pt]
&= - \frac{2\sigma d}{L_\psi} \sum_{k=1}^\infty \frac{\cos(k\theta)}{k}\;
\frac{2\pi^2}{(2\pi)^4} \cdot \frac{3!}{(k L_\psi)^4}
\\[4pt]
&= - \frac{3\,\sigma d}{2\pi^2\, L_\psi^5} \sum_{k=1}^\infty \frac{\cos(k\theta)}{k^5}
\;=\; - \frac{3\,\sigma d}{2\pi^2\, L_\psi^5}\; \Re\, \mathrm{Li}_5\!\big(e^{i\theta}\big),
\qquad \theta = 2\pi a.
\end{align}
Thus, in 5D the massless contribution scales as $1/L_\psi^{5}$ with a universal (non-fitted) coefficient.

\paragraph{Massive case (asymptotics).}
For $m>0$, using $\ln(1-2ct+t^2)= -2\sum_{k\ge 1} \cos(k\theta)\, t^k/k$ with $t=e^{-L_\psi \omega}$ and the known integral
\begin{equation}
\int \!\frac{d^4p}{(2\pi)^4}\, e^{-k L_\psi \sqrt{p^2 + m^2}}
= \frac{m^3}{(2\pi)^2 (k L_\psi)^3}\, K_3(k m L_\psi),
\end{equation}
one finds the exact series
\begin{equation}
V(L_\psi,\theta_H;m)
= - \frac{2\sigma d}{L_\psi} \sum_{k=1}^\infty \frac{\cos(k\theta)}{k}\;
\frac{m^3}{(2\pi)^2 (k L_\psi)^3}\, K_3(k m L_\psi),
\end{equation}
which reduces to the polylogarithmic result above as $m\to 0$ (using $K_\nu(z)\sim 2^{\nu-1}\Gamma(\nu) z^{-\nu}$).


\section*{License}
This work is licensed under a Creative Commons Attribution 4.0 International License (CC BY 4.0).

\end{document}