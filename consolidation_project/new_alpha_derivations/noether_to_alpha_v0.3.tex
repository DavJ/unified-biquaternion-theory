
\documentclass[12pt]{article}
\usepackage{amsmath,amssymb,amsfonts}
\usepackage{bm}
\usepackage{physics}

\title{Noether $\to$ $\alpha$ v0.3: Fixing $L_\psi$, $Z$ and $g_5$ ab-initio}
\author{Draft for UBT Project}
\date{\today}

\begin{document}
\maketitle

\section{Setup: 5D Action, Geometry, BC}
We consider $M^4 \times S^1_\psi$ with coordinates $x^\mu$ and $\psi \sim \psi+L_\psi$.
Let $\Theta(x,\psi)$ denote the unified (biquaternionic) field and $A_M(x,\psi)$ a $U(1)$ gauge field.
The 5D action (natural units $c=\hbar=1$) is
\begin{equation}
S = \int d^4x \int_0^{L_\psi} d\psi\, \sqrt{|g_5|}\left[
\, g^{MN} (D_M\Theta)^\dagger (D_N\Theta)
 - V(\Theta) - \frac{1}{4} g^{MR} g^{NS} F_{MN}F_{RS}\,\right],
\label{eq:5Daction}
\end{equation}
with $D_M=\partial_M + i g_5 A_M$, $F_{MN}=\partial_M A_N-\partial_N A_M$.
We allow a warped background
\begin{equation}
ds^2 = e^{2A(\psi)} \eta_{\mu\nu}dx^\mu dx^\nu + e^{2B(\psi)} d\psi^2,
\qquad \sqrt{|g_5|}=e^{4A(\psi)+B(\psi)}.
\end{equation}
Boundary conditions (BC) along $S^1_\psi$ may be periodic or twisted (phase $\delta$).

\paragraph{Holonomy (Wilson line).}
The gauge-invariant holonomy (Hosotani parameter) is
\begin{equation}
\theta_H \equiv g_5 \oint_{S^1_\psi} \! A_\psi \, d\psi = g_5 \int_0^{L_\psi} d\psi\, A_\psi(x,\psi).
\label{eq:thetaH}
\end{equation}
Large gauge transformations shift $\theta_H \to \theta_H + 2\pi n$, $n\in\mathbb{Z}$.

\section{Reduction and Canonical Normalization}
Assume the photon zero-mode $A_\mu^{(0)}(x)$ is independent of $\psi$ and normalized canonically in 4D.
Reducing the gauge kinetic term in \eqref{eq:5Daction} gives
\begin{equation}
S_{\rm gauge} \supset -\frac{1}{4}\int d^4x\, Z\, F_{\mu\nu}^{(0)}F^{\mu\nu}_{(0)},
\qquad
Z \equiv \int_0^{L_\psi} d\psi\, e^{B(\psi)-2A(\psi)},
\label{eq:Zfactor}
\end{equation}
so the canonically normalized 4D photon is $A_\mu^{(0)} \to A_\mu^{(0)}/\sqrt{Z}$.
The covariant derivative contributes the interaction
\begin{equation}
\int d^4x \int_0^{L_\psi}\! d\psi \,\sqrt{|g_5|}\; J^\mu A_\mu
\;\;\longrightarrow\;\;
\int d^4x \; g_4\, J^\mu_{(0)} A^{(0)}_\mu,
\qquad
g_4 = \frac{g_5}{\sqrt{Z}},
\label{eq:g4}
\end{equation}
where $J^\mu$ is the Noether current density and $J^\mu_{(0)}$ its overlap with the photon zero-mode.
Therefore the fine-structure constant at a reference scale $\mu_0$ is
\begin{equation}
\boxed{ \;\alpha(\mu_0) \;=\; \frac{g_4^2}{4\pi} \;=\; \frac{g_5^2}{4\pi\, Z}\; }.
\label{eq:alpha-master}
\end{equation}

\section{Noether Charge and Current Matching}
Global $U(1)$: $\Theta \to e^{i\lambda} \Theta$ yields the 5D Noether current
\begin{equation}
J^M = i \left[ \Theta^\dagger (D^M \Theta) - (D^M \Theta)^\dagger \Theta \right] \Big|_{A=0}.
\end{equation}
We fix the normalization of $\Theta$ such that the fundamental charged excitation has unit Noether charge
\begin{equation}
Q \equiv \int d^3x \int_0^{L_\psi} d\psi \, \sqrt{|g_5|}\, J^0 = \pm 1.
\label{eq:Q1}
\end{equation}
This fixes the overall scale entering the coupling \eqref{eq:g4}; no additional free normalization survives.

\section{Wilson Line and Quantization}
On the compact $\psi$-cycle the holonomy \eqref{eq:thetaH} is physical.
For a field with $U(1)$ charge $q$ and BC phase $\delta$ the KK momenta are shifted by
\begin{equation}
p_\psi^{(n)} = \frac{2\pi}{L_\psi}\Big(n + a\Big),\qquad a \equiv \frac{q\,\theta_H}{2\pi} + \delta,\qquad n\in\mathbb{Z}.
\label{eq:shift}
\end{equation}
Stationary vacua (Hosotani mechanism) are determined dynamically and may select nontrivial $\theta_H^\star$.
Large gauge invariance enforces periodicity $\theta_H \sim \theta_H + 2\pi$; stable vacua satisfy $\partial V_{\rm eff}/\partial \theta_H=0$.

\section{One-Loop Effective Potential $V_{\rm eff}(L_\psi,\theta_H)$}
For each field $j$ with spin-statistics sign $\sigma_j=\pm 1$ (boson $+1$, fermion $-1$), degeneracy $d_j$, mass $m_j$,
charge $q_j$, and twist $\delta_j$, the one-loop contribution is
\begin{align}
V_j(L_\psi,\theta_H)
&=\frac{\sigma_j d_j}{2}\int \!\frac{d^4p}{(2\pi)^4}\sum_{n\in\mathbb{Z}}
\ln\!\Big[p^2 + m_j^2 + \big(\tfrac{2\pi}{L_\psi}\big)^2\big(n+a_j\big)^2\Big],
\\
a_j &\equiv \frac{q_j\,\theta_H}{2\pi} + \delta_j.
\end{align}
Using standard contour/Matsubara techniques (equivalently Poisson resummation) one obtains an exact resummed form
\begin{equation}
V_j(L_\psi,\theta_H) \;=\; \sigma_j d_j \int \!\frac{d^4p}{(2\pi)^4}\; \frac{1}{L_\psi}\;
\ln\!\Big(1 - 2 e^{-L_\psi \omega_j(p)} \cos 2\pi a_j + e^{-2 L_\psi \omega_j(p)}\Big),
\label{eq:Veff-resummed}
\end{equation}
where $\omega_j(p)=\sqrt{p^2+m_j^2}$. The total potential is
\begin{equation}
V_{\rm eff}(L_\psi,\theta_H) \;=\; \sum_j V_j(L_\psi,\theta_H).
\end{equation}
The vacuum $(L_\psi^\star,\theta_H^\star)$ solves
\begin{equation}
\frac{\partial V_{\rm eff}}{\partial \theta_H}=0,
\qquad
\frac{\partial V_{\rm eff}}{\partial L_\psi}=0.
\label{eq:stationary}
\end{equation}
These conditions fix $L_\psi$ and $\theta_H$ \emph{dynamically} in terms of the field content and BC, with no fits.

\paragraph{Massless limit and modular dependence.}
For $m_j L_\psi \ll 1$ one can expand \eqref{eq:Veff-resummed} to obtain polylogarithms $\mathrm{Li}_5(e^{\pm i 2\pi a_j})$,
reproducing the familiar $1/L_\psi^4$ scaling and periodic dependence on $a_j$. Massive fields yield exponentially
suppressed Bessel-function tails for large $m_j L_\psi$.

\section{Determining $Z$ (Warped Case)}
If the vacuum back-reacts and generates a nontrivial warp, the 4D gauge kinetic factor is
\begin{equation}
Z \;=\; \int_0^{L_\psi}\! d\psi\; e^{B(\psi)-2A(\psi)}\, \abs{\xi_0(\psi)}^2,
\label{eq:Zwarp}
\end{equation}
where $\xi_0(\psi)$ is the photon zero-mode profile (constant in flat space).
The warp factors $A(\psi),B(\psi)$ and $\xi_0(\psi)$ follow from the coupled background equations of motion
derived from \eqref{eq:5Daction} (e.g.\ via a first-order BPS system if available).
In flat space, $\xi_0(\psi)\equiv 1$ and $Z=L_\psi$. In general we write
\begin{equation}
Z \;=\; L_\psi \, f(\tau,\mathrm{BC}) ,
\end{equation}
where $f$ encodes the modular parameter $\tau$ (complex-time sector) and boundary data.

\section{Putting It Together: $\alpha$ from UBT}
Combining \eqref{eq:alpha-master}, \eqref{eq:stationary} and \eqref{eq:Zwarp},
\begin{equation}
\boxed{
\alpha(\mu_0) \;=\; \frac{g_5^2}{4\pi \, Z^\star}
\;=\; \frac{g_5^2}{4\pi \, L_\psi^\star f^\star(\tau,\mathrm{BC})},
\qquad
(L_\psi^\star,\theta_H^\star)\; \text{solve}\; \partial V_{\rm eff}=0.}
\end{equation}
Here $g_5$ is fixed by Noether normalization \eqref{eq:Q1} (unit charge for the fundamental excitation) and current matching \eqref{eq:g4}.
Thus, given the field content, BC, and UBT potential $V(\Theta)$, the pair $(L_\psi^\star,\theta_H^\star)$ and the warp $A,B$ are determined
and \emph{no free fit survives} in $\alpha$. Standard QED running then connects $\alpha(\mu_0)$ to experimental scales.

\section{Electron Mass Consistency (Sketch)}
The same background must produce the lightest charged KK eigenvalue
\begin{equation}
m_e^2 \;=\; \lambda_{\rm min}\Big[ -e^{-B}\partial_\psi\!\big( e^{B-2A}\partial_\psi \cdot \big) + m_\Theta^2 e^{-2A} + \cdots \Big],
\end{equation}
for the appropriate (spinor) sector and BC, including the holonomy shift \eqref{eq:shift}.
This provides a nontrivial cross-check: the \emph{same} $(L_\psi^\star,\theta_H^\star, A,B)$ that gives $\alpha$ must also yield $m_e$.

\section{Checklist to Avoid Numerology}
\begin{itemize}
\item Use \eqref{eq:Veff-resummed} (or its Poisson/Bessel form) \emph{exactly}; minimize \eqref{eq:stationary} without inserting data.
\item Fix Noether charge to $Q=\pm 1$ via \eqref{eq:Q1}; this removes arbitrary rescalings of $\Theta$.
\item Compute $Z$ from \eqref{eq:Zfactor} or \eqref{eq:Zwarp}; do not set $Z=L_\psi$ unless flat space is justified by the background EOM.
\item Only after obtaining $\alpha(\mu_0)$ ab-initio apply perturbative running to compare to $\alpha(M_Z)$.
\end{itemize}


\section*{License}
This work is licensed under a Creative Commons Attribution 4.0 International License (CC BY 4.0).

\end{document}