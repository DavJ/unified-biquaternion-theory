
\appendix
\section*{Appendix K: Consolidation of fundamental constants}

\subsection*{K.1 Input vs. output constants}
Within UBT we distinguish between input parameters, which define the scale and structure, 
and output constants, which the theory predicts. The goal is to minimize the number of inputs.

Inputs: $c$, $\hbar$ (definitions of units), integer parameters (e.g. $n_\mu, n_\tau$ in the original model).

Outputs: $\alpha$, $m_e$, $m_\mu$, $m_\tau$, $\Lambda_{\rm QCD}$, etc.

\subsection*{K.2 Mystery of lepton mass ratios}
The experimentally observed ratios of charged lepton masses are:
\begin{align*}
\frac{m_\mu}{m_e} &\simeq 206.77, \\
\frac{m_\tau}{m_e} &\simeq 3477.2 .
\end{align*}
These numbers, close to the integers 207 and 3477, were long considered a numerological puzzle. 
The original version of UBT postulated that these ratios correspond to internal quantum numbers, 
but lacked a mechanism for their derivation.

\subsection*{K.3 Electron mass as an internal mode}
Building on earlier self–energy models, we show that the electron 
mass arises as the lowest non-trivial eigenvalue of the Dirac operator on 
$T^2(\tau)$ with Hosotani phase $\theta_H=\pi$. 

\paragraph{Eigenvalue problem.}
The internal Dirac operator on the torus reads
\begin{equation}
D = i\gamma^\psi \left(\partial_\psi + i Q \theta_H/L_\psi\right) 
  + i\gamma^\phi \partial_\phi ,
\end{equation}
with eigenmodes $(n,m)\in\mathbb{Z}^2$ shifted by the Hosotani background. 
The eigenenergies are
\begin{equation}
E_{n,m} = \frac{1}{R}\sqrt{(n+\delta)^2 + (m+\delta')^2}.
\end{equation}

\paragraph{Electron as the first excitation.}
For $Q=-1$ and $\theta_H=\pi$, the lowest non-zero mode $(n,m)=(0,1)$ yields 
\begin{equation}
m_e \;=\; \frac{1}{R}\sqrt{\delta^2+1}\;\simeq\;\frac{1}{R},
\end{equation}
where $R$ is tied to the compactification scale fixed in Appendix~V. 

\paragraph{Implication.}
Thus $m_e$ is not an input parameter but an output of the same geometry that 
fixes $\alpha$. Higher modes $(n,m)=(0,2)$ and $(1,0)$ provide natural 
candidates for $m_\mu$ and $m_\tau$, suggesting that lepton mass ratios arise 
from the same internal structure.

\subsection*{K.4 Consolidation}
Together with Appendix~V this shows that both $\alpha$ and $m_e$ 
(and potentially also the heavier lepton masses) emerge from the same 
toroidal geometry.
