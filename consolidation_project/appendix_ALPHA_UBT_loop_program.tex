% =======================================================================
% Appendix: UBT Electromagnetic Sector and Quantum Correction Program for α
% =======================================================================

\section{UBT Electromagnetic Sector and Quantum Correction Program for $\alpha$}
\label{sec:UBT_alpha_loop_program}

In this appendix we define the electromagnetic sector of the Unified Biquaternion Theory (UBT), its effective QED-like limit, and a concrete, non-circular program for computing quantum corrections to the fine-structure constant $\alpha$. The key requirements are:

\begin{itemize}
  \item The \emph{baseline} value $\alpha_0^{-1} = 137$ is fixed by UBT geometry/topology (prime selection and toroidal structure), not by experiment.
  \item All quantum corrections are obtained from loop integrals of the UBT effective field theory, with input couplings and masses determined by UBT, not tuned to match experiment.
  \item Standard QED results (e.g.\ $\Delta \approx 0.036$) may be used only \emph{for comparison}, never as input to the UBT derivation.
\end{itemize}

The final goal is a first-principles prediction
\begin{equation}
  \alpha_{\mathrm{UBT}}^{-1}(\mu_0) = 137 + \Delta_{\mathrm{UBT}}(137;\mu_0),
\end{equation}
where $\Delta_{\mathrm{UBT}}(137;\mu_0)$ is computed from UBT loop integrals.

% -----------------------------------------------------------------------
\subsection{Extended space, complex time, and biquaternionic field}
% -----------------------------------------------------------------------

UBT is formulated on an extended space with coordinates
\begin{equation}
  x^M = (x^\mu, \psi), \qquad \mu = 0,1,2,3, \qquad \psi \sim \psi + 2\pi R_\psi,
\end{equation}
where $\psi$ is the compact internal coordinate associated with the imaginary part of complex time
\begin{equation}
  \tau = t + i \psi.
\end{equation}
The fundamental field is a biquaternionic object
\begin{equation}
  \Theta(x^M) \in \mathbb{B},
\end{equation}
whose internal structure encodes both gauge and spinor degrees of freedom. A $U(1)$ electromagnetic gauge symmetry arises as a phase rotation of an appropriate component of $\Theta$, while spinorial components give an effective Dirac field.

% -----------------------------------------------------------------------
\subsection{UBT electromagnetic action in 5D}
\label{subsec:UBT_EM_action}
% -----------------------------------------------------------------------

We introduce a $U(1)$ gauge field on the extended space,
\begin{equation}
  A_M(x^N) = \bigl(A_\mu(x^N), A_\psi(x^N)\bigr), \qquad M,N = 0,1,2,3,\psi,
\end{equation}
with field strength
\begin{equation}
  F_{MN} = \partial_M A_N - \partial_N A_M.
\end{equation}
The minimal UBT electromagnetic action coupled to a fermionic mode extracted from $\Theta$ is
\begin{equation}
  S_{\mathrm{UBT,EM}}
  =
  \int d^4x \int_0^{2\pi R_\psi} d\psi \,\sqrt{|g|}
  \,\mathcal{L}_{\mathrm{UBT,EM}},
\end{equation}
with Lagrangian
\begin{equation}
  \mathcal{L}_{\mathrm{UBT,EM}}
  =
  -\frac{1}{4 g_0^2} F_{MN} F^{MN}
  + \bar{\Psi}(x^M)\bigl(i \Gamma^M D_M - m_0\bigr)\Psi(x^M)
  + \mathcal{L}_{\mathrm{top}}[\Theta, A_M].
  \label{eq:UBT_L_EM}
\end{equation}
Here:
\begin{itemize}
  \item $\Psi(x^M)$ is an effective 5D spinor extracted from the biquaternionic field $\Theta$.
  \item $\Gamma^M$ are 5D gamma matrices, e.g.\ $\Gamma^\mu = \gamma^\mu$, $\Gamma^\psi = i\gamma^5$.
  \item $D_M = \partial_M + i q A_M$ is the covariant derivative with a $U(1)$ charge $q$ fixed by UBT charge quantization (Dirac/holonomy conditions).
  \item $g_0$ is the bare UBT gauge coupling for the electromagnetic sector.
  \item $m_0$ is the bare fermion mass arising from the biquaternionic/Hopfion sector.
  \item $\mathcal{L}_{\mathrm{top}}[\Theta, A_M]$ encodes the toroidal and biquaternionic topology (effective mode count $N_{\mathrm{eff}}$, compactification radius $R_\psi$, etc.).
\end{itemize}
All these parameters are determined by UBT geometry/topology, not fitted to experiment.

% -----------------------------------------------------------------------
\subsection{Geometric definition of the bare coupling and baseline $\alpha_0$}
\label{subsec:bare_coupling}
% -----------------------------------------------------------------------

The geometric framework of Appendix~C implies that the electromagnetic coupling is related to toroidal radii and effective mode count. We summarize this as
\begin{equation}
  \frac{1}{g_0^2}
  = \frac{N_{\mathrm{eff}}}{4\pi}\,\frac{R_t}{R_\psi},
\end{equation}
where $R_t$ and $R_\psi$ are the principal radii of the complex-time torus and $N_{\mathrm{eff}}$ is the effective number of modes (e.g.\ $N_{\mathrm{eff}} = 12$ from quaternionic mode counting). The electric charge is
\begin{equation}
  e_0 = q\, g_0,
\end{equation}
with $q$ fixed by UBT charge quantization.

We define the \emph{bare} UBT fine-structure constant as
\begin{equation}
  \alpha_0 \equiv \frac{e_0^2}{4\pi}.
\end{equation}
The geometric/topological analysis (prime selection via $V_{\mathrm{eff}}(n)$, Hecke sector stability, etc.) selects
\begin{equation}
  n_\star = 137, \qquad \alpha_0^{-1} = 137.
  \label{eq:alpha0_137_appendix}
\end{equation}
Equation~\eqref{eq:alpha0_137_appendix} is the \emph{baseline UBT prediction} for $\alpha$, obtained without any experimental input.

% -----------------------------------------------------------------------
\subsection{Reduction to 4D QED-like effective theory}
\label{subsec:QED_limit}
% -----------------------------------------------------------------------

We perform a Kaluza--Klein decomposition along the compact $\psi$ direction:
\begin{align}
  A_M(x,\psi) &= \sum_{k\in\mathbb{Z}} A_M^{(k)}(x)\, e^{ik\psi/R_\psi}, \\
  \Psi(x,\psi) &= \sum_{k\in\mathbb{Z}} \Psi^{(k)}(x)\, e^{ik\psi/R_\psi}.
\end{align}
In the low-energy regime $E \ll 1/R_\psi$, the dominant contribution comes from the zero modes $k=0$:
\begin{equation}
  A_\mu^{(0)}(x), \qquad \Psi^{(0)}(x).
\end{equation}
Integrating over $\psi$ and retaining only zero modes yields the 4D effective action
\begin{equation}
  S_{\mathrm{eff}}^{(4\mathrm{D})}
  =
  \int d^4x \left[
    -\frac{1}{4 e_0^2} F_{\mu\nu}^{(0)} F^{\mu\nu}_{(0)}
    + \bar{\Psi}^{(0)}(i\slashed{\partial}-m_0)\Psi^{(0)}
    - e_0\, \bar{\Psi}^{(0)}\slashed{A}^{(0)}\Psi^{(0)}
  \right]
  + S_{\mathrm{KK\text{-}corr}},
  \label{eq:Seff_4D_appendix}
\end{equation}
where $S_{\mathrm{KK\text{-}corr}}$ contains contributions from heavy Kaluza--Klein modes ($k\neq 0$) and UBT-specific corrections. Equation~\eqref{eq:Seff_4D_appendix} has the structure of standard QED, but with:
\begin{itemize}
  \item bare coupling $e_0$ fixed by UBT ($\alpha_0^{-1}=137$),
  \item bare mass $m_0$ fixed by the topological Hopfion sector,
  \item additional higher-dimensional corrections encoded in $S_{\mathrm{KK\text{-}corr}}$.
\end{itemize}

% -----------------------------------------------------------------------
\subsection{Feynman rules in the UBT QED-like limit}
\label{subsec:Feynman_rules_UBT}
% -----------------------------------------------------------------------

For the purposes of loop calculations we extract the effective Feynman rules from~\eqref{eq:Seff_4D_appendix} plus a KK/UBT correction sector.

\paragraph{Gauge choice.}
We work in Feynman gauge for the photon zero mode:
\begin{equation}
  \mathcal{L}_{\mathrm{gauge\,fix}}
  = - \frac{1}{2 e_0^2} (\partial_\mu A^\mu_{(0)})^2.
\end{equation}

\paragraph{Propagators.}
\begin{itemize}
  \item Photon zero-mode propagator:
  \begin{equation}
    D^{\mu\nu}_0(q) = \frac{-i g^{\mu\nu}}{q^2 + i\epsilon}.
  \end{equation}
  \item Fermion zero-mode propagator:
  \begin{equation}
    S_0(p) = \frac{i}{\slashed{p} - m_0 + i\epsilon}.
  \end{equation}
  \item UBT/KK-extended fermion propagator (schematic):
  \begin{equation}
    S_{\mathrm{UBT}}(p)
    = \sum_{k\in\mathbb{Z}} w_k \,\frac{i}{\slashed{p} + \gamma^5 k/R_\psi - m_k + i\epsilon},
    \label{eq:S_UBT_modes_appendix}
  \end{equation}
  where the weights $w_k$ and masses $m_k$ are determined by the biquaternionic and topological sector.
\end{itemize}

\paragraph{Vertices.}
The basic interaction vertex between the photon zero mode and a fermion mode is
\begin{equation}
  -i e_0 \gamma^\mu,
\end{equation}
with additional higher-dimensional vertices (e.g.\ involving $A_\psi$, KK modes, or effective four-fermion terms) contained in $S_{\mathrm{KK\text{-}corr}}$. These give genuine UBT corrections beyond the QED-like sector.

% -----------------------------------------------------------------------
\subsection{Renormalization and definition of the renormalized charge}
\label{subsec:renormalized_charge_appendix}
% -----------------------------------------------------------------------

We introduce renormalization constants
\begin{equation}
  A_\mu^{(0)} = Z_3^{1/2} A_{\mu,R}, \qquad
  \Psi^{(0)} = Z_2^{1/2} \Psi_R, \qquad
  e_0 = Z_e\, e_R,
\end{equation}
and define the renormalized coupling $e_R(\mu)$ at scale $\mu$ in a given scheme.

In the Thomson (on-shell) scheme the photon two-point function has the form
\begin{equation}
  i\Pi^{\mu\nu}(q) = i (q^\mu q^\nu - q^2 g^{\mu\nu})\, \Pi(q^2),
\end{equation}
and the photon field renormalization is
\begin{equation}
  Z_3^{-1} = 1 - \left.\frac{\partial \Pi(q^2)}{\partial q^2}\right|_{q^2=0}.
  \label{eq:Z3_def_appendix}
\end{equation}
The renormalized coupling $e_R(\mu)$ is then fixed by the requirement that the renormalized three-point vertex reproduces the Coulomb interaction with coefficient $e_R$ in the chosen scheme, entirely from UBT loop integrals.

We define the renormalized UBT fine-structure constant at scale $\mu_0$ as
\begin{equation}
  \alpha_{\mathrm{UBT}}(\mu_0) \equiv \frac{e_R^2(\mu_0)}{4\pi}.
\end{equation}

% -----------------------------------------------------------------------
\subsection{One-loop UBT algorithm for the photon self-energy}
\label{subsec:one_loop_algorithm_appendix}
% -----------------------------------------------------------------------

At one loop, the photon self-energy in UBT is given symbolically by
\begin{equation}
  \Pi^{\mu\nu}_{\mathrm{UBT}}(q)
  =
  -(-1)(-ie_0)^2 \int \frac{d^D p}{(2\pi)^D}
  \mathrm{Tr}
  \left[
    \gamma^\mu S_{\mathrm{UBT}}(p)\,
    \gamma^\nu S_{\mathrm{UBT}}(p+q)
  \right],
  \label{eq:Pi_munu_UBT_appendix}
\end{equation}
with $D=4-2\epsilon$ and $S_{\mathrm{UBT}}$ given by~\eqref{eq:S_UBT_modes_appendix}. Splitting off the zero-mode and KK/UBT corrections:
\begin{equation}
  \Pi^{\mu\nu}_{\mathrm{UBT}}(q)
  =
  \Pi^{\mu\nu}_{\mathrm{QED\text{-}like}}(q; e_0, m_0)
  +
  \Delta \Pi^{\mu\nu}_{\mathrm{KK/UBT}}(q; R_\psi, w_k, m_k).
\end{equation}

A concrete one-loop algorithm:

\begin{enumerate}
  \item \textbf{Fix UBT input parameters} from geometry/topology:
    \[
      n_\star = 137, \quad \alpha_0^{-1} = 137, \quad R_\psi, \quad N_{\mathrm{eff}}, \quad m_0, \quad w_k, \quad m_k.
    \]
  \item \textbf{Compute} $\Pi_{\mathrm{UBT}}(q^2)$ at one loop using~\eqref{eq:Pi_munu_UBT_appendix} in a chosen regulator (e.g.\ dimensional regularization).
  \item \textbf{Determine} $Z_3$ from~\eqref{eq:Z3_def_appendix} and similarly $Z_2$, $Z_e$ from fermion self-energy and vertex corrections, all computed from the UBT effective theory.
  \item \textbf{Extract} the renormalized coupling $e_R(\mu_0)$ from the renormalized vertex in the chosen scheme.
  \item \textbf{Define} the one-loop UBT prediction for $\alpha$ at scale $\mu_0$:
    \begin{equation}
      \alpha_{\mathrm{UBT}}(\mu_0)
      = \frac{e_R^2(\mu_0)}{4\pi}
      = \alpha_0 + \delta\alpha_{\mathrm{UBT}}^{(1)}(\mu_0),
    \end{equation}
    and
    \begin{equation}
      \alpha_{\mathrm{UBT}}^{-1}(\mu_0)
      = 137 + \Delta_{\mathrm{UBT}}^{(1)}(137;\mu_0).
    \end{equation}
\end{enumerate}

% -----------------------------------------------------------------------
\subsection{Definition of the UBT quantum correction $\Delta_{\mathrm{UBT}}$}
\label{subsec:Delta_UBT_def_appendix}
% -----------------------------------------------------------------------

\begin{definition}[UBT quantum correction to the fine-structure constant]
Let $\alpha_{\mathrm{UBT}}(\mu_0)$ be the renormalized fine-structure constant obtained from the UBT effective theory at scale $\mu_0$, with bare input $\alpha_0^{-1}=137$ determined solely by UBT geometry/topology. Then we define
\begin{equation}
  \boxed{
    \Delta_{\mathrm{UBT}}(137;\mu_0)
    \equiv
    \alpha_{\mathrm{UBT}}^{-1}(\mu_0) - 137.
  }
\end{equation}
\end{definition}

By construction:
\begin{itemize}
  \item $\Delta_{\mathrm{UBT}}$ is a prediction of the UBT loop structure, not an input.
  \item No experimental value $\alpha_{\mathrm{exp}}$ and no phenomenological $\Delta$ appears anywhere in the derivation.
  \item If future calculations yield $\Delta_{\mathrm{UBT}}(137;\mu_0) \approx 0.036$, then UBT reproduces the experimental $\alpha^{-1} \approx 137.036$ as a derived quantity.
  \item If $\Delta_{\mathrm{UBT}}$ deviates significantly from $0.036$, UBT makes a distinct prediction testable by experiment.
\end{itemize}

% -----------------------------------------------------------------------
\subsection{Extension to two loops and beyond}
\label{subsec:two_loop_program_appendix}
% -----------------------------------------------------------------------

The one-loop program extends systematically to two loops and higher orders:

\begin{enumerate}
  \item Enumerate all two-loop diagrams contributing to the photon self-energy and vertex corrections in the UBT effective theory defined by~\eqref{eq:Seff_4D_appendix} and $S_{\mathrm{KK\text{-}corr}}$.
  \item Express the resulting integrals in terms of scalar integrals and reduce them to master integrals using IBP identities.
  \item Evaluate the master integrals in dimensional regularization and perform renormalization in the chosen scheme, without using experimental $\alpha$.
  \item Extract $\delta\alpha_{\mathrm{UBT}}^{(2)}(\mu_0)$ and thus
    \begin{equation}
      \alpha_{\mathrm{UBT}}^{-1}(\mu_0)
      = 137 + \Delta_{\mathrm{UBT}}^{(1)}(137;\mu_0)
             + \Delta_{\mathrm{UBT}}^{(2)}(137;\mu_0)
             + \cdots.
    \end{equation}
\end{enumerate}

Any comparison with experimental $\alpha$ is an external step, performed only after $\alpha_{\mathrm{UBT}}(\mu_0)$ has been computed from UBT.

% =======================================================================
% End of appendix: UBT Electromagnetic Sector and Quantum Correction Program
% =======================================================================


