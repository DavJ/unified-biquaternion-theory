% © 2025 Ing. David Jaroš — CC BY-NC-ND 4.0
%
% This work is licensed under a Creative Commons Attribution-NonCommercial-NoDerivatives 
% 4.0 International License (CC BY-NC-ND 4.0).
%
% License History: Earlier drafts (up to v0.3) were released under CC BY 4.0. 
% From v0.4 onward, all material is released under CC BY-NC-ND 4.0 to protect 
% the integrity of the theoretical work during ongoing academic development.
%
% See LICENSE.md for full license text.

% VERSION: v1.0
\section{Experimental Predictions and Falsification Protocols for JWST and LHC}
\label{app:jwst_lhc_predictions}

\subsection{Overview: UBT as a Predictive Digital-Physical Model}

The Unified Biquaternion Theory (UBT) can be interpreted through a signal-processing framework, treating spacetime as a discrete information manifold operating over a finite field $GF(2^8)$ with Reed-Solomon error correction encoding. This perspective frames physical observables as emergent from an underlying \textbf{digital substrate} where the biquaternionic field $\Theta(q,\tau)$ represents the decoded state of a $RS(255, k)$ data stream.

From this engineering standpoint, fundamental phenomena—gravitational redshift, particle interactions, and quantum noise—arise from:
\begin{itemize}
\item \textbf{Frame-alignment overhead}: Discrete synchronization boundaries in the spacetime data stream
\item \textbf{Nyquist frequency limits}: Aliasing effects at characteristic harmonic modes of the $GF(2^8)$ manifold
\item \textbf{Parity check saturation}: Error correction failures during high-energy transients
\item \textbf{Clock jitter}: Phase coherence degradation in complex time $\tau = t + i\psi$
\end{itemize}

This section presents \textbf{falsifiable predictions} for JWST (James Webb Space Telescope) and LHC (Large Hadron Collider) observations that directly test the digital-substrate hypothesis.

\subsection{JWST Predictions: Cosmological Structure as Data Stream Artifacts}

\subsubsection{Prediction J1: Redshift Quantization in High-Redshift Galaxies}

\textbf{Physical Basis:} In the digital-substrate interpretation, the Hubble flow is not a continuous expansion but a \textbf{frame-aligned discrete evolution} of the cosmological data stream. The $RS(255, k)$ encoding requires periodic synchronization frames, introducing overhead that manifests as discrete steps in observable redshift.

\textbf{Quantitative Prediction:}

For high-redshift galaxy observations at $z > 10$, JWST will detect \textbf{redshift quantization}—discrete ``steps'' in the redshift distribution rather than a smooth continuum. The step size $\Delta z$ is determined by the frame-alignment overhead:
\begin{equation}
\Delta z = \frac{1}{256} \times (1 + z)
\label{eq:redshift_quantization}
\end{equation}

where the factor $1/256$ arises from the 8-bit symbol structure of $GF(2^8)$ and the overhead ratio of one synchronization frame per 256 data symbols in the $RS(255, k)$ code.

\textbf{Observable Signature:}

The redshift distribution $n(z)$ should exhibit periodic enhancements at:
\begin{equation}
z_n = z_0 \left(1 + \frac{n}{256}\right), \quad n = 1, 2, 3, \ldots
\label{eq:redshift_peaks}
\end{equation}

where $z_0 \approx 10$ is a reference redshift. For $z_0 = 10$, the first three quantization peaks occur at:
\begin{itemize}
\item $z_1 \approx 10.039$ (first frame boundary)
\item $z_2 \approx 10.078$ (second frame boundary)
\item $z_3 \approx 10.117$ (third frame boundary)
\end{itemize}

\textbf{Experimental Method:}
\begin{enumerate}
\item Use JWST NIRSpec to obtain spectroscopic redshifts for $>1000$ galaxies at $10 < z < 15$
\item Bin redshift measurements with precision $\Delta z_{\text{bin}} \ll 0.039$ (i.e., $\Delta z_{\text{bin}} \sim 0.01$)
\item Construct histogram $n(z)$ and search for periodic overdensities
\item Apply Fourier analysis to detect characteristic frequency $f = 1/(0.039 \times (1+z))$
\end{enumerate}

\textbf{Falsification Criterion:}
\begin{itemize}
\item \textbf{If no periodicity detected} with $>3\sigma$ significance after 1000+ galaxies: UBT digital-substrate model \textbf{falsified}
\item \textbf{If periodicity exists but} $\Delta z \neq (1+z)/256$ within 20\%: Frame-alignment hypothesis \textbf{rejected}
\item \textbf{If smooth continuum} persists to $\Delta z < 0.01$ precision: Discrete data stream interpretation \textbf{invalidated}
\end{itemize}

\textbf{Comparison to Standard Cosmology:} $\Lambda$CDM predicts a perfectly smooth redshift distribution (modulo large-scale structure clustering). Any discrete stepping would constitute \textbf{direct evidence} for a digital substrate.

\subsubsection{Prediction J2: Structural Aliasing at Nyquist Frequency}

\textbf{Physical Basis:} The $GF(2^8)$ manifold has a characteristic Nyquist frequency at harmonic mode $\ell \approx 255$. Cosmic structures imprinted above this frequency will exhibit \textbf{aliasing}—spurious correlations and moiré-like patterns in the spatial distribution of early galaxies.

\textbf{Quantitative Prediction:}

The two-point correlation function $\xi(\theta)$ of galaxies at $z > 12$ will show \textbf{artificial enhancement} at angular scales:
\begin{equation}
\theta_{\text{alias}} = \frac{2\pi}{\ell_{\text{Nyquist}}} \times \frac{d_A(z)}{d_H(z)} \approx \frac{2\pi}{255} \quad \text{(radians)}
\label{eq:aliasing_scale}
\end{equation}

where $d_A(z)$ is the angular diameter distance and $d_H(z)$ is the Hubble distance. For $z \approx 13$, this corresponds to:
\begin{equation}
\theta_{\text{alias}} \approx 1.4^\circ \pm 0.2^\circ
\label{eq:aliasing_angle}
\end{equation}

\textbf{Observable Signature:}

The correlation function will exhibit a \textbf{spurious peak} at $\theta \approx 1.4^\circ$ with amplitude:
\begin{equation}
\xi_{\text{alias}}(\theta_{\text{alias}}) = \xi_{\text{true}}(\theta_{\text{alias}}) + A_{\text{alias}} \cos\left(\frac{255 \theta}{\theta_{\text{alias}}}\right)
\label{eq:correlation_aliasing}
\end{equation}

where $A_{\text{alias}} \sim 0.1$--$0.3$ is the aliasing amplitude (10--30\% of the true correlation).

\textbf{Experimental Method:}
\begin{enumerate}
\item Use JWST NIRCam wide-field imaging to map $>10$ sq. degrees at $z > 12$
\item Measure angular two-point correlation function $\xi(\theta)$ for $10^{-3} < \theta < 10^\circ$
\item Compare observed $\xi(\theta)$ to standard $\Lambda$CDM predictions
\item Search for periodic modulation with wavelength $\Delta \theta \sim 2\pi/255$
\end{enumerate}

\textbf{Falsification Criterion:}
\begin{itemize}
\item \textbf{If no aliasing detected} at $\theta \sim 1.4^\circ$ with $>2\sigma$ significance: Nyquist-frequency hypothesis \textbf{falsified}
\item \textbf{If aliasing present but} characteristic scale $\neq 2\pi/255$ within 30\%: $GF(2^8)$ manifold model \textbf{rejected}
\item \textbf{If smooth power spectrum} matches $\Lambda$CDM to $<5\%$: Digital-substrate model \textbf{invalidated}
\end{itemize}

\textbf{Comparison to Standard Cosmology:} $\Lambda$CDM predicts smooth correlation functions with no artificial periodicity. Detection of aliasing patterns would support the discrete manifold hypothesis.

\subsection{Mathematical Formalism: Redshift Drift and Buffer Depth}

The redshift drift $\delta z$ as a function of the 16-channel multiplex buffer depth $D_{\text{buf}}$ is given by:
\begin{equation}
\delta z(D_{\text{buf}}) = \frac{1}{256} \left(1 + \frac{D_{\text{buf}}}{16}\right) (1 + z)
\label{eq:redshift_drift_buffer}
\end{equation}

where:
\begin{itemize}
\item $D_{\text{buf}}$ is the effective buffer depth in units of $RS(255, k)$ codewords
\item The factor $1/16$ accounts for parallel channel processing
\item For $D_{\text{buf}} = 0$ (no buffering): $\delta z = (1+z)/256$ (minimum quantization)
\item For $D_{\text{buf}} = 16$ (one full buffer): $\delta z = 2(1+z)/256$ (doubled step size)
\end{itemize}

This formula enables \textbf{testable predictions} for how observed redshift quantization should vary with cosmological epoch (buffer depth increases with lookback time).

\subsection{LHC Predictions: High-Energy Physics as Error Correction Saturation}

\subsubsection{Prediction L1: Digital Noise Floor in Particle Cross-Sections}

\textbf{Physical Basis:} In the digital-substrate model, the Planck energy $E_{\text{Planck}}$ corresponds to the \textbf{symbol rate} of the fundamental data stream. At energies approaching $E_{\text{Planck}}/256$, the $RS(255, k)$ decoder encounters \textbf{quantization noise} that manifests as a stochastic background in particle production cross-sections.

\textbf{Quantitative Prediction:}

At collision energies $E_{\text{cm}} > E_{\text{threshold}}$, the total inelastic cross-section will exhibit an \textbf{additive noise floor}:
\begin{equation}
\sigma_{\text{total}}(E_{\text{cm}}) = \sigma_{\text{SM}}(E_{\text{cm}}) + \sigma_{\text{noise}}
\label{eq:noise_floor_cross_section}
\end{equation}

where:
\begin{equation}
\sigma_{\text{noise}} = \sigma_0 \times \left(\frac{E_{\text{cm}}}{E_{\text{threshold}}}\right)^2, \quad E_{\text{threshold}} = \frac{E_{\text{Planck}}}{256} \approx 4.8 \times 10^{16} \text{ GeV}
\label{eq:noise_threshold}
\end{equation}

and $\sigma_0 \sim 10^{-42}$ cm$^2$ is the characteristic cross-section scale at threshold.

For LHC energies ($E_{\text{cm}} \sim 13$ TeV $\ll E_{\text{threshold}}$), the noise floor is suppressed by factor:
\begin{equation}
\frac{\sigma_{\text{noise}}}{\sigma_{\text{SM}}} \sim 10^{-26} \left(\frac{13 \text{ TeV}}{4.8 \times 10^{16} \text{ GeV}}\right)^2 \sim 10^{-52}
\label{eq:noise_suppression_lhc}
\end{equation}

This is \textbf{far below} current detector sensitivity, explaining why the LHC has not yet observed the noise floor.

\textbf{Testable Prediction for Future Colliders:}

A \textbf{100 TeV} Future Circular Collider (FCC) would probe energies where:
\begin{equation}
\frac{\sigma_{\text{noise}}}{\sigma_{\text{SM}}} \sim 10^{-46} \quad \text{(potentially observable with } 10^{10} \text{ events)}
\label{eq:noise_fcc}
\end{equation}

\textbf{Experimental Method:}
\begin{enumerate}
\item Measure total inelastic cross-section at multiple $\sqrt{s}$ values from 13 TeV to 100 TeV
\item Search for \textbf{excess events} in minimum-bias triggers after subtracting Standard Model predictions
\item Look for quadratic energy dependence: $\sigma_{\text{excess}} \propto E_{\text{cm}}^2$
\item Integrated luminosity $>10$ ab$^{-1}$ required for $3\sigma$ detection at FCC energies
\end{enumerate}

\textbf{Falsification Criterion:}
\begin{itemize}
\item \textbf{If no excess observed} at FCC energies with sensitivity $\sigma_{\text{excess}} < 10^{-48}$ cm$^2$: Digital noise floor hypothesis \textbf{falsified}
\item \textbf{If excess present but} $E$-dependence $\neq E^2$: Quantization noise model \textbf{rejected}
\item \textbf{If threshold energy} $\neq E_{\text{Planck}}/256$ within factor of 2: Frame-alignment interpretation \textbf{invalidated}
\end{itemize}

\subsubsection{Prediction L2: Parity Check Violations as Error Overflows}

\textbf{Physical Basis:} The $RS(255, 201)$ code can correct up to $t = (255-201)/2 = 27$ symbol errors per codeword. During \textbf{high-energy transient events} (e.g., TeV-scale heavy ion collisions, top-quark pair production), the error correction system may be \textbf{saturated}, leading to uncorrectable errors that manifest as apparent violations of conservation laws.

\textbf{Quantitative Prediction:}

In events where the instantaneous energy density exceeds:
\begin{equation}
\rho_{\text{crit}} = \frac{27}{256} \times \rho_{\text{Planck}} \approx 0.105 \times \rho_{\text{Planck}} \approx 5.4 \times 10^{92} \text{ g/cm}^3
\label{eq:critical_density}
\end{equation}

the $RS(255, 201)$ parity check will \textbf{fail}, producing \textbf{extremely rare} anomalous decays with:
\begin{itemize}
\item $\Delta Q \neq 0$ (electric charge non-conservation)
\item $\Delta L \neq 0$ (lepton number violation)
\item $\Delta B \neq 0$ (baryon number violation)
\end{itemize}

The predicted rate is:
\begin{equation}
\Gamma_{\text{anomaly}} = \Gamma_{\text{SM}} \times \exp\left(-\frac{\rho_{\text{crit}}}{\langle \rho \rangle}\right)
\label{eq:anomaly_rate}
\end{equation}

where $\langle \rho \rangle$ is the average energy density in the collision.

For LHC Pb+Pb collisions at $\sqrt{s_{NN}} = 5.02$ TeV:
\begin{equation}
\langle \rho \rangle \sim 10^{2} \text{ GeV/fm}^3 \sim 10^{39} \text{ g/cm}^3 \ll \rho_{\text{crit}}
\label{eq:lhc_density}
\end{equation}

Thus:
\begin{equation}
\Gamma_{\text{anomaly}} \sim \Gamma_{\text{SM}} \times \exp(-10^{53}) \approx 0 \quad \text{(effectively unobservable)}
\label{eq:lhc_anomaly_rate}
\end{equation}

\textbf{Observable Signature:}

Despite exponential suppression, \textbf{stacking} $>10^{12}$ events might reveal:
\begin{itemize}
\item One or two events with $\Delta Q = \pm 1$ and no corresponding charged track
\item Missing energy $E_{\text{miss}} > 100$ GeV inconsistent with neutrinos
\item Unusual particle production: e.g., $pp \to pppp$ (baryon number violation by $\Delta B = +2$)
\end{itemize}

\textbf{Experimental Method:}
\begin{enumerate}
\item Analyze full Run 3 + High-Luminosity LHC dataset ($>10^{16}$ proton-proton collisions)
\item Search for events with \textbf{apparent conservation law violations}
\item Apply strict quality cuts to eliminate detector artifacts
\item Look for clustering around high-$p_T$ jets or heavy flavor production
\end{enumerate}

\textbf{Falsification Criterion:}
\begin{itemize}
\item \textbf{If no anomalies observed} after $10^{16}$ events: Error overflow hypothesis \textbf{not testable} at LHC (requires higher energy density)
\item \textbf{If anomalies detected but} rate $\neq \exp(-\rho_{\text{crit}}/\langle\rho\rangle)$ form: Parity check saturation model \textbf{rejected}
\item \textbf{If violations occur} at low energy density $\rho \ll 10^{39}$ g/cm$^3$: Critical density prediction \textbf{falsified}
\end{itemize}

\textbf{Note on Detectability:} This prediction is \textbf{extremely challenging} to test at current LHC energies due to exponential suppression. Future ultra-high-energy colliders (e.g., muon collider at 10+ TeV center-of-mass energy) would be required for meaningful constraints.

\subsection{Mathematical Formalism: Fine Structure Constant and Vacuum Polarization Noise}

The fine structure constant $\alpha$ receives a 2-loop correction from vacuum polarization noise in the digital-substrate model:
\begin{equation}
\alpha(Q^2) = \alpha_0 \left[1 + \frac{\alpha_0}{\pi} \ln\left(\frac{Q^2}{m_e^2}\right) + \frac{\alpha_0^2}{\pi^2} f_{\text{noise}}(Q^2)\right]
\label{eq:alpha_running_noise}
\end{equation}

where the noise function is:
\begin{equation}
f_{\text{noise}}(Q^2) = \frac{1}{256} \sum_{n=1}^{255} \frac{1}{n^2} \sin^2\left(\frac{n Q}{E_{\text{Planck}}}\right)
\label{eq:vacuum_noise_function}
\end{equation}

This sum represents \textbf{interference between 256 discrete virtual states} in the $GF(2^8)$ vacuum structure.

\textbf{Testable Prediction:}

High-precision QED experiments (e.g., electron $g-2$ measurements, Lamb shift spectroscopy) should detect \textbf{oscillatory corrections} to $\alpha$ at energy scales:
\begin{equation}
Q_n = \frac{n E_{\text{Planck}}}{256}, \quad n = 1, 2, 3, \ldots
\label{eq:alpha_resonances}
\end{equation}

The first resonance occurs at $Q_1 \sim 4.8 \times 10^{16}$ GeV (inaccessible). However, the \textbf{summed effect} produces a small constant offset:
\begin{equation}
\Delta \alpha_{\text{noise}} = \frac{\alpha_0^2}{\pi^2} \times \frac{\pi^2}{6 \times 256} \approx \frac{\alpha_0^2}{1536}
\label{eq:alpha_offset}
\end{equation}

For $\alpha_0^{-1} \approx 137.036$:
\begin{equation}
\Delta \alpha_{\text{noise}} \approx 3.5 \times 10^{-8}
\label{eq:alpha_correction_value}
\end{equation}

This is \textbf{within the sensitivity} of current $g-2$ experiments and could explain residual discrepancies between theory and experiment.

\subsection{Tone and Framing: Engineering Perspective on Physical Phenomena}

Throughout this section, we have deliberately adopted an \textbf{engineering-focused interpretation} of UBT, framing physical observables through the language of:
\begin{itemize}
\item \textbf{Signal Processing}: Redshift as frame-alignment artifacts, correlation functions as aliasing patterns
\item \textbf{Error Correction}: Conservation law violations as parity check saturation, vacuum fluctuations as decoding noise
\item \textbf{Clock Jitter}: Complex time $\tau = t + i\psi$ as phase-locked loop dynamics, $\psi$ drift as timing uncertainty
\item \textbf{Digital Substrates}: $GF(2^8)$ as the hardware layer, biquaternionic fields as the decoded information
\end{itemize}

This perspective is \textbf{isomorphic} to the standard geometric formulation of UBT but emphasizes testability and falsifiability through quantitative, measurable signatures.

\subsection{Summary of Falsifiable Predictions}

\begin{table}[h]
\centering
\caption{Summary of JWST and LHC Predictions}
\label{tab:predictions_summary}
\begin{tabular}{llcl}
\hline
\textbf{Observable} & \textbf{Instrument} & \textbf{Signature} & \textbf{Falsification Threshold} \\
\hline
Redshift quantization & JWST & $\Delta z = (1+z)/256$ & No periodicity at $3\sigma$ \\
Structural aliasing & JWST & Peak at $\theta \sim 1.4^\circ$ & No peak at $2\sigma$ \\
Digital noise floor & FCC (future) & $\sigma \propto E^2$ & No excess with $\mathcal{L} > 10$ ab$^{-1}$ \\
Parity violations & HL-LHC & $\sim 1$ event per $10^{16}$ collisions & Zero events (inconclusive) \\
$\alpha$ offset & $g-2$ experiments & $\Delta \alpha \sim 3.5 \times 10^{-8}$ & Residual $< 10^{-9}$ \\
\hline
\end{tabular}
\end{table}

All predictions are \textbf{quantitative}, \textbf{measurable}, and \textbf{falsifiable} within the next 10--20 years of observational cosmology and high-energy physics experiments.

\subsection{Concluding Remarks}

The digital-substrate interpretation of UBT provides a concrete, testable framework for experimental validation. Key advantages include:
\begin{itemize}
\item \textbf{Quantitative predictions} with numerical values (not just qualitative trends)
\item \textbf{Clear falsification criteria} (specific observational thresholds)
\item \textbf{Near-term testability} (JWST already operational, HL-LHC commencing)
\item \textbf{Engineering-based reasoning} (familiar to experimental collaborations)
\end{itemize}

While the digital-substrate model is \textbf{interpretive} (the same mathematics underlies the geometric biquaternionic formulation), it offers a valuable complementary perspective for designing experimental tests and communicating with non-theoretical audiences.

Future work should:
\begin{enumerate}
\item Develop detailed Monte Carlo simulations of JWST redshift surveys under the quantization hypothesis
\item Collaborate with ATLAS/CMS trigger groups to optimize searches for parity-violating events
\item Refine vacuum polarization noise calculations for precision QED comparisons
\item Explore connections between buffer depth $D_{\text{buf}}$ and observed large-scale structure
\end{enumerate}

By grounding UBT in measurable, falsifiable predictions, this framework advances the theory from speculative mathematical elegance to empirically testable science.
