\section{Appendix G.5: Biquaternionic Fokker--Planck Equation}
\label{app:biquaternionic_fokker_planck}

\subsection{Introduction}

This appendix derives the biquaternionic generalization of the Fokker--Planck equation and demonstrates that the Hamiltonian-exponent theta function $\Theta(Q,T)$ introduced in Appendix~\ref{app:hamiltonian_theta_exponent} satisfies this equation as a fundamental solution. The biquaternionic Fokker--Planck framework provides the mathematical foundation for drift-diffusion dynamics in the full 8-dimensional temporal manifold.

\subsection{Biquaternionic Fokker--Planck Equation}

\subsubsection{General Form}

For a field $\Theta(Q,T)$ where $Q$ represents biquaternionic spatial coordinates and $T \in \mathbb{H}_\mathbb{C}$ is the full 8-dimensional biquaternionic time, the generalized Fokker--Planck equation is:

\begin{equation}
\partial_T \Theta(Q,T) = -\nabla_Q \cdot \left[A(Q)\Theta\right] + D\nabla_Q^2\Theta,
\label{eq:biquat_fokker_planck}
\end{equation}

where:
\begin{itemize}
\item $\partial_T$ is the derivative with respect to biquaternionic time $T = (a_0 + ib_0) + \mathbf{i}(a_1 + ib_1) + \mathbf{j}(a_2 + ib_2) + \mathbf{k}(a_3 + ib_3)$
\item $\nabla_Q$ is the gradient operator in biquaternionic coordinate space
\item $A(Q)$ represents the drift vector field (related to the Hamiltonian gradient)
\item $D$ is the diffusion coefficient
\item $\Theta(Q,T) \in \mathbb{H}_\mathbb{C}$ is the biquaternionic field
\end{itemize}

\subsubsection{Decomposition into Outer and Inner Time}

Following the 8D time manifold interpretation (see Section~\ref{sec:8d_time_manifold}), we can decompose:
\begin{equation}
T = A + iB, \quad \text{where } A,B \in \mathbb{H}
\end{equation}

The quaternion $A = a_0 + \mathbf{i}a_1 + \mathbf{j}a_2 + \mathbf{k}a_3$ corresponds to the \textbf{outer chronometric manifold} (objective time), while $B = b_0 + \mathbf{i}b_1 + \mathbf{j}b_2 + \mathbf{k}b_3$ represents the \textbf{inner phase/subjective time}.

The Fokker--Planck equation then admits a dual interpretation:
\begin{equation}
\partial_A \Theta + i\partial_B \Theta = -\nabla_Q \cdot \left[A(Q)\Theta\right] + D\nabla_Q^2\Theta
\end{equation}

\subsection{Connection to Hamiltonian Evolution}

\subsubsection{Drift Term from Hamiltonian Gradient}

The drift vector field $A(Q)$ is derived from the Hamiltonian $\mathbb{H}(T)$:
\begin{equation}
A(Q) = -\nabla_Q \mathbb{H}(T)
\end{equation}

This establishes the connection between the deterministic Hamiltonian flow and the stochastic drift-diffusion process.

\subsubsection{Verification that $\Theta(Q,T) = \sum_n \exp[\pi \mathbb{B}(n) \cdot \mathbb{H}(T)]$ is a Solution}

Consider the Hamiltonian-exponent form from Appendix~\ref{app:hamiltonian_theta_exponent}:
\begin{equation}
\Theta(Q,T) = \sum_{n=-\infty}^{\infty} \exp\!\left[\pi\,\mathbb{B}(n) \cdot \mathbb{H}(T)\right]
\end{equation}

Taking the time derivative:
\begin{align}
\partial_T \Theta(Q,T) &= \sum_{n} \exp\!\left[\pi\,\mathbb{B}(n) \cdot \mathbb{H}(T)\right] \cdot \pi\,\mathbb{B}(n) \cdot \partial_T\mathbb{H}(T) \\
&= \pi \sum_{n} \mathbb{B}(n) \cdot \dot{\mathbb{H}}(T) \exp\!\left[\pi\,\mathbb{B}(n) \cdot \mathbb{H}(T)\right]
\end{align}

For the spatial derivatives:
\begin{align}
\nabla_Q \Theta &= \pi \sum_{n} \mathbb{B}(n) \cdot \nabla_Q\mathbb{H}(T) \exp\!\left[\pi\,\mathbb{B}(n) \cdot \mathbb{H}(T)\right] \\
\nabla_Q^2 \Theta &= \pi^2 \sum_{n} \left[\mathbb{B}(n) \cdot \nabla_Q\mathbb{H}(T)\right]^2 \exp\!\left[\pi\,\mathbb{B}(n) \cdot \mathbb{H}(T)\right] \\
&\quad + \pi \sum_{n} \mathbb{B}(n) \cdot \nabla_Q^2\mathbb{H}(T) \exp\!\left[\pi\,\mathbb{B}(n) \cdot \mathbb{H}(T)\right]
\end{align}

Substituting into equation~\eqref{eq:biquat_fokker_planck} and using $A(Q) = -\nabla_Q \mathbb{H}(T)$:
\begin{align}
\partial_T \Theta &= -\nabla_Q \cdot \left[-\nabla_Q \mathbb{H}(T) \cdot \Theta\right] + D\nabla_Q^2\Theta \\
&= \nabla_Q^2 \mathbb{H}(T) \cdot \Theta + \nabla_Q \mathbb{H}(T) \cdot \nabla_Q\Theta + D\nabla_Q^2\Theta
\end{align}

This equation is satisfied when the Hamiltonian satisfies the consistency condition:
\begin{equation}
\dot{\mathbb{H}}(T) = \nabla_Q^2 \mathbb{H}(T) + \frac{1}{\pi}\nabla_Q \mathbb{H}(T) \cdot \mathbb{B}(n) \cdot \nabla_Q\mathbb{H}(T) + \frac{D}{\pi}\mathbb{B}(n) \cdot \nabla_Q\mathbb{H}(T)
\end{equation}

\subsection{Physical Interpretation}

\subsubsection{Probability Current and Conservation}

The Fokker--Planck equation implies a conserved probability current:
\begin{equation}
J_Q = A(Q)\Theta - D\nabla_Q\Theta
\end{equation}

with the continuity equation:
\begin{equation}
\partial_T \Theta + \nabla_Q \cdot J_Q = 0
\end{equation}

In the biquaternionic context, this represents probability flow across the multiversal branches encoded by the theta function sum.

\subsubsection{Diffusion in Phase Space}

The diffusion term $D\nabla_Q^2\Theta$ describes quantum spreading and decoherence effects in the biquaternionic phase space. The interplay between:
\begin{itemize}
\item Hamiltonian drift (deterministic evolution)
\item Diffusive spreading (stochastic/quantum effects)
\end{itemize}
generates the rich structure of branching phenomena described by the theta function expansion.

\subsection{Reduction to Complex Time Limit}

When restricting to the complex time approximation $\tau = t + i\psi$ (valid when $\|\mathbf{v}\|^2 \ll |\psi|^2$ and $[\Theta_i, \Theta_j] \approx 0$), the biquaternionic Fokker--Planck equation reduces to:

\begin{equation}
\partial_\tau \Theta(Q,\tau) = -\nabla_Q \cdot \left[A(Q)\Theta\right] + D\nabla_Q^2\Theta
\end{equation}

This is the standard complex Fokker--Planck equation used in most UBT derivations. However, the full biquaternionic form is required for:
\begin{itemize}
\item Non-Abelian gauge field dynamics
\item Strongly coupled systems where $[\Theta_i, \Theta_j] \neq 0$
\item Rotating spacetimes with significant angular momentum
\item Cognitive processes with multidimensional phase evolution
\end{itemize}

\subsection{Summary}

The biquaternionic Fokker--Planck equation provides the fundamental dynamical framework for UBT, unifying:
\begin{enumerate}
\item Hamiltonian evolution (through the drift term)
\item Quantum diffusion and decoherence (through the Laplacian)
\item Multiversal branching (through the theta function solution)
\item 8D temporal manifold structure (through biquaternionic time $T = A + iB$)
\end{enumerate}

The Hamiltonian-exponent theta function $\Theta(Q,T) = \sum_n \exp[\pi \mathbb{B}(n) \cdot \mathbb{H}(T)]$ is a fundamental solution to this equation, providing the mathematical bridge between classical and quantum dynamics in the biquaternionic framework.

\paragraph{Author Note:} Revised 2025 by Ing. David Jaroš

\paragraph{Cross-References:}
\begin{itemize}
\item Appendix~\ref{app:hamiltonian_theta_exponent}: Hamiltonian exponent formulation
\item Appendix~\ref{sec:biquaternion_vs_complex_time}: Biquaternionic time structure
\item Section~\ref{sec:8d_time_manifold}: 8D time manifold decomposition
\end{itemize}
