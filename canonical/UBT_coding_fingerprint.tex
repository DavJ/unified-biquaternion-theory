
\documentclass[11pt]{article}
\usepackage{amsmath,amssymb,physics,geometry}
\geometry{margin=1in}

\title{Unified Biquaternion Theory as a Spectral Multiplex System\\
Coding-Layer Fingerprints without Spatial Addressing}
\author{David Jaro\v{s}}
\date{\today}

\begin{document}
\maketitle

\section*{Axiom O1: Ontological Equivalence}
\textbf{Axiom O1.}
\emph{No experiment performed entirely within a physical system can
distinguish whether that system constitutes a fundamental reality
or an exact simulation.}

Operationally, this axiom asserts that physical law is exhausted by
internal relational structure and invariant observables.

\section{Paradigm Shift: From Spatial Addressing to Spectral Multiplexing}

UBT abandons the notion that physical information is stored at spatial
addresses. Instead, reality is represented as a superposition of
orthogonal spectral modes. Discrete indices label \emph{channels in a
multiplex}, not locations in space.

This removes the need to define a fixed coordinate grid on the sphere,
resolving the tension between discretization and rotational invariance.

\section{Hilbert Space of Modes}

Let $T(\hat n)$ denote an observable scalar field on the celestial sphere,
$\hat n\in S^2$. Its expansion in spherical harmonics reads
\begin{equation}
T(\hat n)=\sum_{\ell=0}^{\infty}\sum_{m=-\ell}^{\ell} a_{\ell m}Y_{\ell m}(\hat n).
\end{equation}

Under rotations $R\in SO(3)$, the coefficients transform unitarily:
\begin{equation}
a'_{\ell m}=\sum_{m'=-\ell}^{\ell}D^{(\ell)}_{m m'}(R)\,a_{\ell m'}.
\end{equation}

The multipole index $\ell$ labels invariant spectral \emph{slabs}, while
$m$ labels internal degrees of freedom mixed by rotations. This structure
is directly analogous to OFDM systems with unitary channel mixing.

\section{Gauge-Fixed Spectral Multiplex}

To access phase information while preserving rotational covariance, we
introduce a data-driven gauge fixing per multipole $\ell$.

Define the power tensor
\begin{equation}
Q^{(\ell)}_{ij}=\sum_{m,m'} a_{\ell m}^*\,a_{\ell m'}\,
\langle \ell m|L_iL_j|\ell m'\rangle,
\end{equation}
where $L_i$ are generators of $SO(3)$ in the $\ell$ representation.

Let $\hat u_\ell$ be the principal eigenvector of $Q^{(\ell)}$. A rotation
$R_\ell$ mapping $\hat u_\ell\mapsto \hat z$ defines gauge-fixed coefficients
\begin{equation}
\tilde a_{\ell m}=\sum_{m'=-\ell}^{\ell}D^{(\ell)}_{m m'}(R_\ell)\,a_{\ell m'}.
\end{equation}

\paragraph{Remark.}
This procedure introduces no external sky grid. The ``orientation'' is
selected entirely from the field's own multipole content.

\section{Symbolization and Multiplex Ordering}

Phases of gauge-fixed coefficients are quantized into a finite alphabet:
\begin{equation}
s_{\ell m}=\left\lfloor
\frac{256}{2\pi}\big(\arg(\tilde a_{\ell m})+\pi\big)
\right\rfloor
\in\mathbb Z_{256}.
\end{equation}

We define a deterministic multiplex ordering that linearizes spectral modes:
\begin{equation}
k=\kappa(\ell,m)=\ell^2+(m+\ell),
\end{equation}
and the corresponding symbol stream
\begin{equation}
x_k=s_{\ell m}.
\end{equation}

\paragraph{Interpretation.}
The index $k$ labels a spectral \emph{channel}. For example, $k=137$ denotes
the 137th multiplexed mode, not a spatial address.

\section{Coding-Layer Fingerprints}

\subsection{Fingerprint 1: Spectral Hamming Parity}

Group the stream into blocks of eight symbols,
\begin{equation}
\mathbf x^{(b)}=(x_{8b},x_{8b+1},\dots,x_{8b+7}).
\end{equation}

Extract a bit-level feature per symbol, for example the least significant bit
\begin{equation}
\mathbf b^{(b)}=(\mathrm{LSB}(x_{8b}),\dots,\mathrm{LSB}(x_{8b+7}))\in\{0,1\}^8.
\end{equation}

Let $H$ denote the parity-check matrix of the extended Hamming $(8,4,4)$ code.
The syndrome is
\begin{equation}
\mathbf r^{(b)}=H(\mathbf b^{(b)})^T\pmod2.
\end{equation}

Define the fingerprint statistic as the fraction of zero syndromes:
\begin{equation}
P_0=\frac{1}{B}\sum_{b=0}^{B-1}\mathbf 1[\mathbf r^{(b)}=0].
\end{equation}

\paragraph{Detection rule.}
A statistically significant excess of $P_0$ relative to phase-randomized
surrogates (preserving $|a_{\ell m}|$ and $C_\ell$ while randomizing phases)
constitutes a positive detection.

\subsection{Fingerprint 2: Reed--Solomon Consistency across Channels}

Embed $x_k$ into $GF(256)$ (via a fixed byte-to-field isomorphism) and consider
blocks of length $n=255$:
\begin{equation}
\mathbf X^{(b)}=(X_{255b},\dots,X_{255b+254})\in GF(256)^{255}.
\end{equation}

A Reed--Solomon code RS$(n,k)$ can be expressed by the existence of a polynomial
$p(t)$ of degree $<k$ such that
\begin{equation}
X_i = p(\alpha_i),\quad i=0,\dots,n-1,
\end{equation}
for distinct evaluation points $\alpha_i\in GF(256)$.

Operationally, one may compute RS syndromes or attempt bounded-distance decoding.
A non-random excess of near-codeword blocks provides an independent fingerprint.

\section{Time as Phase Drift (Candidate Signature)}

A mismatch between data frame length (255) and a synchronization period (256)
induces a cumulative phase drift per frame
\begin{equation}
\Delta\phi=\frac{2\pi}{256}.
\end{equation}

In a spectral-multiplex picture, such drift manifests as a linear phase slope
in mode space (a shear-like signature in Fourier representations). This is a
candidate (model-dependent) signature of Layer~2/3 timing structure.

\section{Consequences of Non-Detection}

Non-detection of any specific fingerprint falsifies only the corresponding
coding-layer realization (Layer~2/3) and does \emph{not} invalidate the
biquaternionic geometric core (Layer~1). Alternative symbolizations,
alternative algebraic codes, or different invariants may still realize the
UBT coding layer.

\section{Conclusion}

UBT admits a formulation as a high-dimensional OFDM-like multiplex in spectral
space. Discrete indices label channels (modes), not spatial positions. Coding
constraints act across spectral channels, while rotational invariance is handled
by per-multipole gauge fixing derived from the field itself. This yields
concrete, falsifiable fingerprints that do not require spatial addressing.

\end{document}
