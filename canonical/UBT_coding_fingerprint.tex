
\documentclass[11pt]{article}
\usepackage{amsmath,amssymb,physics,geometry}
\geometry{margin=1in}

\title{Unified Biquaternion Theory:\\
Coding Layers, Ontological Equivalence, and Observable Fingerprints}
\author{David Jaro\v{s}}
\date{\today}

\begin{document}
\maketitle

\section*{Axiom O1: Ontological Equivalence}

\textbf{Axiom O1 (Ontological Equivalence).}
\emph{No experiment performed entirely within a physical system can
distinguish whether that system constitutes a fundamental reality
or an exact simulation.}

This axiom does not assert that reality is a simulation, nor does it
postulate an external simulator. It states that the ontological origin
of a physical system has no operational meaning for internal observers.
All physical theories must therefore be formulated purely in terms of
intrinsic structure and observable invariants.

This principle plays a role analogous to Einstein's equivalence
principle in general relativity: it constrains admissible theoretical
structures without introducing additional dynamical assumptions.

\section{Layered Structure of UBT}

Unified Biquaternion Theory (UBT) is organized into three conceptually
distinct layers:

\begin{itemize}
\item \textbf{Layer 1 (Geometric Core):} a biquaternionic formulation of
spacetime and fields, redefining the geometric foundations of general
relativity and quantum field theory.
\item \textbf{Layer 2 (Coding and Stability):} discrete selection rules
and redundancy structures that stabilize admissible physical
configurations.
\item \textbf{Layer 3 (Global Coherence):} long-range algebraic and
renormalization constraints ensuring macroscopic consistency.
\end{itemize}

The fingerprint analysis presented in this work probes exclusively
Layers 2 and 3. Failure of any specific fingerprint falsifies only the
corresponding realization of these layers and does not invalidate the
biquaternionic geometric core.

\section{Geometric Anchoring of Discrete Symbols}

Although the fingerprints studied here probe only Layers~2--3 (coding, stability, and global coherence), it is useful to anchor the discrete symbols to a minimal geometric/variational bridge. The goal of this section is not to \emph{derive} a unique code, but to specify a conservative path by which spherical data may acquire a naturally discretized phase description.

\subsection{Spherical observables and a dimensionless action phase}

Observations on the sky are naturally represented as fields on the celestial sphere $S^2$ at an effectively fixed radius $r=r_\ast$ (e.g., a last-scattering surface). We parametrize a direction by $\hat{n}=(\theta,\phi)$.

In UBT, a natural dimensionless phase variable is the action measured in units of $\hbar$. We therefore introduce an \emph{action-phase map}
\begin{equation}
z(\hat{n}) \equiv \frac{S(\hat{n})}{\hbar} = z_0 + \epsilon\,\delta z(\hat{n}),
\end{equation}
where $z_0$ is the isotropic component and $\delta z(\hat{n})$ encodes directional modulation (projection/selection effects). In the simplest theta-ansatz one may write an effective complex amplitude on the sphere as
\begin{equation}
\Psi(\hat{n}) = \Psi_0\,\vartheta\!\big(z(\hat{n}),\tau\big),
\end{equation}
where $\vartheta$ is a Jacobi theta function and $\tau$ is the complex-time modulus (UBT uses $\tau=t+i\psi$).

\subsection{Quantization hypothesis and $N$-ary phase sectors}

To connect geometry to discrete symbol streams, we adopt a conservative quantization hypothesis: if physical evolution is fundamentally discretized at some time step $\Delta t$ (Planck time being a limiting case), then frequencies and phases become effectively discretized in finite observation windows. Independently of the detailed mechanism, this motivates testing finite alphabets
\begin{equation}
\varphi(\hat{n}) \equiv \arg\!\big(\Psi(\hat{n})\big) \approx \frac{2\pi}{N}\,k(\hat{n}),\qquad k(\hat{n})\in\{0,\dots,N-1\}.
\end{equation}
In this paper we focus on $N\in\{16,256\}$ as practical candidates motivated by multiplex sectorization (16) and high-resolution phase coding (256). The mapping from spherical data to symbols is part of the Layer~2/3 realization and is therefore falsifiable without affecting Layer~1.

\subsection{Symbol extraction from complex coefficients}

Given complex observational coefficients (e.g., spherical harmonic coefficients $a_{\ell m}$), define the discretized symbol
\begin{equation}
s_{\ell m} = \left\lfloor \frac{N}{2\pi}\big(\arg(a_{\ell m})+\pi\big) \right\rfloor \in \mathbb{Z}_N.
\end{equation}
This converts a spherical (or harmonic) observable into an $N$-ary symbol stream suitable for testing parity/redundancy constraints.

\subsection{A note on spherical ``packing'' of a 256-frame}

If a preferred frame length $N=256$ exists in the underlying dynamics, a natural geometric question is how such a frame could manifest on $S^2$. The simplest heuristic is to associate one period with a great-circle azimuthal partition,
\begin{equation}
\Delta\phi = \frac{2\pi}{256},\qquad \Delta\phi \approx 1.40625^{\circ},
\end{equation}
which corresponds to a characteristic angular scale. In multipole space this suggests examining anomalies or phase-structure changes near a characteristic band $\ell\sim 2\pi/\Delta\phi\approx 256$ (and nearby harmonics). This is a \emph{candidate} spherical fingerprint, not a necessity: realistic sky maps use nontrivial pixelizations (e.g., HEALPix) and the projection from UBT variables to observables may smear or shift the effective scale.

\subsection{2D Fourier skew fingerprint (candidate)}

If an $N$-frame is realized as a stream of $N-1$ ``data'' symbols plus one synchronization symbol (a common coding pattern, e.g. $255+1$), then an unwrapped 2D indexing of the stream can exhibit a small shear. A minimal model is a linear shear transform in the 2D Fourier domain,
\begin{equation}
\mathcal{M}_{\text{skew}} =
\begin{pmatrix}
1 & 1/256\\
0 & 1
\end{pmatrix},
\end{equation}
corresponding to a skew angle
\begin{equation}
\theta_{\text{skew}} = \arctan\!\left(\frac{1}{256}\right) \approx 0.2238^{\circ}.
\end{equation}
Operationally, this fingerprint would be sought in planar projections of spherical data after a controlled unwrapping (and careful null tests). We treat it as an exploratory signature that may guide where to look, while emphasizing that its non-detection does not impact Layer~1.

\section{Minimal Coding-Layer Fingerprint}

We now state the simplest operational fingerprint using the discretized
symbols $s_{\ell m}\in\mathbb{Z}_N$ defined in the previous section.

\subsection{Block Formation and Parity Constraints}

The symbols are grouped into blocks of length eight,
\begin{equation}
\mathbf{s}_k = (s_{k,1},\dots,s_{k,8}).
\end{equation}

We test each block against the parity constraints of the extended
Hamming $(8,4,4)$ code. Let $H$ denote its parity-check matrix. The
syndrome vector is
\begin{equation}
\mathbf{r}_k = H \mathbf{s}_k^{T}.
\end{equation}

\subsection{Fingerprint Statistic}

The primary fingerprint statistic is the empirical probability
\begin{equation}
P_0 = \frac{1}{K}\sum_{k=1}^K \delta(\mathbf{r}_k,0),
\end{equation}
where $\delta$ is the Kronecker delta. A statistically significant
excess of $P_0$ relative to phase-randomized surrogate data constitutes
a positive detection of the coding-layer fingerprint.

\paragraph{Interpretation.}
A positive detection supports the presence of discrete stabilizing
selection in Layer~2. Non-detection excludes this specific realization
but leaves the UBT framework intact.

\section{Additional Fingerprint Classes}

\subsection{Phase Quantization Fingerprint}

UBT predicts clustering of phases at discrete values
\begin{equation}
\arg(a_{\ell m}) \in \left\{\frac{2\pi k}{N}\right\},
\end{equation}
testable via circular entropy and Kuiper statistics.

\subsection{Multiscale Fractal Fingerprint}

Hierarchical stability implies scale-invariant correlations in the
fingerprint statistic,
\begin{equation}
P_0(L) \sim L^{-\alpha},
\end{equation}
where $L$ denotes block scale. Detection of nontrivial $\alpha$ supports
Layer~3 coherence.

\section{Consequences of Non-Detection}

Failure to detect the minimal fingerprint rules out the specific
$(8,4,4)$ coding realization. It does not falsify:
\begin{itemize}
\item the biquaternionic geometric layer,
\item alternative coding structures,
\item or global coherence mechanisms.
\end{itemize}

Thus non-detection constrains the model space without collapsing the
theoretical framework.

\section{Conclusion}

UBT predicts observable statistical fingerprints arising from discrete
stabilization layers beyond its geometric core. The minimal fingerprint
presented here is intentionally conservative and falsifiable. Future
work will explore alternative coding realizations and higher-order
fingerprints.

\end{document}
