
\documentclass[11pt]{article}
\usepackage{amsmath,amssymb,physics,geometry}
\geometry{margin=1in}

\title{Unified Biquaternion Theory:\\
Coding Layers, Ontological Equivalence, and Observable Fingerprints}
\author{David Jaro\v{s}}
\date{\today}

\begin{document}
\maketitle

\section*{Axiom O1: Ontological Equivalence}

\textbf{Axiom O1 (Ontological Equivalence).}
\emph{No experiment performed entirely within a physical system can
distinguish whether that system constitutes a fundamental reality
or an exact simulation.}

This axiom does not assert that reality is a simulation, nor does it
postulate an external simulator. It states that the ontological origin
of a physical system has no operational meaning for internal observers.
All physical theories must therefore be formulated purely in terms of
intrinsic structure and observable invariants.

This principle plays a role analogous to Einstein's equivalence
principle in general relativity: it constrains admissible theoretical
structures without introducing additional dynamical assumptions.

\section{Layered Structure of UBT}

Unified Biquaternion Theory (UBT) is organized into three conceptually
distinct layers:

\begin{itemize}
\item \textbf{Layer 1 (Geometric Core):} a biquaternionic formulation of
spacetime and fields, redefining the geometric foundations of general
relativity and quantum field theory.
\item \textbf{Layer 2 (Coding and Stability):} discrete selection rules
and redundancy structures that stabilize admissible physical
configurations.
\item \textbf{Layer 3 (Global Coherence):} long-range algebraic and
renormalization constraints ensuring macroscopic consistency.
\end{itemize}

The fingerprint analysis presented in this work probes exclusively
Layers 2 and 3. Failure of any specific fingerprint falsifies only the
corresponding realization of these layers and does not invalidate the
biquaternionic geometric core.

\section{Minimal Coding-Layer Fingerprint}

\subsection{Discretized Phase Symbols}

Let $a_{\ell m}$ denote complex observational coefficients (for example,
spherical harmonic coefficients of a cosmological field). We define
discretized phase symbols by
\begin{equation}
s_{\ell m} =
\left\lfloor
\frac{N}{2\pi}\left(\arg(a_{\ell m})+\pi\right)
\right\rfloor
\in \mathbb{Z}_N,
\end{equation}
where $N=16$ or $N=256$.

\subsection{Block Formation and Parity Constraints}

The symbols are grouped into blocks of length eight,
\begin{equation}
\mathbf{s}_k = (s_{k,1},\dots,s_{k,8}).
\end{equation}

We test each block against the parity constraints of the extended
Hamming $(8,4,4)$ code. Let $H$ denote its parity-check matrix. The
syndrome vector is
\begin{equation}
\mathbf{r}_k = H \mathbf{s}_k^{T}.
\end{equation}

\subsection{Fingerprint Statistic}

The primary fingerprint statistic is the empirical probability
\begin{equation}
P_0 = \frac{1}{K}\sum_{k=1}^K \delta(\mathbf{r}_k,0),
\end{equation}
where $\delta$ is the Kronecker delta. A statistically significant
excess of $P_0$ relative to phase-randomized surrogate data constitutes
a positive detection of the coding-layer fingerprint.

\paragraph{Interpretation.}
A positive detection supports the presence of discrete stabilizing
selection in Layer~2. Non-detection excludes this specific realization
but leaves the UBT framework intact.

\section{Additional Fingerprint Classes}

\subsection{Phase Quantization Fingerprint}

UBT predicts clustering of phases at discrete values
\begin{equation}
\arg(a_{\ell m}) \in \left\{\frac{2\pi k}{N}\right\},
\end{equation}
testable via circular entropy and Kuiper statistics.

\subsection{Multiscale Fractal Fingerprint}

Hierarchical stability implies scale-invariant correlations in the
fingerprint statistic,
\begin{equation}
P_0(L) \sim L^{-\alpha},
\end{equation}
where $L$ denotes block scale. Detection of nontrivial $\alpha$ supports
Layer~3 coherence.

\section{Consequences of Non-Detection}

Failure to detect the minimal fingerprint rules out the specific
$(8,4,4)$ coding realization. It does not falsify:
\begin{itemize}
\item the biquaternionic geometric layer,
\item alternative coding structures,
\item or global coherence mechanisms.
\end{itemize}

Thus non-detection constrains the model space without collapsing the
theoretical framework.

\section{Conclusion}

UBT predicts observable statistical fingerprints arising from discrete
stabilization layers beyond its geometric core. The minimal fingerprint
presented here is intentionally conservative and falsifiable. Future
work will explore alternative coding realizations and higher-order
fingerprints.

\end{document}
