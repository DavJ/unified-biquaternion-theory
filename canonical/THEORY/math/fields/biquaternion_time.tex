% Canonical Biquaternion Time Definition  
% Version: 2.1 - Phase 2 (Corrected Per User Requirement)
% Date: 2025-11-14
% Status: Canonical - DO NOT DUPLICATE

\section{Biquaternion Time $T_B$}
\label{sec:canonical:biquaternion_time}

\subsection{Canonical Definition}

The fundamental time coordinate in Unified Biquaternion Theory is a \textbf{biquaternion}:

\begin{equation}
\label{eq:canonical:biquaternion_time}
\boxed{T_B = t + i\psi + j\chi + k\xi}
\end{equation}

\noindent where:
\begin{itemize}
    \item $t \in \mathbb{R}$ is the \textbf{real time coordinate} (standard physical time)
    \item $\psi, \chi, \xi \in \mathbb{R}$ are \textbf{imaginary time components}
    \item $i, j, k$ are quaternion units satisfying $i^2 = j^2 = k^2 = ijk = -1$
\end{itemize}

Equivalently, using vector notation:
\begin{equation}
\label{eq:canonical:biquaternion_time_vector}
T_B = t + i(\psi + \mathbf{v} \cdot \boldsymbol{\sigma}) = t + i\psi + j\chi + k\xi
\end{equation}

where $\mathbf{v} = (\chi, \xi, 0)$ and $\boldsymbol{\sigma} = (j, k, ij)$ are quaternion vector components.

\textbf{Note}: This is the canonical, most general formulation. Complex time $\tau = t + i\psi$ is a limiting/simplified case valid when directional components are negligible.

\subsection{Physical Interpretation}

\subsubsection{Real Component: Standard Time}

The real part $t$ represents:
\begin{itemize}
    \item Ordinary physical time
    \item Observable temporal evolution
    \item Causality structure of spacetime
    \item Time measured by physical clocks
\end{itemize}

\subsubsection{Scalar Imaginary Component: Isotropic Phase}

The scalar imaginary part $\psi$ represents:
\begin{itemize}
    \item \textbf{Isotropic phase structure} of the $\Theta$ field
    \item \textbf{Internal auxiliary sector} (speculative; see \texttt{speculative\_extensions/})
    \item \textbf{Scalar dark energy} contributions
    \item \textbf{Universal quantum coherence} (direction-independent)
\end{itemize}

\subsubsection{Vector Imaginary Components: Directional Phases}

The vector imaginary parts $(\chi, \xi)$ represent:
\begin{itemize}
    \item \textbf{Directional phase structures} in spacetime
    \item \textbf{Torsion effects} and spin-orbit coupling
    \item \textbf{Anisotropic dark matter} distributions
    \item \textbf{Directional phase modes} (collective internal sector)
\end{itemize}

\textbf{Critical}: All components ($\psi, \chi, \xi$) are \textbf{dynamical variables} with physical consequences, not mathematical artifacts.

\subsection{Dynamics of Imaginary Components}

The imaginary time components have their own dynamics. The full equations emerge from variation of the action, but schematically:

\begin{equation}
\label{eq:canonical:psi_dynamics}
\Box \psi + m_\psi^2 \psi = J_\psi^{\text{scalar}}
\end{equation}

\begin{equation}
\label{eq:canonical:vec_dynamics}
\nabla^2 \mathbf{v} + m_v^2 \mathbf{v} = \mathbf{J}_v^{\text{vector}}
\end{equation}

\noindent where:
\begin{itemize}
    \item $\Box = \partial_\mu\partial^\mu$ is the d'Alembertian operator
    \item $m_\psi, m_v$ are effective mass scales
    \item $J_\psi, \mathbf{J}_v$ are sources coupling to matter/internal sector
    \item $\mathbf{v} = (\chi, \xi, 0)$ is the vector imaginary component
\end{itemize}

These equations couple through the full $\Theta$ field dynamics.

\subsection{Hierarchical Reduction Structure}

\subsubsection{Complex Time Limit}

When the vector components are negligible ($\|\mathbf{v}\|^2 \ll \psi^2$), or when there is directional isotropy:

\begin{equation}
\label{eq:canonical:complex_limit}
\boxed{\chi, \xi \to 0 \quad \Rightarrow \quad T_B \to \tau = t + i\psi}
\end{equation}

This \textbf{complex time limit} is valid for:
\begin{itemize}
    \item Spherically symmetric systems
    \item Weakly coupled systems
    \item Weakly-coupled isotropic sector
    \item Isotropic cosmological backgrounds
\end{itemize}

\subsubsection{Classical Time Limit}

When all imaginary components vanish:

\begin{equation}
\label{eq:canonical:real_limit}
\psi, \chi, \xi \to 0 \quad \Rightarrow \quad T_B \to t \quad \Rightarrow \quad \text{UBT reduces to standard GR/QFT}
\end{equation}

This ensures:
\begin{itemize}
    \item ✅ Einstein equations recovered exactly
    \item ✅ Standard Model preserved
    \item ✅ All experimental tests of GR/QFT satisfied
\end{itemize}

\subsubsection{Hierarchy Summary}

\begin{equation}
\boxed{T_B = t + i\psi + j\chi + k\xi \quad \xrightarrow{\|\mathbf{v}\| \to 0} \quad \tau = t + i\psi \quad \xrightarrow{\psi \to 0} \quad t}
\end{equation}

\textbf{Biquaternion time} (full) → \textbf{Complex time} (isotropic limit) → \textbf{Classical time} (GR limit)

\subsection{Measurement and Observability}

\subsubsection{Direct Observability}

The imaginary components $(\psi, \chi, \xi)$ are \textbf{not directly observable} in classical measurements because:
\begin{itemize}
    \item Ordinary matter couples only to real metric $g_{\mu\nu} := \text{Re}(\mathcal{G}_{\mu\nu})$ where $\mathcal{G}_{\mu\nu}$ is the biquaternionic metric
    \item Classical observables involve $\text{Re}\,\text{Tr}(\ldots)$
    \item Phase information is hidden in quantum coherence
\end{itemize}

\subsubsection{Indirect Detection}

The imaginary time components can be detected through:
\begin{itemize}
    \item Psychon excitations (via $\Theta$-resonator)
    \item Quantum entanglement signatures
    \item Consciousness-correlated phenomena
    \item Dark sector interactions
    \item Anisotropic dark matter distributions (vector components)
    \item Torsion effects in strong gravity (vector components)
\end{itemize}

See Section~\ref{sec:canonical:theta_resonator} for experimental protocols.

\subsection{Mathematical Properties}

\subsubsection{Biquaternion Conjugation}

The full conjugate combines complex and quaternion conjugation:

\begin{equation}
\label{eq:canonical:tb_conjugate}
T_B^\dagger = t - i\psi - j\chi - k\xi
\end{equation}

\subsubsection{Norm}

\begin{equation}
\label{eq:canonical:tb_norm}
|T_B|^2 = T_B \cdot T_B^\dagger = t^2 + \psi^2 + \chi^2 + \xi^2
\end{equation}

\subsubsection{Scalar and Vector Parts}

\begin{equation}
\label{eq:canonical:tb_decomposition}
T_B = t + i\psi_{\text{scalar}} + \mathbf{v}_{\text{vector}}
\end{equation}

where $\mathbf{v}_{\text{vector}} = j\chi + k\xi$.

\subsection{Derivatives and Calculus}

\subsubsection{Partial Derivatives}

For a function $f(T_B) = f(t,\psi,\chi,\xi)$:

\begin{equation}
\label{eq:canonical:tb_derivatives}
\frac{\partial f}{\partial T_B} = \frac{1}{4}\left(\frac{\partial f}{\partial t} - i\frac{\partial f}{\partial \psi} - j\frac{\partial f}{\partial \chi} - k\frac{\partial f}{\partial \xi}\right)
\end{equation}

In the complex time limit ($\chi, \xi \to 0$), this reduces to:

\begin{equation}
\frac{\partial f}{\partial \tau} = \frac{1}{2}\left(\frac{\partial f}{\partial t} - i\frac{\partial f}{\partial \psi}\right)
\end{equation}

\subsubsection{Integration}

Biquaternion time integration:

\begin{equation}
\label{eq:canonical:tb_integration}
\int dT_B = \int dt + i \int d\psi + j \int d\chi + k \int d\xi
\end{equation}

For physical observables, we typically integrate over real time only:

\begin{equation}
\mathcal{O}_{\text{phys}} = \int dt \, \text{Re}\,\mathcal{O}(T_B)|_{\psi,\chi,\xi=f(t)}
\end{equation}

\subsection{Relation to Other Formulations}

\subsubsection{Euclidean Time (Wick Rotation)}

Biquaternion time $T_B$ is \textbf{NOT} the same as Euclidean time from Wick rotation:

\begin{itemize}
    \item \textbf{Wick rotation}: $t \to -i t_E$ (analytical continuation)
    \item \textbf{UBT biquaternion time}: $T_B = t + i\psi + j\chi + k\xi$ (physical extension)
\end{itemize}

In UBT, \textit{all} components $t, \psi, \chi, \xi$ are real, physical parameters.

\subsubsection{Thermal Field Theory}

At finite temperature $T$, the scalar component $\psi$ may be related to thermal time:

\begin{equation}
\label{eq:canonical:thermal_psi}
\psi \sim \beta = \frac{1}{k_B T}
\end{equation}

but this is a special case, not a general requirement. The vector components $(\chi, \xi)$ are independent of thermal effects.

\subsection{Conflict Resolution}

This definition supersedes the following conflicting versions:

\begin{enumerate}
    \item ❌ \textbf{Drift-diffusion Fokker-Planck variant}: imaginary components as stochastic variables only
    \item ❌ \textbf{Toroidal variant with $\theta$-functions}: time as modular parameter only
    \item ❌ \textbf{Hermitized variant (Appendix F)}: imaginary components as purely mathematical
    \item ❌ \textbf{Complex time only variants}: missing directional structure
\end{enumerate}

\textbf{Canonical resolution}: $T_B = t + i\psi + j\chi + k\xi$ where \textit{all imaginary components} are \textbf{dynamical physical fields} with equations of motion. Complex time $\tau = t + i\psi$ is the limiting case valid for isotropic systems.

\subsection{Coordinate Systems}

\subsubsection{Biquaternion Coordinates}

The full coordinate set is:

\begin{equation}
\label{eq:canonical:full_coordinates}
(q, T_B) = (q^0, q^1, q^2, q^3, t, \psi, \chi, \xi)
\end{equation}

where $q^\mu$ ($\mu=0,1,2,3$) are biquaternion spatial coordinates.

In the complex time limit:
\begin{equation}
(q, \tau) = (q^0, q^1, q^2, q^3, t, \psi)
\end{equation}

\subsubsection{Metric Signature}

The extended metric in full biquaternion time has signature:

\begin{equation}
\label{eq:canonical:extended_signature}
\text{signature}(g_{AB}) = (+, -, -, -, -, +, +, +) \quad \text{or similar}
\end{equation}

depending on convention. The imaginary time coordinates typically have mixed signature from spatial coordinates.

\subsection{Conservation Laws}

Energy-momentum conservation in biquaternion time:

\begin{equation}
\label{eq:canonical:biq_conservation}
\frac{\partial T^{\mu\nu}}{\partial T_B} = 0
\end{equation}

This expands to separate conservation laws for each component:
\begin{align}
\frac{\partial T^{\mu\nu}}{\partial t} &= \text{sources from } \psi, \chi, \xi \\
\frac{\partial T^{\mu\nu}}{\partial \psi} &= \text{phase sector coupling} \\
\frac{\partial T^{\mu\nu}}{\partial \chi}, \frac{\partial T^{\mu\nu}}{\partial \xi} &= \text{torsion/anisotropic sources}
\end{align}

\subsection{Causality Structure}

The causal structure is determined by:

\begin{equation}
\label{eq:canonical:causality}
ds^2 = g_{\mu\nu}(T_B) dx^\mu dx^\nu
\end{equation}

where $g_{\mu\nu} := \text{Re}(\mathcal{G}_{\mu\nu}(T_B))$ is the real projection of the biquaternionic metric. Light cones may rotate in the extended time space, but physical causality (in real time $t$) is preserved through appropriate boundary conditions.

\subsection{Symbol Standardization}

\begin{tabular}{ll}
\textbf{Symbol} & \textbf{Meaning} \\
\hline
$T_B$ & Biquaternion time coordinate (canonical) \\
$\tau$ & Complex time coordinate (isotropic limit) \\
$t$ & Real time component \\
$\psi$ & Scalar imaginary time component \\
$\chi, \xi$ & Vector imaginary time components \\
$T_B^\dagger$ & Biquaternion conjugate of $T_B$ \\
\end{tabular}

\vspace{1em}

\textbf{Forbidden uses}:
\begin{itemize}
    \item ❌ $\psi$ for wavefunction (use $\Psi$ instead)
    \item ❌ $\psi$ for spinor (use $\psi_\text{spinor}$ if needed)
    \item ❌ $\tau$ for proper time (use $s$ or $\lambda$)
    \item ❌ $T$ for temperature (use $\mathcal{T}$ or specify $T_{\text{temp}}$)
\end{itemize}

\subsection{Units}

In natural units ($\hbar = c = 1$):

\begin{equation}
[t] = [\psi] = [\chi] = [\xi] = [\text{length}] = [\text{time}]
\end{equation}

All components have dimension of length (or inverse energy).

\subsection{Historical Note}

This canonical definition establishes biquaternion time $T_B = t + i\psi + j\chi + k\xi$ as the \textbf{fundamental, most general time coordinate} in UBT. The imaginary components $(\psi, \chi, \xi)$ are \textbf{dynamical physical fields}, not passive parameters. 

Complex time $\tau = t + i\psi$ is a \textbf{simplification/limiting case} valid when directional structure is negligible. This hierarchical structure resolves ambiguities in earlier formulations and provides a consistent foundation for:
\begin{itemize}
    \item Internal auxiliary sector physics (speculative; see \texttt{speculative\_extensions/})
    \item Dark sector coupling (scalar dark energy, anisotropic dark matter)
    \item Spacetime torsion and spin effects
    \item Directional quantum coherence
\end{itemize}

\subsection{Simplification: Complex Time Limit}

\textbf{When to use complex time}: For systems with directional isotropy or weak coupling, the complex time limit $\tau = t + i\psi$ is sufficient. This includes:

\begin{itemize}
    \item Spherically symmetric gravitational systems
    \item Cosmological models with isotropy
    \item Individual weakly-coupled sector (isotropic modes)
    \item Weak-field approximations
\end{itemize}

\textbf{When full biquaternion time is needed}:
\begin{itemize}
    \item Strong torsion effects
    \item Anisotropic dark matter distributions
    \item Collective phase-sector phenomena (speculative)
    \item Non-Abelian gauge dynamics beyond SM
    \item Spin-orbit coupling in strong fields
\end{itemize}

The decision to use $\tau$ vs $T_B$ is problem-dependent, but $T_B$ is always the more fundamental, canonical formulation.

% End of canonical complex time definition
