% Canonical Symbol Dictionary for UBT
% Version: 1.0
% Date: 2025-11-14
% Status: Canonical - DO NOT DUPLICATE

\section{Symbol Dictionary and Notation Conventions}
\label{sec:canonical:symbols}

\subsection{Purpose}

This section establishes the \textbf{unique, canonical meaning} of all symbols used in the Unified Biquaternion Theory. Each symbol has \textbf{exactly one meaning} to avoid confusion and ensure consistency across all documents.

\subsection{Reserved Symbols — Single Meaning Only}

\begin{longtable}{|l|p{6cm}|p{5cm}|}
\hline
\textbf{Symbol} & \textbf{Canonical Meaning} & \textbf{Notes} \\
\hline
\endfirsthead
\hline
\textbf{Symbol} & \textbf{Canonical Meaning} & \textbf{Notes} \\
\hline
\endhead

$\alpha$ & Fine structure constant $\approx 1/137.036$ & NO other uses (angle, decay rate, etc.) \\
\hline
$\psi$ & Imaginary component of complex time & NOT wavefunction, NOT spinor \\
\hline
$\tau$ & Complex time $= t + i\psi$ & NOT proper time \\
\hline
$\Theta$ & Fundamental biquaternion field & Capital theta only for field \\
\hline
$q$ & Biquaternion coordinate (4 DOF) & NOT charge \\
\hline
$g_{\mu\nu}$ & Metric tensor & NO other metric symbols \\
\hline
$T_{\mu\nu}$ & Stress-energy tensor & Canonical form only \\
\hline
$F_{\mu\nu}$ & Electromagnetic field strength & QED only \\
\hline
$G^a_{\mu\nu}$ & Gluon field strength & QCD only \\
\hline
$W^i_{\mu\nu}$ & Weak field strength & Weak interactions \\
\hline
$B_{\mu\nu}$ & Hypercharge field strength & Electroweak \\
\hline

\caption{Reserved symbols with unique canonical meanings}
\label{tab:canonical:reserved_symbols}
\end{longtable}

\subsection{Index Conventions}

\subsubsection{Spacetime Indices}

\begin{itemize}
    \item \textbf{Greek indices} $\mu, \nu, \rho, \sigma, \lambda = 0, 1, 2, 3$ for spacetime coordinates
    \item \textbf{Coordinates}: $x^\mu = (x^0, x^1, x^2, x^3) = (t, x, y, z)$ or $(ct, x, y, z)$
    \item \textbf{Summation convention}: Repeated indices are summed (Einstein convention)
\end{itemize}

\subsubsection{Spatial Indices}

\begin{itemize}
    \item \textbf{Latin indices} $i, j, k = 1, 2, 3$ for spatial coordinates only
    \item \textbf{Coordinates}: $x^i = (x^1, x^2, x^3) = (x, y, z)$
\end{itemize}

\subsubsection{Gauge Indices}

\begin{itemize}
    \item \textbf{Color indices} $a, b, c = 1, 2, \ldots, 8$ for $SU(3)$ adjoint representation
    \item \textbf{Weak isospin indices} $i, j, k = 1, 2, 3$ for $SU(2)$ generators
    \item \textbf{Flavor indices} $f, g = u, d, s, c, b, t$ for quark flavors
    \item \textbf{Generation indices} $\alpha, \beta = 1, 2, 3$ for fermion generations
\end{itemize}

\subsubsection{Matrix Indices}

\begin{itemize}
    \item \textbf{Matrix elements} $A, B, C = 0, 1, 2, 3$ for $\Theta$ components: $\Theta_{AB}$
    \item \textbf{Internal indices} $m, n = 1, 2, \ldots, N$ for general matrix dimensions
\end{itemize}

\subsection{Field and Operator Notation}

\begin{longtable}{|l|p{6cm}|p{5cm}|}
\hline
\textbf{Symbol} & \textbf{Meaning} & \textbf{Context} \\
\hline
\endfirsthead
\hline
\textbf{Symbol} & \textbf{Meaning} & \textbf{Context} \\
\hline
\endhead

$\Theta(q,\tau)$ & Fundamental biquaternion field & Core UBT field \\
\hline
$\Theta^\dagger$ & Hermitian conjugate of $\Theta$ & $(\bar{\Theta})^T$ \\
\hline
$\Psi$ & Wavefunction (if needed) & Use capital psi \\
\hline
$A_\mu$ & Electromagnetic potential & QED photon field \\
\hline
$G^a_\mu$ & Gluon field & QCD, $a=1,\ldots,8$ \\
\hline
$W^i_\mu$ & Weak boson field & Electroweak, $i=1,2,3$ \\
\hline
$B_\mu$ & Hypercharge field & Electroweak \\
\hline
$\Phi$ & Higgs field & Electroweak symmetry breaking \\
\hline

\caption{Field and operator notation}
\label{tab:canonical:field_notation}
\end{longtable}

\subsection{Derivative Notation}

\begin{longtable}{|l|p{6cm}|p{5cm}|}
\hline
\textbf{Symbol} & \textbf{Meaning} & \textbf{Notes} \\
\hline
\endfirsthead
\hline
\textbf{Symbol} & \textbf{Meaning} & \textbf{Notes} \\
\hline
\endhead

$\partial_\mu$ & Partial derivative $\frac{\partial}{\partial x^\mu}$ & Coordinate derivative \\
\hline
$\partial^\mu$ & Contravariant derivative $g^{\mu\nu}\partial_\nu$ & Raised index \\
\hline
$\nabla_\mu$ & Covariant derivative (gravity) & Includes Christoffel symbols \\
\hline
$D_\mu$ & Gauge covariant derivative & Includes gauge connection \\
\hline
$\nabla^\dagger$ & Biquaternionic conjugate derivative & UBT-specific \\
\hline
$\Box$ & d'Alembertian $\partial^\mu\partial_\mu$ & Wave operator \\
\hline

\caption{Derivative notation}
\label{tab:canonical:derivative_notation}
\end{longtable}

\subsection{Coupling Constants}

\begin{longtable}{|l|p{4cm}|p{3cm}|p{3cm}|}
\hline
\textbf{Symbol} & \textbf{Meaning} & \textbf{Value} & \textbf{Status} \\
\hline
\endfirsthead
\hline
\textbf{Symbol} & \textbf{Meaning} & \textbf{Value} & \textbf{Status} \\
\hline
\endhead

$\alpha$ & Fine structure constant & $\approx 1/137.036$ & Predicted \\
\hline
$\alpha_s$ & Strong coupling & $\approx 0.118$ at $M_Z$ & Predicted \\
\hline
$g$ & Weak coupling & $SU(2)_L$ & Derived \\
\hline
$g'$ & Hypercharge coupling & $U(1)_Y$ & Derived \\
\hline
$g_s$ & Strong coupling & $SU(3)_c$ & Derived \\
\hline
$e$ & Elementary charge & $\sqrt{4\pi\alpha}$ & Input \\
\hline
$G$ & Newton's constant & $6.674 \times 10^{-11}$ m³/kg/s² & Input \\
\hline

\caption{Coupling constants}
\label{tab:canonical:coupling_constants}
\end{longtable}

\subsection{Mass Scales}

\begin{longtable}{|l|p{5cm}|p{4cm}|p{3cm}|}
\hline
\textbf{Symbol} & \textbf{Meaning} & \textbf{Approximate Value} & \textbf{Status} \\
\hline
\endfirsthead
\hline
\textbf{Symbol} & \textbf{Meaning} & \textbf{Approximate Value} & \textbf{Status} \\
\hline
\endhead

$m_e$ & Electron mass & $0.511$ MeV & Predicted \\
\hline
$m_\mu$ & Muon mass & $105.7$ MeV & Predicted \\
\hline
$m_\tau$ & Tau mass & $1.777$ GeV & Predicted \\
\hline
$m_p$ & Proton mass & $938.3$ MeV & Input \\
\hline
$\Lambda_{\text{QCD}}$ & QCD scale & $\sim 200$ MeV & Predicted \\
\hline
$M_W$ & W boson mass & $80.4$ GeV & Derived \\
\hline
$M_Z$ & Z boson mass & $91.2$ GeV & Derived \\
\hline
$m_H$ & Higgs mass & $125$ GeV & Input \\
\hline
$M_{\text{Pl}}$ & Planck mass & $1.22 \times 10^{19}$ GeV & Fundamental \\
\hline

\caption{Mass scales}
\label{tab:canonical:mass_scales}
\end{longtable}

\subsection{Forbidden Symbol Uses}

To maintain clarity and avoid conflicts, the following uses are \textbf{explicitly forbidden}:

\begin{enumerate}
    \item ❌ $\alpha$ for any angle, decay rate, or parameter other than fine structure constant
    \item ❌ $\psi$ for wavefunction (use $\Psi$ instead) or spinor (use $\psi_{\text{spinor}}$)
    \item ❌ $\tau$ for proper time (use $s$ or $\lambda$)
    \item ❌ $q$ for electric charge (use $Q$ or $e$)
    \item ❌ $\Theta$ (lowercase theta) for field (reserved for angles if needed)
    \item ❌ Multiple definitions of metric (only $g_{\mu\nu}$)
    \item ❌ Alternative stress-energy symbols (only $T_{\mu\nu}$)
\end{enumerate}

\subsection{Metric Signature Convention}

\textbf{Default signature}: $(+, -, -, -)$ (mostly minus, timelike positive)

\begin{itemize}
    \item $g_{00} > 0$ (timelike positive)
    \item $g_{11}, g_{22}, g_{33} < 0$ (spacelike negative)
\end{itemize}

Alternative signature $(-, +, +, +)$ may be used in specific contexts with explicit notice.

\subsection{Unit Conventions}

\subsubsection{Natural Units}

Default: $\hbar = c = 1$

\begin{itemize}
    \item Energy, mass, temperature have dimension $[\text{mass}]$
    \item Length, time have dimension $[\text{mass}]^{-1}$
    \item Action is dimensionless
\end{itemize}

\subsubsection{SI Units}

When presenting results:
\begin{itemize}
    \item Masses in eV, keV, MeV, GeV
    \item Lengths in meters, nm, fm
    \item Times in seconds
    \item Energies in eV, joules
\end{itemize}

\subsection{Special Function Notation}

\begin{longtable}{|l|p{6cm}|p{5cm}|}
\hline
\textbf{Symbol} & \textbf{Meaning} & \textbf{Notes} \\
\hline
\endfirsthead
\hline
\textbf{Symbol} & \textbf{Meaning} & \textbf{Notes} \\
\hline
\endhead

$\text{Tr}(A)$ & Matrix trace & $\sum_i A_{ii}$ \\
\hline
$\det(A)$ & Matrix determinant & Determinant \\
\hline
$\text{Re}(z)$ & Real part & Real component \\
\hline
$\text{Im}(z)$ & Imaginary part & Imaginary component \\
\hline
$\bar{z}$ & Complex conjugate & Conjugation \\
\hline
$\theta_i(z,\tau)$ & Jacobi theta functions & $i=1,2,3,4$ \\
\hline
$\zeta(s)$ & Riemann zeta function & Number theory \\
\hline

\caption{Special function notation}
\label{tab:canonical:special_functions}
\end{longtable}

\subsection{Tensor and Matrix Operations}

\begin{longtable}{|l|p{6cm}|p{5cm}|}
\hline
\textbf{Operation} & \textbf{Notation} & \textbf{Meaning} \\
\hline
\endfirsthead
\hline
\textbf{Operation} & \textbf{Notation} & \textbf{Meaning} \\
\hline
\endhead

Matrix product & $AB$ & $(AB)_{ij} = \sum_k A_{ik}B_{kj}$ \\
\hline
Tensor product & $A \otimes B$ & Outer product \\
\hline
Commutator & $[A,B]$ & $AB - BA$ \\
\hline
Anticommutator & $\{A,B\}$ & $AB + BA$ \\
\hline
Covariant derivative & $\nabla_\mu T_{\nu\rho}$ & Includes connection \\
\hline
Lie derivative & $\mathcal{L}_X T$ & Along vector field $X$ \\
\hline

\caption{Tensor and matrix operations}
\label{tab:canonical:tensor_operations}
\end{longtable}

\subsection{Abbreviations and Acronyms}

\begin{longtable}{|l|p{8cm}|}
\hline
\textbf{Acronym} & \textbf{Meaning} \\
\hline
\endfirsthead
\hline
\textbf{Acronym} & \textbf{Meaning} \\
\hline
\endhead

UBT & Unified Biquaternion Theory \\
\hline
GR & General Relativity \\
\hline
QFT & Quantum Field Theory \\
\hline
QED & Quantum Electrodynamics \\
\hline
QCD & Quantum Chromodynamics \\
\hline
SM & Standard Model \\
\hline
EW & Electroweak \\
\hline
GUT & Grand Unified Theory \\
\hline
CKM & Cabibbo-Kobayashi-Maskawa (quark mixing matrix) \\
\hline
PMNS & Pontecorvo-Maki-Nakagawa-Sakata (neutrino mixing matrix) \\
\hline
EWSB & Electroweak Symmetry Breaking \\
\hline
VEV & Vacuum Expectation Value \\
\hline
DOF & Degrees of Freedom \\
\hline

\caption{Abbreviations and acronyms}
\label{tab:canonical:abbreviations}
\end{longtable}

\subsection{Version Control}

This symbol dictionary is version-controlled and canonical. Any proposed changes must:
\begin{enumerate}
    \item Be justified by theoretical necessity
    \item Not conflict with existing usage
    \item Be documented in changelog
    \item Update all affected documents
\end{enumerate}

\subsection{Enforcement}

All UBT documents must:
\begin{itemize}
    \item Comply with this symbol dictionary
    \item Report conflicts as errors
    \item Reference this section for definitions
    \item Avoid introducing new symbol meanings
\end{itemize}

% End of canonical symbol dictionary
