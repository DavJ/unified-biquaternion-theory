% Fundamental Biquaternionic Metric Definition
% Version: 1.0
% Date: 2026-01-07
% Status: Canonical - FUNDAMENTAL GEOMETRY

\section{The Fundamental Biquaternionic Metric $\mathcal{G}_{\mu\nu}$}
\label{sec:canonical:biquaternion_metric}

\subsection{Fundamental Postulate}

\textbf{The spacetime metric is fundamentally a biquaternionic object, not a real tensor.}

The fundamental geometric object in UBT is the \textbf{biquaternionic metric}:

\begin{equation}
\label{eq:canonical:biq_metric_fundamental}
\boxed{\mathcal{G}_{\mu\nu}(x) \in \mathbb{B} = \mathbb{H} \otimes \mathbb{C}}
\end{equation}

\noindent where $\mathbb{B}$ denotes the algebra of biquaternions (complex quaternions).

\subsection{Biquaternionic Decomposition}

The biquaternionic metric admits the decomposition:

\begin{equation}
\label{eq:canonical:biq_metric_decomposition}
\boxed{\mathcal{G}_{\mu\nu} = g_{\mu\nu} + \mathbf{I} h_{\mu\nu} + \mathbf{J} \cdot \mathbf{k}_{\mu\nu}}
\end{equation}

\noindent where:
\begin{itemize}
    \item $g_{\mu\nu} \in \mathbb{R}$ is the \textbf{real projection} (classical GR sector)
    \item $h_{\mu\nu} \in \mathbb{R}$ represents \textbf{phase curvature} (imaginary scalar component)
    \item $\mathbf{k}_{\mu\nu} = (k^1_{\mu\nu}, k^2_{\mu\nu}, k^3_{\mu\nu}) \in \mathbb{R}^3$ represents \textbf{inertial and causal geometry} (quaternionic vector components)
    \item $\mathbf{I}$ is the imaginary unit ($\mathbf{I}^2 = -1$)
    \item $\mathbf{J} = (j_1, j_2, j_3)$ are the quaternionic basis elements ($j_i j_j = -\delta_{ij} + \epsilon_{ijk} j_k$)
\end{itemize}

\textbf{Non-commutativity}: The biquaternionic components do NOT commute. The ordering of products matters:
\begin{equation}
\mathcal{G}_{\mu\nu} \mathcal{G}_{\rho\sigma} \neq \mathcal{G}_{\rho\sigma} \mathcal{G}_{\mu\nu} \quad \text{(in general)}
\end{equation}

\subsection{Mandatory Projection Rule}

\begin{tcolorbox}[colback=red!5!white,colframe=red!75!black,title=Mandatory Rule]
The classical metric $g_{\mu\nu}$ used in General Relativity is \textbf{NOT fundamental}. It is defined exclusively as the real projection of the biquaternionic metric:

\begin{equation}
\label{eq:canonical:metric_projection_rule}
\boxed{g_{\mu\nu} := \text{Re}(\mathcal{G}_{\mu\nu})}
\end{equation}

\textbf{It is FORBIDDEN to introduce $g_{\mu\nu}$ without explicit reference to $\mathcal{G}_{\mu\nu}$.}
\end{tcolorbox}

\subsection{Derivation from Tetrad}

The biquaternionic metric is \textbf{not postulated directly}. Instead, it is derived from the more fundamental biquaternionic tetrad field $E_\mu(x)$ (see Section~\ref{sec:canonical:biquaternion_tetrad}):

\begin{equation}
\label{eq:canonical:biq_metric_from_tetrad}
\mathcal{G}_{\mu\nu} = \text{Sc}(E_\mu E_\nu^\dagger)
\end{equation}

\noindent where:
\begin{itemize}
    \item $E_\mu(x) \in \mathbb{B}$ is the biquaternionic tetrad
    \item $E_\nu^\dagger$ is the Hermitian conjugate
    \item $\text{Sc}(\cdot)$ denotes the scalar part of the biquaternion
\end{itemize}

This construction ensures:
\begin{enumerate}
    \item \textbf{Hermiticity}: $\mathcal{G}_{\mu\nu}^\dagger = \mathcal{G}_{\nu\mu}$ (generalized symmetry)
    \item \textbf{Positive signature}: Appropriate for Lorentzian geometry
    \item \textbf{Gauge covariance}: Tetrad transformations preserve metric structure
\end{enumerate}

\subsection{Properties of the Biquaternionic Metric}

\subsubsection{Hermiticity}

\begin{equation}
\label{eq:canonical:biq_metric_hermiticity}
\mathcal{G}_{\mu\nu}^\dagger = \mathcal{G}_{\nu\mu}
\end{equation}

This replaces the classical symmetry $g_{\mu\nu} = g_{\nu\mu}$ with a more general Hermitian condition.

\subsubsection{Inverse Metric}

The biquaternionic inverse metric $\mathcal{G}^{\mu\nu}$ satisfies:

\begin{equation}
\label{eq:canonical:biq_inverse_metric}
\mathcal{G}^{\mu\rho} \star \mathcal{G}_{\rho\nu} = \delta^\mu_\nu
\end{equation}

where $\star$ denotes the biquaternionic product (non-commutative).

\subsubsection{Signature}

In the real limit ($h_{\mu\nu} \to 0$, $\mathbf{k}_{\mu\nu} \to 0$):

\begin{equation}
\text{signature}(\mathcal{G}_{\mu\nu}) \to \text{signature}(g_{\mu\nu}) = (+, -, -, -)
\end{equation}

The Lorentzian signature emerges from the real projection.

\subsection{Physical Interpretation of Components}

\subsubsection{Real Component: $g_{\mu\nu}$}

\begin{itemize}
    \item \textbf{Classical spacetime geometry}
    \item Couples to ordinary matter and energy
    \item Satisfies Einstein's equations in the real limit
    \item \textbf{Observable} via standard GR experiments
\end{itemize}

\subsubsection{Phase Curvature: $h_{\mu\nu}$}

\begin{itemize}
    \item \textbf{Imaginary time curvature}
    \item Couples to phase structure of the $\Theta$ field
    \item Represents non-local energy configurations
    \item \textbf{Invisible to classical observations} (ordinary matter couples only to $g_{\mu\nu}$)
    \item Responsible for:
    \begin{itemize}
        \item Dark energy effects
        \item Consciousness substrate (via psychon fields)
        \item Phase-locked coherent states
    \end{itemize}
\end{itemize}

\subsubsection{Inertial Geometry: $\mathbf{k}_{\mu\nu}$}

\begin{itemize}
    \item \textbf{Quaternionic directional components}
    \item Encodes torsion and non-metricity
    \item Couples to spin and angular momentum in strong gravity
    \item Responsible for:
    \begin{itemize}
        \item Dark matter halos (via p-adic extensions)
        \item Frame-dragging modifications
        \item Directional asymmetries in gravitational effects
    \end{itemize}
\end{itemize}

\subsection{Line Element}

The fundamental line element in biquaternionic spacetime is:

\begin{equation}
\label{eq:canonical:biq_line_element}
d\mathcal{S}^2 = \mathcal{G}_{\mu\nu} dx^\mu \otimes dx^\nu
\end{equation}

where $\otimes$ denotes the biquaternionic tensor product.

In the real limit:
\begin{equation}
\text{Re}(d\mathcal{S}^2) = ds^2 = g_{\mu\nu} dx^\mu dx^\nu
\end{equation}

This recovers the classical proper distance.

\subsection{Relation to Einstein Metric}

\begin{tcolorbox}[colback=blue!5!white,colframe=blue!75!black,title=General Relativity as Real Projection]
\textbf{General Relativity arises as the real, commutative projection of the fundamental biquaternionic geometry of spacetime.}

Einstein's metric tensor is:
\begin{equation}
g_{\mu\nu}^{\text{GR}} = \text{Re}(\mathcal{G}_{\mu\nu})
\end{equation}

\textbf{Apparent violations} such as antigravity or causal drift correspond to non-real sectors of the metric and curvature, not to exotic matter.

In the limit where imaginary components vanish:
\begin{equation}
h_{\mu\nu} \to 0, \quad \mathbf{k}_{\mu\nu} \to 0 \quad \Rightarrow \quad \mathcal{G}_{\mu\nu} \to g_{\mu\nu}
\end{equation}

UBT \textbf{exactly reproduces} General Relativity. This includes:
\begin{itemize}
    \item Schwarzschild solution (black holes)
    \item Kerr solution (rotating black holes)
    \item FLRW cosmology
    \item Gravitational waves
    \item Perihelion precession
    \item Light bending
\end{itemize}

\textbf{UBT does not contradict GR. It generalizes and embeds it.}
\end{tcolorbox}

\subsection{Exotic Regimes}

Solutions with non-vanishing imaginary components:

\begin{equation}
\text{Im}(\mathcal{G}_{\mu\nu}) \neq 0
\end{equation}

are \textbf{physically consistent within UBT} but remain \textbf{invisible to classical GR observations}.

These exotic regimes are responsible for:
\begin{enumerate}
    \item \textbf{Pseudo-antigravitational behavior}: Apparent repulsion due to phase curvature
    \item \textbf{Phase invisibility}: Dark matter and dark energy effects
    \item \textbf{Local temporal drift}: Imaginary time flow variations
    \item \textbf{Consciousness coupling}: Psychon field interactions
    \item \textbf{Quantum coherence preservation}: Phase-locked states in biological systems
\end{enumerate}

See Section~\ref{sec:canonical:exotic_regimes} for detailed analysis.

\subsection{Coordinate Transformations}

Under diffeomorphisms $x^\mu \to x'^\mu$:

\begin{equation}
\label{eq:canonical:biq_metric_diffeomorphism}
\mathcal{G}'_{\alpha\beta}(x') = \frac{\partial x^\mu}{\partial x'^\alpha} \frac{\partial x^\nu}{\partial x'^\beta} \mathcal{G}_{\mu\nu}(x)
\end{equation}

This is the standard tensor transformation law, extended to biquaternionic objects.

\subsection{Gauge Transformations}

Under local biquaternionic gauge transformations:

\begin{equation}
\mathcal{G}_{\mu\nu} \to U \mathcal{G}_{\mu\nu} U^\dagger
\end{equation}

where $U \in \text{SU(2)} \times \text{U(1)}$ acts on the internal biquaternionic structure.

The real projection is gauge-invariant:
\begin{equation}
\text{Re}(U \mathcal{G}_{\mu\nu} U^\dagger) = \text{Re}(\mathcal{G}_{\mu\nu}) = g_{\mu\nu}
\end{equation}

\subsection{Curvature Coupling}

The biquaternionic metric determines the biquaternionic connection $\Omega_\mu$ (see Section~\ref{sec:canonical:biquaternion_connection}) and hence the biquaternionic curvature $\mathcal{R}_{\mu\nu\rho\sigma}$ (see Section~\ref{sec:canonical:biquaternion_curvature}).

\subsection{Field Equation}

The fundamental field equation of UBT is:

\begin{equation}
\label{eq:canonical:biq_field_equation}
\boxed{\mathcal{G}_{\mu\nu} = \kappa \mathcal{T}_{\mu\nu}}
\end{equation}

where:
\begin{itemize}
    \item $\mathcal{G}_{\mu\nu}$ is the biquaternionic Einstein tensor (defined from $\mathcal{G}_{\mu\nu}$ via biquaternionic curvature)
    \item $\mathcal{T}_{\mu\nu}$ is the biquaternionic stress-energy tensor (Section~\ref{sec:canonical:biquaternion_stress_energy})
    \item $\kappa = 8\pi G/c^4$ (in SI units) or $\kappa = 8\pi G$ (in natural units)
\end{itemize}

Taking the real part:
\begin{equation}
\text{Re}(\mathcal{G}_{\mu\nu}) = \kappa \text{Re}(\mathcal{T}_{\mu\nu}) \quad \Rightarrow \quad G_{\mu\nu} = 8\pi G T_{\mu\nu}
\end{equation}

This recovers Einstein's field equations.

\subsection{Conflict Resolution}

This canonical definition supersedes all previous formulations that treat $g_{\mu\nu}$ as fundamental.

\textbf{Deprecated approaches:}
\begin{enumerate}
    \item ❌ Direct postulation of real metric $g_{\mu\nu}$ as fundamental
    \item ❌ Deriving $g_{\mu\nu}$ from $\Theta$ without biquaternionic structure
    \item ❌ Treating GR as an independent axiom rather than a limiting case
\end{enumerate}

\textbf{Canonical resolution}: Use Eq.~\ref{eq:canonical:biq_metric_fundamental} with decomposition Eq.~\ref{eq:canonical:biq_metric_decomposition} and projection rule Eq.~\ref{eq:canonical:metric_projection_rule}.

\subsection{Computational Formula}

For numerical calculations, expand the biquaternionic metric:

\begin{equation}
\label{eq:canonical:biq_metric_computational}
\mathcal{G}_{\mu\nu} = g_{\mu\nu} + i h_{\mu\nu} + j_1 k^1_{\mu\nu} + j_2 k^2_{\mu\nu} + j_3 k^3_{\mu\nu} + \text{(mixed terms)}
\end{equation}

where mixed terms involve products like $i j_k$.

\subsection{Historical Note}

This definition establishes the biquaternionic metric as the fundamental geometric object in UBT, from which all classical GR results emerge as real projections. This represents a paradigm shift from treating GR as foundational to recognizing it as a limiting case of a richer biquaternionic geometry.

% End of fundamental biquaternionic metric definition
