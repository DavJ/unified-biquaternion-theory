% Biquaternionic Curvature and Ricci Tensor
% Version: 1.0
% Date: 2026-01-07
% Status: Canonical - FUNDAMENTAL GEOMETRY

\section{Biquaternionic Curvature $\mathcal{R}_{\mu\nu}$ and Ricci Tensor}
\label{sec:canonical:biquaternion_curvature}

\subsection{Fundamental Postulate}

\textbf{The Riemann curvature tensor is fundamentally a biquaternionic object.}

The fundamental curvature in UBT is the \textbf{biquaternionic curvature tensor}:

\begin{equation}
\label{eq:canonical:biq_curvature_fundamental}
\boxed{\mathcal{R}_{\mu\nu\rho\sigma}(x) \in \mathbb{B} = \mathbb{H} \otimes \mathbb{C}}
\end{equation}

\subsection{Definition from Connection}

The biquaternionic curvature is defined as the field strength of the biquaternionic connection:

\begin{equation}
\label{eq:canonical:biq_curvature_definition}
\boxed{\mathcal{R}_{\mu\nu} = \partial_\mu \Omega_\nu - \partial_\nu \Omega_\mu + [\Omega_\mu, \Omega_\nu]_\star}
\end{equation}

\noindent where:
\begin{itemize}
    \item $\Omega_\mu$ is the biquaternionic connection (Section~\ref{sec:canonical:biquaternion_connection})
    \item $[\Omega_\mu, \Omega_\nu]_\star = \Omega_\mu \star \Omega_\nu - \Omega_\nu \star \Omega_\mu$ is the biquaternionic commutator
    \item $\star$ denotes the full biquaternionic product (non-commutative, non-associative)
\end{itemize}

\begin{tcolorbox}[colback=red!5!white,colframe=red!75!black,title=Critical: Non-Commutativity]
\textbf{Do NOT simplify commutators.}

\textbf{Do NOT assume associativity.}

The biquaternionic commutator $[\Omega_\mu, \Omega_\nu]_\star = \Omega_\mu \star \Omega_\nu - \Omega_\nu \star \Omega_\mu$ is generally non-zero even when components commute individually.

Biquaternionic multiplication is:
\begin{itemize}
    \item \textbf{Non-commutative}: $AB \neq BA$ in general
    \item \textbf{Non-associative}: $(AB)C \neq A(BC)$ in general
\end{itemize}

The full algebraic structure must be preserved. Simplifications valid only in the real limit must be labeled "GR limit".
\end{tcolorbox}

\subsection{Full Riemann Tensor}

The complete biquaternionic Riemann curvature tensor is:

\begin{equation}
\label{eq:canonical:biq_riemann}
\mathcal{R}^\rho{}_{\sigma\mu\nu} = \partial_\mu \Omega^\rho_{\nu\sigma} - \partial_\nu \Omega^\rho_{\mu\sigma} + \Omega^\rho_{\mu\lambda} \star \Omega^\lambda_{\nu\sigma} - \Omega^\rho_{\nu\lambda} \star \Omega^\lambda_{\mu\sigma}
\end{equation}

This generalizes the classical Riemann tensor to biquaternionic geometry.

\subsection{Biquaternionic Ricci Tensor}

The biquaternionic Ricci tensor is obtained by contraction with the tetrad field:

\begin{equation}
\label{eq:canonical:biq_ricci}
\boxed{\mathcal{R}_{\nu\sigma} = E^{\mu} \star \mathcal{R}_{\mu\nu} \star E_\sigma}
\end{equation}

\noindent where:
\begin{itemize}
    \item $E^\mu$ is the inverse biquaternionic tetrad (Section~\ref{sec:canonical:biquaternion_tetrad})
    \item $\star$ denotes biquaternionic multiplication
    \item The ordering matters due to non-commutativity
\end{itemize}

Alternatively, using index contraction:

\begin{equation}
\mathcal{R}_{\mu\nu} = \mathcal{R}^\lambda{}_{\mu\lambda\nu} = \mathcal{G}^{\rho\sigma} \mathcal{R}_{\rho\mu\sigma\nu}
\end{equation}

where $\mathcal{G}^{\rho\sigma}$ is the inverse biquaternionic metric.

\subsection{Classical Ricci Tensor as Projection}

The classical Ricci tensor used in GR is obtained by projection:

\begin{equation}
\label{eq:canonical:ricci_projection}
\boxed{R_{\mu\nu} := \text{Re}(\mathcal{R}_{\mu\nu})}
\end{equation}

\textbf{Only after this projection} is the classical Ricci tensor defined.

\begin{tcolorbox}[colback=red!5!white,colframe=red!75!black,title=Prohibition]
\textbf{It is FORBIDDEN to postulate $R_{\mu\nu}$ directly without deriving it from the biquaternionic Ricci tensor $\mathcal{R}_{\mu\nu}$.}
\end{tcolorbox}

\subsection{Biquaternionic Ricci Scalar}

The biquaternionic Ricci scalar (scalar curvature) is:

\begin{equation}
\label{eq:canonical:biq_ricci_scalar}
\mathcal{R} = \mathcal{G}^{\mu\nu} \mathcal{R}_{\mu\nu}
\end{equation}

The classical scalar curvature is:

\begin{equation}
R = \text{Re}(\mathcal{R}) = g^{\mu\nu} R_{\mu\nu}
\end{equation}

\subsection{Biquaternionic Einstein Tensor}

The biquaternionic Einstein tensor is defined as:

\begin{equation}
\label{eq:canonical:biq_einstein_tensor}
\boxed{\mathcal{E}_{\mu\nu} = \mathcal{R}_{\mu\nu} - \frac{1}{2} \mathcal{G}_{\mu\nu} \mathcal{R}}
\end{equation}

\noindent where:
\begin{itemize}
    \item $\mathcal{R}_{\mu\nu}$ is the biquaternionic Ricci tensor
    \item $\mathcal{R}$ is the biquaternionic Ricci scalar
    \item $\mathcal{G}_{\mu\nu}$ is the biquaternionic metric
\end{itemize}

The classical Einstein tensor is:

\begin{equation}
G_{\mu\nu} = \text{Re}(\mathcal{E}_{\mu\nu}) = R_{\mu\nu} - \frac{1}{2} g_{\mu\nu} R
\end{equation}

\subsection{Biquaternionic Decomposition}

The curvature tensor decomposes as:

\begin{equation}
\mathcal{R}_{\mu\nu\rho\sigma} = R_{\mu\nu\rho\sigma} + \mathbf{I} H_{\mu\nu\rho\sigma} + \mathbf{J} \cdot \mathbf{K}_{\mu\nu\rho\sigma}
\end{equation}

where:
\begin{itemize}
    \item $R_{\mu\nu\rho\sigma}$ is the \textbf{real curvature} (classical Riemann tensor)
    \item $H_{\mu\nu\rho\sigma}$ is the \textbf{phase curvature} (imaginary time component)
    \item $\mathbf{K}_{\mu\nu\rho\sigma}$ is the \textbf{quaternionic curvature} (torsional components)
\end{itemize}

\subsection{Properties}

\subsubsection{Bianchi Identities}

The biquaternionic curvature satisfies generalized Bianchi identities:

\textbf{First Bianchi identity:}
\begin{equation}
\mathcal{R}_{\rho\sigma\mu\nu} + \mathcal{R}_{\rho\mu\nu\sigma} + \mathcal{R}_{\rho\nu\sigma\mu} = 0
\end{equation}

\textbf{Second Bianchi identity (contracted):}
\begin{equation}
\nabla^\mu \mathcal{E}_{\mu\nu} = 0
\end{equation}

where $\nabla^\mu$ is the biquaternionic covariant derivative.

\subsubsection{Hermiticity}

\begin{equation}
\mathcal{R}_{\mu\nu\rho\sigma}^\dagger = \mathcal{R}_{\nu\mu\sigma\rho}
\end{equation}

This generalizes the classical symmetry properties.

\subsubsection{Symmetries}

In the biquaternionic case, symmetries are modified:

\begin{align}
\mathcal{R}_{\rho\sigma\mu\nu} &= -\mathcal{R}_{\sigma\rho\mu\nu}^\dagger \quad \text{(Hermitian antisymmetry)} \\
\mathcal{R}_{\rho\sigma\mu\nu} &= -\mathcal{R}_{\rho\sigma\nu\mu}^\dagger \quad \text{(Hermitian antisymmetry)} \\
\text{Re}(\mathcal{R}_{\rho\sigma\mu\nu}) &= \text{Re}(\mathcal{R}_{\mu\nu\rho\sigma}) \quad \text{(real part exchange)}
\end{align}

\subsection{Physical Interpretation}

\subsubsection{Real Component: $R_{\mu\nu\rho\sigma}$}

\begin{itemize}
    \item Classical spacetime curvature
    \item Observable via gravitational effects
    \item Satisfies Einstein's equations
    \item Produces:
    \begin{itemize}
        \item Gravitational lensing
        \item Perihelion precession
        \item Gravitational waves
        \item Black hole formation
    \end{itemize}
\end{itemize}

\subsubsection{Phase Curvature: $H_{\mu\nu\rho\sigma}$}

\begin{itemize}
    \item Imaginary time curvature
    \item Invisible to classical observations
    \item Couples to consciousness fields
    \item Responsible for:
    \begin{itemize}
        \item Dark energy effects (cosmological acceleration)
        \item Psychon field dynamics
        \item Quantum coherence in biological systems
        \item Phase-locked states
    \end{itemize}
\end{itemize}

\subsubsection{Quaternionic Curvature: $\mathbf{K}_{\mu\nu\rho\sigma}$}

\begin{itemize}
    \item Torsional and non-metric curvature
    \item Couples to spin and angular momentum
    \item Responsible for:
    \begin{itemize}
        \item Dark matter halos (via p-adic extensions)
        \item Modified frame-dragging
        \item Spin-orbit coupling in strong gravity
        \item Directional gravitational asymmetries
    \end{itemize}
\end{itemize}

\subsection{Field Equations}

The fundamental UBT field equation in terms of curvature is:

\begin{equation}
\label{eq:canonical:biq_field_equation}
\boxed{\mathcal{E}_{\mu\nu} = \kappa \mathcal{T}_{\mu\nu}}
\end{equation}

\noindent where:
\begin{itemize}
    \item $\mathcal{E}_{\mu\nu} = \mathcal{R}_{\mu\nu} - \frac{1}{2}\mathcal{G}_{\mu\nu}\mathcal{R}$ is the biquaternionic Einstein tensor
    \item $\mathcal{T}_{\mu\nu}$ is the biquaternionic stress-energy tensor (Section~\ref{sec:canonical:biquaternion_stress_energy})
    \item $\kappa = 8\pi G$ (in natural units)
\end{itemize}

\subsection{Einstein Equations as Real Projection}

Taking the real part of Eq.~\ref{eq:canonical:biq_field_equation}:

\begin{equation}
\text{Re}(\mathcal{E}_{\mu\nu}) = \kappa \text{Re}(\mathcal{T}_{\mu\nu})
\end{equation}

This yields:

\begin{equation}
\label{eq:canonical:einstein_from_biq}
\boxed{G_{\mu\nu} = R_{\mu\nu} - \frac{1}{2} g_{\mu\nu} R = 8\pi G T_{\mu\nu}}
\end{equation}

\textbf{Einstein's field equations are recovered exactly in the real limit.}

\begin{tcolorbox}[colback=blue!5!white,colframe=blue!75!black,title=General Relativity Recovery]
The fundamental field equation of UBT is:
\begin{equation}
\mathcal{E}_{\mu\nu} = \kappa \mathcal{T}_{\mu\nu} \quad \text{(biquaternionic)}
\end{equation}

In the real limit ($\text{Im}(\mathcal{G}_{\mu\nu}) \to 0$, $\text{Im}(\mathcal{T}_{\mu\nu}) \to 0$):
\begin{equation}
G_{\mu\nu} = 8\pi G T_{\mu\nu} \quad \text{(Einstein's equations)}
\end{equation}

\textbf{UBT generalizes GR. It does not contradict it.}
\end{tcolorbox}

\subsection{Exotic Regimes}

When imaginary components are non-zero:

\begin{equation}
\text{Im}(\mathcal{R}_{\mu\nu}) \neq 0
\end{equation}

we obtain \textbf{exotic gravitational behavior}:

\begin{enumerate}
    \item \textbf{Pseudo-antigravity}: Phase curvature produces apparent repulsion
    \begin{equation}
    H_{\mu\nu} < 0 \quad \Rightarrow \quad \text{effective repulsive gravity}
    \end{equation}
    
    \item \textbf{Dark energy}: Imaginary vacuum energy drives cosmological acceleration
    \begin{equation}
    \text{Im}(\mathcal{T}_{00}) \sim \rho_{\text{dark}}
    \end{equation}
    
    \item \textbf{Phase invisibility}: Matter coupling only to $g_{\mu\nu}$ cannot detect $H_{\mu\nu}$
    
    \item \textbf{Local time drift}: Imaginary time curvature produces temporal flow variations
    \begin{equation}
    \frac{d\psi}{dt} \propto H_{00}
    \end{equation}
\end{enumerate}

See Section~\ref{sec:canonical:exotic_regimes} for detailed analysis.

\subsection{Weyl Tensor}

The biquaternionic Weyl tensor (conformal curvature) is:

\begin{equation}
\mathcal{C}_{\rho\sigma\mu\nu} = \mathcal{R}_{\rho\sigma\mu\nu} - \text{(trace terms)}
\end{equation}

This represents the trace-free part of the curvature, encoding tidal forces and gravitational waves.

\subsection{Kretschmann Scalar}

The biquaternionic Kretschmann scalar is:

\begin{equation}
\mathcal{K} = \mathcal{R}^{\rho\sigma\mu\nu} \mathcal{R}_{\rho\sigma\mu\nu}
\end{equation}

The classical Kretschmann scalar is:

\begin{equation}
K = \text{Re}(\mathcal{K}) = R^{\rho\sigma\mu\nu} R_{\rho\sigma\mu\nu}
\end{equation}

\subsection{Geodesic Deviation}

The biquaternionic geodesic deviation equation is:

\begin{equation}
\frac{D^2 \xi^\mu}{D\lambda^2} = \mathcal{R}^\mu{}_{\nu\rho\sigma} u^\nu u^\rho \xi^\sigma
\end{equation}

where:
\begin{itemize}
    \item $\xi^\mu$ is the deviation vector (biquaternionic)
    \item $u^\mu$ is the 4-velocity
    \item $D/D\lambda$ is the biquaternionic covariant derivative along the curve
\end{itemize}

Imaginary components produce phase-dependent tidal forces.

\subsection{Explicit Computation}

For practical calculations:

\textbf{Step 1:} Compute the biquaternionic connection $\Omega_\mu$ from the metric.

\textbf{Step 2:} Calculate the curvature tensor:
\begin{equation}
\mathcal{R}^\rho{}_{\sigma\mu\nu} = \partial_\mu \Omega^\rho_{\nu\sigma} - \partial_\nu \Omega^\rho_{\mu\sigma} + \Omega^\rho_{\mu\lambda} \Omega^\lambda_{\nu\sigma} - \Omega^\rho_{\nu\lambda} \Omega^\lambda_{\mu\sigma}
\end{equation}

\textbf{Step 3:} Contract to get Ricci tensor:
\begin{equation}
\mathcal{R}_{\mu\nu} = \mathcal{R}^\lambda{}_{\mu\lambda\nu}
\end{equation}

\textbf{Step 4:} Compute Ricci scalar:
\begin{equation}
\mathcal{R} = \mathcal{G}^{\mu\nu} \mathcal{R}_{\mu\nu}
\end{equation}

\textbf{Step 5:} Form Einstein tensor:
\begin{equation}
\mathcal{E}_{\mu\nu} = \mathcal{R}_{\mu\nu} - \frac{1}{2} \mathcal{G}_{\mu\nu} \mathcal{R}
\end{equation}

\textbf{Step 6:} Extract classical quantities:
\begin{equation}
R_{\mu\nu} = \text{Re}(\mathcal{R}_{\mu\nu}), \quad G_{\mu\nu} = \text{Re}(\mathcal{E}_{\mu\nu})
\end{equation}

\subsection{Consistency with General Relativity}

When all imaginary components vanish:

\begin{equation}
H_{\mu\nu\rho\sigma} \to 0, \quad \mathbf{K}_{\mu\nu\rho\sigma} \to 0
\end{equation}

The biquaternionic curvature reduces to the classical Riemann tensor:

\begin{equation}
\mathcal{R}_{\mu\nu\rho\sigma} \to R_{\mu\nu\rho\sigma}
\end{equation}

All GR predictions are exactly recovered:
\begin{itemize}
    \item Schwarzschild metric: $R_{\mu\nu\rho\sigma}$ identical to GR
    \item Gravitational waves: $h_{\mu\nu}$ perturbations match GR
    \item Cosmology: FLRW solutions identical to GR
    \item Black holes: Event horizons at same locations as GR
\end{itemize}

\textbf{UBT is a consistent extension of General Relativity.}

\subsection{Conflict Resolution}

This canonical definition supersedes:

\begin{enumerate}
    \item ❌ Direct postulation of Riemann tensor from Christoffel symbols
    \item ❌ Treating $R_{\mu\nu}$ as fundamental rather than derived
    \item ❌ Einstein equations as axioms rather than projections
\end{enumerate}

\textbf{Canonical resolution}: The biquaternionic curvature $\mathcal{R}_{\mu\nu\rho\sigma} \in \mathbb{B}$ is fundamental. Classical curvature $R_{\mu\nu\rho\sigma} = \text{Re}(\mathcal{R}_{\mu\nu\rho\sigma})$ is its real projection.

\subsection{Historical Note}

The biquaternionic curvature formalism provides the deepest geometric description in UBT, unifying classical curvature with phase curvature and torsional effects in a single biquaternionic object.

% End of biquaternionic curvature definition
