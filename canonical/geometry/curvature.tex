% Classical Curvature Tensor Definitions (Derived Quantities)
% Version: 2.0
% Date: 2026-01-07
% Status: Canonical - DERIVED FROM BIQUATERNIONIC GEOMETRY

\section{Classical Curvature Tensors and GR Equivalence (Derived Quantities)}
\label{sec:canonical:curvature}

\begin{tcolorbox}[colback=yellow!10!white,colframe=orange!75!black,title=Important: These are Derived Quantities]
\textbf{The Riemann tensor $R_{\mu\nu\rho\sigma}$, Ricci tensor $R_{\mu\nu}$, and Christoffel symbols $\Gamma^\lambda_{\mu\nu}$ are NOT fundamental in UBT.}

They are real projections of fundamental biquaternionic objects:
\begin{itemize}
    \item $\Omega_\mu \in \mathbb{B}$ (biquaternionic connection) → $\Gamma^\lambda_{\mu\nu} = \text{Re}(\Omega^\lambda_{\mu\nu})$
    \item $\mathcal{R}_{\mu\nu\rho\sigma} \in \mathbb{B}$ (biquaternionic curvature) → $R_{\mu\nu\rho\sigma} = \text{Re}(\mathcal{R}_{\mu\nu\rho\sigma})$
    \item $\mathcal{R}_{\mu\nu} \in \mathbb{B}$ (biquaternionic Ricci) → $R_{\mu\nu} = \text{Re}(\mathcal{R}_{\mu\nu})$
\end{itemize}

For the fundamental geometric description, see:
\begin{itemize}
    \item Section~\ref{sec:canonical:biquaternion_connection}: Biquaternionic connection $\Omega_\mu$
    \item Section~\ref{sec:canonical:biquaternion_curvature}: Biquaternionic curvature $\mathcal{R}_{\mu\nu\rho\sigma}$
\end{itemize}
\end{tcolorbox}

\subsection{Metric and Connection}

\subsubsection{Canonical Metric}

From the biquaternion field $\Theta(q,T_B)$, the spacetime metric emerges as:

\begin{equation}
\label{eq:canonical:metric_curvature}
g_{\mu\nu}(x) = \text{Re}[(\partial_\mu\Theta)(\partial_\nu\Theta^\dagger)]
\end{equation}

where $\partial_\mu$ denotes biquaternion-valued partial derivatives.

\textbf{Signature}: $(+,-,-,-)$ (mostly minus convention)

\subsubsection{Christoffel Symbols (Derived)}

The Levi-Civita connection (Christoffel symbols) is computed from the classical metric as the real projection of the biquaternionic connection:

\begin{equation}
\label{eq:canonical:christoffel_derived}
\Gamma^\lambda_{\mu\nu} = \text{Re}(\Omega^\lambda_{\mu\nu}) = \frac{1}{2} g^{\lambda\rho} \left( \partial_\mu g_{\nu\rho} + \partial_\nu g_{\rho\mu} - \partial_\rho g_{\mu\nu} \right)
\end{equation}

\textbf{Important:} The Christoffel symbols are NOT fundamental. They are derived from:
\begin{enumerate}
    \item The biquaternionic connection $\Omega_\mu$ (fundamental) via real projection, OR
    \item The classical metric $g_{\mu\nu}$ (itself derived from $\mathcal{G}_{\mu\nu} = \text{Re}(\mathcal{G}_{\mu\nu})$)
\end{enumerate}

See Section~\ref{sec:canonical:biquaternion_connection} for the fundamental connection formalism.

Properties:
\begin{itemize}
    \item \textbf{Symmetric}: $\Gamma^\lambda_{\mu\nu} = \Gamma^\lambda_{\nu\mu}$
    \item \textbf{Metric-compatible}: $\nabla_\lambda g_{\mu\nu} = 0$
    \item \textbf{Torsion-free}: $\Gamma^\lambda_{\mu\nu} - \Gamma^\lambda_{\nu\mu} = 0$
\end{itemize}

\subsection{Riemann Curvature Tensor (Derived)}

\subsubsection{Definition as Real Projection}

The classical Riemann curvature tensor is the real projection of the biquaternionic curvature:

\begin{equation}
\label{eq:canonical:riemann_projection}
R^\rho{}_{\sigma\mu\nu} = \text{Re}(\mathcal{R}^\rho{}_{\sigma\mu\nu})
\end{equation}

where $\mathcal{R}^\rho{}_{\sigma\mu\nu} \in \mathbb{B}$ is the biquaternionic Riemann tensor (Section~\ref{sec:canonical:biquaternion_curvature}).

Alternatively, computed from Christoffel symbols (which are themselves derived):

\begin{equation}
\label{eq:canonical:riemann_from_christoffel}
R^\rho{}_{\sigma\mu\nu} = \partial_\mu \Gamma^\rho_{\nu\sigma} - \partial_\nu \Gamma^\rho_{\mu\sigma} + \Gamma^\rho_{\mu\lambda}\Gamma^\lambda_{\nu\sigma} - \Gamma^\rho_{\nu\lambda}\Gamma^\lambda_{\mu\sigma}
\end{equation}

\textbf{Note:} This classical formula is valid because the real projection commutes with derivatives and products.

Alternatively, via the commutator of covariant derivatives:
\begin{equation}
[\nabla_\mu, \nabla_\nu] V^\rho = R^\rho{}_{\sigma\mu\nu} V^\sigma
\end{equation}

\subsubsection{Symmetries}

The Riemann tensor has the following symmetries:

\begin{align}
R_{\rho\sigma\mu\nu} &= -R_{\sigma\rho\mu\nu} \quad \text{(antisymmetric in first pair)} \\
R_{\rho\sigma\mu\nu} &= -R_{\rho\sigma\nu\mu} \quad \text{(antisymmetric in second pair)} \\
R_{\rho\sigma\mu\nu} &= R_{\mu\nu\rho\sigma} \quad \text{(pair exchange symmetry)} \\
R_{\rho\sigma\mu\nu} + R_{\rho\mu\nu\sigma} + R_{\rho\nu\sigma\mu} &= 0 \quad \text{(first Bianchi identity)}
\end{align}

where $R_{\rho\sigma\mu\nu} = g_{\rho\lambda} R^\lambda{}_{\sigma\mu\nu}$.

\subsubsection{Independent Components}

In 4D spacetime, the Riemann tensor has:
\begin{itemize}
    \item Total components: $4^4 = 256$
    \item Independent components: $\frac{4^2(4^2-1)}{12} = 20$
\end{itemize}

\subsection{Ricci Tensor and Scalar (Derived)}

\subsubsection{Ricci Tensor as Projection}

The classical Ricci tensor is the real projection of the biquaternionic Ricci tensor:

\begin{equation}
\label{eq:canonical:ricci_tensor_projection}
R_{\mu\nu} = \text{Re}(\mathcal{R}_{\mu\nu})
\end{equation}

where $\mathcal{R}_{\mu\nu} \in \mathbb{B}$ is defined in Section~\ref{sec:canonical:biquaternion_curvature}.

Alternatively, via contraction of the classical Riemann tensor:

\begin{equation}
\label{eq:canonical:ricci_tensor_contraction}
R_{\mu\nu} = R^\lambda{}_{\mu\lambda\nu} = g^{\rho\sigma} R_{\rho\mu\sigma\nu}
\end{equation}

Properties:
\begin{itemize}
    \item \textbf{Symmetric}: $R_{\mu\nu} = R_{\nu\mu}$
    \item \textbf{Independent components}: 10 in 4D
\end{itemize}

\subsubsection{Ricci Scalar as Projection}

The classical Ricci scalar (scalar curvature) is the real projection of the biquaternionic Ricci scalar:

\begin{equation}
\label{eq:canonical:ricci_scalar_projection}
R = \text{Re}(\mathcal{R})
\end{equation}

where $\mathcal{R} = \mathcal{G}^{\mu\nu} \mathcal{R}_{\mu\nu}$ is the biquaternionic Ricci scalar.

Alternatively, via trace of classical Ricci tensor:

\begin{equation}
\label{eq:canonical:ricci_scalar_trace}
R = g^{\mu\nu} R_{\mu\nu} = R^\mu{}_\mu
\end{equation}

This is a scalar invariant measuring the average curvature of spacetime.

\subsection{Einstein Tensor (Derived)}

The classical Einstein tensor is the real projection of the biquaternionic Einstein tensor:

\begin{equation}
\label{eq:canonical:einstein_tensor_projection}
G_{\mu\nu} = \text{Re}(\mathcal{E}_{\mu\nu})
\end{equation}

where $\mathcal{E}_{\mu\nu} = \mathcal{R}_{\mu\nu} - \frac{1}{2}\mathcal{G}_{\mu\nu}\mathcal{R}$ is the biquaternionic Einstein tensor
and $\mathcal{G}_{\mu\nu}$ is the biquaternionic \textbf{metric} (distinct symbols).

Alternatively, constructed from classical curvature:

\begin{equation}
\label{eq:canonical:einstein_tensor_classical}
G_{\mu\nu} = R_{\mu\nu} - \frac{1}{2} g_{\mu\nu} R
\end{equation}

\subsubsection{Properties}

\begin{enumerate}
    \item \textbf{Symmetric}: $G_{\mu\nu} = G_{\nu\mu}$
    
    \item \textbf{Divergence-free} (contracted Bianchi identity):
    \begin{equation}
    \nabla^\mu G_{\mu\nu} = 0
    \end{equation}
    
    \item \textbf{Traceless in vacuum}: When $T_{\mu\nu} = 0$, we have $G = g^{\mu\nu}G_{\mu\nu} = -R$
\end{enumerate}

\subsection{Einstein Field Equations (Real Projection)}

\subsubsection{Standard Form as Real Limit}

The Einstein field equations are the real projection of the fundamental biquaternionic field equation:

\begin{equation}
\label{eq:canonical:einstein_eqn_derived}
\boxed{
G_{\mu\nu} = R_{\mu\nu} - \frac{1}{2} g_{\mu\nu} R = 8\pi G \, T_{\mu\nu}
}
\end{equation}

This is obtained by taking $\text{Re}(\cdot)$ of the fundamental equation:

\begin{equation}
\mathcal{E}_{\mu\nu} = \kappa \mathcal{T}_{\mu\nu} \quad \Rightarrow \quad \text{Re}(\mathcal{E}_{\mu\nu}) = \kappa \text{Re}(\mathcal{T}_{\mu\nu})
\end{equation}

where:
\begin{itemize}
    \item $G$ = Newton's gravitational constant
    \item $T_{\mu\nu} = \text{Re}(\mathcal{T}_{\mu\nu})$ = classical stress-energy tensor (real projection of biquaternionic stress-energy)
    \item $c = 1$ (natural units)
\end{itemize}

\begin{tcolorbox}[colback=blue!5!white,colframe=blue!75!black,title=Einstein's Equations as Real Projection]
\textbf{The fundamental UBT field equation is:}
\begin{equation}
\mathcal{E}_{\mu\nu} = \kappa \mathcal{T}_{\mu\nu} \quad \text{(biquaternionic)}
\end{equation}

\textbf{Einstein's field equations emerge as the real projection:}
\begin{equation}
G_{\mu\nu} = 8\pi G T_{\mu\nu} \quad \text{(classical GR)}
\end{equation}

\textbf{This establishes that:}
\begin{itemize}
    \item GR is NOT an axiom—it is a limiting case
    \item All GR tests are automatically satisfied (they probe only the real sector)
    \item Imaginary components $\text{Im}(\mathcal{G}_{\mu\nu}) \neq 0$ produce dark sector effects
    \item UBT generalizes GR without contradicting it
\end{itemize}
\end{tcolorbox}

\subsubsection{Alternative Form with Cosmological Constant}

Including the cosmological constant $\Lambda$:

\begin{equation}
R_{\mu\nu} - \frac{1}{2} g_{\mu\nu} R + \Lambda g_{\mu\nu} = 8\pi G \, T_{\mu\nu}
\end{equation}

\subsection{UBT Field Equation Connection}

\subsubsection{T-shirt Formula}

The fundamental UBT equation is:

\begin{equation}
\label{eq:canonical:tshirt_curvature}
\nabla^\dagger \nabla \Theta(q,T_B) = \kappa \mathcal{T}(q,T_B)
\end{equation}

where:
\begin{itemize}
    \item $\nabla = \partial + \Gamma^{\text{grav}} + A^{\text{SM}}$ (full covariant derivative)
    \item $\kappa \propto 8\pi G$ (gravitational-gauge coupling)
    \item $\mathcal{T}$ = biquaternion stress-energy source
\end{itemize}

\subsubsection{Derivation of Einstein Equations}

The Einstein equations emerge from the T-shirt formula in the following limit:

\textbf{Step 1}: Take $T_B \to t$ (real time limit, $\psi, \chi, \xi \to 0$)

\textbf{Step 2}: Extract real part of the field equation

\textbf{Step 3}: Identify metric $g_{\mu\nu} = \text{Re}[(\partial_\mu\Theta)(\partial_\nu\Theta^\dagger)]$

\textbf{Step 4}: The field equation becomes:
\begin{equation}
\nabla^\mu \nabla_\mu \Theta = \text{(curvature terms)} + \text{(source terms)}
\end{equation}

\textbf{Step 5}: Projecting onto metric components yields:
\begin{equation}
G_{\mu\nu} = 8\pi G \, T_{\mu\nu}
\end{equation}

\textbf{Result}: UBT generalizes GR. Einstein's equations are recovered exactly in the classical limit.

\subsection{Weyl Tensor}

The Weyl curvature tensor (conformal curvature) represents the trace-free part of the Riemann tensor:

\begin{equation}
\label{eq:canonical:weyl}
C_{\rho\sigma\mu\nu} = R_{\rho\sigma\mu\nu} - \frac{1}{2}(g_{\rho\mu}R_{\sigma\nu} - g_{\rho\nu}R_{\sigma\mu} + g_{\sigma\nu}R_{\rho\mu} - g_{\sigma\mu}R_{\rho\nu}) + \frac{1}{6}R(g_{\rho\mu}g_{\sigma\nu} - g_{\rho\nu}g_{\sigma\mu})
\end{equation}

Properties:
\begin{itemize}
    \item Vanishes in 3D
    \item Represents tidal forces (gravity waves)
    \item Conformally invariant: $C'_{\rho\sigma\mu\nu} = \Omega^{-2} C_{\rho\sigma\mu\nu}$ under $g_{\mu\nu} \to \Omega^2 g_{\mu\nu}$
    \item 10 independent components in 4D
\end{itemize}

\subsection{Geodesic Equation}

Particles follow geodesics in curved spacetime:

\begin{equation}
\label{eq:canonical:geodesic}
\frac{d^2 x^\mu}{d\lambda^2} + \Gamma^\mu_{\rho\sigma} \frac{dx^\rho}{d\lambda} \frac{dx^\sigma}{d\lambda} = 0
\end{equation}

where $\lambda$ is an affine parameter.

Equivalently:
\begin{equation}
\nabla_u u^\mu = u^\nu \nabla_\nu u^\mu = 0
\end{equation}
where $u^\mu = dx^\mu/d\lambda$ is the 4-velocity.

\subsection{Parallel Transport}

A vector $V^\mu$ is parallel-transported along a curve $x^\mu(\lambda)$ if:

\begin{equation}
\frac{DV^\mu}{D\lambda} = \frac{dV^\mu}{d\lambda} + \Gamma^\mu_{\rho\sigma} \frac{dx^\rho}{d\lambda} V^\sigma = 0
\end{equation}

Curvature manifests as the failure of parallel transport around closed loops.

\subsection{Torsion (UBT Extension)}

In standard GR, torsion vanishes. In UBT, imaginary components of $T_B$ can generate torsion:

\begin{equation}
T^\lambda_{\mu\nu} = \Gamma^\lambda_{\mu\nu} - \Gamma^\lambda_{\nu\mu}
\end{equation}

\textbf{UBT prediction}: Non-zero torsion when $\|\mathbf{v}\| = \sqrt{\chi^2 + \xi^2} \neq 0$ (vector imaginary time components active).

Physical effects:
\begin{itemize}
    \item Spin-orbit coupling in strong gravity
    \item Modifications to frame-dragging
    \item Directional dark matter signatures
\end{itemize}

\subsection{Kretschmann Scalar}

The Kretschmann scalar is a curvature invariant:

\begin{equation}
\label{eq:canonical:kretschmann}
K = R^{\rho\sigma\mu\nu} R_{\rho\sigma\mu\nu}
\end{equation}

Properties:
\begin{itemize}
    \item Scalar invariant under all coordinate transformations
    \item Diverges at true singularities (e.g., $r=0$ in Schwarzschild)
    \item Used to identify physical vs. coordinate singularities
\end{itemize}

\subsection{Summary: GR Equivalence in Real Limit}

\begin{tcolorbox}[title=UBT Recovers General Relativity]
In the limit where all imaginary components vanish ($\text{Im}(\mathcal{G}_{\mu\nu}) \to 0$, $\text{Im}(\mathcal{T}_{\mu\nu}) \to 0$):

\begin{enumerate}
    \item Biquaternionic metric $\mathcal{G}_{\mu\nu} \to g_{\mu\nu}$ (classical metric)
    \item Biquaternionic connection $\Omega_\mu \to \omega_\mu$ (Christoffel symbols)
    \item Biquaternionic curvature $\mathcal{R}^\rho{}_{\sigma\mu\nu} \to R^\rho{}_{\sigma\mu\nu}$ (Riemann tensor)
    \item Einstein tensor $\mathcal{E}_{\mu\nu} \to G_{\mu\nu}$ (classical Einstein tensor)
    \item Field equation $\mathcal{E}_{\mu\nu} = \kappa \mathcal{T}_{\mu\nu} \to G_{\mu\nu} = 8\pi G T_{\mu\nu}$ (Einstein's equations)
\end{enumerate}

\textbf{All GR predictions are exactly recovered:}
\begin{itemize}
    \item Schwarzschild solution (black holes)
    \item Kerr solution (rotating black holes)
    \item FLRW cosmology
    \item Gravitational waves
    \item Perihelion precession of Mercury
    \item Gravitational lensing
    \item Frame dragging (Lense-Thirring effect)
\end{itemize}

\textbf{UBT does not contradict GR. It generalizes it by adding biquaternionic structure that becomes invisible in the real limit.}
\end{tcolorbox}

\textbf{This establishes UBT as a consistent extension of General Relativity, not a replacement or alternative theory.}
