% Canonical Curvature Tensor Definitions
% Version: 1.0
% Date: 2025-11-14
% Status: Canonical - Geometric Foundation

\section{Curvature Tensors and GR Equivalence}
\label{sec:canonical:curvature}

\subsection{Metric and Connection}

\subsubsection{Canonical Metric}

From the biquaternion field $\Theta(q,T_B)$, the spacetime metric emerges as:

\begin{equation}
\label{eq:canonical:metric_curvature}
g_{\mu\nu}(x) = \text{Re}[(\partial_\mu\Theta)(\partial_\nu\Theta^\dagger)]
\end{equation}

where $\partial_\mu$ denotes biquaternion-valued partial derivatives.

\textbf{Signature}: $(+,-,-,-)$ (mostly minus convention)

\subsubsection{Christoffel Symbols}

The Levi-Civita connection (Christoffel symbols) is computed from the metric:

\begin{equation}
\label{eq:canonical:christoffel}
\Gamma^\lambda_{\mu\nu} = \frac{1}{2} g^{\lambda\rho} \left( \partial_\mu g_{\nu\rho} + \partial_\nu g_{\rho\mu} - \partial_\rho g_{\mu\nu} \right)
\end{equation}

Properties:
\begin{itemize}
    \item \textbf{Symmetric}: $\Gamma^\lambda_{\mu\nu} = \Gamma^\lambda_{\nu\mu}$
    \item \textbf{Metric-compatible}: $\nabla_\lambda g_{\mu\nu} = 0$
    \item \textbf{Torsion-free}: $\Gamma^\lambda_{\mu\nu} - \Gamma^\lambda_{\nu\mu} = 0$
\end{itemize}

\subsection{Riemann Curvature Tensor}

\subsubsection{Definition}

The Riemann curvature tensor measures the failure of parallel transport to be path-independent:

\begin{equation}
\label{eq:canonical:riemann}
R^\rho{}_{\sigma\mu\nu} = \partial_\mu \Gamma^\rho_{\nu\sigma} - \partial_\nu \Gamma^\rho_{\mu\sigma} + \Gamma^\rho_{\mu\lambda}\Gamma^\lambda_{\nu\sigma} - \Gamma^\rho_{\nu\lambda}\Gamma^\lambda_{\mu\sigma}
\end{equation}

Alternatively, via the commutator of covariant derivatives:
\begin{equation}
[\nabla_\mu, \nabla_\nu] V^\rho = R^\rho{}_{\sigma\mu\nu} V^\sigma
\end{equation}

\subsubsection{Symmetries}

The Riemann tensor has the following symmetries:

\begin{align}
R_{\rho\sigma\mu\nu} &= -R_{\sigma\rho\mu\nu} \quad \text{(antisymmetric in first pair)} \\
R_{\rho\sigma\mu\nu} &= -R_{\rho\sigma\nu\mu} \quad \text{(antisymmetric in second pair)} \\
R_{\rho\sigma\mu\nu} &= R_{\mu\nu\rho\sigma} \quad \text{(pair exchange symmetry)} \\
R_{\rho\sigma\mu\nu} + R_{\rho\mu\nu\sigma} + R_{\rho\nu\sigma\mu} &= 0 \quad \text{(first Bianchi identity)}
\end{align}

where $R_{\rho\sigma\mu\nu} = g_{\rho\lambda} R^\lambda{}_{\sigma\mu\nu}$.

\subsubsection{Independent Components}

In 4D spacetime, the Riemann tensor has:
\begin{itemize}
    \item Total components: $4^4 = 256$
    \item Independent components: $\frac{4^2(4^2-1)}{12} = 20$
\end{itemize}

\subsection{Ricci Tensor and Scalar}

\subsubsection{Ricci Tensor}

The Ricci tensor is the contraction of the Riemann tensor:

\begin{equation}
\label{eq:canonical:ricci_tensor}
R_{\mu\nu} = R^\lambda{}_{\mu\lambda\nu} = g^{\rho\sigma} R_{\rho\mu\sigma\nu}
\end{equation}

Properties:
\begin{itemize}
    \item \textbf{Symmetric}: $R_{\mu\nu} = R_{\nu\mu}$
    \item \textbf{Independent components}: 10 in 4D
\end{itemize}

\subsubsection{Ricci Scalar}

The Ricci scalar (scalar curvature) is the trace of the Ricci tensor:

\begin{equation}
\label{eq:canonical:ricci_scalar}
R = g^{\mu\nu} R_{\mu\nu} = R^\mu{}_\mu
\end{equation}

This is a scalar invariant measuring the average curvature of spacetime.

\subsection{Einstein Tensor}

The Einstein tensor is defined as:

\begin{equation}
\label{eq:canonical:einstein_tensor}
G_{\mu\nu} = R_{\mu\nu} - \frac{1}{2} g_{\mu\nu} R
\end{equation}

\subsubsection{Properties}

\begin{enumerate}
    \item \textbf{Symmetric}: $G_{\mu\nu} = G_{\nu\mu}$
    
    \item \textbf{Divergence-free} (contracted Bianchi identity):
    \begin{equation}
    \nabla^\mu G_{\mu\nu} = 0
    \end{equation}
    
    \item \textbf{Traceless in vacuum}: When $T_{\mu\nu} = 0$, we have $G = g^{\mu\nu}G_{\mu\nu} = -R$
\end{enumerate}

\subsection{Einstein Field Equations}

\subsubsection{Standard Form}

The Einstein field equations relate geometry to matter:

\begin{equation}
\label{eq:canonical:einstein_eqn}
\boxed{
G_{\mu\nu} = R_{\mu\nu} - \frac{1}{2} g_{\mu\nu} R = 8\pi G \, T_{\mu\nu}
}
\end{equation}

where:
\begin{itemize}
    \item $G$ = Newton's gravitational constant
    \item $T_{\mu\nu}$ = stress-energy tensor (from canonical definition)
    \item $c = 1$ (natural units)
\end{itemize}

\subsubsection{Alternative Form with Cosmological Constant}

Including the cosmological constant $\Lambda$:

\begin{equation}
R_{\mu\nu} - \frac{1}{2} g_{\mu\nu} R + \Lambda g_{\mu\nu} = 8\pi G \, T_{\mu\nu}
\end{equation}

\subsection{UBT Field Equation Connection}

\subsubsection{T-shirt Formula}

The fundamental UBT equation is:

\begin{equation}
\label{eq:canonical:tshirt_curvature}
\nabla^\dagger \nabla \Theta(q,T_B) = \kappa \mathcal{T}(q,T_B)
\end{equation}

where:
\begin{itemize}
    \item $\nabla = \partial + \Gamma^{\text{grav}} + A^{\text{SM}}$ (full covariant derivative)
    \item $\kappa \propto 8\pi G$ (gravitational-gauge coupling)
    \item $\mathcal{T}$ = biquaternion stress-energy source
\end{itemize}

\subsubsection{Derivation of Einstein Equations}

The Einstein equations emerge from the T-shirt formula in the following limit:

\textbf{Step 1}: Take $T_B \to t$ (real time limit, $\psi, \chi, \xi \to 0$)

\textbf{Step 2}: Extract real part of the field equation

\textbf{Step 3}: Identify metric $g_{\mu\nu} = \text{Re}[(\partial_\mu\Theta)(\partial_\nu\Theta^\dagger)]$

\textbf{Step 4}: The field equation becomes:
\begin{equation}
\nabla^\mu \nabla_\mu \Theta = \text{(curvature terms)} + \text{(source terms)}
\end{equation}

\textbf{Step 5}: Projecting onto metric components yields:
\begin{equation}
G_{\mu\nu} = 8\pi G \, T_{\mu\nu}
\end{equation}

\textbf{Result}: UBT generalizes GR. Einstein's equations are recovered exactly in the classical limit.

\subsection{Weyl Tensor}

The Weyl curvature tensor (conformal curvature) represents the trace-free part of the Riemann tensor:

\begin{equation}
\label{eq:canonical:weyl}
C_{\rho\sigma\mu\nu} = R_{\rho\sigma\mu\nu} - \frac{1}{2}(g_{\rho\mu}R_{\sigma\nu} - g_{\rho\nu}R_{\sigma\mu} + g_{\sigma\nu}R_{\rho\mu} - g_{\sigma\mu}R_{\rho\nu}) + \frac{1}{6}R(g_{\rho\mu}g_{\sigma\nu} - g_{\rho\nu}g_{\sigma\mu})
\end{equation}

Properties:
\begin{itemize}
    \item Vanishes in 3D
    \item Represents tidal forces (gravity waves)
    \item Conformally invariant: $C'_{\rho\sigma\mu\nu} = \Omega^{-2} C_{\rho\sigma\mu\nu}$ under $g_{\mu\nu} \to \Omega^2 g_{\mu\nu}$
    \item 10 independent components in 4D
\end{itemize}

\subsection{Geodesic Equation}

Particles follow geodesics in curved spacetime:

\begin{equation}
\label{eq:canonical:geodesic}
\frac{d^2 x^\mu}{d\lambda^2} + \Gamma^\mu_{\rho\sigma} \frac{dx^\rho}{d\lambda} \frac{dx^\sigma}{d\lambda} = 0
\end{equation}

where $\lambda$ is an affine parameter.

Equivalently:
\begin{equation}
\nabla_u u^\mu = u^\nu \nabla_\nu u^\mu = 0
\end{equation}
where $u^\mu = dx^\mu/d\lambda$ is the 4-velocity.

\subsection{Parallel Transport}

A vector $V^\mu$ is parallel-transported along a curve $x^\mu(\lambda)$ if:

\begin{equation}
\frac{DV^\mu}{D\lambda} = \frac{dV^\mu}{d\lambda} + \Gamma^\mu_{\rho\sigma} \frac{dx^\rho}{d\lambda} V^\sigma = 0
\end{equation}

Curvature manifests as the failure of parallel transport around closed loops.

\subsection{Torsion (UBT Extension)}

In standard GR, torsion vanishes. In UBT, imaginary components of $T_B$ can generate torsion:

\begin{equation}
T^\lambda_{\mu\nu} = \Gamma^\lambda_{\mu\nu} - \Gamma^\lambda_{\nu\mu}
\end{equation}

\textbf{UBT prediction}: Non-zero torsion when $\|\mathbf{v}\| = \sqrt{\chi^2 + \xi^2} \neq 0$ (vector imaginary time components active).

Physical effects:
\begin{itemize}
    \item Spin-orbit coupling in strong gravity
    \item Modifications to frame-dragging
    \item Directional dark matter signatures
\end{itemize}

\subsection{Kretschmann Scalar}

The Kretschmann scalar is a curvature invariant:

\begin{equation}
\label{eq:canonical:kretschmann}
K = R^{\rho\sigma\mu\nu} R_{\rho\sigma\mu\nu}
\end{equation}

Properties:
\begin{itemize}
    \item Scalar invariant under all coordinate transformations
    \item Diverges at true singularities (e.g., $r=0$ in Schwarzschild)
    \item Used to identify physical vs. coordinate singularities
\end{itemize}

\subsection{Summary: GR Equivalence}

\begin{tcolorbox}[title=UBT Recovers General Relativity]
In the limit $T_B \to t$ (biquaternion time → real time):

\begin{enumerate}
    \item Metric $g_{\mu\nu}$ is identical to GR metric
    \item Christoffel symbols $\Gamma^\lambda_{\mu\nu}$ match GR
    \item Riemann tensor $R^\rho{}_{\sigma\mu\nu}$ is standard GR
    \item Einstein equations $G_{\mu\nu} = 8\pi G T_{\mu\nu}$ recovered exactly
    \item All GR predictions (perihelion precession, light bending, gravitational waves) preserved
\end{enumerate}

\textbf{UBT does not contradict GR. It generalizes it by adding imaginary time structure.}
\end{tcolorbox}

\textbf{This establishes UBT as a consistent extension of General Relativity.}
