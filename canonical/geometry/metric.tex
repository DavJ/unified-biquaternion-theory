% Classical Metric Tensor as Derived Quantity
% Version: 2.0
% Date: 2026-01-07
% Status: Canonical - DERIVED FROM BIQUATERNIONIC METRIC

\section{The Classical Metric Tensor $g_{\mu\nu}$ (Derived Quantity)}
\label{sec:canonical:metric}

\begin{tcolorbox}[colback=yellow!10!white,colframe=orange!75!black,title=Important: This is a Derived Quantity]
\textbf{The metric $g_{\mu\nu}$ is NOT fundamental in UBT.}

It is the real projection of the fundamental biquaternionic metric $\mathcal{G}_{\mu\nu} \in \mathbb{B}$ (see Section~\ref{sec:canonical:biquaternion_metric}).

For the fundamental geometric description, see:
\begin{itemize}
    \item Section~\ref{sec:canonical:biquaternion_metric}: Biquaternionic metric $\mathcal{G}_{\mu\nu}$
    \item Section~\ref{sec:canonical:biquaternion_tetrad}: Biquaternionic tetrad $E_\mu$
    \item Section~\ref{sec:canonical:biquaternion_connection}: Biquaternionic connection $\Omega_\mu$
\end{itemize}
\end{tcolorbox}

\subsection{Canonical Definition (as Projection)}

The spacetime metric tensor used in General Relativity is derived from the fundamental biquaternionic metric via projection:

\begin{equation}
\label{eq:canonical:metric_projection}
\boxed{g_{\mu\nu} = \text{Re}(\mathcal{G}_{\mu\nu})}
\end{equation}

\noindent where $\mathcal{G}_{\mu\nu}$ is the biquaternionic metric (Section~\ref{sec:canonical:biquaternion_metric}).

Alternatively, from the $\Theta$ field directly:

\begin{equation}
\label{eq:canonical:metric_from_theta}
g_{\mu\nu}(\Theta) = \text{Re}\,\text{Tr}\left(\partial_\mu\Theta \, \partial_\nu\Theta^\dagger\right)
\end{equation}

\noindent where:
\begin{itemize}
    \item $\partial_\mu = \frac{\partial}{\partial x^\mu}$ is the partial derivative with respect to spacetime coordinate $x^\mu$
    \item $\Theta^\dagger$ is the Hermitian conjugate of $\Theta$ (see Eq.~\ref{eq:canonical:theta_hermitian})
    \item $\text{Tr}$ denotes the matrix trace
    \item $\text{Re}$ denotes the real part
\end{itemize}

\subsection{Index Convention}

Throughout this work, we use:

\begin{itemize}
    \item \textbf{Greek indices} $\mu, \nu, \rho, \sigma = 0, 1, 2, 3$ for spacetime coordinates
    \item \textbf{Signature}: $(+, -, -, -)$ (mostly minus, spacelike negative)
    \item \textbf{Coordinates}: $x^\mu = (x^0, x^1, x^2, x^3) = (t, x, y, z)$ or $(ct, x, y, z)$
\end{itemize}

Alternative signature $(-, +, +, +)$ (mostly plus) may be used in some contexts; the choice will be clearly stated.

\subsection{Explicit Form}

Expanding the matrix product:

\begin{equation}
\label{eq:canonical:metric_explicit}
g_{\mu\nu} = \text{Re}\sum_{i,j=0}^{3} (\partial_\mu\Theta)_{ij} \overline{(\partial_\nu\Theta)_{ij}}
\end{equation}

This is manifestly:
\begin{itemize}
    \item \textbf{Real-valued}: by construction via $\text{Re}$
    \item \textbf{Symmetric}: $g_{\mu\nu} = g_{\nu\mu}$
    \item \textbf{Dynamic}: depends on $\Theta$ field configuration
\end{itemize}

\subsection{Properties}

\subsubsection{Symmetry}

\begin{equation}
\label{eq:canonical:metric_symmetry}
g_{\mu\nu} = g_{\nu\mu}
\end{equation}

Proof: Since $\text{Tr}(AB) = \text{Tr}(BA)$ and both $\partial_\mu\Theta$ and $\partial_\nu\Theta^\dagger$ are matrices.

\subsubsection{Positive Definiteness}

For spacelike separations, $g_{ij}$ ($i,j=1,2,3$) is negative definite in signature $(+,-,-,-)$:

\begin{equation}
\label{eq:canonical:metric_spacelike}
g_{ij} v^i v^j < 0 \quad \text{for } v^i \neq 0
\end{equation}

\subsubsection{Determinant}

\begin{equation}
\label{eq:canonical:metric_determinant}
g = \det(g_{\mu\nu}) < 0
\end{equation}

for Lorentzian signature. The quantity $\sqrt{-g}$ appears in the volume element.

\subsection{Inverse Metric}

The inverse metric $g^{\mu\nu}$ satisfies:

\begin{equation}
\label{eq:canonical:inverse_metric}
g^{\mu\rho} g_{\rho\nu} = \delta^\mu_\nu
\end{equation}

Raising and lowering indices:
\begin{align}
V_\mu &= g_{\mu\nu} V^\nu \label{eq:canonical:lower_index} \\
V^\mu &= g^{\mu\nu} V_\nu \label{eq:canonical:raise_index}
\end{align}

\subsection{Connection to Curvature}

The metric determines the Christoffel symbols (affine connection), which are themselves derived from the biquaternionic connection (see Section~\ref{sec:canonical:biquaternion_connection}):

\begin{equation}
\label{eq:canonical:christoffel_from_metric}
\Gamma^\lambda_{\mu\nu} = \text{Re}(\Omega^\lambda_{\mu\nu}) = \frac{1}{2} g^{\lambda\rho} \left(\partial_\mu g_{\nu\rho} + \partial_\nu g_{\mu\rho} - \partial_\rho g_{\mu\nu}\right)
\end{equation}

\textbf{Note:} The Christoffel symbols are NOT fundamental. They are the real projection of the biquaternionic connection $\Omega_\mu$.

From which the Riemann curvature tensor follows:

\begin{equation}
\label{eq:canonical:riemann}
R^\rho_{\sigma\mu\nu} = \partial_\mu\Gamma^\rho_{\nu\sigma} - \partial_\nu\Gamma^\rho_{\mu\sigma} + \Gamma^\rho_{\mu\lambda}\Gamma^\lambda_{\nu\sigma} - \Gamma^\rho_{\nu\lambda}\Gamma^\lambda_{\mu\sigma}
\end{equation}

See Section~\ref{sec:canonical:curvature} for details.

\subsection{Special Cases}

\subsubsection{Minkowski Limit}

When $\Theta = \Theta_0$ (constant vacuum configuration):

\begin{equation}
\label{eq:canonical:minkowski_limit}
g_{\mu\nu} \to \eta_{\mu\nu} = \text{diag}(+1, -1, -1, -1)
\end{equation}

This is the flat spacetime limit.

\subsubsection{Weak Field Approximation}

For small perturbations $\Theta = \Theta_0 + h$:

\begin{equation}
\label{eq:canonical:weak_field}
g_{\mu\nu} = \eta_{\mu\nu} + h_{\mu\nu} + O(h^2)
\end{equation}

where $h_{\mu\nu}$ is the gravitational wave perturbation.

\subsection{Physical Interpretation}

The metric $g_{\mu\nu}$ encodes:

\begin{enumerate}
    \item \textbf{Spacetime geometry}: distances, angles, volumes
    \item \textbf{Gravitational field}: curvature of spacetime
    \item \textbf{Causal structure}: light cones, timelike/spacelike separation
    \item \textbf{Matter coupling}: via minimal coupling prescription
\end{enumerate}

\subsection{Line Element}

The infinitesimal proper distance is:

\begin{equation}
\label{eq:canonical:line_element}
ds^2 = g_{\mu\nu} dx^\mu dx^\nu
\end{equation}

For timelike paths ($ds^2 > 0$), this gives proper time:
\begin{equation}
d\tau_{\text{proper}}^2 = \frac{1}{c^2} ds^2
\end{equation}

Note: $\tau_{\text{proper}}$ here is proper time, distinct from complex time $\tau = t + i\psi$.

\subsection{Volume Element}

The invariant volume element is:

\begin{equation}
\label{eq:canonical:volume_element}
d^4x \sqrt{-g} = dt\, dx\, dy\, dz\, \sqrt{-\det(g_{\mu\nu})}
\end{equation}

This is used in the action:

\begin{equation}
S = \int d^4x \sqrt{-g}\, \mathcal{L}
\end{equation}

\subsection{Compatibility with General Relativity}

In the limit where imaginary components of the biquaternionic metric vanish ($\text{Im}(\mathcal{G}_{\mu\nu}) \to 0$), the metric derived from $\Theta$ \textbf{exactly reproduces} the Einstein metric satisfying:

\begin{equation}
\label{eq:canonical:einstein_equation_from_real}
R_{\mu\nu} - \frac{1}{2} g_{\mu\nu} R = 8\pi G T_{\mu\nu}
\end{equation}

where $T_{\mu\nu} = \text{Re}(\mathcal{T}_{\mu\nu})$ is the real projection of the biquaternionic stress-energy tensor (see Section~\ref{sec:canonical:biquaternion_stress_energy}).

\begin{tcolorbox}[colback=blue!5!white,colframe=blue!75!black,title=General Relativity as Real Projection]
\textbf{General Relativity arises as the real, commutative projection of the fundamental biquaternionic geometry of spacetime.}

The classical metric is:
\begin{equation}
g_{\mu\nu}^{\text{GR}} = \text{Re}(\mathcal{G}_{\mu\nu})
\end{equation}

\textbf{Apparent violations} such as antigravity or causal drift correspond to non-real sectors of the metric and curvature ($\text{Im}(\mathcal{G}_{\mu\nu}) \neq 0$), not to exotic matter.

\textbf{UBT generalizes GR rather than contradicting it. This ensures UBT is compatible with all experimental confirmations of General Relativity.}
\end{tcolorbox}

\subsection{Complex Extension}

For full complex time $\tau = t + i\psi$, we can define extended metric:

\begin{equation}
\label{eq:canonical:complex_metric}
\tilde{g}_{\mu\nu}(\tau) = \text{Re}\,\text{Tr}\left(\frac{\partial\Theta}{\partial x^\mu} \frac{\partial\Theta^\dagger}{\partial x^\nu}\right)
\end{equation}

where derivatives may include $\partial/\partial\psi$ components.

The imaginary part:

\begin{equation}
\label{eq:canonical:metric_imaginary}
\tilde{h}_{\mu\nu} = \text{Im}\,\text{Tr}\left(\partial_\mu\Theta \, \partial_\nu\Theta^\dagger\right)
\end{equation}

encodes phase curvature and couples to consciousness/dark sectors.

\subsection{Gauge Invariance}

Under gauge transformations (Eq.~\ref{eq:canonical:theta_gauge}):

\begin{equation}
\Theta \to U\Theta V^\dagger
\end{equation}

The metric transforms as:

\begin{equation}
\label{eq:canonical:metric_gauge}
g_{\mu\nu} \to \text{Re}\,\text{Tr}\left(\partial_\mu(U\Theta V^\dagger) \partial_\nu(U\Theta V^\dagger)^\dagger\right)
\end{equation}

For $U(1)$ gauge transformations, the metric is invariant:

\begin{equation}
g_{\mu\nu}[U\Theta V^\dagger] = g_{\mu\nu}[\Theta]
\end{equation}

\subsection{Coordinate Transformations}

Under diffeomorphisms $x^\mu \to x'^\mu$:

\begin{equation}
\label{eq:canonical:metric_diffeomorphism}
g'_{\alpha\beta}(x') = \frac{\partial x^\mu}{\partial x'^\alpha} \frac{\partial x^\nu}{\partial x'^\beta} g_{\mu\nu}(x)
\end{equation}

This is the standard tensor transformation law.

\subsection{Conflict Resolution}

This canonical definition supersedes:

\begin{enumerate}
    \item ❌ \textbf{Old derivation (Appendix B)}: Different index conventions
    \item ❌ \textbf{New derivation (consolidation K2/K5)}: Alternative normalization
    \item ❌ \textbf{Experimental holographic version}: Non-standard signature
\end{enumerate}

\textbf{Canonical resolution}: Use Eq.~\ref{eq:canonical:metric} with signature $(+,-,-,-)$ and index convention $\mu,\nu = 0,1,2,3$ consistently throughout.

\subsection{Computational Formula}

For practical calculations with $\Theta \in \mathbb{C}^{4 \times 4}$:

\begin{equation}
\label{eq:canonical:metric_computational}
g_{\mu\nu} = \sum_{i=0}^{3}\sum_{j=0}^{3} \text{Re}\left[(\partial_\mu\Theta_{ij}) \overline{(\partial_\nu\Theta_{ij})}\right]
\end{equation}

In component form:
\begin{equation}
g_{\mu\nu} = \sum_{i,j} \left[\frac{\partial(\text{Re}\,\Theta_{ij})}{\partial x^\mu}\frac{\partial(\text{Re}\,\Theta_{ij})}{\partial x^\nu} + \frac{\partial(\text{Im}\,\Theta_{ij})}{\partial x^\mu}\frac{\partial(\text{Im}\,\Theta_{ij})}{\partial x^\nu}\right]
\end{equation}

\subsection{Units}

In natural units ($\hbar = c = 1$):

\begin{equation}
[g_{\mu\nu}] = \text{dimensionless}
\end{equation}

as expected for a metric tensor.

\subsection{Historical Note}

This definition provides a unique, unambiguous metric derivation from the $\Theta$ field, resolving all previous inconsistencies in signature, normalization, and index conventions.

% End of canonical metric definition
