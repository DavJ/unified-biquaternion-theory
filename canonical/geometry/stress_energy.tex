% Classical Stress-Energy Tensor (Derived Quantity)
% Version: 2.0
% Date: 2026-01-07
% Status: Canonical - DERIVED FROM BIQUATERNIONIC STRESS-ENERGY

\section{The Classical Stress-Energy Tensor $T_{\mu\nu}$ (Derived Quantity)}
\label{sec:canonical:stress_energy}

\begin{tcolorbox}[colback=yellow!10!white,colframe=orange!75!black,title=Important: This is a Derived Quantity]
\textbf{The stress-energy tensor $T_{\mu\nu}$ is NOT fundamental in UBT.}

It is the real projection of the fundamental biquaternionic stress-energy tensor $\mathcal{T}_{\mu\nu} \in \mathbb{B}$ (see Section~\ref{sec:canonical:biquaternion_stress_energy}):

\begin{equation}
T_{\mu\nu} = \text{Re}(\mathcal{T}_{\mu\nu})
\end{equation}

For the fundamental description, see Section~\ref{sec:canonical:biquaternion_stress_energy}.
\end{tcolorbox}

\subsection{Canonical Definition (as Projection)}

The classical energy-momentum (stress-energy) tensor for the biquaternion field $\Theta(q,\tau)$ is obtained as the real projection of the biquaternionic stress-energy tensor:

\begin{equation}
\label{eq:canonical:stress_energy_projection}
\boxed{T_{\mu\nu} = \text{Re}(\mathcal{T}_{\mu\nu})}
\end{equation}

where $\mathcal{T}_{\mu\nu} = \langle D_\mu \Theta, D_\nu \Theta \rangle_\mathbb{B} - \frac{1}{2}\mathcal{G}_{\mu\nu}\langle D\Theta, D\Theta \rangle$ is the fundamental biquaternionic stress-energy.

Alternatively, computed directly from the $\Theta$ field:

\begin{equation}
\label{eq:canonical:stress_energy_from_theta}
T_{\mu\nu} = \partial_\mu\Theta \, \partial_\nu\Theta^\dagger - \frac{1}{2} g_{\mu\nu} g^{\alpha\beta} \partial_\alpha\Theta \, \partial_\beta\Theta^\dagger
\end{equation}

\noindent where:
\begin{itemize}
    \item $\partial_\mu = \frac{\partial}{\partial x^\mu}$ is the partial derivative
    \item $g_{\mu\nu}$ is the metric tensor (Eq.~\ref{eq:canonical:metric})
    \item $g^{\alpha\beta}$ is the inverse metric
    \item Matrix multiplication is implied for $\Theta$ products
\end{itemize}

\subsection{Alternative Equivalent Form}

Using the trace, an equivalent expression is:

\begin{equation}
\label{eq:canonical:stress_energy_alt}
T_{\mu\nu} = \partial_\mu\Theta \, \partial_\nu\Theta^\dagger - \frac{1}{2} g_{\mu\nu} \text{Tr}\left(\partial^\alpha\Theta \, \partial_\alpha\Theta^\dagger\right)
\end{equation}

where $\partial^\alpha = g^{\alpha\beta}\partial_\beta$ is the contravariant derivative.

\subsection{Derivation from Lagrangian}

The stress-energy tensor is derived from the Lagrangian density:

\begin{equation}
\label{eq:canonical:lagrangian_theta}
\mathcal{L} = \text{Tr}\left[(\partial_\mu\Theta)^\dagger (\partial^\mu\Theta)\right]
\end{equation}

via Noether's theorem for spacetime translation invariance:

\begin{equation}
\label{eq:canonical:stress_energy_noether}
T_{\mu\nu} = \frac{\partial\mathcal{L}}{\partial(\partial^\mu\Theta)} \partial_\nu\Theta - g_{\mu\nu}\mathcal{L}
\end{equation}

This yields Eq.~\ref{eq:canonical:stress_energy}.

\subsection{Properties}

\subsubsection{Symmetry}

\begin{equation}
\label{eq:canonical:stress_symmetry}
T_{\mu\nu} = T_{\nu\mu}
\end{equation}

This follows from symmetry of the metric $g_{\mu\nu}$ and commutativity of partial derivatives.

\subsubsection{Conservation}

In flat spacetime:

\begin{equation}
\label{eq:canonical:stress_conservation_flat}
\partial^\mu T_{\mu\nu} = 0
\end{equation}

In curved spacetime:

\begin{equation}
\label{eq:canonical:stress_conservation_curved}
\nabla^\mu T_{\mu\nu} = 0
\end{equation}

where $\nabla^\mu$ is the covariant derivative.

\subsubsection{Reality}

\begin{equation}
\label{eq:canonical:stress_reality}
T_{\mu\nu} \in \mathbb{R}
\end{equation}

The stress-energy tensor is real-valued (can be verified by explicit calculation).

\subsubsection{Trace}

The trace of the stress-energy tensor is:

\begin{equation}
\label{eq:canonical:stress_trace}
T = g^{\mu\nu} T_{\mu\nu} = -\frac{1}{2} g^{\alpha\beta} \partial_\alpha\Theta \, \partial_\beta\Theta^\dagger
\end{equation}

\subsection{Physical Interpretation}

The components of $T_{\mu\nu}$ represent:

\begin{itemize}
    \item $T_{00}$ = energy density $\rho c^2$
    \item $T_{0i}$ = energy flux / momentum density $c p_i$
    \item $T_{i0}$ = momentum flux
    \item $T_{ij}$ = stress tensor (pressure, shear stress)
\end{itemize}

In detail:
\begin{align}
T_{00} &= \text{energy density} \label{eq:canonical:T00} \\
T_{0i} &= \frac{1}{c}(\text{momentum density})_i \label{eq:canonical:T0i} \\
T_{ij} &= (\text{stress tensor})_{ij} \label{eq:canonical:Tij}
\end{align}

\subsection{Connection to Einstein Equation}

The classical stress-energy tensor sources spacetime curvature via Einstein's field equation, which is the real projection of the fundamental biquaternionic equation:

\begin{equation}
\label{eq:canonical:einstein_field_from_stress}
R_{\mu\nu} - \frac{1}{2} g_{\mu\nu} R = 8\pi G T_{\mu\nu}
\end{equation}

This is obtained from:

\begin{equation}
\mathcal{E}_{\mu\nu} = \kappa \mathcal{T}_{\mu\nu} \quad \Rightarrow \quad \text{Re}(\mathcal{E}_{\mu\nu}) = \kappa \text{Re}(\mathcal{T}_{\mu\nu})
\end{equation}

where:
\begin{itemize}
    \item $R_{\mu\nu} = \text{Re}(\mathcal{R}_{\mu\nu})$ is the classical Ricci curvature tensor (real projection)
    \item $R = \text{Re}(\mathcal{R})$ is the classical Ricci scalar (real projection)
    \item $G$ is Newton's gravitational constant
    \item $\kappa = 8\pi G$ is the Einstein coupling constant
\end{itemize}

This is Einstein's field equation of General Relativity, recovered as the real limit of UBT.

\subsection{Energy Conditions}

\subsubsection{Weak Energy Condition (WEC)}

For any timelike vector $u^\mu$ ($u^\mu u_\mu = 1$):

\begin{equation}
\label{eq:canonical:wec}
T_{\mu\nu} u^\mu u^\nu \geq 0
\end{equation}

This ensures positive energy density in all frames.

\subsubsection{Dominant Energy Condition (DEC)}

\begin{equation}
\label{eq:canonical:dec}
T_{\mu\nu} u^\mu u^\nu \geq 0 \quad \text{and} \quad T^\mu_\nu u^\mu \text{ is timelike or null}
\end{equation}

This ensures energy flows at or below the speed of light.

\subsubsection{Strong Energy Condition (SEC)}

\begin{equation}
\label{eq:canonical:sec}
\left(T_{\mu\nu} - \frac{1}{2} T g_{\mu\nu}\right) u^\mu u^\nu \geq 0
\end{equation}

This condition is relevant for cosmology and singularity theorems.

\textbf{Note}: For exotic matter or dark energy, some energy conditions may be violated.

\subsection{Explicit Component Form}

For $\Theta \in \mathbb{C}^{4 \times 4}$:

\begin{equation}
\label{eq:canonical:stress_explicit}
T_{\mu\nu} = \sum_{i,j=0}^{3} \left[\partial_\mu\Theta_{ij} \overline{\partial_\nu\Theta_{ij}} - \frac{1}{2} g_{\mu\nu} g^{\alpha\beta} \partial_\alpha\Theta_{ij} \overline{\partial_\beta\Theta_{ij}}\right]
\end{equation}

\subsection{Perfect Fluid Form}

For a perfect fluid, the stress-energy tensor has the form:

\begin{equation}
\label{eq:canonical:perfect_fluid}
T_{\mu\nu}^{\text{fluid}} = (\rho + p) u_\mu u_\nu - p g_{\mu\nu}
\end{equation}

where:
\begin{itemize}
    \item $\rho$ = energy density
    \item $p$ = pressure
    \item $u^\mu$ = 4-velocity of fluid
\end{itemize}

The $\Theta$ field can reproduce this in appropriate limits.

\subsection{Electromagnetic Contribution}

For electromagnetic fields coupled to $\Theta$:

\begin{equation}
\label{eq:canonical:stress_em}
T_{\mu\nu}^{\text{EM}} = \frac{1}{4\pi}\left(F_{\mu\alpha}F_\nu^{\ \alpha} - \frac{1}{4}g_{\mu\nu}F_{\alpha\beta}F^{\alpha\beta}\right)
\end{equation}

where $F_{\mu\nu}$ is the electromagnetic field strength tensor.

Total stress-energy:
\begin{equation}
T_{\mu\nu}^{\text{total}} = T_{\mu\nu}[\Theta] + T_{\mu\nu}^{\text{EM}} + \ldots
\end{equation}

\subsection{Conflict Resolution}

This canonical definition supersedes:

\begin{enumerate}
    \item ❌ $T_{\mu\nu} = \Theta\Theta^\dagger$ (incorrect, wrong tensor structure)
    \item ❌ $T_{\mu\nu} = \frac{d\Theta}{d\tau} \times \frac{d\Theta^\dagger}{d\tau}$ (incorrect derivative)
    \item ❌ $T_{\mu\nu}$ from Lagrangian variation (different form, inconsistent normalization)
    \item ❌ Direct postulation of $T_{\mu\nu}$ without biquaternionic origin
\end{enumerate}

\textbf{Canonical resolution}: The classical stress-energy $T_{\mu\nu}$ is the real projection of the fundamental biquaternionic stress-energy $\mathcal{T}_{\mu\nu} \in \mathbb{B}$. Use either:
\begin{itemize}
    \item $T_{\mu\nu} = \text{Re}(\mathcal{T}_{\mu\nu})$ (projection method), OR
    \item Eq.~\ref{eq:canonical:stress_energy_from_theta} (direct computation from $\Theta$)
\end{itemize}

\begin{tcolorbox}[colback=blue!5!white,colframe=blue!75!black,title=Stress-Energy and Dark Sector]
\textbf{The classical stress-energy $T_{\mu\nu}$ describes only ordinary matter and energy.}

Dark sector components arise from imaginary parts of the biquaternionic stress-energy:
\begin{itemize}
    \item \textbf{Dark energy}: $S_{\mu\nu} = \text{Im}_{\text{scalar}}(\mathcal{T}_{\mu\nu})$ (phase energy)
    \item \textbf{Dark matter}: $\mathbf{P}_{\mu\nu} = \text{Im}_{\text{quaternion}}(\mathcal{T}_{\mu\nu})$ (quaternionic momentum)
\end{itemize}

These components are invisible to classical GR but affect global geometry and cosmology.

See Section~\ref{sec:canonical:exotic_regimes} for physical effects of imaginary stress-energy components.
\end{tcolorbox}

\subsection{Consistency Checks}

\subsubsection{Dimensional Analysis}

In natural units ($\hbar = c = 1$):

\begin{equation}
[T_{\mu\nu}] = [\text{energy density}] = [\text{mass}]^4 = [\text{length}]^{-4}
\end{equation}

Verified:
\begin{equation}
[\partial_\mu\Theta \partial_\nu\Theta^\dagger] = [\text{mass}]^2 \cdot [\text{length}]^{-2} = [\text{mass}]^4 \cdot [\text{length}]^{-4} \quad \checkmark
\end{equation}

(using $[\Theta] = [\text{mass}]$, $[\partial_\mu] = [\text{length}]^{-1}$).

\subsubsection{Minkowski Limit}

In flat spacetime ($g_{\mu\nu} = \eta_{\mu\nu}$) with constant $\Theta_0$:

\begin{equation}
T_{\mu\nu}[\Theta_0] = 0
\end{equation}

as expected for vacuum.

\subsection{Numerical Evaluation}

For computational purposes, use:

\begin{equation}
\label{eq:canonical:stress_numerical}
T_{\mu\nu} = \text{Re}\left[\sum_{i,j} \partial_\mu\Theta_{ij} \overline{\partial_\nu\Theta_{ij}}\right] - \frac{1}{2} g_{\mu\nu} \text{Re}\left[\sum_{i,j} g^{\alpha\beta} \partial_\alpha\Theta_{ij} \overline{\partial_\beta\Theta_{ij}}\right]
\end{equation}

\subsection{Relation to Biquaternionic Form}

In biquaternionic notation:

\begin{equation}
\label{eq:canonical:stress_biquaternion}
\mathcal{T}(q,\tau) = \nabla\Theta \otimes \nabla^\dagger\Theta^\dagger - \frac{1}{2} \mathbf{g} \text{Tr}(\nabla\Theta \nabla^\dagger\Theta^\dagger)
\end{equation}

where $\nabla$ is the biquaternionic gradient and $\mathbf{g}$ is the biquaternionic metric.

This reduces to $T_{\mu\nu}$ upon taking the real part and projecting to 4D spacetime.

\subsection{Complex Time Dependence}

For complex time $\tau = t + i\psi$:

\begin{equation}
\label{eq:canonical:stress_complex_time}
T_{\mu\nu}(\tau) = T_{\mu\nu}^{(R)}(t,\psi) + i T_{\mu\nu}^{(I)}(t,\psi)
\end{equation}

The real part $T_{\mu\nu}^{(R)}$ sources classical gravity. The imaginary part $T_{\mu\nu}^{(I)}$ couples to dark sector and internal auxiliary sector.

\subsection{Gauge Transformations}

Under $U(1)$ gauge transformations:

\begin{equation}
T_{\mu\nu}[U\Theta V^\dagger] = T_{\mu\nu}[\Theta]
\end{equation}

The stress-energy tensor is gauge-invariant.

\subsection{Historical Note}

This canonical form is the standard field-theoretic stress-energy tensor derived from the $\Theta$ field Lagrangian. It ensures compatibility with General Relativity and provides the correct source term for Einstein's equations.

% End of canonical stress-energy tensor definition
