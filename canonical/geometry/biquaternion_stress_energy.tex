% Biquaternionic Stress-Energy Tensor
% Version: 1.0
% Date: 2026-01-07
% Status: Canonical - FUNDAMENTAL GEOMETRY

\section{The Biquaternionic Stress-Energy Tensor $\mathcal{T}_{\mu\nu}$}
\label{sec:canonical:biquaternion_stress_energy}

\subsection{Fundamental Postulate}

\textbf{The stress-energy tensor is fundamentally a biquaternionic object, not a real tensor.}

The fundamental energy-momentum tensor in UBT is the \textbf{biquaternionic stress-energy tensor}:

\begin{equation}
\label{eq:canonical:biq_stress_fundamental}
\boxed{\mathcal{T}_{\mu\nu}(x) \in \mathbb{B} = \mathbb{H} \otimes \mathbb{C}}
\end{equation}

\subsection{Definition}

The biquaternionic stress-energy tensor is defined as:

\begin{equation}
\label{eq:canonical:biq_stress_definition}
\boxed{\mathcal{T}_{\mu\nu} = \langle D_\mu \Theta, D_\nu \Theta \rangle_\mathbb{B} - \frac{1}{2} \mathcal{G}_{\mu\nu} \langle D\Theta, D\Theta \rangle}
\end{equation}

\noindent where:
\begin{itemize}
    \item $\Theta(q,\tau)$ is the fundamental biquaternionic field
    \item $D_\mu \Theta$ is the biquaternionic covariant derivative (defined below)
    \item $\langle \cdot, \cdot \rangle_\mathbb{B}$ is the biquaternionic inner product
    \item $\mathcal{G}_{\mu\nu}$ is the biquaternionic metric (Section~\ref{sec:canonical:biquaternion_metric})
\end{itemize}

\subsection{Biquaternionic Covariant Derivative}

The covariant derivative of the $\Theta$ field is:

\begin{equation}
\label{eq:canonical:biq_covariant_derivative}
\boxed{D_\mu \Theta = \partial_\mu \Theta + \Omega_\mu \Theta}
\end{equation}

\noindent where:
\begin{itemize}
    \item $\partial_\mu = \frac{\partial}{\partial x^\mu}$ is the partial derivative
    \item $\Omega_\mu$ is the biquaternionic connection (Section~\ref{sec:canonical:biquaternion_connection})
    \item The product $\Omega_\mu \Theta$ is the full biquaternionic multiplication (non-commutative)
\end{itemize}

\textbf{Note:} The ordering matters. In general:
\begin{equation}
\Omega_\mu \Theta \neq \Theta \Omega_\mu
\end{equation}

\subsection{Biquaternionic Inner Product}

The inner product on biquaternion space is defined as:

\begin{equation}
\label{eq:canonical:biq_inner_product}
\langle A, B \rangle_\mathbb{B} = \text{Sc}(A^\dagger B)
\end{equation}

where:
\begin{itemize}
    \item $A^\dagger$ is the Hermitian conjugate (complex conjugation + quaternionic conjugation)
    \item $\text{Sc}(\cdot)$ extracts the scalar part
\end{itemize}

For the stress-energy tensor:

\begin{equation}
\langle D_\mu \Theta, D_\nu \Theta \rangle_\mathbb{B} = \text{Sc}[(D_\mu \Theta)^\dagger (D_\nu \Theta)]
\end{equation}

\subsection{Explicit Form}

Expanding the definition:

\begin{equation}
\mathcal{T}_{\mu\nu} = \text{Sc}[(D_\mu \Theta)^\dagger (D_\nu \Theta)] - \frac{1}{2} \mathcal{G}_{\mu\nu} \mathcal{G}^{\alpha\beta} \text{Sc}[(D_\alpha \Theta)^\dagger (D_\beta \Theta)]
\end{equation}

In components:

\begin{align}
\mathcal{T}_{\mu\nu} = &\text{Sc}[(\partial_\mu \Theta + \Omega_\mu \Theta)^\dagger (\partial_\nu \Theta + \Omega_\nu \Theta)] \\
&- \frac{1}{2} \mathcal{G}_{\mu\nu} \mathcal{G}^{\alpha\beta} \text{Sc}[(\partial_\alpha \Theta + \Omega_\alpha \Theta)^\dagger (\partial_\beta \Theta + \Omega_\beta \Theta)]
\end{align}

\subsection{Classical Stress-Energy as Projection}

\begin{tcolorbox}[colback=red!5!white,colframe=red!75!black,title=Prohibition]
\textbf{The classical stress-energy tensor $T_{\mu\nu}$ is NOT fundamental.}

It is defined exclusively as the real projection:

\begin{equation}
\label{eq:canonical:stress_projection}
\boxed{T_{\mu\nu} := \text{Re}(\mathcal{T}_{\mu\nu})}
\end{equation}

\textbf{It is FORBIDDEN to define $T_{\mu\nu}$ without reference to the biquaternionic stress-energy tensor $\mathcal{T}_{\mu\nu}$.}
\end{tcolorbox}

\subsection{Biquaternionic Decomposition}

The stress-energy tensor decomposes as:

\begin{equation}
\mathcal{T}_{\mu\nu} = T_{\mu\nu} + \mathbf{I} S_{\mu\nu} + \mathbf{J} \cdot \mathbf{P}_{\mu\nu}
\end{equation}

where:
\begin{itemize}
    \item $T_{\mu\nu} \in \mathbb{R}$ is the \textbf{real stress-energy} (classical matter-energy)
    \item $S_{\mu\nu} \in \mathbb{R}$ is the \textbf{phase energy-momentum} (dark energy, consciousness)
    \item $\mathbf{P}_{\mu\nu} \in \mathbb{R}^3$ is the \textbf{quaternionic momentum} (dark matter, spin currents)
\end{itemize}

\subsection{Properties}

\subsubsection{Hermiticity}

\begin{equation}
\mathcal{T}_{\mu\nu}^\dagger = \mathcal{T}_{\nu\mu}
\end{equation}

This is the biquaternionic generalization of symmetry.

\subsubsection{Conservation}

The biquaternionic stress-energy tensor satisfies the conservation law:

\begin{equation}
\label{eq:canonical:biq_conservation}
\nabla^\mu \mathcal{T}_{\mu\nu} = 0
\end{equation}

where $\nabla^\mu$ is the biquaternionic covariant derivative.

Taking the real part:
\begin{equation}
\nabla^\mu T_{\mu\nu} = 0 \quad \text{(classical energy-momentum conservation)}
\end{equation}

\subsubsection{Reality of Classical Part}

By construction:
\begin{equation}
T_{\mu\nu} = \text{Re}(\mathcal{T}_{\mu\nu}) \in \mathbb{R}
\end{equation}

\subsection{Physical Interpretation}

\subsubsection{Real Component: $T_{\mu\nu}$}

\begin{itemize}
    \item Classical matter and energy density
    \item Observable via gravitational coupling
    \item Satisfies Einstein's equations
    \item Components:
    \begin{itemize}
        \item $T_{00}$: energy density $\rho c^2$
        \item $T_{0i}$: momentum density / energy flux
        \item $T_{ij}$: stress tensor (pressure, shear)
    \end{itemize}
\end{itemize}

\subsubsection{Phase Energy: $S_{\mu\nu}$}

\begin{itemize}
    \item Imaginary time energy-momentum
    \item Invisible to classical observations
    \item Couples only to phase curvature $H_{\mu\nu}$
    \item Responsible for:
    \begin{itemize}
        \item Dark energy (cosmological constant-like behavior)
        \item Psychon field energy (consciousness substrate)
        \item Quantum coherence energy in biological systems
        \item Phase-locked state energy
    \end{itemize}
\end{itemize}

\subsubsection{Quaternionic Momentum: $\mathbf{P}_{\mu\nu}$}

\begin{itemize}
    \item Directional energy-momentum
    \item Couples to quaternionic curvature $\mathbf{K}_{\mu\nu\rho\sigma}$
    \item Responsible for:
    \begin{itemize}
        \item Dark matter density (via p-adic extensions)
        \item Spin currents in strong gravity
        \item Angular momentum transport
        \item Torsional energy flow
    \end{itemize}
\end{itemize}

\subsection{Field Equation}

The fundamental UBT field equation is:

\begin{equation}
\label{eq:canonical:biq_field_eqn}
\boxed{\mathcal{G}_{\mu\nu} = \kappa \mathcal{T}_{\mu\nu}}
\end{equation}

\noindent where:
\begin{itemize}
    \item $\mathcal{G}_{\mu\nu} = \mathcal{R}_{\mu\nu} - \frac{1}{2}\mathcal{G}_{\mu\nu}\mathcal{R}$ is the biquaternionic Einstein tensor
    \item $\kappa = 8\pi G$ (in natural units)
\end{itemize}

This is the \textbf{fundamental field equation of UBT}.

\subsection{Einstein's Equations as Real Projection}

Taking the real part of Eq.~\ref{eq:canonical:biq_field_eqn}:

\begin{equation}
\text{Re}(\mathcal{G}_{\mu\nu}) = \kappa \text{Re}(\mathcal{T}_{\mu\nu})
\end{equation}

This yields Einstein's field equations:

\begin{equation}
\label{eq:canonical:einstein_from_stress}
\boxed{G_{\mu\nu} = R_{\mu\nu} - \frac{1}{2} g_{\mu\nu} R = 8\pi G T_{\mu\nu}}
\end{equation}

\begin{tcolorbox}[colback=blue!5!white,colframe=blue!75!black,title=General Relativity as Real Projection]
\textbf{General Relativity arises as the real, commutative projection of fundamental biquaternionic geometry.}

The fundamental field equation is:
\begin{equation}
\mathcal{G}_{\mu\nu} = \kappa \mathcal{T}_{\mu\nu}
\end{equation}

Projecting to the real sector:
\begin{equation}
G_{\mu\nu} = 8\pi G T_{\mu\nu}
\end{equation}

\textbf{Apparent violations} (antigravity, causal drift) correspond to non-real sectors of the metric and stress-energy, not to exotic matter.

\textbf{UBT generalizes GR. It does not contradict it.}
\end{tcolorbox}

\subsection{Energy Conditions}

\subsubsection{Weak Energy Condition}

For timelike vector $u^\mu$:

\begin{equation}
\text{Re}(\mathcal{T}_{\mu\nu} u^\mu u^\nu) \geq 0
\end{equation}

This ensures positive energy density in all frames for the real sector.

\subsubsection{Dominant Energy Condition}

The real part satisfies:
\begin{equation}
T_{\mu\nu} u^\mu u^\nu \geq 0 \quad \text{and} \quad T^\mu_\nu u^\mu \text{ is timelike or null}
\end{equation}

\subsubsection{Violations in Imaginary Sector}

The imaginary components can violate energy conditions:

\begin{equation}
S_{\mu\nu} u^\mu u^\nu < 0 \quad \text{(possible)}
\end{equation}

This produces:
\begin{itemize}
    \item Pseudo-negative energy density (dark energy)
    \item Apparent antigravitational effects
    \item Accelerated cosmic expansion
\end{itemize}

These violations are \textbf{invisible to classical observations} but affect global geometry.

\subsection{Perfect Fluid Form}

For a perfect fluid in the real sector:

\begin{equation}
T_{\mu\nu}^{\text{fluid}} = (\rho + p) u_\mu u_\nu - p g_{\mu\nu}
\end{equation}

The full biquaternionic form is:

\begin{equation}
\mathcal{T}_{\mu\nu}^{\text{fluid}} = (\varrho + \mathcal{P}) \mathcal{U}_\mu \mathcal{U}_\nu - \mathcal{P} \mathcal{G}_{\mu\nu}
\end{equation}

where $\varrho, \mathcal{P}, \mathcal{U}_\mu \in \mathbb{B}$.

\subsection{Electromagnetic Contribution}

For electromagnetic fields:

\begin{equation}
\mathcal{T}_{\mu\nu}^{\text{EM}} = \frac{1}{4\pi}\left(\mathcal{F}_{\mu\alpha}\mathcal{F}_\nu^{\ \alpha} - \frac{1}{4}\mathcal{G}_{\mu\nu}\mathcal{F}_{\alpha\beta}\mathcal{F}^{\alpha\beta}\right)
\end{equation}

where $\mathcal{F}_{\mu\nu}$ is the biquaternionic field strength.

Real projection:
\begin{equation}
T_{\mu\nu}^{\text{EM}} = \text{Re}(\mathcal{T}_{\mu\nu}^{\text{EM}}) = \frac{1}{4\pi}\left(F_{\mu\alpha}F_\nu^{\ \alpha} - \frac{1}{4}g_{\mu\nu}F_{\alpha\beta}F^{\alpha\beta}\right)
\end{equation}

\subsection{Total Stress-Energy}

The total biquaternionic stress-energy is:

\begin{equation}
\mathcal{T}_{\mu\nu}^{\text{total}} = \mathcal{T}_{\mu\nu}[\Theta] + \mathcal{T}_{\mu\nu}^{\text{EM}} + \mathcal{T}_{\mu\nu}^{\text{matter}} + \cdots
\end{equation}

\subsection{Trace}

The trace of the biquaternionic stress-energy tensor is:

\begin{equation}
\mathcal{T} = \mathcal{G}^{\mu\nu} \mathcal{T}_{\mu\nu}
\end{equation}

Classical trace:
\begin{equation}
T = g^{\mu\nu} T_{\mu\nu} = \text{Re}(\mathcal{T})
\end{equation}

\subsection{Noether Theorem Derivation}

The stress-energy tensor arises from Noether's theorem for spacetime translation invariance.

Biquaternionic Lagrangian:
\begin{equation}
\mathcal{L} = \text{Sc}[(D_\mu \Theta)^\dagger (D^\mu \Theta)]
\end{equation}

Stress-energy from variation:
\begin{equation}
\mathcal{T}_{\mu\nu} = \frac{\partial \mathcal{L}}{\partial (D^\mu \Theta)} D_\nu \Theta - \mathcal{G}_{\mu\nu} \mathcal{L}
\end{equation}

This ensures consistency with variational principles.

\subsection{Exotic Regimes}

When imaginary components are significant:

\begin{equation}
\text{Im}(\mathcal{T}_{\mu\nu}) \neq 0
\end{equation}

we obtain exotic matter/energy behavior:

\begin{enumerate}
    \item \textbf{Dark energy}: $S_{00} < 0$ produces negative pressure
    \begin{equation}
    p_{\text{eff}} = -\rho_{\text{eff}} \quad \text{(cosmological constant-like)}
    \end{equation}
    
    \item \textbf{Dark matter}: $\mathbf{P}_{\mu\nu}$ produces mass without light
    \begin{equation}
    \rho_{\text{DM}} \propto |\mathbf{P}_{00}|
    \end{equation}
    
    \item \textbf{Consciousness energy}: Psychon fields contribute to $S_{\mu\nu}$
    
    \item \textbf{Quantum coherence}: Phase-locked states have $S_{\mu\nu} \neq 0$
\end{enumerate}

See Section~\ref{sec:canonical:exotic_regimes} for details.

\subsection{Explicit Computation}

For practical calculations:

\textbf{Step 1:} Compute covariant derivatives:
\begin{equation}
D_\mu \Theta = \partial_\mu \Theta + \Omega_\mu \Theta
\end{equation}

\textbf{Step 2:} Form the stress-energy tensor:
\begin{equation}
\mathcal{T}_{\mu\nu} = \text{Sc}[(D_\mu \Theta)^\dagger (D_\nu \Theta)] - \frac{1}{2} \mathcal{G}_{\mu\nu} \mathcal{G}^{\alpha\beta} \text{Sc}[(D_\alpha \Theta)^\dagger (D_\beta \Theta)]
\end{equation}

\textbf{Step 3:} Extract classical stress-energy:
\begin{equation}
T_{\mu\nu} = \text{Re}(\mathcal{T}_{\mu\nu})
\end{equation}

\textbf{Step 4:} Solve field equations:
\begin{equation}
\mathcal{G}_{\mu\nu} = \kappa \mathcal{T}_{\mu\nu}
\end{equation}

\subsection{Consistency with General Relativity}

When all imaginary components vanish:

\begin{equation}
S_{\mu\nu} \to 0, \quad \mathbf{P}_{\mu\nu} \to 0
\end{equation}

The biquaternionic stress-energy reduces to classical form:

\begin{equation}
\mathcal{T}_{\mu\nu} \to T_{\mu\nu}
\end{equation}

Einstein's equations are exactly recovered:

\begin{equation}
G_{\mu\nu} = 8\pi G T_{\mu\nu}
\end{equation}

\textbf{UBT is a consistent extension of General Relativity.}

\subsection{Conflict Resolution}

This canonical definition supersedes:

\begin{enumerate}
    \item ❌ Direct postulation of real $T_{\mu\nu}$ as fundamental
    \item ❌ Stress-energy without biquaternionic structure
    \item ❌ Ad-hoc dark energy or dark matter sources
\end{enumerate}

\textbf{Canonical resolution}: The biquaternionic stress-energy tensor $\mathcal{T}_{\mu\nu} \in \mathbb{B}$ is fundamental. Classical $T_{\mu\nu} = \text{Re}(\mathcal{T}_{\mu\nu})$ is its real projection.

\subsection{Historical Note}

The biquaternionic stress-energy formalism unifies ordinary matter, dark energy, and dark matter within a single geometric object, with classical matter emerging as the real projection.

% End of biquaternionic stress-energy tensor definition
