% Canonical QED Lagrangian Definition
% Version: 1.0
% Date: 2025-11-14
% Status: Canonical - DO NOT DUPLICATE

\section{Quantum Electrodynamics (QED) Lagrangian}
\label{sec:canonical:qed}

\subsection{Canonical Definition}

The complete QED Lagrangian in the Unified Biquaternion Theory framework is:

\begin{equation}
\label{eq:canonical:qed_lagrangian}
\boxed{\mathcal{L}_{\text{QED}} = \text{Tr}\left[(D_\mu\Theta)^\dagger (D^\mu\Theta)\right] - \frac{1}{4} F_{\mu\nu} F^{\mu\nu}}
\end{equation}

\noindent where:
\begin{itemize}
    \item $D_\mu = \partial_\mu + ig A_\mu$ is the electromagnetic covariant derivative
    \item $A_\mu$ is the electromagnetic gauge potential (photon field)
    \item $g = e$ is the electromagnetic coupling constant (elementary charge)
    \item $F_{\mu\nu} = \partial_\mu A_\nu - \partial_\nu A_\mu$ is the electromagnetic field strength tensor
    \item $\Theta \in \mathbb{C}^{4 \times 4}$ is the biquaternion field
\end{itemize}

\subsection{Components}

\subsubsection{Matter Field Term}

\begin{equation}
\label{eq:canonical:qed_matter}
\mathcal{L}_{\text{matter}} = \text{Tr}\left[(D_\mu\Theta)^\dagger (D^\mu\Theta)\right]
\end{equation}

Expanding the covariant derivative:

\begin{equation}
\mathcal{L}_{\text{matter}} = \text{Tr}\left[(\partial_\mu\Theta - ig A_\mu\Theta)^\dagger (\partial^\mu\Theta + ig A^\mu\Theta)\right]
\end{equation}

\begin{equation}
= \text{Tr}\left[(\partial_\mu\Theta)^\dagger (\partial^\mu\Theta)\right] + ig \text{Tr}\left[(\partial_\mu\Theta)^\dagger A^\mu\Theta - A_\mu\Theta^\dagger (\partial^\mu\Theta)\right] + g^2 A_\mu A^\mu \text{Tr}(\Theta^\dagger\Theta)
\end{equation}

\subsubsection{Photon Field Term}

\begin{equation}
\label{eq:canonical:qed_photon}
\mathcal{L}_{\text{photon}} = -\frac{1}{4} F_{\mu\nu} F^{\mu\nu}
\end{equation}

where the field strength tensor is:

\begin{equation}
F_{\mu\nu} = \partial_\mu A_\nu - \partial_\nu A_\mu
\end{equation}

Equivalently, in terms of electric and magnetic fields:

\begin{equation}
\mathcal{L}_{\text{photon}} = \frac{1}{2}(\mathbf{E}^2 - \mathbf{B}^2)
\end{equation}

in natural units with $c = 1$.

\subsection{Gauge Symmetry}

\subsubsection{U(1) Gauge Transformation}

The Lagrangian is invariant under local $U(1)$ gauge transformations:

\begin{align}
\Theta(x) &\to e^{i\alpha(x)} \Theta(x) \label{eq:canonical:qed_gauge_theta} \\
A_\mu(x) &\to A_\mu(x) - \frac{1}{g}\partial_\mu\alpha(x) \label{eq:canonical:qed_gauge_photon}
\end{align}

where $\alpha(x)$ is an arbitrary real scalar function.

Proof of invariance:
\begin{equation}
D_\mu\Theta \to e^{i\alpha} D_\mu\Theta \quad \Rightarrow \quad (D_\mu\Theta)^\dagger(D^\mu\Theta) \to (D_\mu\Theta)^\dagger(D^\mu\Theta)
\end{equation}

and $F_{\mu\nu} \to F_{\mu\nu}$ (gauge-invariant).

\subsubsection{Gauge Fixing}

For quantization, we choose a gauge. Common choices:

\begin{itemize}
    \item \textbf{Lorenz gauge}: $\partial^\mu A_\mu = 0$
    \item \textbf{Coulomb gauge}: $\nabla \cdot \mathbf{A} = 0$
    \item \textbf{Temporal gauge}: $A_0 = 0$
\end{itemize}

In curved spacetime:
\begin{equation}
\nabla^\mu A_\mu = 0 \quad \text{(covariant Lorenz gauge)}
\end{equation}

\subsection{Equations of Motion}

\subsubsection{Maxwell Equations}

Varying $\mathcal{L}_{\text{QED}}$ with respect to $A_\mu$ yields:

\begin{equation}
\label{eq:canonical:maxwell}
\partial_\nu F^{\mu\nu} = j^\mu
\end{equation}

where the electromagnetic current is:

\begin{equation}
\label{eq:canonical:em_current}
j^\mu = ig \text{Tr}\left[\Theta^\dagger D^\mu\Theta - (D^\mu\Theta)^\dagger \Theta\right]
\end{equation}

In curved spacetime:

\begin{equation}
\label{eq:canonical:maxwell_curved}
\nabla_\nu F^{\mu\nu} = j^\mu
\end{equation}

\subsubsection{Bianchi Identity}

The field strength satisfies:

\begin{equation}
\label{eq:canonical:bianchi}
\partial_\lambda F_{\mu\nu} + \partial_\mu F_{\nu\lambda} + \partial_\nu F_{\lambda\mu} = 0
\end{equation}

Equivalently:
\begin{equation}
\nabla_{[\lambda} F_{\mu\nu]} = 0
\end{equation}

\subsubsection{Theta Field Equation}

Varying with respect to $\Theta$:

\begin{equation}
\label{eq:canonical:theta_qed_equation}
D^\mu D_\mu\Theta = 0
\end{equation}

This is the Klein-Gordon-like equation for the charged biquaternion field.

\subsection{Interaction Term}

The interaction between matter and electromagnetic field is:

\begin{equation}
\label{eq:canonical:qed_interaction}
\mathcal{L}_{\text{int}} = ig \text{Tr}\left[A^\mu \left(\Theta^\dagger \partial_\mu\Theta - (\partial_\mu\Theta)^\dagger \Theta\right)\right] + g^2 A_\mu A^\mu \text{Tr}(\Theta^\dagger\Theta)
\end{equation}

The first term is the minimal coupling (current-photon interaction). The second term is the Proca-like mass term if $\Theta$ has non-zero vacuum expectation value.

\subsection{Current Conservation}

The electromagnetic current satisfies:

\begin{equation}
\label{eq:canonical:current_conservation}
\partial_\mu j^\mu = 0
\end{equation}

In curved spacetime:
\begin{equation}
\nabla_\mu j^\mu = 0
\end{equation}

This follows from gauge invariance via Noether's theorem.

\subsection{Coupling to Complex Time}

For complex time $\tau = t + i\psi$, the Lagrangian extends to:

\begin{equation}
\label{eq:canonical:qed_complex_time}
\mathcal{L}_{\text{QED}}(\tau) = \text{Tr}\left[(D_\mu\Theta(\tau))^\dagger (D^\mu\Theta(\tau))\right] - \frac{1}{4} F_{\mu\nu}(\tau) F^{\mu\nu}(\tau)
\end{equation}

where all fields depend on $\tau = t + i\psi$.

The imaginary time component $\psi$ introduces:
\begin{itemize}
    \item Phase modulation of electromagnetic interactions
    \item Coupling to consciousness fields (psychons)
    \item Nonlocal quantum correlations
\end{itemize}

\subsection{Fine Structure Constant}

The electromagnetic coupling strength is characterized by:

\begin{equation}
\label{eq:canonical:alpha_em}
\alpha = \frac{g^2}{4\pi\hbar c} = \frac{e^2}{4\pi\epsilon_0\hbar c} \approx \frac{1}{137.036}
\end{equation}

In UBT, $\alpha$ is \textbf{predicted} from geometric and topological properties of the $\Theta$ field (see Section~\ref{sec:canonical:alpha_derivation}).

\subsection{Renormalization}

\subsubsection{Running Coupling}

The effective coupling $\alpha$ depends on energy scale $Q$:

\begin{equation}
\label{eq:canonical:alpha_running}
\alpha(Q^2) = \frac{\alpha(\mu^2)}{1 - \frac{\alpha(\mu^2)}{3\pi}\ln\left(\frac{Q^2}{\mu^2}\right)}
\end{equation}

where $\mu$ is a reference scale.

\subsubsection{Charge Renormalization}

The bare charge $g_0$ relates to renormalized charge $g$ via:

\begin{equation}
\label{eq:canonical:charge_renormalization}
g_0 = Z_3^{-1/2} g
\end{equation}

where $Z_3$ is the photon field renormalization constant.

\subsection{Energy-Momentum Tensor}

The QED contribution to the stress-energy tensor is:

\begin{equation}
\label{eq:canonical:qed_stress_energy}
T_{\mu\nu}^{\text{QED}} = T_{\mu\nu}[\Theta] + T_{\mu\nu}^{\text{EM}}
\end{equation}

where:

\begin{equation}
T_{\mu\nu}^{\text{EM}} = \frac{1}{4\pi}\left(F_{\mu\alpha}F_\nu^{\ \alpha} - \frac{1}{4}g_{\mu\nu}F_{\alpha\beta}F^{\alpha\beta}\right)
\end{equation}

is the electromagnetic stress-energy tensor.

\subsection{Curved Spacetime Extension}

In curved spacetime with biquaternionic metric $\mathcal{G}_{\mu\nu}$ and its real projection $g_{\mu\nu} := \text{Re}(\mathcal{G}_{\mu\nu})$:

\begin{equation}
\label{eq:canonical:qed_curved}
\mathcal{L}_{\text{QED}}^{\text{curved}} = \sqrt{-g}\left[\text{Tr}\left[(D_\mu\Theta)^\dagger (D^\mu\Theta)\right] - \frac{1}{4} F_{\mu\nu} F^{\mu\nu}\right]
\end{equation}

where $g = \det(g_{\mu\nu})$ and indices are raised/lowered with $g^{\mu\nu}$ and $g_{\mu\nu}$.

The covariant derivative on $\Theta$ includes both gauge and gravitational connections:

\begin{equation}
D_\mu\Theta = \partial_\mu\Theta + ig A_\mu\Theta + \Gamma_\mu\Theta
\end{equation}

where $\Gamma_\mu$ is the spin connection.

\subsection{Conflict Resolution}

This canonical definition supersedes:

\begin{enumerate}
    \item ❌ \textbf{Lagrangian without complex time integration}: Incomplete
    \item ❌ \textbf{E/B from Maxwell in flat space}: Inconsistent with curved spacetime
    \item ❌ \textbf{Alternative gauge choices}: Use Lorenz gauge as default
\end{enumerate}

\textbf{Canonical resolution}: Use Eq.~\ref{eq:canonical:qed_lagrangian} with:
\begin{itemize}
    \item Covariant derivatives in curved spacetime
    \item $F_{\mu\nu}$ and $g_{\mu\nu} := \text{Re}(\mathcal{G}_{\mu\nu})$ from biquaternionic metric Eq.~\ref{eq:canonical:metric}
    \item Complex time $\tau$ enters through $\Theta(q,\tau)$
\end{itemize}

\subsection{Computational Implementation}

For numerical calculations:

\begin{enumerate}
    \item Define $\Theta(x,t,\psi) \in \mathbb{C}^{4 \times 4}$
    \item Compute $D_\mu\Theta = \partial_\mu\Theta + ig A_\mu\Theta$
    \item Evaluate $\text{Tr}[(D_\mu\Theta)^\dagger(D^\mu\Theta)]$
    \item Compute $F_{\mu\nu} = \partial_\mu A_\nu - \partial_\nu A_\mu$
    \item Assemble $\mathcal{L}_{\text{QED}}$
\end{enumerate}

\subsection{Physical Predictions}

From this Lagrangian, UBT predicts:

\begin{itemize}
    \item \textbf{Electron mass}: $m_e \approx 0.511$ MeV (from $\Theta$ self-energy)
    \item \textbf{Fine structure constant}: $\alpha \approx 1/137.036$ (geometric derivation)
    \item \textbf{Magnetic moment}: Anomalous magnetic moment of electron
    \item \textbf{Lamb shift}: Energy level corrections in hydrogen
\end{itemize}

All standard QED predictions are recovered in the limit $\psi \to 0$.

\subsection{Extensions}

\subsubsection{QED + Weak Interactions}

For electroweak unification, extend to:

\begin{equation}
\mathcal{L}_{\text{EW}} = \text{Tr}[(D_\mu\Theta)^\dagger(D^\mu\Theta)] - \frac{1}{4}W_{\mu\nu}^a W^{a\mu\nu} - \frac{1}{4}F_{\mu\nu}F^{\mu\nu}
\end{equation}

where $W_{\mu\nu}^a$ are the weak field strengths and $D_\mu$ includes $SU(2)_L$ covariant derivative.

\subsubsection{QED + Gravity}

Full gravitational coupling:

\begin{equation}
\mathcal{L}_{\text{QED+GR}} = \sqrt{-g}\left[\frac{R}{16\pi G} + \text{Tr}[(D_\mu\Theta)^\dagger(D^\mu\Theta)] - \frac{1}{4}F_{\mu\nu}F^{\mu\nu}\right]
\end{equation}

where $R$ is the Ricci scalar.

\subsection{Historical Note}

This canonical QED Lagrangian provides a complete, consistent formulation of electromagnetism within UBT, incorporating complex time, curved spacetime, and biquaternion field structure while maintaining compatibility with standard QED.

% End of canonical QED Lagrangian definition
