% Canonical QCD Lagrangian Definition
% Version: 1.1
% Date: 2025-11-14
% Status: Canonical - DO NOT DUPLICATE

\section{Quantum Chromodynamics (QCD) Lagrangian}
\label{sec:canonical:qcd}

\begin{center}
\fbox{\begin{minipage}{0.92\textwidth}
\textbf{Status: Embedded (Track B — Octonionic Completion Hypothesis)}\\
This section embeds standard QCD in UBT-compatible notation.
Derivation of $SU(3)$ from $\C \otimes \H$ alone is \textbf{not yet established}.
$SU(3)$ appears upon extension to $\C \otimes \O$; it is therefore part of the
Octonionic Completion Hypothesis (Track B).
See \texttt{research\_tracks/octonionic\_completion/hypothesis.md}.
\end{minipage}}
\end{center}

\subsection{Canonical Definition}

\paragraph{Metric convention.} Throughout this section, the fundamental geometry is described by the biquaternionic metric $\mathcal{G}_{\mu\nu}$. The physical metric appearing in covariant derivatives and stress-energy tensors is its real projection $g_{\mu\nu} := \text{Re}(\mathcal{G}_{\mu\nu})$.

The complete QCD Lagrangian in the Unified Biquaternion Theory framework is:

\begin{equation}
\label{eq:canonical:qcd_lagrangian}
\boxed{\mathcal{L}_{\text{QCD}} = \text{Tr}\left[(D_\mu\Theta)^\dagger (D^\mu\Theta)\right] - \frac{1}{4} G^a_{\mu\nu} G^{a\mu\nu}}
\end{equation}

\noindent where:
\begin{itemize}
    \item $D_\mu = \partial_\mu + ig_s T^a G^a_\mu$ is the QCD covariant derivative
    \item $G^a_\mu$ is the gluon field for color index $a = 1, \ldots, 8$
    \item $T^a$ are the $SU(3)$ generators (Gell-Mann matrices)
    \item $g_s$ is the strong coupling constant
    \item $G^a_{\mu\nu}$ is the gluon field strength tensor
\end{itemize}

\subsection{Gluon Field Strength}

The non-Abelian field strength tensor is:

\begin{equation}
\label{eq:canonical:gluon_field_strength}
G^a_{\mu\nu} = \partial_\mu G^a_\nu - \partial_\nu G^a_\mu + g_s f^{abc} G^b_\mu G^c_\nu
\end{equation}

\noindent where $f^{abc}$ are the $SU(3)$ structure constants.

The last term represents gluon self-interaction (a key feature of non-Abelian gauge theories).

\subsection{SU(3) Color Symmetry}

\subsubsection{Gauge Group}

QCD is based on the gauge group:

\begin{equation}
\label{eq:canonical:su3_group}
G_{\text{color}} = SU(3)_c
\end{equation}

\subsubsection{Generators}

The $SU(3)$ generators $T^a$ satisfy:

\begin{equation}
\label{eq:canonical:su3_commutator}
[T^a, T^b] = i f^{abc} T^c
\end{equation}

\begin{equation}
\label{eq:canonical:su3_normalization}
\text{Tr}(T^a T^b) = \frac{1}{2}\delta^{ab}
\end{equation}

\subsubsection{Gell-Mann Matrices}

The standard representation uses the eight Gell-Mann matrices $\lambda^a$ ($a=1,\ldots,8$):

\begin{equation}
T^a = \frac{\lambda^a}{2}
\end{equation}

Explicit forms (for reference):
\begin{align}
\lambda^1 &= \begin{pmatrix} 0 & 1 & 0 \\ 1 & 0 & 0 \\ 0 & 0 & 0 \end{pmatrix}, \quad
\lambda^2 = \begin{pmatrix} 0 & -i & 0 \\ i & 0 & 0 \\ 0 & 0 & 0 \end{pmatrix}, \quad
\lambda^3 = \begin{pmatrix} 1 & 0 & 0 \\ 0 & -1 & 0 \\ 0 & 0 & 0 \end{pmatrix} \\
\lambda^4 &= \begin{pmatrix} 0 & 0 & 1 \\ 0 & 0 & 0 \\ 1 & 0 & 0 \end{pmatrix}, \quad
\lambda^5 = \begin{pmatrix} 0 & 0 & -i \\ 0 & 0 & 0 \\ i & 0 & 0 \end{pmatrix}, \quad
\lambda^6 = \begin{pmatrix} 0 & 0 & 0 \\ 0 & 0 & 1 \\ 0 & 1 & 0 \end{pmatrix} \\
\lambda^7 &= \begin{pmatrix} 0 & 0 & 0 \\ 0 & 0 & -i \\ 0 & i & 0 \end{pmatrix}, \quad
\lambda^8 = \frac{1}{\sqrt{3}}\begin{pmatrix} 1 & 0 & 0 \\ 0 & 1 & 0 \\ 0 & 0 & -2 \end{pmatrix}
\end{align}

\subsection{Color Indices}

\textbf{Standard convention}:
\begin{itemize}
    \item Latin indices $a, b, c = 1, 2, \ldots, 8$ for adjoint representation (gluons)
    \item Latin indices $i, j, k = 1, 2, 3$ for fundamental representation (quarks)
    \item Summation convention applies: repeated indices are summed
\end{itemize}

\subsection{Structure Constants}

The $SU(3)$ structure constants $f^{abc}$ are totally antisymmetric:

\begin{equation}
f^{abc} = -f^{bac} = -f^{acb}
\end{equation}

Non-zero values (selection):
\begin{align}
f^{123} &= 1, \quad f^{147} = f^{246} = f^{257} = f^{345} = \frac{1}{2} \\
f^{156} &= f^{367} = -\frac{1}{2}, \quad f^{458} = f^{678} = \frac{\sqrt{3}}{2}
\end{align}

\subsection{Gauge Transformations}

\subsubsection{Local SU(3) Transformation}

The Lagrangian is invariant under local $SU(3)$ gauge transformations:

\begin{align}
\Theta(x) &\to U(x) \Theta(x) U^\dagger(x) \label{eq:canonical:qcd_gauge_theta} \\
G^a_\mu(x) T^a &\to U(x) G^a_\mu(x) T^a U^\dagger(x) - \frac{i}{g_s}(\partial_\mu U(x)) U^\dagger(x) \label{eq:canonical:qcd_gauge_gluon}
\end{align}

where $U(x) \in SU(3)$ is a local gauge transformation:

\begin{equation}
U(x) = \exp\left(i\alpha^a(x) T^a\right)
\end{equation}

with $\alpha^a(x)$ arbitrary real scalar functions.

\subsection{Equations of Motion}

\subsubsection{Yang-Mills Equations}

Varying $\mathcal{L}_{\text{QCD}}$ with respect to $G^a_\mu$ yields:

\begin{equation}
\label{eq:canonical:yang_mills}
D_\nu G^{a\mu\nu} = j^{a\mu}
\end{equation}

where the color current is:

\begin{equation}
\label{eq:canonical:color_current}
j^{a\mu} = g_s \text{Tr}\left[T^a \left(\Theta^\dagger D^\mu\Theta - (D^\mu\Theta)^\dagger \Theta\right)\right]
\end{equation}

The covariant derivative of the field strength is:

\begin{equation}
D_\nu G^{a\mu\nu} = \partial_\nu G^{a\mu\nu} + g_s f^{abc} G^b_\nu G^{c\mu\nu}
\end{equation}

\subsubsection{Bianchi Identity}

\begin{equation}
\label{eq:canonical:bianchi_qcd}
D_{[\lambda} G^a_{\mu\nu]} = 0
\end{equation}

Explicitly:
\begin{equation}
D_\lambda G^a_{\mu\nu} + D_\mu G^a_{\nu\lambda} + D_\nu G^a_{\lambda\mu} = 0
\end{equation}

\subsection{Asymptotic Freedom}

\subsubsection{Running Coupling}

The strong coupling $\alpha_s = g_s^2/(4\pi)$ runs with energy scale $Q$:

\begin{equation}
\label{eq:canonical:alpha_s_running}
\alpha_s(Q^2) = \frac{\alpha_s(\mu^2)}{1 + \frac{\beta_0}{4\pi}\alpha_s(\mu^2)\ln(Q^2/\mu^2)}
\end{equation}

where the one-loop beta function coefficient is:

\begin{equation}
\beta_0 = 11 - \frac{2n_f}{3}
\end{equation}

with $n_f$ the number of active quark flavors.

For $SU(3)$, $\beta_0 = 11 - 2n_f/3 > 0$ (assuming $n_f \leq 16$), leading to \textbf{asymptotic freedom}:

\begin{equation}
\alpha_s(Q^2) \to 0 \quad \text{as} \quad Q^2 \to \infty
\end{equation}

\subsubsection{QCD Scale $\Lambda_{\text{QCD}}$}

The dimensional transmutation scale:

\begin{equation}
\label{eq:canonical:lambda_qcd}
\Lambda_{\text{QCD}} \approx 200 \text{ MeV}
\end{equation}

In UBT, this is \textbf{predicted} from emergent $SU(3)$ structure (see Section~\ref{sec:canonical:emergent_su3}).

\subsection{Confinement}

At low energies ($Q \lesssim \Lambda_{\text{QCD}}$), the strong coupling becomes large:

\begin{equation}
\alpha_s(Q^2) \to \infty \quad \text{as} \quad Q^2 \to \Lambda_{\text{QCD}}^2
\end{equation}

This leads to \textbf{color confinement}: quarks and gluons are confined within hadrons.

UBT provides a geometric mechanism for confinement via complex time dynamics (see Section~\ref{sec:canonical:confinement_mechanism}).

\subsection{Quark Masses}

The QCD Lagrangian includes quark mass terms:

\begin{equation}
\label{eq:canonical:qcd_mass}
\mathcal{L}_{\text{mass}} = -\sum_{f} m_f \bar{q}_f q_f
\end{equation}

where $f$ labels quark flavors (up, down, strange, charm, bottom, top) and $q_f$ are quark fields.

In UBT, quark masses emerge from $\Theta$ field structure:

\begin{itemize}
    \item $m_u \approx 2.2$ MeV
    \item $m_d \approx 4.7$ MeV
    \item $m_s \approx 95$ MeV
    \item $m_c \approx 1.28$ GeV
    \item $m_b \approx 4.18$ GeV
    \item $m_t \approx 173$ GeV
\end{itemize}

\subsection{Gluon Condensate}

The QCD vacuum has non-zero gluon condensate:

\begin{equation}
\label{eq:canonical:gluon_condensate}
\langle 0 | \frac{\alpha_s}{\pi} G^a_{\mu\nu} G^{a\mu\nu} | 0 \rangle \approx 0.012 \text{ GeV}^4
\end{equation}

This contributes to hadron masses and QCD vacuum energy.

\subsection{Chiral Symmetry Breaking}

For light quarks ($m_u, m_d, m_s \ll \Lambda_{\text{QCD}}$), the Lagrangian has approximate chiral symmetry:

\begin{equation}
SU(n_f)_L \times SU(n_f)_R
\end{equation}

which is spontaneously broken to:

\begin{equation}
SU(n_f)_V
\end{equation}

leading to Goldstone bosons (pions, kaons, eta).

\subsection{Energy-Momentum Tensor}

The QCD contribution to stress-energy:

\begin{equation}
\label{eq:canonical:qcd_stress_energy}
T_{\mu\nu}^{\text{QCD}} = T_{\mu\nu}[\Theta] + T_{\mu\nu}^{\text{gluon}}
\end{equation}

where:

\begin{equation}
T_{\mu\nu}^{\text{gluon}} = \frac{1}{4\pi}\left(G^a_{\mu\alpha}G^{a\nu\alpha} - \frac{1}{4}g_{\mu\nu}G^a_{\alpha\beta}G^{a\alpha\beta}\right)
\end{equation}

\subsection{Curved Spacetime Extension}

In curved spacetime:

\begin{equation}
\label{eq:canonical:qcd_curved}
\mathcal{L}_{\text{QCD}}^{\text{curved}} = \sqrt{-g}\left[\text{Tr}\left[(D_\mu\Theta)^\dagger (D^\mu\Theta)\right] - \frac{1}{4} G^a_{\mu\nu} G^{a\mu\nu}\right]
\end{equation}

The covariant derivative includes both color and gravitational connections.

\subsection{Emergent SU(3) in UBT}
\label{sec:canonical:emergent_su3}

In UBT, the $SU(3)$ color symmetry appears upon octonionic extension ($\C \otimes \O$) of the internal structure of the $\Theta$ field:

\subsubsection{Mechanism}

The 8×8 extended $\Theta$ field decomposes into:

\begin{equation}
\Theta_{8 \times 8} = \sum_{a=1}^{8} \Theta_a \otimes T^a
\end{equation}

where $\Theta_a$ are $3 \times 3$ blocks and $T^a$ are $SU(3)$ generators.

\subsubsection{Derivation}

The emergent gauge fields $G^a_\mu$ arise from phase gradients:

\begin{equation}
G^a_\mu = \frac{1}{g_s} \text{Tr}\left[T^a \left(\Theta^\dagger \partial_\mu \Theta - (\partial_\mu\Theta^\dagger)\Theta\right)\right]
\end{equation}

This produces $SU(3)$ gauge structure via octonionic extension (Track~B hypothesis).

\subsection{Conflict Resolution}

This canonical definition supersedes:

\begin{enumerate}
    \item ❌ \textbf{Appendix G (emergent SU(3))}: Inconsistent color indices
    \item ❌ \textbf{Appendix K5 ($\Lambda_{\text{QCD}}$)}: Different normalization
    \item ❌ \textbf{Old main article text}: Incompatible with complex time
\end{enumerate}

\textbf{Canonical resolution}: Use Eq.~\ref{eq:canonical:qcd_lagrangian} with:
\begin{itemize}
    \item Standard $SU(3)$ generators (Gell-Mann matrices)
    \item Color indices $a, b, c = 1, \ldots, 8$
    \item Normalization $\text{Tr}(T^a T^b) = \frac{1}{2}\delta^{ab}$
    \item Emergent mechanism from $\Theta$ field structure
\end{itemize}

\subsection{Experimental Predictions}

From this Lagrangian, UBT predicts:

\begin{itemize}
    \item \textbf{$\Lambda_{\text{QCD}}$}: $\approx 200$ MeV (from emergent $SU(3)$)
    \item \textbf{Quark masses}: From $\Theta$ field configurations
    \item \textbf{Confinement scale}: Related to $\psi$ dynamics
    \item \textbf{Glueball spectrum}: From pure glue sector
\end{itemize}

\subsection{Lattice QCD Comparison}

Predictions can be tested against lattice QCD simulations:
\begin{itemize}
    \item Hadron masses
    \item Glueball masses
    \item String tension
    \item Phase transitions
\end{itemize}

\subsection{Historical Note}

This canonical QCD Lagrangian provides the strong interaction theory within UBT, where $SU(3)$ color symmetry appears via octonionic extension $\C \otimes \O$ (Octonionic Completion Hypothesis, Track~B) rather than being postulated. Derivation from the associative sector $\C \otimes \H$ alone is an open research question (Track~A).

% End of canonical QCD Lagrangian definition
