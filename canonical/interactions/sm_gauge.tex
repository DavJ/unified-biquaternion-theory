% Canonical Standard Model Gauge Structure
% Version: 1.0
% Date: 2025-11-14
% Status: Canonical - DO NOT DUPLICATE

\section{Standard Model Gauge Structure}
\label{sec:canonical:sm_gauge}

\subsection{Canonical Definition}

The complete Standard Model gauge Lagrangian in the Unified Biquaternion Theory framework is:

\begin{equation}
\label{eq:canonical:sm_lagrangian}
\boxed{\mathcal{L}_{\text{SM}} = \text{Tr}\left[(D_\mu\Theta)^\dagger (D^\mu\Theta)\right] - \frac{1}{4}W^i_{\mu\nu}W^{i\mu\nu} - \frac{1}{4}B_{\mu\nu}B^{\mu\nu} - \frac{1}{4}G^a_{\mu\nu}G^{a\mu\nu}}
\end{equation}

\noindent where the covariant derivative is:

\begin{equation}
\label{eq:canonical:sm_covariant}
D_\mu = \partial_\mu + ig_s T^a G^a_\mu + ig \tau^i W^i_\mu + ig' Y B_\mu
\end{equation}

\subsection{Gauge Group}

\begin{equation}
\label{eq:canonical:sm_group}
\boxed{G_{\text{SM}} = SU(3)_c \times SU(2)_L \times U(1)_Y}
\end{equation}

This is the \textbf{Standard Model gauge group} with three factors:

\begin{enumerate}
    \item $SU(3)_c$ = color symmetry (strong interactions, QCD)
    \item $SU(2)_L$ = weak isospin (left-handed weak interactions)
    \item $U(1)_Y$ = hypercharge (electromagnetic and weak)
\end{enumerate}

\subsection{Gauge Fields and Couplings}

\begin{table}[h]
\centering
\begin{tabular}{|l|l|l|l|}
\hline
\textbf{Group} & \textbf{Field} & \textbf{Coupling} & \textbf{Index Range} \\
\hline
$SU(3)_c$ & $G^a_\mu$ (gluons) & $g_s$ & $a = 1, \ldots, 8$ \\
$SU(2)_L$ & $W^i_\mu$ (weak bosons) & $g$ & $i = 1, 2, 3$ \\
$U(1)_Y$ & $B_\mu$ (hypercharge) & $g'$ & (single field) \\
\hline
\end{tabular}
\caption{Standard Model gauge fields and couplings}
\label{tab:canonical:sm_fields}
\end{table}

\subsection{Field Strength Tensors}

\subsubsection{SU(3) Gluon Field Strength}

\begin{equation}
\label{eq:canonical:sm_gluon}
G^a_{\mu\nu} = \partial_\mu G^a_\nu - \partial_\nu G^a_\mu + g_s f^{abc} G^b_\mu G^c_\nu
\end{equation}

where $f^{abc}$ are $SU(3)$ structure constants (see Section~\ref{sec:canonical:qcd}).

\subsubsection{SU(2) Weak Field Strength}

\begin{equation}
\label{eq:canonical:sm_weak}
W^i_{\mu\nu} = \partial_\mu W^i_\nu - \partial_\nu W^i_\mu + g \epsilon^{ijk} W^j_\mu W^k_\nu
\end{equation}

where $\epsilon^{ijk}$ is the Levi-Civita symbol ($\epsilon^{123} = 1$).

\subsubsection{U(1) Hypercharge Field Strength}

\begin{equation}
\label{eq:canonical:sm_hypercharge}
B_{\mu\nu} = \partial_\mu B_\nu - \partial_\nu B_\mu
\end{equation}

This is Abelian (no self-interaction term).

\subsection{Generators}

\subsubsection{SU(3) Generators}

\begin{equation}
T^a = \frac{\lambda^a}{2}, \quad a = 1, \ldots, 8
\end{equation}

where $\lambda^a$ are Gell-Mann matrices (see Section~\ref{sec:canonical:qcd}).

Normalization:
\begin{equation}
\text{Tr}(T^a T^b) = \frac{1}{2}\delta^{ab}
\end{equation}

\subsubsection{SU(2) Generators}

\begin{equation}
\tau^i = \frac{\sigma^i}{2}, \quad i = 1, 2, 3
\end{equation}

where $\sigma^i$ are Pauli matrices:

\begin{equation}
\sigma^1 = \begin{pmatrix} 0 & 1 \\ 1 & 0 \end{pmatrix}, \quad
\sigma^2 = \begin{pmatrix} 0 & -i \\ i & 0 \end{pmatrix}, \quad
\sigma^3 = \begin{pmatrix} 1 & 0 \\ 0 & -1 \end{pmatrix}
\end{equation}

Commutation relations:
\begin{equation}
[\tau^i, \tau^j] = i\epsilon^{ijk}\tau^k
\end{equation}

Normalization:
\begin{equation}
\text{Tr}(\tau^i \tau^j) = \frac{1}{2}\delta^{ij}
\end{equation}

\subsubsection{U(1) Generator}

\begin{equation}
Y = \text{hypercharge operator}
\end{equation}

For fermions:
\begin{equation}
Y = Q - T_3
\end{equation}

where $Q$ is electric charge and $T_3 = \tau^3$ is the third component of weak isospin.

\subsection{Electroweak Unification}

\subsubsection{Electroweak Lagrangian}

Combining $SU(2)_L \times U(1)_Y$:

\begin{equation}
\label{eq:canonical:ew_lagrangian}
\mathcal{L}_{\text{EW}} = \text{Tr}\left[(D_\mu\Theta)^\dagger (D^\mu\Theta)\right] - \frac{1}{4}W^i_{\mu\nu}W^{i\mu\nu} - \frac{1}{4}B_{\mu\nu}B^{\mu\nu}
\end{equation}

\subsubsection{Physical Gauge Bosons}

After electroweak symmetry breaking, the physical bosons are:

\begin{align}
W^\pm_\mu &= \frac{1}{\sqrt{2}}(W^1_\mu \mp i W^2_\mu) \label{eq:canonical:W_bosons} \\
Z^0_\mu &= \cos\theta_W W^3_\mu - \sin\theta_W B_\mu \label{eq:canonical:Z_boson} \\
A_\mu &= \sin\theta_W W^3_\mu + \cos\theta_W B_\mu \label{eq:canonical:photon}
\end{align}

where $\theta_W$ is the \textbf{weak mixing angle} (Weinberg angle).

\subsubsection{Weak Mixing Angle}

\begin{equation}
\label{eq:canonical:weinberg_angle}
\tan\theta_W = \frac{g'}{g}
\end{equation}

Experimental value:
\begin{equation}
\sin^2\theta_W \approx 0.23122
\end{equation}

In UBT, this is \textbf{derived} from the internal structure of $\Theta$ (see Section~\ref{sec:canonical:theta_w_derivation}).

\subsection{Coupling Constants}

\subsubsection{Gauge Couplings}

\begin{table}[h]
\centering
\begin{tabular}{|l|l|l|}
\hline
\textbf{Coupling} & \textbf{Symbol} & \textbf{Approximate Value} \\
\hline
Strong coupling & $\alpha_s(M_Z)$ & $\approx 0.118$ \\
Weak coupling & $\alpha_2 = g^2/(4\pi)$ & $\approx 1/30$ \\
Hypercharge coupling & $\alpha_1 = g'^2/(4\pi)$ & $\approx 1/59$ \\
Electromagnetic & $\alpha = e^2/(4\pi)$ & $\approx 1/137.036$ \\
\hline
\end{tabular}
\caption{Standard Model coupling constants at $M_Z$}
\label{tab:canonical:sm_couplings}
\end{table}

\subsubsection{Electromagnetic Coupling}

The electromagnetic coupling relates to weak couplings via:

\begin{equation}
\label{eq:canonical:em_coupling}
\frac{1}{\alpha} = \frac{1}{\alpha_2} + \frac{1}{\alpha_1}
\end{equation}

or equivalently:
\begin{equation}
e = g \sin\theta_W = g' \cos\theta_W
\end{equation}

\subsection{Fermion Representations}

\subsubsection{Quark Doublets (Left-Handed)}

\begin{equation}
Q_L = \begin{pmatrix} u_L \\ d_L \end{pmatrix}, \quad
Q_L' = \begin{pmatrix} c_L \\ s_L \end{pmatrix}, \quad
Q_L'' = \begin{pmatrix} t_L \\ b_L \end{pmatrix}
\end{equation}

Quantum numbers:
\begin{equation}
(SU(3), SU(2), Y) = (\mathbf{3}, \mathbf{2}, +\frac{1}{6})
\end{equation}

\subsubsection{Quark Singlets (Right-Handed)}

\begin{align}
u_R, c_R, t_R &: \quad (\mathbf{3}, \mathbf{1}, +\frac{2}{3}) \\
d_R, s_R, b_R &: \quad (\mathbf{3}, \mathbf{1}, -\frac{1}{3})
\end{align}

\subsubsection{Lepton Doublets (Left-Handed)}

\begin{equation}
L_L = \begin{pmatrix} \nu_e \\ e_L \end{pmatrix}, \quad
L_L' = \begin{pmatrix} \nu_\mu \\ \mu_L \end{pmatrix}, \quad
L_L'' = \begin{pmatrix} \nu_\tau \\ \tau_L \end{pmatrix}
\end{equation}

Quantum numbers:
\begin{equation}
(SU(3), SU(2), Y) = (\mathbf{1}, \mathbf{2}, -\frac{1}{2})
\end{equation}

\subsubsection{Lepton Singlets (Right-Handed)}

\begin{equation}
e_R, \mu_R, \tau_R : \quad (\mathbf{1}, \mathbf{1}, -1)
\end{equation}

\subsection{Higgs Mechanism}

\subsubsection{Higgs Doublet}

\begin{equation}
\Phi = \begin{pmatrix} \phi^+ \\ \phi^0 \end{pmatrix}, \quad (SU(3), SU(2), Y) = (\mathbf{1}, \mathbf{2}, +\frac{1}{2})
\end{equation}

\subsubsection{Higgs Potential}

\begin{equation}
V(\Phi) = -\mu^2 \Phi^\dagger\Phi + \lambda(\Phi^\dagger\Phi)^2
\end{equation}

\subsubsection{Vacuum Expectation Value}

\begin{equation}
\langle \Phi \rangle = \frac{1}{\sqrt{2}}\begin{pmatrix} 0 \\ v \end{pmatrix}, \quad v \approx 246 \text{ GeV}
\end{equation}

\subsubsection{Gauge Boson Masses}

\begin{align}
M_W &= \frac{gv}{2} \approx 80.4 \text{ GeV} \label{eq:canonical:W_mass} \\
M_Z &= \frac{v}{2}\sqrt{g^2 + g'^2} = \frac{M_W}{\cos\theta_W} \approx 91.2 \text{ GeV} \label{eq:canonical:Z_mass} \\
M_\gamma &= 0 \quad \text{(photon remains massless)} \label{eq:canonical:photon_mass}
\end{align}

\subsection{Yukawa Couplings}

Fermion masses arise from Yukawa interactions:

\begin{equation}
\mathcal{L}_{\text{Yukawa}} = -y_u \bar{Q}_L \tilde{\Phi} u_R - y_d \bar{Q}_L \Phi d_R - y_e \bar{L}_L \Phi e_R + \text{h.c.}
\end{equation}

where $\tilde{\Phi} = i\sigma^2\Phi^*$ and $y_u, y_d, y_e$ are Yukawa coupling matrices.

Fermion masses:
\begin{equation}
m_f = \frac{y_f v}{\sqrt{2}}
\end{equation}

\subsection{CKM Matrix}

Quark mixing is described by the Cabibbo-Kobayashi-Maskawa (CKM) matrix:

\begin{equation}
V_{\text{CKM}} = \begin{pmatrix}
V_{ud} & V_{us} & V_{ub} \\
V_{cd} & V_{cs} & V_{cb} \\
V_{td} & V_{ts} & V_{tb}
\end{pmatrix}
\end{equation}

Unitarity triangle relations provide CP violation tests.

\subsection{PMNS Matrix}

Neutrino mixing is described by the Pontecorvo-Maki-Nakagawa-Sakata (PMNS) matrix:

\begin{equation}
U_{\text{PMNS}} = \begin{pmatrix}
U_{e1} & U_{e2} & U_{e3} \\
U_{\mu 1} & U_{\mu 2} & U_{\mu 3} \\
U_{\tau 1} & U_{\tau 2} & U_{\tau 3}
\end{pmatrix}
\end{equation}

\subsection{Emergence in UBT}

\subsubsection{Gauge Group from Theta Field}

In UBT, the full SM gauge group \textbf{emerges} from the $8 \times 8$ extended $\Theta$ field:

\begin{equation}
\Theta_{8 \times 8} \quad \Rightarrow \quad SU(3)_c \times SU(2)_L \times U(1)_Y
\end{equation}

The decomposition:
\begin{equation}
\Theta = \sum_a \Theta_a^{\text{color}} \otimes T^a + \sum_i \Theta_i^{\text{weak}} \otimes \tau^i + \Theta^Y \otimes Y
\end{equation}

naturally produces the SM gauge structure.

\subsubsection{Predictions}

From the $\Theta$ field geometry, UBT predicts:
\begin{itemize}
    \item $\sin^2\theta_W$ (weak mixing angle)
    \item $\alpha_s(M_Z)$ (strong coupling)
    \item Fermion mass ratios
    \item CKM/PMNS matrix elements
\end{itemize}

\subsection{Grand Unification}

\subsubsection{GUT Scale}

At high energies ($E \sim 10^{16}$ GeV), the three couplings may unify:

\begin{equation}
\alpha_s(M_{\text{GUT}}) = \alpha_2(M_{\text{GUT}}) = \alpha_1(M_{\text{GUT}}) = \alpha_{\text{GUT}}
\end{equation}

Possible GUT groups:
\begin{itemize}
    \item $SU(5)$
    \item $SO(10)$
    \item $E_6$
\end{itemize}

\subsubsection{UBT and GUT}

The $\Theta$ field structure may naturally accommodate GUT symmetries at high energies while breaking to $SU(3) \times SU(2) \times U(1)$ at low energies.

\subsection{Conflict Resolution}

This canonical definition supersedes all previous inconsistent SM formulations by:

\begin{itemize}
    \item Unifying notation for all three gauge groups
    \item Standardizing generator normalizations
    \item Consistent index conventions throughout
    \item Embedding in $\Theta$ field structure
\end{itemize}

\subsection{Experimental Status}

All SM predictions have been confirmed by experiment, including:
\begin{itemize}
    \item Higgs boson discovery (2012, $m_H \approx 125$ GeV)
    \item Precision electroweak tests
    \item Quark and lepton masses
    \item CKM matrix elements
    \item Neutrino oscillations (PMNS matrix)
\end{itemize}

UBT must reproduce all these results in the limit $\psi \to 0$.

\subsection{Historical Note}

This canonical SM gauge structure provides the complete Standard Model within UBT, where all gauge symmetries emerge from the fundamental $\Theta$ field rather than being postulated \textit{a priori}.

% End of canonical SM gauge structure definition
