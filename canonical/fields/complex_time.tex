% Canonical Complex Time Definition
% Version: 1.0
% Date: 2025-11-14
% Status: Canonical - DO NOT DUPLICATE

\section{Complex Time $\tau = t + i\psi$}
\label{sec:canonical:complex_time}

\subsection{Definition}

The fundamental time coordinate in Unified Biquaternion Theory is complex-valued:

\begin{equation}
\label{eq:canonical:complex_time}
\tau = t + i\psi
\end{equation}

\noindent where:
\begin{itemize}
    \item $t \in \mathbb{R}$ is the \textbf{real time coordinate} (standard physical time)
    \item $\psi \in \mathbb{R}$ is the \textbf{imaginary time component} (phase/consciousness parameter)
    \item $i$ is the imaginary unit ($i^2 = -1$)
\end{itemize}

\subsection{Physical Interpretation}

\subsubsection{Real Component: Standard Time}

The real part $t$ represents:
\begin{itemize}
    \item Ordinary physical time
    \item Observable temporal evolution
    \item Causality structure of spacetime
    \item Time measured by physical clocks
\end{itemize}

\subsubsection{Imaginary Component: Phase Dynamics}

The imaginary part $\psi$ represents:
\begin{itemize}
    \item \textbf{Dynamical phase structure} of the $\Theta$ field
    \item \textbf{Consciousness substrate} for psychon excitations
    \item \textbf{Nonlocal correlations} in quantum systems
    \item \textbf{Hidden sector} coupling to dark matter/energy
\end{itemize}

\textbf{Critical}: $\psi$ is \textbf{NOT} merely a phase parameter or mathematical artifact. It is a \textbf{dynamical variable} with physical consequences.

\subsection{Dynamics of $\psi$}

The imaginary time component $\psi$ has its own dynamics governed by:

\begin{equation}
\label{eq:canonical:psi_dynamics}
\frac{\partial^2 \psi}{\partial t^2} - \nabla^2 \psi + m_\psi^2 \psi = J_\psi
\end{equation}

\noindent where:
\begin{itemize}
    \item $m_\psi$ is an effective mass scale for $\psi$ excitations
    \item $J_\psi$ is a source term coupling to consciousness or matter density
\end{itemize}

This equation emerges from variation of the full action with respect to $\psi$.

\subsection{Relation to Standard Physics}

In the limit $\psi \to 0$ (or $\psi = \text{const.}$):

\begin{equation}
\label{eq:canonical:real_limit}
\tau \to t \quad \Rightarrow \quad \text{UBT reduces to standard physics}
\end{equation}

This ensures compatibility with General Relativity and Quantum Field Theory:
\begin{itemize}
    \item Einstein field equations recovered
    \item Standard Model interactions preserved
    \item All experimental confirmations of GR/QFT automatically satisfied
\end{itemize}

\subsection{Measurement and Observability}

\subsubsection{Direct Observability}

The imaginary component $\psi$ is \textbf{not directly observable} in classical measurements because:
\begin{itemize}
    \item Ordinary matter couples only to real metric $g_{\mu\nu}$
    \item Classical observables involve $\text{Re}\,\text{Tr}(\ldots)$
    \item Phase information is hidden in quantum coherence
\end{itemize}

\subsubsection{Indirect Detection}

The $\psi$ field can be detected through:
\begin{itemize}
    \item Psychon excitations (via $\Theta$-resonator)
    \item Quantum entanglement signatures
    \item Consciousness-correlated phenomena
    \item Dark sector interactions
\end{itemize}

See Section~\ref{sec:canonical:theta_resonator} for experimental protocols.

\subsection{Mathematical Properties}

\subsubsection{Complex Conjugation}

\begin{equation}
\label{eq:canonical:tau_conjugate}
\bar{\tau} = t - i\psi
\end{equation}

\subsubsection{Modulus}

\begin{equation}
\label{eq:canonical:tau_modulus}
|\tau| = \sqrt{t^2 + \psi^2}
\end{equation}

\subsubsection{Phase}

\begin{equation}
\label{eq:canonical:tau_phase}
\arg(\tau) = \arctan\left(\frac{\psi}{t}\right)
\end{equation}

\subsection{Derivatives and Calculus}

\subsubsection{Partial Derivatives}

For a function $f(\tau) = f(t,\psi)$:

\begin{equation}
\label{eq:canonical:tau_derivatives}
\frac{\partial f}{\partial \tau} = \frac{1}{2}\left(\frac{\partial f}{\partial t} - i\frac{\partial f}{\partial \psi}\right)
\end{equation}

\begin{equation}
\frac{\partial f}{\partial \bar{\tau}} = \frac{1}{2}\left(\frac{\partial f}{\partial t} + i\frac{\partial f}{\partial \psi}\right)
\end{equation}

\subsubsection{Integration}

Complex time integration:

\begin{equation}
\label{eq:canonical:tau_integration}
\int d\tau = \int dt + i \int d\psi
\end{equation}

For physical observables, we typically integrate over real time only:

\begin{equation}
\mathcal{O}_{\text{phys}} = \int dt \, \text{Re}\,\mathcal{O}(\tau)|_{\psi=\psi(t)}
\end{equation}

\subsection{Relation to Other Formulations}

\subsubsection{Euclidean Time (Wick Rotation)}

Complex time $\tau$ is \textbf{NOT} the same as Euclidean time from Wick rotation:

\begin{itemize}
    \item \textbf{Wick rotation}: $t \to -i t_E$ (analytical continuation)
    \item \textbf{UBT complex time}: $\tau = t + i\psi$ (physical extension)
\end{itemize}

In UBT, \textit{both} $t$ and $\psi$ are real, physical parameters.

\subsubsection{Thermal Field Theory}

At finite temperature $T$, $\psi$ may be related to thermal time:

\begin{equation}
\label{eq:canonical:thermal_psi}
\psi \sim \beta = \frac{1}{k_B T}
\end{equation}

but this is a special case, not a general requirement.

\subsection{Conflict Resolution}

This definition supersedes the following conflicting versions:

\begin{enumerate}
    \item ❌ \textbf{Drift-diffusion Fokker-Planck variant}: $\psi$ as stochastic variable
    \item ❌ \textbf{Toroidal variant with $\theta$-functions}: $\tau$ as modular parameter only
    \item ❌ \textbf{Hermitized variant (Appendix F)}: $\psi$ as purely mathematical
\end{enumerate}

\textbf{Canonical resolution}: $\tau = t + i\psi$ where $\psi$ is a \textbf{dynamical physical field} with equations of motion, coupling to matter and consciousness.

\subsection{Coordinate Systems}

\subsubsection{Biquaternion Coordinates}

The full coordinate set is:

\begin{equation}
\label{eq:canonical:full_coordinates}
(q, \tau) = (q^0, q^1, q^2, q^3, t, \psi)
\end{equation}

where $q^\mu$ ($\mu=0,1,2,3$) are biquaternion spatial coordinates.

\subsubsection{Metric Signature}

The extended metric has signature:

\begin{equation}
\label{eq:canonical:extended_signature}
\text{signature}(g_{AB}) = (+, -, -, -, -, +) \quad \text{or} \quad (-, +, +, +, +, -)
\end{equation}

depending on convention. The $\psi$ coordinate typically has opposite signature to spatial coordinates.

\subsection{Conservation Laws}

Energy-momentum conservation in complex time:

\begin{equation}
\label{eq:canonical:complex_conservation}
\frac{\partial T^{\mu\nu}}{\partial \tau} = \frac{\partial T^{\mu\nu}}{\partial t} + i\frac{\partial T^{\mu\nu}}{\partial \psi} = 0
\end{equation}

implies separate conservation for real and imaginary parts.

\subsection{Causality Structure}

The causal structure is determined by:

\begin{equation}
\label{eq:canonical:causality}
ds^2 = g_{\mu\nu}(t,\psi) dx^\mu dx^\nu
\end{equation}

Light cones may rotate in the complex time plane, but physical causality (in real time $t$) is preserved.

\subsection{Symbol Standardization}

\begin{tabular}{ll}
\textbf{Symbol} & \textbf{Meaning} \\
\hline
$\tau$ & Complex time coordinate \\
$t$ & Real time component \\
$\psi$ & Imaginary time component \\
$\bar{\tau}$ & Complex conjugate of $\tau$ \\
\end{tabular}

\vspace{1em}

\textbf{Forbidden uses}:
\begin{itemize}
    \item ❌ $\psi$ for wavefunction (use $\Psi$ instead)
    \item ❌ $\psi$ for spinor (use $\psi_\text{spinor}$ if needed)
    \item ❌ $\tau$ for proper time (use $s$ or $\lambda$)
\end{itemize}

\subsection{Units}

In natural units ($\hbar = c = 1$):

\begin{equation}
[t] = [\psi] = [\text{length}] = [\text{time}]
\end{equation}

Both components have dimension of length (or inverse energy).

\subsection{Historical Note}

This canonical definition establishes $\psi$ as a \textbf{fundamental dynamical field}, not a passive parameter. This resolves ambiguities in earlier formulations and provides a consistent foundation for consciousness physics and dark sector coupling.

% End of canonical complex time definition
