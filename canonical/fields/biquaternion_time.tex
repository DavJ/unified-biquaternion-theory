% Canonical Biquaternion Time Definition
% Version: 2.0
% Date: 2025-11-14
% Status: Canonical - DO NOT DUPLICATE

\section{Biquaternion Time $T_B = t + i\psi + j\chi + k\xi$}
\label{sec:canonical:biquaternion_time}

\subsection{Definition}

The fundamental time coordinate in Unified Biquaternion Theory is \textbf{biquaternion-valued}:

\begin{equation}
\label{eq:canonical:biquaternion_time}
T_B = t + i\psi + j\chi + k\xi = t + i(\psi + \mathbf{v} \cdot \boldsymbol{\sigma})
\end{equation}

\noindent where:
\begin{itemize}
    \item $t \in \mathbb{R}$ is the \textbf{real time coordinate} (standard physical time)
    \item $\psi, \chi, \xi \in \mathbb{R}$ are \textbf{imaginary time components}
    \item $i, j, k$ are quaternion units satisfying $i^2 = j^2 = k^2 = ijk = -1$
    \item $\mathbf{v} = (\chi, \xi, \psi)$ is the vector imaginary component
    \item $\boldsymbol{\sigma} = (\sigma_1, \sigma_2, \sigma_3)$ are Pauli matrices
\end{itemize}

\subsection{Hierarchical Structure}

Biquaternion time admits a hierarchical reduction:

\begin{equation}
\label{eq:canonical:time_hierarchy}
\boxed{
\begin{array}{c}
T_B = t + i\psi + j\chi + k\xi \quad \text{(full biquaternion time)} \\
\downarrow \quad [\|\mathbf{v}\| \to 0] \\
\tau = t + i\psi \quad \text{(complex time limit)} \\
\downarrow \quad [\psi \to 0] \\
t \quad \text{(classical real time / GR)}
\end{array}
}
\end{equation}

\subsection{Complex Time as Limiting Case}

\textbf{Complex time} $\tau = t + i\psi$ is obtained as a limiting case when:

\begin{equation}
\label{eq:canonical:complex_limit}
\|\mathbf{v}\|^2 = \chi^2 + \xi^2 \ll \psi^2
\end{equation}

\textbf{Projection criterion}:
\begin{itemize}
    \item \textbf{Complex time valid}: $\|\mathbf{v}\|^2 \ll \psi^2$ (directional isotropy)
    \item \textbf{Full biquaternion required}: $\|\mathbf{v}\|^2 \sim \psi^2$ or directional anisotropy
\end{itemize}

The reduction corresponds to a holographic projection:
\begin{equation}
\pi_H: T_B \to \tau, \quad \pi_H(t + i\mathbf{v}) = t + i\langle \mathbf{v}, \mathbf{n} \rangle
\end{equation}
where $\mathbf{n}$ is the normal to a holographic boundary.

\subsection{Physical Interpretation}

\subsubsection{Real Component: Standard Time}

The real part $t$ represents:
\begin{itemize}
    \item Ordinary physical time
    \item Observable temporal evolution
    \item Causality structure of spacetime
    \item Time measured by physical clocks
\end{itemize}

\subsubsection{Scalar Imaginary Component: $\psi$}

The scalar imaginary part $\psi$ represents:
\begin{itemize}
    \item \textbf{Isotropic phase structure} of the $\Theta$ field
    \item \textbf{Consciousness substrate} for psychon excitations (scalar mode)
    \item \textbf{Scalar dark energy} contribution
    \item \textbf{Universal quantum phase}
\end{itemize}

\subsubsection{Vector Imaginary Components: $(\chi, \xi)$ or $\mathbf{v}$}

The vector imaginary parts represent:
\begin{itemize}
    \item \textbf{Directional phase structure} and anisotropies
    \item \textbf{Spacetime torsion} and twisting
    \item \textbf{Spin-dependent effects} in gravity
    \item \textbf{Anisotropic dark matter} distributions
    \item \textbf{Frame-dragging} beyond Lense-Thirring
\end{itemize}

\textbf{Critical}: All imaginary components ($\psi, \chi, \xi$) are \textbf{dynamical variables} with physical consequences, NOT mere mathematical artifacts.

\subsection{Dynamics of Biquaternion Time}

The imaginary time components have coupled dynamics:

\begin{equation}
\label{eq:canonical:biquat_time_dynamics}
\Box T_B + M^2 T_B = \mathcal{J}
\end{equation}

where $\Box = \partial_\mu\partial^\mu$ is the d'Alembertian, $M^2$ is a mass matrix, and $\mathcal{J}$ contains source terms from matter and consciousness.

In component form:
\begin{align}
\Box \psi + m_\psi^2 \psi &= J_\psi \\
\Box \chi + m_\chi^2 \chi &= J_\chi \\
\Box \xi + m_\xi^2 \xi &= J_\xi
\end{align}

\subsection{Relation to Standard Physics}

The reduction to standard physics occurs hierarchically:

\textbf{Step 1}: Complex time limit ($\chi, \xi \to 0$):
\begin{equation}
T_B \to \tau = t + i\psi \quad \text{(sufficient for weakly-coupled systems, QED)}
\end{equation}

\textbf{Step 2}: Classical limit ($\psi \to 0$):
\begin{equation}
\tau \to t \quad \text{(General Relativity, classical physics)}
\end{equation}

This ensures compatibility:
\begin{itemize}
    \item Einstein field equations recovered (at $T_B \to t$)
    \item Standard Model valid (at $T_B \to \tau$ or $T_B \to t$)
    \item All experimental confirmations of GR/QFT/SM automatically satisfied
\end{itemize}

\subsection{Measurement and Observability}

\subsubsection{Direct Observability}

The imaginary component $\psi$ is \textbf{not directly observable} in classical measurements because:
\begin{itemize}
    \item Ordinary matter couples only to real metric $g_{\mu\nu}$
    \item Classical observables involve $\text{Re}\,\text{Tr}(\ldots)$
    \item Phase information is hidden in quantum coherence
\end{itemize}

\subsubsection{Indirect Detection}

The $\psi$ field can be detected through:
\begin{itemize}
    \item Psychon excitations (via $\Theta$-resonator)
    \item Quantum entanglement signatures
    \item Consciousness-correlated phenomena
    \item Dark sector interactions
\end{itemize}

See Section~\ref{sec:canonical:theta_resonator} for experimental protocols.

\subsection{Mathematical Properties}

\subsubsection{Complex Conjugation}

\begin{equation}
\label{eq:canonical:tau_conjugate}
\bar{\tau} = t - i\psi
\end{equation}

\subsubsection{Modulus}

\begin{equation}
\label{eq:canonical:tau_modulus}
|\tau| = \sqrt{t^2 + \psi^2}
\end{equation}

\subsubsection{Phase}

\begin{equation}
\label{eq:canonical:tau_phase}
\arg(\tau) = \arctan\left(\frac{\psi}{t}\right)
\end{equation}

\subsection{Derivatives and Calculus}

\subsubsection{Partial Derivatives}

For a function $f(\tau) = f(t,\psi)$:

\begin{equation}
\label{eq:canonical:tau_derivatives}
\frac{\partial f}{\partial \tau} = \frac{1}{2}\left(\frac{\partial f}{\partial t} - i\frac{\partial f}{\partial \psi}\right)
\end{equation}

\begin{equation}
\frac{\partial f}{\partial \bar{\tau}} = \frac{1}{2}\left(\frac{\partial f}{\partial t} + i\frac{\partial f}{\partial \psi}\right)
\end{equation}

\subsubsection{Integration}

Complex time integration:

\begin{equation}
\label{eq:canonical:tau_integration}
\int d\tau = \int dt + i \int d\psi
\end{equation}

For physical observables, we typically integrate over real time only:

\begin{equation}
\mathcal{O}_{\text{phys}} = \int dt \, \text{Re}\,\mathcal{O}(\tau)|_{\psi=\psi(t)}
\end{equation}

\subsection{Relation to Other Formulations}

\subsubsection{Euclidean Time (Wick Rotation)}

Complex time $\tau$ is \textbf{NOT} the same as Euclidean time from Wick rotation:

\begin{itemize}
    \item \textbf{Wick rotation}: $t \to -i t_E$ (analytical continuation)
    \item \textbf{UBT complex time}: $\tau = t + i\psi$ (physical extension)
\end{itemize}

In UBT, \textit{both} $t$ and $\psi$ are real, physical parameters.

\subsubsection{Thermal Field Theory}

At finite temperature $T$, $\psi$ may be related to thermal time:

\begin{equation}
\label{eq:canonical:thermal_psi}
\psi \sim \beta = \frac{1}{k_B T}
\end{equation}

but this is a special case, not a general requirement.

\subsection{Conflict Resolution}

This definition supersedes the following conflicting versions:

\begin{enumerate}
    \item ❌ \textbf{Drift-diffusion Fokker-Planck variant}: $\psi$ as stochastic variable
    \item ❌ \textbf{Toroidal variant with $\theta$-functions}: $\tau$ as modular parameter only
    \item ❌ \textbf{Hermitized variant (Appendix F)}: $\psi$ as purely mathematical
\end{enumerate}

\textbf{Canonical resolution}: $\tau = t + i\psi$ where $\psi$ is a \textbf{dynamical physical field} with equations of motion, coupling to matter and consciousness.

\subsection{Coordinate Systems}

\subsubsection{Biquaternion Coordinates}

The full coordinate set is:

\begin{equation}
\label{eq:canonical:full_coordinates}
(q, \tau) = (q^0, q^1, q^2, q^3, t, \psi)
\end{equation}

where $q^\mu$ ($\mu=0,1,2,3$) are biquaternion spatial coordinates.

\subsubsection{Metric Signature}

The extended metric has signature:

\begin{equation}
\label{eq:canonical:extended_signature}
\text{signature}(g_{AB}) = (+, -, -, -, -, +) \quad \text{or} \quad (-, +, +, +, +, -)
\end{equation}

depending on convention. The $\psi$ coordinate typically has opposite signature to spatial coordinates.

\subsection{Conservation Laws}

Energy-momentum conservation in complex time:

\begin{equation}
\label{eq:canonical:complex_conservation}
\frac{\partial T^{\mu\nu}}{\partial \tau} = \frac{\partial T^{\mu\nu}}{\partial t} + i\frac{\partial T^{\mu\nu}}{\partial \psi} = 0
\end{equation}

implies separate conservation for real and imaginary parts.

\subsection{Causality Structure}

The causal structure is determined by:

\begin{equation}
\label{eq:canonical:causality}
ds^2 = g_{\mu\nu}(t,\psi) dx^\mu dx^\nu
\end{equation}

Light cones may rotate in the complex time plane, but physical causality (in real time $t$) is preserved.

\subsection{Symbol Standardization}

\begin{tabular}{ll}
\textbf{Symbol} & \textbf{Meaning} \\
\hline
$\tau$ & Complex time coordinate \\
$t$ & Real time component \\
$\psi$ & Imaginary time component \\
$\bar{\tau}$ & Complex conjugate of $\tau$ \\
\end{tabular}

\vspace{1em}

\textbf{Forbidden uses}:
\begin{itemize}
    \item ❌ $\psi$ for wavefunction (use $\Psi$ instead)
    \item ❌ $\psi$ for spinor (use $\psi_\text{spinor}$ if needed)
    \item ❌ $\tau$ for proper time (use $s$ or $\lambda$)
\end{itemize}

\subsection{Units}

In natural units ($\hbar = c = 1$):

\begin{equation}
[t] = [\psi] = [\text{length}] = [\text{time}]
\end{equation}

Both components have dimension of length (or inverse energy).

\subsection{Historical Note}

This canonical definition establishes $\psi$ as a \textbf{fundamental dynamical field}, not a passive parameter. This resolves ambiguities in earlier formulations and provides a consistent foundation for consciousness physics and dark sector coupling.

% End of canonical complex time definition
