% ======================== CT Scheme Definition =========================
\section{Complex-Time (CT) Renormalization Scheme}
\label{sec:ct-scheme}

\paragraph{Purpose:} This section formalizes assumption \textbf{A2} from the CT Two-Loop Baseline
(Appendix~CT, Section~\ref{app:ct-baseline-R1}), detailing the CT renormalization prescription
and validating the conditions required for the rigorous result $\mathcal{R}_{\mathrm{UBT}} = 1$.

\paragraph{Scope.}
The CT scheme specifies (i) time-contour/analytic continuation, (ii) regularization,
(iii) subtraction/renormalization conditions, and (iv) Ward-identity enforcement.
It is used to evaluate higher-loop corrections that yield the factor
\(\mathcal R_{\mathrm{UBT}}\) in the UBT expression \( B=\frac{2\pi N_{\mathrm{eff}}}{3R_\psi}\times \mathcal R_{\mathrm{UBT}} \).
Under the standard assumptions detailed below, Theorem~\ref{thm:RUBT-equals-one} proves 
$\mathcal{R}_{\mathrm{UBT}} = 1$ with no fitting parameters.

\subsection{Contour and continuation}
We work with a two-leg complex-time contour \(\mathcal C\) that reduces to standard
real-time QED in the limit \(\psi\to 0\). Propagators are defined by contour-ordered
correlation functions \( \langle \mathcal T_{\mathcal C} \cdots \rangle \) and their
spectral representations. The prescription satisfies KMS-like analyticity and recovers
Feynman boundary conditions at \(\psi\to 0\).

\subsection{Regularization and subtractions}
Dimensional regularization with \(d=4-2\epsilon\) is used throughout. We define a
CT-\(\overline{\mathrm{MS}}\) subtraction:
\[
Z_i = 1 + \sum_{\ell\ge1}\frac{z_i^{(\ell)}(\xi)}{\epsilon^\ell},\qquad
\mu\frac{d}{d\mu}\log Z_i \;\text{finite},\quad i\in\{A,\psi,e\}.
\]
Here \(\xi\) is the covariant gauge parameter. Counterterms are chosen to preserve
Ward identities. Fields are normalized to match the Thomson limit in the QED reduction.

\subsection{Ward identities}
We require \(Z_1=Z_2\) to all perturbative orders (vertex vs.\ fermion wavefunction).
The photon Ward identity fixes the longitudinal part of the vacuum polarization.
These constraints eliminate \(\xi\)-dependence from the renormalized charge and ensure
gauge-parameter independence of \(B\) at the order considered.

\subsection{QED/real-time limit}
In the limit \(\psi\to 0\) and with the real-time contour, CT reduces to standard
\(\overline{\mathrm{MS}}\) QED. The two-loop correction to \(\alpha\) becomes the known
small QED value; any finite enhancement in \(\mathcal R_{\mathrm{UBT}}\) is thus a bona fide
CT effect and must be derived from first principles in this scheme.

\subsection{Scheme statement}
The quantity \(\mathcal R_{\mathrm{UBT}}\) is defined as the finite, scheme-stable,
gauge-parameter independent factor extracted from the renormalized two-loop corrections
to the photon vacuum polarization and charge renormalization in the CT-\(\overline{\mathrm{MS}}\)
prescription at the specified reference scale \(\mu\).
% =======================================================================
