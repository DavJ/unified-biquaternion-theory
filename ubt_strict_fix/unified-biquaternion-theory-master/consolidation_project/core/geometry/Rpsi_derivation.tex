% R_psi Derivation: Complex Time Fibre Quantization
% Geometric determination of compactification radius
% Author: UBT Team
% Purpose: Prove R_psi is fixed by geometry, independent of matter parameters

\subsection{\texorpdfstring{\(R_\psi\)}{R\_psi} Derivation: CT Fibre Periodicity}
\label{sec:Rpsi-derivation}

The compactification radius \(R_\psi\) of the imaginary time coordinate \(\psi\)
is determined by canonical normalization of the zero-mode photon field. It is
\textbf{not a physical scale}, but rather a normalization convention that cancels
in observable quantities.

\subsubsection{Definition: CT Fibre Structure}

\begin{definition}[Complex Time Fibre]
\label{def:CT-fibre}
In the biquaternionic formulation, the time coordinate extends to:
\begin{equation}
\label{eq:tau-structure}
\tau = t + i\psi + j\chi + k\xi \in \mathbb{H}_{\mathbb{C}}
\end{equation}
where:
\begin{itemize}
    \item \(t \in \mathbb{R}\): physical real time
    \item \(\psi, \chi, \xi \in S^1\): compact imaginary directions
\end{itemize}

The \textbf{CT fibre} is the \(S^1\) component corresponding to \(\psi\),
with periodicity:
\begin{equation}
\label{eq:psi-periodicity}
\psi \sim \psi + 2\pi R_\psi
\end{equation}
\end{definition}

\textbf{Note:} The factor \(2\pi\) is conventional. We can always rescale \(\psi\)
to absorb \(R_\psi\), which is why it's a \emph{normalization choice}, not a
physical parameter.

\subsubsection{Quantization from Boundary Conditions}

\begin{lemma}[Quantization of \texorpdfstring{\(\psi\)}{psi}-Cycles]
\label{lem:psi-quantization}
Regularity of the biquaternionic field \(\Theta(q, \tau)\) under \(\psi \to \psi + 2\pi\)
requires:
\begin{equation}
\label{eq:psi-quantization}
e^{2\pi i \hat{L}_\psi} \Theta = \Theta
\end{equation}
where \(\hat{L}_\psi\) is the generator of \(\psi\)-translations.

This quantizes the allowed field configurations and fixes the normalization
of the \(\psi\)-direction.
\end{lemma}

\begin{proof}
\textbf{Step 1:} The field \(\Theta\) must be single-valued on the compactified
\(\psi\) circle:
\[\Theta(q, t, \psi + 2\pi, \chi, \xi) = \Theta(q, t, \psi, \chi, \xi)\]

\textbf{Step 2:} Expand in Fourier modes along \(\psi\):
\[\Theta(q, t, \psi, \ldots) = \sum_{n \in \mathbb{Z}} \Theta_n(q, t, \chi, \xi) e^{in\psi}\]

\textbf{Step 3:} Each mode \(n\) transforms under \(\psi\)-translation by:
\[\Theta_n \to \Theta_n e^{2\pi i n}\]

Single-valuedness requires \(e^{2\pi i n} = 1\), which holds for all \(n \in \mathbb{Z}\).

\textbf{Step 4:} The allowed winding numbers \(n\) are quantized (integers),
independent of any physical scale.

\textbf{Conclusion:} The \(\psi\)-periodicity is \emph{intrinsic} to the
geometry, not set by matter content.
\end{proof}

\subsubsection{Determination of \texorpdfstring{\(R_\psi\)}{R\_psi} via Zero-Mode Normalization}

\begin{theorem}[Determination of \texorpdfstring{\(R_\psi\)}{R\_psi}]
\label{thm:Rpsi-determination}
The compactification radius \(R_\psi\) is fixed by requiring canonical
normalization of the photon zero-mode:
\begin{equation}
\label{eq:Rpsi-normalization}
\int_0^{2\pi} d\psi \, |\xi_0(\psi)|^2 = 1
\end{equation}
where \(\xi_0(\psi)\) is the photon zero-mode profile.

This gives:
\begin{equation}
\label{eq:Rpsi-value}
R_\psi = 1 \quad \text{(dimensionless, canonical choice)}
\end{equation}
\end{theorem}

\begin{proof}
\textbf{Setup:} Perform Kaluza-Klein reduction from 5D (4D spacetime + \(\psi\))
to 4D. The 5D photon field \(A_M(x, \psi)\) (\(M = 0,1,2,3,5\)) decomposes as:
\[A_\mu(x, \psi) = \sum_{n \in \mathbb{Z}} A_\mu^{(n)}(x) \, \xi_n(\psi)\]
where \(\xi_n(\psi) = e^{in\psi}/\sqrt{2\pi}\) are normalized modes.

\textbf{Zero-mode sector (\(n=0\)):}
\[\xi_0(\psi) = \frac{1}{\sqrt{2\pi}} = \text{constant}\]

\textbf{Normalization integral:}
\[\int_0^{2\pi} d\psi \, |\xi_0(\psi)|^2 = \int_0^{2\pi} d\psi \, \frac{1}{2\pi} = 1\]

This is satisfied \emph{by construction} when we set the period to \(2\pi\).

\textbf{Effective 4D gauge coupling:}
From the 5D action:
\[S_5 = \int d^4x \, d\psi \, \left[ -\frac{1}{4g_5^2} F_{MN}^2 \right]\]

Integrating out \(\psi\):
\[S_4 = \int d^4x \left[ -\frac{1}{4g_4^2} F_{\mu\nu}^2 \right]\]
where:
\[\frac{1}{g_4^2} = \frac{2\pi R_\psi}{g_5^2}\]

\textbf{Physical interpretation:} The 4D coupling \(g_4\) is what we measure.
Changing \(R_\psi\) is equivalent to changing \(g_5\), but the ratio
\(g_4\) (the observable) is \emph{unaffected}.

\textbf{Convention:} Set \(R_\psi = 1\) as the canonical choice. This is like
choosing units where \(\hbar = c = 1\)—a normalization, not physics.

\textbf{Conclusion:} \(R_\psi\) is determined by normalization convention.
The standard choice is \(R_\psi = 1\).
\end{proof}

\subsubsection{Independence from Physical Parameters}

\begin{corollary}[Independence of \texorpdfstring{\(R_\psi\)}{R\_psi}]
\label{cor:Rpsi-independence}
The compactification radius \(R_\psi\) is independent of:
\begin{enumerate}
    \item Renormalization scale \(\mu\)
    \item Gauge parameter \(\xi\)
    \item Fermion mass \(m_e\)
    \item Any other dynamical matter parameter
\end{enumerate}
\end{corollary}

\begin{proof}
\textbf{Part 1 (\(\mu\) independence):}
\(R_\psi\) is a geometric property of the compactification, set at the
\emph{classical} level (before quantization). The renormalization scale \(\mu\)
is introduced during loop calculations (quantum corrections).

Geometrical structures are \(\mu\)-independent: \(\frac{\partial R_\psi}{\partial \mu} = 0\).

\textbf{Part 2 (\(\xi\) independence):}
Gauge parameter \(\xi\) affects the photon propagator structure, not the
underlying geometry. The periodicity \(\psi \sim \psi + 2\pi\) is gauge-independent.

Formally: BRST cohomology (which determines physical content) is gauge-independent,
and \(R_\psi\) is part of the cohomology structure.

\textbf{Part 3 (\(m_e\) independence):}
Fermion mass is a Yukawa coupling between fermion and Higgs (or a bare mass term).
Neither affects the geometry of the \(\psi\) circle.

Changing \(m_e\) changes loop integrals (via fermion propagator), but not
the compactification radius: \(\frac{\partial R_\psi}{\partial m_e} = 0\).

\textbf{Conclusion:} \(R_\psi\) is purely geometric, decoupled from dynamics.
\end{proof}

\subsubsection{Observability and Gauge Invariance}

\begin{remark}[Non-Observability of \texorpdfstring{\(R_\psi\)}{R\_psi}]
\label{rem:Rpsi-non-observable}
The compactification radius \(R_\psi\) is \textbf{not observable}. Only
dimensionless ratios like:
\begin{equation}
\label{eq:observable-ratio}
\frac{R_\psi}{R_\chi}, \quad \frac{R_\psi}{R_\xi}
\end{equation}
or derived quantities like \(B = 2\pi N_{\mathrm{eff}}/(3R_\psi)\) appear
in physical predictions.

Changing \(R_\psi \to \lambda R_\psi\) is equivalent to rescaling the zero-mode
field, which cancels in observables:
\begin{equation}
\label{eq:rescaling-cancellation}
A_\mu^{(0)} \to \lambda^{-1/2} A_\mu^{(0)}, \quad R_\psi \to \lambda R_\psi
\end{equation}
but \(g_4^2 = (2\pi R_\psi/g_5^2)\) remains unchanged.
\end{remark}

This is analogous to choosing natural units (\(\hbar = c = 1\)). We can set
\(R_\psi = 1\) without loss of generality.

\subsubsection{Tests and Verification}

The independence claims can be tested:

\begin{enumerate}
    \item \textbf{Test \(\partial R_\psi / \partial \mu = 0\):}
    
    Vary \(\mu\) in symbolic computation and verify \(R_\psi\) unchanged.
    
    Test: \texttt{test\_Rpsi\_independence.py}
    
    \item \textbf{Test \(\partial R_\psi / \partial \xi = 0\):}
    
    Vary gauge parameter and verify no effect on \(R_\psi\).
    
    \item \textbf{Test gauge coupling scaling:}
    
    Verify \(g_4^2 \propto R_\psi\) but observables (like \(\alpha\)) are
    \(R_\psi\)-independent.
\end{enumerate}

\subsubsection{Summary}

\begin{table}[h]
\centering
\caption{Derivation of \(R_\psi\): Normalization Convention}
\label{tab:Rpsi-summary}
\begin{tabular}{lll}
\hline
Property & Value & Reference \\
\hline
Definition & \(\psi\)-fibre compactification radius & Def.~\ref{def:CT-fibre} \\
Periodicity & \(\psi \sim \psi + 2\pi R_\psi\) & Eq.~\eqref{eq:psi-periodicity} \\
Quantization & Integer winding \(n \in \mathbb{Z}\) & Lemma~\ref{lem:psi-quantization} \\
Normalization & Zero-mode: \(\int |\xi_0|^2 = 1\) & Theorem~\ref{thm:Rpsi-determination} \\
Canonical value & \(R_\psi = 1\) (dimensionless) & Eq.~\eqref{eq:Rpsi-value} \\
\(\mu\) independence & \(\partial R_\psi/\partial \mu = 0\) & Cor.~\ref{cor:Rpsi-independence} \\
\(\xi\) independence & \(\partial R_\psi/\partial \xi = 0\) & Cor.~\ref{cor:Rpsi-independence} \\
\(m_e\) independence & \(\partial R_\psi/\partial m_e = 0\) & Cor.~\ref{cor:Rpsi-independence} \\
Observability & Not directly observable & Rem.~\ref{rem:Rpsi-non-observable} \\
\hline
\end{tabular}
\end{table}

\textbf{Conclusion:} \(R_\psi\) is a normalization convention (like \(\hbar = 1\)),
not a tunable parameter. Setting \(R_\psi = 1\) is the canonical choice and
introduces no physics beyond the geometry.
