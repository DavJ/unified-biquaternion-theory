% ================== CT Two-Loop Renormalization for Fermion Masses ==================
% VERSION: v1.0 - Program Sketch
% AUTHOR: UBT Team
% PURPOSE: CT renormalization conditions that tie Yukawa couplings to geometric invariants
%
% DEPENDENCIES:
% Requires: \usepackage{amsmath,amssymb,amsthm}
% References: Appendix CT (CT baseline), yukawa_in_HC.tex

\section{CT Two-Loop Renormalization Conditions for Yukawa Sector}
\label{app:ct-two-loop-yukawa}

\subsection{Overview}

The derivation of fermion masses from first principles in UBT requires extending the CT renormalization framework (established for $\alpha$ in Appendix CT) to the Yukawa sector. The key insight is that CT renormalization conditions, combined with Ward identities and Thomson-limit matching, can constrain Yukawa coupling parameters without introducing fitted values.

This appendix outlines the program. Full calculations are in progress.

\subsection{CT Renormalization at Two Loops}

In the Standard Model, fermion masses arise from Yukawa couplings to the Higgs field:
\[
\mathcal L_{\text{Yukawa}} = -Y_{\ell}^{ij} \bar{L}_i \phi e_R^j + \text{h.c.}
\]
where $Y_{\ell}^{ij}$ is the charged-lepton Yukawa matrix.

At the bare level, these couplings are divergent and require renormalization:
\[
Y_{\ell,\text{bare}} = Z_Y \cdot Y_{\ell,\text{ren}}(\mu),
\]
where $Z_Y$ is the Yukawa renormalization constant and $\mu$ is the renormalization scale.

\paragraph{Ward identities and mixed terms.}
In QED, the renormalization of the photon-fermion vertex is constrained by the Ward identity $Z_1 = Z_2$. This same identity applies in the CT scheme (Theorem~\ref{thm:ward-ct} in Appendix CT).

For the Yukawa sector, there exist analogous constraints from:
\begin{itemize}
  \item Gauge invariance under $\mathrm{SU}(2)_L \times \mathrm{U}(1)_Y$
  \item Flavor symmetry (if not broken by Yukawa textures)
  \item Discrete symmetries (P, T, C) enforced by biquaternionic involutions
\end{itemize}

\subsection{Counterterms as Functions of Geometric Invariants}

The central goal of this program is to show that certain Yukawa counterterms become functions of $\mathbb H_{\mathbb C}$ invariants rather than independent parameters.

\paragraph{Locking mechanism (analogous to $\alpha$).}
For the fine-structure constant, we established (Appendix CT):
\begin{itemize}
  \item Geometric locking fixes $N_{\mathrm{eff}}$ and $R_\psi$ without tunable parameters (A1)
  \item Ward identities enforce $\mathcal R_{\mathrm{UBT}} = 1$ at two loops (A2)
  \item Thomson-limit extraction yields gauge-invariant $B$ (A3)
\end{itemize}

For fermion masses, we seek an analogous structure:
\begin{itemize}
  \item \textbf{Geometric constraints} from Hermitian slice structure determine allowed Yukawa textures
  \item \textbf{CT renormalization conditions} at two loops relate counterterms to geometric invariants
  \item \textbf{Sum rules} emerge as falsifiable predictions (mass ratios independent of overall scale)
  \item \textbf{Absolute scale} fixed by geometric normalization (e.g., measure normalization used for $N_{\mathrm{eff}}$)
\end{itemize}

\subsection{Sketch of Two-Loop Calculation}

At one loop, the Yukawa renormalization receives contributions from:
\begin{enumerate}
  \item Higgs self-energy corrections
  \item Fermion self-energy corrections
  \item Gauge boson loops (photon, $W^\pm$, $Z^0$)
\end{enumerate}

At two loops, additional diagrams involve:
\begin{enumerate}
  \item Mixed gauge-Yukawa vertex corrections
  \item Two-loop fermion self-energies
  \item Higgs-gauge mixing terms
\end{enumerate}

In the CT scheme, each diagram is evaluated with complex-time propagators:
\[
D(p) = D(p_0 + i\psi, \mathbf{p}),
\]
where the imaginary component $\psi$ acts as a regulator.

\paragraph{Ward identity constraints.}
The key observation is that Ward identities (enforced by gauge invariance) reduce the number of independent counterterms. In particular:
\begin{itemize}
  \item The photon contribution to Yukawa renormalization is fixed by $Z_1 = Z_2$
  \item The $W^\pm$ and $Z^0$ contributions are constrained by $\mathrm{SU}(2)_L$ gauge invariance
  \item The Higgs contribution is related to the Higgs quartic coupling via the scalar Ward identity
\end{itemize}

After imposing these constraints, the remaining freedom is parametrized by:
\[
Y_{\ell,\text{ren}}(\mu) = Y_{\ell,\text{geom}} \times f(\mu, N_{\mathrm{eff}}, R_\psi, \ldots),
\]
where $Y_{\ell,\text{geom}}$ encodes the geometric structure from $\mathbb H_{\mathbb C}$ and $f$ is a known renormalization group function.

\subsection{Sketch of Constraints}

If the biquaternionic structure imposes sufficient constraints (via involutions and Hermitian slice restrictions), the Yukawa matrix takes a restricted form:
\[
Y_{\ell} = a \cdot \mathbb{1}_{3\times 3} + b \cdot \Sigma + c \cdot T + \ldots,
\]
where:
\begin{itemize}
  \item $a, b, c$ are determined by geometric invariants (not free parameters)
  \item $\Sigma, T$ are fixed matrices determined by discrete symmetries
  \item The ellipsis indicates possible higher-order terms constrained similarly
\end{itemize}

The two-loop CT renormalization then fixes ratios among $a, b, c$ by relating them to:
\begin{itemize}
  \item The same $N_{\mathrm{eff}}$ and $R_\psi$ used in $\alpha$ derivation
  \item Overlap integrals of mode functions on the Hermitian slice
  \item Topological winding numbers or other invariants
\end{itemize}

\paragraph{No free parameters.}
Crucially, if this program succeeds, there are \textbf{no adjustable knobs}. All coupling ratios are fixed by geometry and renormalization consistency, analogous to the $\mathcal R_{\mathrm{UBT}} = 1$ baseline for $\alpha$.

\subsection{Testability and Falsifiability}

The predictions from this framework are:
\begin{enumerate}
  \item \textbf{Mass ratios} (e.g., $m_\mu/m_e$, $m_\tau/m_\mu$) computed from Yukawa eigenvalues
  \item \textbf{Sum rules} relating different sectors (leptons vs. quarks, if applicable)
  \item \textbf{Absolute scale} for fermion masses tied to geometric normalization
\end{enumerate}

Each prediction is falsifiable:
\begin{itemize}
  \item If computed ratios disagree with PDG values $\to$ theory falsified
  \item If sum rules fail experimental tests $\to$ theory falsified
  \item If absolute scale prediction is off by more than error budget $\to$ theory falsified
\end{itemize}

\subsection{Current Status and Roadmap}

\textbf{Status:} Program outlined; detailed two-loop calculations not yet complete.

\textbf{Next steps:}
\begin{enumerate}
  \item Formalize the Yukawa coupling structure on the Hermitian slice (Appendix~\ref{app:yukawa-in-HC})
  \item Compute two-loop Yukawa counterterms in CT scheme using dimensional regularization
  \item Verify Ward identities hold and constrain independent parameters
  \item Extract mass ratios and compare to experimental data
  \item Derive absolute mass scale from geometric normalization
\end{enumerate}

\textbf{Timeline estimate:} 18--24 months for first falsifiable predictions.

\subsection{Relation to Existing UBT Literature}

This appendix extends the CT baseline framework (Appendix CT) from the gauge sector to the Yukawa sector. The key principles are:
\begin{itemize}
  \item No fitted parameters
  \item Geometric locking via $\mathbb H_{\mathbb C}$ structure
  \item Ward identities enforce consistency
  \item Falsifiable predictions prioritized over parameter fitting
\end{itemize}

% ================== END ==================
