% © 2025 Ing. David Jaroš — CC BY-NC-ND 4.0
%
% This work is licensed under a Creative Commons Attribution-NonCommercial-NoDerivatives 
% 4.0 International License (CC BY-NC-ND 4.0).
%
% License History: Earlier drafts (up to v0.3) were released under CC BY 4.0. 
% From v0.4 onward, all material is released under CC BY-NC-ND 4.0 to protect 
% the integrity of the theoretical work during ongoing academic development.
%
% See LICENSE.md for full license text.

\documentclass[12pt]{article}
\usepackage{amsmath,amssymb,amsfonts}
\usepackage{geometry}
\usepackage{hyperref}
\usepackage{booktabs}
\geometry{a4paper, margin=1in}

% Define keywords command
\providecommand{\keywords}[1]{\textbf{Keywords:} #1}

\title{Unified Biquaternion Theory (UBT): Complex Time, Consciousness, and Field Unification}
\author{Ing.~David~Jaroš}
\date{November 2025}

\begin{document}
\maketitle

% License Notice - Visible in PDF
\noindent
\textbf{License:} © 2025 Ing. David Jaroš. This work is licensed under a Creative Commons Attribution-NonCommercial-NoDerivatives 4.0 International License (CC BY-NC-ND 4.0). Earlier drafts (up to v0.3) were released under CC BY 4.0. From v0.4 onward, all material is released under CC BY-NC-ND 4.0 to protect the integrity of the theoretical work during ongoing academic development. See \url{https://creativecommons.org/licenses/by-nc-nd/4.0/} for details.

\vspace{1em}

\begin{abstract}
The Unified Biquaternion Theory (UBT) presents a unified framework combining General Relativity, Quantum Field Theory, and Standard Model symmetries within a biquaternionic field defined over complex time $\tau = t + i\psi$. This theory generalizes Einstein's General Relativity and, in the real-valued limit, exactly reproduces Einstein's field equations for all curvature regimes. UBT extends the framework through additional biquaternionic degrees of freedom that may correspond to dark sector physics and quantum gravitational corrections.
\end{abstract}

\keywords{biquaternion, complex time, consciousness, unified field theory, Hermitian gravity, SU(3) symmetry, theta function}

\tableofcontents

\section{Introduction}

The Unified Biquaternion Theory (UBT) aims to provide a comprehensive framework for understanding fundamental physics by embedding General Relativity, Quantum Field Theory, and the Standard Model within a single mathematical structure based on biquaternionic fields over complex time.

The core equation of UBT is:
\begin{equation}
\nabla^\dagger\nabla\Theta(q,\tau) = \kappa \mathcal{T}(q,\tau)
\end{equation}
where $\Theta(q,\tau)$ is the unified biquaternionic field, $q \in \mathbb{C}\otimes\mathbb{H}$ represents biquaternion coordinates, $\tau = t + i\psi$ is complex time, $\nabla^\dagger\nabla$ is the gauge-covariant d'Alembertian, $\kappa$ is the coupling constant related to $8\pi G$, and $\mathcal{T}$ is the energy-momentum tensor.

\section{Mathematical Foundations}

\subsection{Biquaternionic Structure}

The biquaternion algebra $\mathbb{B} = \mathbb{C}\otimes\mathbb{H}$ combines complex numbers with quaternions, providing a rich algebraic structure capable of encoding both spacetime geometry and internal gauge symmetries.

\subsection{Complex Time}

Complex time $\tau = t + i\psi$ extends the real time coordinate $t$ by adding an imaginary component $\psi$ that represents phase-like degrees of freedom. This extension enables the unified treatment of quantum phases and gravitational curvature within a single geometric framework.

\subsection{Biquaternionic Metric and Projection}

The fundamental geometry is encoded in the biquaternionic metric $\mathcal{G}_{\mu\nu}(\Theta)$ built from the field $\Theta$. All curvature tensors $\mathcal{R}_{\mu\nu}$ and connections are derived from $\mathcal{G}_{\mu\nu}$. The classical metric is a projection
\[
g_{\mu\nu} \;:=\; \Re\!\left(\mathcal{G}_{\mu\nu}\right),
\qquad
R_{\mu\nu} \;:=\; \Re\!\left(\mathcal{R}_{\mu\nu}\right),
\qquad
T_{\mu\nu} \;:=\; \Re\!\left(\mathcal{T}_{\mu\nu}\right),
\]
used only to describe the Hermitian (real) sector.

\section{Physical Content}

\subsection{General Relativity Recovery}

In the Hermitian (real) projection $\mathcal{G}_{\mu\nu}\mapsto g_{\mu\nu}$ and $\mathcal{R}_{\mu\nu}\mapsto R_{\mu\nu}$, UBT exactly reproduces Einstein's field equations:
\begin{equation}
R_{\mu\nu} - \frac{1}{2}g_{\mu\nu}R = 8\pi G T_{\mu\nu}
\end{equation}
This ensures full compatibility with all experimental confirmations of General Relativity, including perihelion precession, gravitational waves, and cosmological solutions.

\subsection{Why GR Misses Exotic Regimes}

Classical GR observes only $g_{\mu\nu}=\Re(\mathcal{G}_{\mu\nu})$, so any solution with $\Im(\mathcal{G}_{\mu\nu})\neq 0$ projects to the same real geometry and remains invisible. Antigravitational responses, metric cloaking, or temporal drift arise from these imaginary components and are suppressed in the projection. They are not forbidden by physics; they are hidden by the Hermitian reduction. UBT therefore extends GR without contradiction: the real sector matches Einsteinian dynamics, while the full biquaternionic geometry supports exotic but observationally hidden regimes.

\subsection{Standard Model Symmetries}

The gauge group $SU(3) \times SU(2) \times U(1)$ of the Standard Model emerges naturally from the automorphism group of the biquaternionic structure, providing a geometric origin for the strong, weak, and electromagnetic interactions.

\subsection{Dark Sector Physics}

The additional degrees of freedom in the biquaternionic field, beyond those required for reproducing General Relativity, may provide candidates for dark matter and dark energy phenomena through p-adic extensions of the theory.

\section{Acknowledgments}

The author acknowledges A.H. Chamseddine (2025) and E. Verlinde (2025).  
The present biquaternionic formulation extends these approaches through a unified complex-time and toroidal topology model.

\appendix

% Include appendices
% Note: This file requires \usepackage{booktabs} in the main document
\appendix
\section*{Appendix F — Hermitian Gravity Limit of UBT}
\addcontentsline{toc}{section}{Appendix F — Hermitian Gravity Limit of UBT}

\subsection*{F.1 Motivation}

As shown in A.H. Chamseddine, \textit{Gravity in Complex Hermitian Space}, Phys. Rev. D (2025) \cite{chamseddine2025hermitian},
the Hermitian metric corresponds to the limit of the biquaternionic field 
$\Theta(q,\tau)$ where imaginary components commute pairwise.  
In UBT, this corresponds to the constrained subspace:
\[
\mathrm{Im}(i)=\mathrm{Im}(j)=\mathrm{Im}(k),
\]
recovering Chamseddine's Hermitian geometry as a projection of the full $\mathbb{B}$-space.

\textbf{Disclaimer:} The Hermitian correspondence discussed here is mathematical and speculative. 
No physical or experimental realization of complex-metric gravity or FTL propagation 
has been demonstrated to date.

\bigskip

The Unified Biquaternion Theory (UBT) naturally extends the Hermitian and complex formulations of gravity by embedding them in a richer algebraic framework. 
A. H. Chamseddine's \textit{Hermitian Gravity} \cite{Chamseddine2005,Chamseddine2006,ChamseddineMukhanov2012,chamseddine2025hermitian}
is recovered as a limiting case of UBT when the biquaternionic field $\Theta$ is projected onto a complex subspace. 
This appendix clarifies the precise mathematical and physical correspondence between both frameworks.

UBT employs the \textbf{biquaternionic field}
\[
\Theta_\mu \in \mathbb{C}\otimes\mathbb{H},
\]
defined over the extended complex-time manifold $\tau = t + i\psi$. 
Its internal structure carries four complex degrees of freedom, naturally encompassing both metric and antisymmetric components.

Chamseddine's Hermitian gravity, by contrast, uses a \textbf{Hermitian metric tensor}
\[
H_{\mu\nu} = g_{\mu\nu} + i B_{\mu\nu},
\]
defined on a complex manifold with a holomorphic connection compatible with the $U(1,3)$ group.

\subsection*{F.2 Mapping between UBT and Hermitian Formulations}

The correspondence arises from the projection operator
\[
\Pi : \mathbb{C}\otimes\mathbb{H} \longrightarrow \mathbb{C},
\]
applied to the composite $\Theta$-field:
\[
H_{\mu\nu} = \Pi\!\left(\Theta_\mu^\dagger \Theta_\nu\right).
\]
Taking the real and imaginary parts yields the identification:
\[
g_{\mu\nu} = \Re(H_{\mu\nu}), \qquad B_{\mu\nu} = \Im(H_{\mu\nu}).
\]

\paragraph{Transition criterion.}
The projection $\Pi$ is \emph{admissible} if and only if the noncommutativity of the biquaternionic components is negligible:
\[
[\Theta_i, \Theta_j] \approx 0 \quad \Rightarrow \quad \text{complex (Hermitian) limit valid.}
\]
If this condition fails, the full biquaternionic description must be retained, as the complex projection would destroy essential nonlocal or topological information.

\subsection*{F.3 Comparison of Notations}

\begin{table}[h]
\centering
\begin{tabular}{lll}
\toprule
Concept & UBT (Biquaternionic) & Hermitian Gravity (Chamseddine) \\
\midrule
Fundamental field & $\Theta_\mu \in \mathbb{C}\otimes\mathbb{H}$ & $H_{\mu\nu}=g_{\mu\nu}+iB_{\mu\nu}$ \\
Metric sector & $\Re(\Theta^\dagger_\mu\Theta_\nu)$ & $g_{\mu\nu}$ \\
Antisymmetric sector & $\Im(\Theta^\dagger_\mu\Theta_\nu)$ & $B_{\mu\nu}$ \\
Connection & $\nabla_\mu\Theta_\nu$ (UBT covariant derivative) & Chern-type $U(1,3)$ connection \\
Local symmetry & $\mathrm{Aut}(\mathbb{C}\otimes\mathbb{H})$ & $U(1,3)$ \\
Time coordinate & $\tau = t + i\psi$ (biquaternionic time) & complex $z^\mu$ \\
\bottomrule
\end{tabular}
\caption{Correspondence between UBT and Hermitian formulations.}
\end{table}

\subsection*{F.4 Field Equations and Curvature}

In the Hermitian limit, the UBT covariant derivative reduces to the Chern-compatible connection:
\[
\nabla_\mu \Theta_\nu \; \longrightarrow \; \partial_\mu \Theta_\nu + \Gamma^\lambda_{\mu\nu}\Theta_\lambda,
\]
with curvature two-form
\[
F_{\mu\nu} = \partial_\mu A_\nu - \partial_\nu A_\mu + [A_\mu, A_\nu],
\]
where $A_\mu$ represents the effective $U(1,3)$ connection extracted from the projected $\Theta$-fields.

Variation of the UBT action in this limit reproduces the Hermitian gravity equations of Chamseddine up to higher-order biquaternionic corrections.

\subsection*{F.5 Fine-Structure Constant and Geometric Consistency}

The geometric fine-structure parameter $\alpha$ remains uniquely defined in UBT as
\[
\alpha^{-1} = 4\pi \oint_{\mathcal{C}} \! B \, d\ell,
\]
where the loop $\mathcal{C}$ encloses a closed $\Theta$-flux tube of radius $R_\Theta$. 
In the Hermitian limit, this expression maps to the invariant integral of the $B_{\mu\nu}$ field:
\[
\alpha^{-1} = \frac{1}{4\pi} \int B_{\mu\nu}\,dx^\mu\wedge dx^\nu.
\]
Hence $\alpha$ remains fixed by topology rather than a free scale parameter.

\subsection*{F.6 Speculative: FTL and Invisible Metric Domains}

The biquaternionic structure of UBT allows formal solutions in which the metric acquires an imaginary signature component in the $\psi$-direction of biquaternionic time. 
In these sectors, null geodesics may locally exceed $c$ or become ``invisible'' in the real-projected metric.

\begin{quote}
\textbf{Important notice:} 
These effects are purely theoretical and have not been observed in any experiment. 
No empirical evidence currently supports faster-than-light travel, metric cloaking, or hyperdimensional transport. 
The discussion here is speculative and provided only to illustrate how UBT geometrically permits such solutions without violating internal consistency.
\end{quote}

\subsection*{F.7 Summary}

The Hermitian Gravity limit demonstrates that Chamseddine's formulation is a special case of UBT where the noncommutative biquaternionic structure is suppressed. 
UBT therefore generalizes Hermitian gravity in the same way as Hermitian gravity generalizes General Relativity: 
it adds algebraic depth and potentially new topological degrees of freedom.

\bigskip
\noindent\textbf{References:} see file \texttt{references\_Hermitian.bib} for detailed bibliography.


% =====================================================================
% Appendix G: Emergent SU(3) Color Symmetry
% =====================================================================

\appendix
\section*{Appendix G — Emergent SU(3) Color Symmetry}
\addcontentsline{toc}{section}{Appendix G — Emergent SU(3) Color Symmetry}

\subsection*{G.1 Overview}

The Unified Biquaternion Theory (UBT) provides a natural geometric origin for the $\mathrm{SU}(3)$ color symmetry of quantum chromodynamics (QCD). Rather than postulating this gauge group as an external structure, UBT derives it from the internal phase structure of the biquaternionic field $\Theta(q,\tau)$ over biquaternionic time.

\subsection*{G.2 Quaternionic Phase Structure and Color Degrees of Freedom}

The biquaternionic field $\Theta \in \mathbb{C}\otimes\mathbb{H}$ possesses three independent imaginary quaternionic axes: $i$, $j$, and $k$. Each of these axes can carry an independent complex phase $\psi_n$ within the biquaternionic time structure $\tau = t + i\psi$.

When the field is promoted to include multi-dimensional phase degrees of freedom, these three independent phase directions naturally correspond to the three color charges of QCD. The mapping is given in the following table:

\begin{table}[h]
\centering
\caption{Mapping between quaternionic and SU(3) color degrees of freedom.}
\begin{tabular}{lll}
\hline
Quaternion Axis & Complex Phase & SU(3) Color Equivalent \\
\hline
$i$ & $\psi_1$ & Red \\
$j$ & $\psi_2$ & Green \\
$k$ & $\psi_3$ & Blue \\
\hline
\end{tabular}
\label{tab:quaternion_color_map}
\end{table}

\subsection*{G.3 Modular Realization via Theta Functions}

A concrete realization of this correspondence uses multi-variable theta functions:
\begin{equation}
\Theta(x,\tau) = \sum_{n\in \mathbb{Z}^3}\exp\Big(i\pi\, n^{\!\top}\,\Omega(x,\tau)\,n + 2\pi i\, n^{\!\top} z(x,\tau)\Big)\, \Xi(x,\tau),
\end{equation}
where $\Omega\in \mathrm{Mat}_{3\times 3}(\mathbb{C})$ is a symmetric period matrix with $\mathrm{Im}\,\Omega>0$, and $z\in \mathbb{C}^3$ is the internal phase coordinate corresponding to the three quaternionic axes.

The $\mathrm{SU}(3)$ color symmetry emerges as the traceless unitary automorphism group of this three-dimensional internal phase space:
\begin{equation}
\mathrm{SU}(3)_{\text{color}} = \{U\in \mathrm{U}(3)\,|\,\det U=1\}.
\end{equation}

\subsection*{G.4 Color Connection and Field Strength}

The color gauge connection arises naturally from the phase geometry:
\begin{equation}
A_\mu = \mathcal{U}^\dagger \partial_\mu \mathcal{U} - \tfrac{1}{3}\mathrm{tr}(\mathcal{U}^\dagger \partial_\mu \mathcal{U})\,\mathbf{1}_3 \in \mathfrak{su}(3),
\end{equation}
where $\mathcal{U}(x,\tau) \in \mathrm{U}(3)$ is the unitary phase frame on the internal fiber.

The field strength is the standard Yang-Mills curvature:
\begin{equation}
F_{\mu\nu} = \partial_\mu A_\nu - \partial_\nu A_\mu + [A_\mu, A_\nu] \in \mathfrak{su}(3).
\end{equation}

\subsection*{G.5 Compatibility with General Relativity}

Because the color transformations act on the internal phase fiber and $\mathcal{U}$ is unitary, the metric tensor remains invariant:
\begin{equation}
g_{\mu\nu} = \mathrm{Re}(\Theta^\dagger \Theta)
\end{equation}
is unchanged under color rotations. Therefore, the Einstein field equations and all tests of General Relativity remain unaffected by the color sector.

\subsection*{G.6 Summary}

The emergent SU(3) symmetry arises as a phase-locking pattern of the imaginary quaternionic subspace, corresponding to color confinement in QCD. The three quaternionic axes $i$, $j$, $k$ provide the three color degrees of freedom (red, green, blue), while the traceless condition (removal of the overall U(1) trace) reduces the nine U(3) generators to the eight gluons of $\mathrm{SU}(3)$.

This derivation demonstrates that color symmetry is not an arbitrary external gauge structure but rather an intrinsic consequence of the biquaternionic geometry of spacetime in UBT.


% =====================================================================
% Appendix H: Theta Phase Emergence
% =====================================================================

\appendix
\section*{Appendix H — Theta Phase Emergence}
\addcontentsline{toc}{section}{Appendix H — Theta Phase Emergence}

\subsection*{H.1 Overview}

This appendix describes the dynamical emergence of the phase field $\psi$ within the complex time structure $\tau = t + i\psi$ of the Unified Biquaternion Theory (UBT). The phase dynamics are governed by a drift-diffusion equation whose steady-state solutions naturally correspond to Jacobi theta functions, providing a self-consistent foundation for the theory.

\subsection*{H.2 Drift-Diffusion Dynamics of the Phase Field}

The phase field $\psi$ follows the drift–diffusion law:
\begin{equation}
\label{eq:drift_diffusion}
\frac{\partial \psi}{\partial t} = D \nabla^2 \psi - \alpha \frac{\partial V}{\partial \psi},
\end{equation}
where:
\begin{itemize}
\item $D$ is the diffusion coefficient in phase space
\item $\alpha$ is the drift coefficient coupling the phase to potential energy
\item $V(\psi)$ is the effective potential governing phase dynamics
\end{itemize}

This equation describes how the imaginary component of complex time evolves under the competing influences of diffusion (spreading) and drift (potential-driven flow).

\subsection*{H.3 Theta Function Attractor}

The steady-state solutions of equation~\eqref{eq:drift_diffusion} correspond to configurations where:
\begin{equation}
D \nabla^2 \psi = \alpha \frac{\partial V}{\partial \psi}.
\end{equation}

For appropriate choices of the potential $V(\psi)$ consistent with the toroidal topology of the phase space, these steady-state solutions are precisely the Jacobi theta functions:
\begin{equation}
\Theta_n(z,\tau) = \sum_{k=-\infty}^{\infty} \exp\left(\pi i k^2 \tau + 2\pi i k z\right),
\end{equation}
where $\tau$ plays the dual role of complex time parameter and modular parameter for the theta function.

\subsection*{H.4 Physical Interpretation}

The drift-diffusion equation provides a dynamical mechanism for the emergence of the biquaternionic field structure:

\begin{enumerate}
\item \textbf{Diffusion term} ($D \nabla^2 \psi$): Represents quantum uncertainty and phase spreading in the complex time direction.

\item \textbf{Drift term} ($-\alpha \frac{\partial V}{\partial \psi}$): Represents the gradient flow driven by the effective potential, which encodes the geometric and topological constraints of the theory.

\item \textbf{Theta function solutions}: These special functions naturally encode:
\begin{itemize}
\item Modular symmetry (related to gauge transformations)
\item Toroidal topology (compact phase space)
\item Quantization conditions (integer winding numbers)
\item Holomorphic structure (compatibility with complex analysis)
\end{itemize}
\end{enumerate}

\subsection*{H.5 Connection to Field Dynamics}

The phase field $\psi$ couples to the full biquaternionic field $\Theta(q,\tau)$ through the complex time structure. The drift-diffusion equation ensures that the phase evolves toward configurations that minimize the combined energy functional:
\begin{equation}
E[\psi] = \int \left[\frac{D}{2}|\nabla \psi|^2 + V(\psi)\right] d^4x,
\end{equation}
subject to the constraints imposed by the toroidal topology of the internal phase manifold.

\subsection*{H.6 Relation to Gauge Fields}

The theta function structure of the phase field has profound implications for gauge field emergence:

\begin{itemize}
\item The modular transformations of theta functions correspond to gauge transformations
\item The period matrix $\Omega$ of the theta function relates to the gauge field strength
\item Different theta characteristics correspond to different topological sectors
\end{itemize}

This provides a unified origin for both the geometric structure (through the metric derived from $\Theta^\dagger\Theta$) and the gauge structure (through the modular properties of the phase field).

\subsection*{H.7 Stability and Attractors}

The drift-diffusion equation exhibits attractor dynamics: generic initial phase configurations $\psi(x,t=0)$ evolve toward theta function solutions under the flow. This provides dynamical stability for the theory—perturbations away from theta function configurations are damped by the combined action of diffusion and potential-driven drift.

The characteristic relaxation time scale is:
\begin{equation}
\tau_{\text{relax}} \sim \frac{L^2}{D},
\end{equation}
where $L$ is the characteristic spatial scale of phase variations.

\subsection*{H.8 Summary}

The drift-diffusion equation~\eqref{eq:drift_diffusion} provides a dynamical foundation for the emergence of theta function structure in UBT. The phase field $\psi$ naturally evolves toward Jacobi theta function configurations, which encode the modular symmetry, toroidal topology, and quantization conditions essential to the theory. This establishes UBT not merely as a kinematic framework but as a dynamical theory with self-consistent phase evolution.



\bibliographystyle{plain}
\bibliography{references}

\section*{License}
© 2025 Ing. David Jaroš — CC BY-NC-ND 4.0

This work is licensed under a Creative Commons Attribution-NonCommercial-NoDerivatives 4.0 International License (CC BY-NC-ND 4.0).

\textbf{License History:} Earlier drafts (up to v0.3) were released under CC BY 4.0. From v0.4 onward, all material is released under CC BY-NC-ND 4.0 to protect the integrity of the theoretical work during ongoing academic development.

\end{document}
