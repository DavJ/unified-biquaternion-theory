% Note: This file requires \usepackage{booktabs} in the main document
\appendix
\section*{Appendix F — Hermitian Gravity Limit of UBT}
\addcontentsline{toc}{section}{Appendix F — Hermitian Gravity Limit of UBT}

\subsection*{F.1 Motivation}

As shown in A.H. Chamseddine, \textit{Gravity in Complex Hermitian Space}, Phys. Rev. D (2025) \cite{chamseddine2025hermitian},
the Hermitian metric corresponds to the limit of the biquaternionic field 
$\Theta(q,\tau)$ where imaginary components commute pairwise.  
In UBT, this corresponds to the constrained subspace:
\[
\mathrm{Im}(i)=\mathrm{Im}(j)=\mathrm{Im}(k),
\]
recovering Chamseddine's Hermitian geometry as a projection of the full $\mathbb{B}$-space.

\textbf{Disclaimer:} The Hermitian correspondence discussed here is mathematical and speculative. 
No physical or experimental realization of complex-metric gravity or FTL propagation 
has been demonstrated to date.

\bigskip

The Unified Biquaternion Theory (UBT) naturally extends the Hermitian and complex formulations of gravity by embedding them in a richer algebraic framework. 
A. H. Chamseddine's \textit{Hermitian Gravity} \cite{Chamseddine2005,Chamseddine2006,ChamseddineMukhanov2012,chamseddine2025hermitian}
is recovered as a limiting case of UBT when the biquaternionic field $\Theta$ is projected onto a complex subspace. 
This appendix clarifies the precise mathematical and physical correspondence between both frameworks.

UBT employs the \textbf{biquaternionic field}
\[
\Theta_\mu \in \mathbb{C}\otimes\mathbb{H},
\]
defined over the extended complex-time manifold $\tau = t + i\psi$. 
Its internal structure carries four complex degrees of freedom, naturally encompassing both metric and antisymmetric components.

Chamseddine's Hermitian gravity, by contrast, uses a \textbf{Hermitian metric tensor}
\[
H_{\mu\nu} = g_{\mu\nu} + i B_{\mu\nu},
\]
defined on a complex manifold with a holomorphic connection compatible with the $U(1,3)$ group.

\subsection*{F.2 Mapping between UBT and Hermitian Formulations}

The correspondence arises from the projection operator
\[
\Pi : \mathbb{C}\otimes\mathbb{H} \longrightarrow \mathbb{C},
\]
applied to the composite $\Theta$-field:
\[
H_{\mu\nu} = \Pi\!\left(\Theta_\mu^\dagger \Theta_\nu\right).
\]
Taking the real and imaginary parts yields the identification:
\[
g_{\mu\nu} = \Re(H_{\mu\nu}), \qquad B_{\mu\nu} = \Im(H_{\mu\nu}).
\]

\paragraph{Transition criterion.}
The projection $\Pi$ is \emph{admissible} if and only if the noncommutativity of the biquaternionic components is negligible:
\[
[\Theta_i, \Theta_j] \approx 0 \quad \Rightarrow \quad \text{complex (Hermitian) limit valid.}
\]
If this condition fails, the full biquaternionic description must be retained, as the complex projection would destroy essential nonlocal or topological information.

\subsection*{F.3 Comparison of Notations}

\begin{table}[h]
\centering
\begin{tabular}{lll}
\toprule
Concept & UBT (Biquaternionic) & Hermitian Gravity (Chamseddine) \\
\midrule
Fundamental field & $\Theta_\mu \in \mathbb{C}\otimes\mathbb{H}$ & $H_{\mu\nu}=g_{\mu\nu}+iB_{\mu\nu}$ \\
Metric sector & $\Re(\Theta^\dagger_\mu\Theta_\nu)$ & $g_{\mu\nu}$ \\
Antisymmetric sector & $\Im(\Theta^\dagger_\mu\Theta_\nu)$ & $B_{\mu\nu}$ \\
Connection & $\nabla_\mu\Theta_\nu$ (UBT covariant derivative) & Chern-type $U(1,3)$ connection \\
Local symmetry & $\mathrm{Aut}(\mathbb{C}\otimes\mathbb{H})$ & $U(1,3)$ \\
Time coordinate & $\tau = t + i\psi$ (biquaternionic time) & complex $z^\mu$ \\
\bottomrule
\end{tabular}
\caption{Correspondence between UBT and Hermitian formulations.}
\end{table}

\subsection*{F.4 Field Equations and Curvature}

In the Hermitian limit, the UBT covariant derivative reduces to the Chern-compatible connection:
\[
\nabla_\mu \Theta_\nu \; \longrightarrow \; \partial_\mu \Theta_\nu + \Gamma^\lambda_{\mu\nu}\Theta_\lambda,
\]
with curvature two-form
\[
F_{\mu\nu} = \partial_\mu A_\nu - \partial_\nu A_\mu + [A_\mu, A_\nu],
\]
where $A_\mu$ represents the effective $U(1,3)$ connection extracted from the projected $\Theta$-fields.

Variation of the UBT action in this limit reproduces the Hermitian gravity equations of Chamseddine up to higher-order biquaternionic corrections.

\subsection*{F.5 Fine-Structure Constant and Geometric Consistency}

The geometric fine-structure parameter $\alpha$ remains uniquely defined in UBT as
\[
\alpha^{-1} = 4\pi \oint_{\mathcal{C}} \! B \, d\ell,
\]
where the loop $\mathcal{C}$ encloses a closed $\Theta$-flux tube of radius $R_\Theta$. 
In the Hermitian limit, this expression maps to the invariant integral of the $B_{\mu\nu}$ field:
\[
\alpha^{-1} = \frac{1}{4\pi} \int B_{\mu\nu}\,dx^\mu\wedge dx^\nu.
\]
Hence $\alpha$ remains fixed by topology rather than a free scale parameter.

\subsection*{F.6 Speculative: FTL and Invisible Metric Domains}

The biquaternionic structure of UBT allows formal solutions in which the metric acquires an imaginary signature component in the $\psi$-direction of biquaternionic time. 
In these sectors, null geodesics may locally exceed $c$ or become ``invisible'' in the real-projected metric.

\begin{quote}
\textbf{Important notice:} 
These effects are purely theoretical and have not been observed in any experiment. 
No empirical evidence currently supports faster-than-light travel, metric cloaking, or hyperdimensional transport. 
The discussion here is speculative and provided only to illustrate how UBT geometrically permits such solutions without violating internal consistency.
\end{quote}

\subsection*{F.7 Summary}

The Hermitian Gravity limit demonstrates that Chamseddine's formulation is a special case of UBT where the noncommutative biquaternionic structure is suppressed. 
UBT therefore generalizes Hermitian gravity in the same way as Hermitian gravity generalizes General Relativity: 
it adds algebraic depth and potentially new topological degrees of freedom.

\bigskip
\noindent\textbf{References:} see file \texttt{references\_Hermitian.bib} for detailed bibliography.

