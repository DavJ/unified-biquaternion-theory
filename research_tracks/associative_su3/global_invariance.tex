% © 2025 Ing. David Jaroš — CC BY-NC-ND 4.0
%
% This work is licensed under a Creative Commons Attribution-NonCommercial-NoDerivatives
% 4.0 International License (CC BY-NC-ND 4.0).
%
% Track A Dynamic Test: Global SU(3) Invariance of Θ-Triplet Kinetics
% Status: Proved for constant U; local extension conditional.
% Version: 1.0
% Date: 2026-03-01

\section{Global $\mathrm{SU}(3)$ Invariance of the $\Theta$-Triplet Dynamics}
\label{sec:global_invariance}

\subsection{Setup: Triplet Lift of $\Theta$}

Let $V_c = \mathrm{span}_\mathbb{C}\{\mathbf{I},\mathbf{J},\mathbf{K}\}\subset
\mathcal{B}$ be the canonical carrier space defined in
\S\ref{subsec:inv:carrier_space}.  A \emph{$\Theta$-triplet} is a
$V_c$-valued field on spacetime $M^4$:
\begin{equation}
  \label{eq:Theta_triplet}
  \boldsymbol{\Theta}(x) = \Theta^a(x)\,\mathbf{e}_a, \quad a = 1,2,3,
\end{equation}
where $\{\mathbf{e}_1,\mathbf{e}_2,\mathbf{e}_3\}=\{\mathbf{I},\mathbf{J},\mathbf{K}\}$
is the $\mathbb{C}$-basis of $V_c$ and $\Theta^a(x)\in\mathbb{C}$.  Equivalently,
$\boldsymbol{\Theta}$ is a section of the trivial bundle $M^4\times\mathbb{C}^3$.

\subsection{Kinetic Operator for the Triplet}

On flat Minkowski spacetime $(M^4,\eta_{\mu\nu})$, define the kinetic operator
\begin{equation}
  \label{eq:kinetic_operator}
  \mathcal{K}[\boldsymbol{\Theta}]
  \;:=\;
  \eta^{\mu\nu}\partial_\mu\partial_\nu\boldsymbol{\Theta}
  \;=\;
  \Box\boldsymbol{\Theta}
  \;=\;
  (\Box\Theta^a)\,\mathbf{e}_a,
\end{equation}
where $\Box = \partial_t^2 - \nabla^2$ is the d'Alembertian acting
component-wise on each $\Theta^a\in\mathbb{C}$.  The action functional is
\begin{equation}
  \label{eq:triplet_action}
  S[\boldsymbol{\Theta}]
  = \int_{M^4} d^4x\;
    \langle\partial_\mu\boldsymbol{\Theta},\,\partial^\mu\boldsymbol{\Theta}\rangle,
\end{equation}
where $\langle\cdot,\cdot\rangle$ is the Hermitian form on $V_c\cong\mathbb{C}^3$
from Definition~\ref{def:inv:Vc}:
\begin{equation}
  \langle\boldsymbol{\Theta},\boldsymbol{\Phi}\rangle
  = \sum_{a=1}^3 \overline{\Theta^a}\,\Phi^a.
\end{equation}

\subsection{Global $\mathrm{SU}(3)$ Invariance}

\begin{theorem}[Global $\mathrm{SU}(3)$ invariance]
  \label{thm:global_su3}
  The action \eqref{eq:triplet_action} and kinetic operator
  \eqref{eq:kinetic_operator} are invariant under constant
  $U\in\mathrm{SU}(3)_{V_c}$:
  \begin{equation}
    \boldsymbol{\Theta}(x) \;\mapsto\; U\boldsymbol{\Theta}(x)
    \qquad\text{(U constant, } U^\dagger U = \mathbb{1},\; \det U = 1).
  \end{equation}
\end{theorem}

\begin{proof}
  Since $U$ is constant (independent of $x$), $\partial_\mu(U\boldsymbol{\Theta})
  = U(\partial_\mu\boldsymbol{\Theta})$.  Then
  \begin{align}
    \langle\partial_\mu(U\boldsymbol{\Theta}),\,\partial^\mu(U\boldsymbol{\Theta})\rangle
    &= \langle U\partial_\mu\boldsymbol{\Theta},\,U\partial^\mu\boldsymbol{\Theta}\rangle
    \nonumber\\
    &= \langle\partial_\mu\boldsymbol{\Theta},\,\partial^\mu\boldsymbol{\Theta}\rangle,
  \end{align}
  where the last equality uses unitarity $\langle Ux,Uy\rangle=\langle x,y\rangle$
  (Theorem~\ref{thm:su3_action}).  Integrating over $M^4$,
  $S[U\boldsymbol{\Theta}]=S[\boldsymbol{\Theta}]$.  The equation of motion
  $\Box\boldsymbol{\Theta}=0$ transforms as $U\Box\boldsymbol{\Theta}=0$, which
  holds iff $\Box\boldsymbol{\Theta}=0$ (since $U$ is invertible).
\end{proof}

\begin{remark}[Scope]
  Theorem~\ref{thm:global_su3} establishes invariance for \emph{constant}
  $U\in\mathrm{SU}(3)$.  The extension to local (spacetime-dependent) gauge
  transformations $U(x)\in\mathrm{SU}(3)$ requires the introduction of a
  gauge connection and is addressed separately (conditionally) in
  \S\ref{sec:local_gauge} (if proved) or stated as an open problem.
\end{remark}

\subsection{Symmetry Under Curved Background}

On a general curved background $(M^4,g_{\mu\nu})$, the kinetic operator
becomes the covariant d'Alembertian
\begin{equation}
  \Box_g\boldsymbol{\Theta} = g^{\mu\nu}\nabla_\mu\nabla_\nu\boldsymbol{\Theta},
\end{equation}
where $\nabla_\mu$ involves only the Levi-Civita connection of $g_{\mu\nu}$
(acting trivially on the internal $V_c$ indices for constant $U$).

\begin{corollary}[Background independence of global $\mathrm{SU}(3)$]
  \label{cor:curved_invariance}
  Global $\mathrm{SU}(3)_{V_c}$ invariance holds on any Riemannian or
  Lorentzian background $(M^4,g_{\mu\nu})$, since constant $U$ commutes
  with any background covariant derivative.
\end{corollary}

\begin{proof}
  $\nabla_\mu(U\boldsymbol{\Theta}) = U\nabla_\mu\boldsymbol{\Theta}$ for
  constant $U$; the argument of Theorem~\ref{thm:global_su3} applies verbatim.
\end{proof}

\subsection{Summary}

\begin{center}
\begin{tabular}{lll}
  \hline
  Statement & Status & Reference \\
  \hline
  $\mathrm{SU}(3)_{V_c}$ exists as symmetry of $(V_c,\langle\cdot,\cdot\rangle)$ & Proved & Thm.~\ref{thm:su3_action} \\
  Global SU(3) invariance (flat background) & \textbf{Proved} & Thm.~\ref{thm:global_su3} \\
  Global SU(3) invariance (curved background) & Proved & Cor.~\ref{cor:curved_invariance} \\
  Local SU(3) gauge invariance & Conditional & \S\ref{sec:local_gauge} \\
  \hline
\end{tabular}
\end{center}

% End of global invariance section
