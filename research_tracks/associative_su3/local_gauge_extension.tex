% © 2025 Ing. David Jaroš — CC BY-NC-ND 4.0
%
% This work is licensed under a Creative Commons Attribution-NonCommercial-NoDerivatives
% 4.0 International License (CC BY-NC-ND 4.0).
%
% Track A Dynamic Test: Local SU(3) Gauge Extension (Candidate)
% Condition: only_if_global_invariance_proven
% Status: CANDIDATE EXTENSION — clearly labelled as such.
% Version: 1.0
% Date: 2026-03-01

\section{Local $\mathrm{SU}(3)$ Gauge Extension (Candidate)}
\label{sec:local_gauge}

\begin{tcolorbox}[colback=yellow!5!white,colframe=orange!80!black,
  title=Status: Candidate Extension]
  This section is a \textbf{candidate construction} conditional on
  Theorem~\ref{thm:global_su3} (global $\mathrm{SU}(3)$ invariance,
  proved in \S\ref{sec:global_invariance}).  The local extension is a
  standard gauge-theoretic step; no new claims about physical interpretation
  are made.  All results are labelled explicitly.
\end{tcolorbox}

\subsection{Motivation}

Theorem~\ref{thm:global_su3} establishes invariance of the $\Theta$-triplet
action under \emph{constant} $U\in\mathrm{SU}(3)_{V_c}$.  A natural candidate
extension is to allow $U=U(x)$ to be spacetime-dependent.  This requires
introducing a gauge connection.

\subsection{Gauge Field Introduction}

\begin{definition}[Gauge connection — candidate]
  \label{def:gauge_field}
  Introduce an $\mathfrak{su}(3)$-valued gauge field
  \begin{equation}
    A_\mu(x) = A_\mu^A(x)\,T^A,
    \qquad A_\mu^A\in\mathbb{R},\quad T^A\text{ Gell-Mann generators},
  \end{equation}
  transforming under a local $U(x)\in\mathrm{SU}(3)$ as
  \begin{equation}
    \label{eq:gauge_transform_A}
    A_\mu \;\mapsto\; U A_\mu U^\dagger - \tfrac{i}{g}(\partial_\mu U)\,U^\dagger,
  \end{equation}
  where $g$ is a coupling constant.
\end{definition}

\subsection{Covariant Derivative}

\begin{definition}[Covariant derivative — candidate]
  \label{def:covariant_derivative}
  Define the covariant derivative of the triplet field by
  \begin{equation}
    \label{eq:covariant_derivative}
    D_\mu\boldsymbol{\Theta}
    \;:=\;
    \partial_\mu\boldsymbol{\Theta} - ig\,A_\mu\boldsymbol{\Theta}
    \;=\;
    \partial_\mu\boldsymbol{\Theta} - ig\,A_\mu^A T^A\boldsymbol{\Theta}.
  \end{equation}
\end{definition}

\begin{proposition}[Covariance — candidate]
  \label{prop:covariance}
  Under the local transformation
  $\boldsymbol{\Theta}(x)\mapsto U(x)\boldsymbol{\Theta}(x)$ combined with
  \eqref{eq:gauge_transform_A}, the covariant derivative transforms
  covariantly:
  \begin{equation}
    D_\mu\boldsymbol{\Theta} \;\mapsto\; U(x)\,D_\mu\boldsymbol{\Theta}.
  \end{equation}
\end{proposition}

\begin{proof}
  Standard gauge-theoretic calculation:
  \begin{align}
    D_\mu(U\boldsymbol{\Theta})
    &= \partial_\mu(U\boldsymbol{\Theta}) - igA_\mu'(U\boldsymbol{\Theta})
    \nonumber\\
    &= (\partial_\mu U)\boldsymbol{\Theta} + U(\partial_\mu\boldsymbol{\Theta})
      - ig\Bigl(UA_\mu U^\dagger - \tfrac{i}{g}(\partial_\mu U)U^\dagger\Bigr)
      U\boldsymbol{\Theta}
    \nonumber\\
    &= (\partial_\mu U)\boldsymbol{\Theta} + U(\partial_\mu\boldsymbol{\Theta})
      - igUA_\mu\boldsymbol{\Theta}
      - (\partial_\mu U)\boldsymbol{\Theta}
    \nonumber\\
    &= U\bigl(\partial_\mu\boldsymbol{\Theta} - igA_\mu\boldsymbol{\Theta}\bigr)
    \;=\; U(D_\mu\boldsymbol{\Theta}).
  \end{align}
\end{proof}

\subsection{Gauge-Invariant Action (Candidate)}

\begin{definition}[Locally gauge-invariant action — candidate]
  \label{def:gauged_action}
  Replace $\partial_\mu$ with $D_\mu$ in the triplet action:
  \begin{equation}
    \label{eq:gauged_action}
    S_\mathrm{gauged}[\boldsymbol{\Theta},A]
    = \int d^4x\;
    \langle D_\mu\boldsymbol{\Theta},\,D^\mu\boldsymbol{\Theta}\rangle
    - \tfrac{1}{2g^2}\int d^4x\;\mathrm{tr}(F_{\mu\nu}F^{\mu\nu}),
  \end{equation}
  where
  $F_{\mu\nu} = \partial_\mu A_\nu - \partial_\nu A_\mu
  - ig[A_\mu,A_\nu]$
  is the field strength tensor.
\end{definition}

\begin{remark}[Status and caveats]
  \label{rem:local_gauge_status}
  \begin{enumerate}
    \item This is a \textbf{candidate extension}, not an independently derived
      consequence of the biquaternion algebra.  It uses the standard
      Yang-Mills construction applied to the $\mathrm{SU}(3)_{V_c}$ symmetry
      group.
    \item The coupling constant $g$ is a free parameter; no prediction of its
      value is made here.
    \item Whether $A_\mu$ has any interpretation in terms of the biquaternion
      geometry (e.g.\ as a component of the biquaternionic connection
      $\Omega_\mu$) is an open problem.
    \item This section does not establish any connection to QCD; the physical
      interpretation of $A_\mu$ and $F_{\mu\nu}$ is left open.
  \end{enumerate}
\end{remark}

\subsection{Equations of Motion (Candidate)}

Varying \eqref{eq:gauged_action} with respect to $\bar{\boldsymbol{\Theta}}$:
\begin{equation}
  D^\mu D_\mu\boldsymbol{\Theta} = 0.
\end{equation}
Varying with respect to $A_\mu^A$:
\begin{equation}
  D_\nu F^{\nu\mu\,A} = g\,J^{\mu\,A},
  \qquad
  J^{\mu\,A}
  = i\!\left(
      \bar\Theta^a(T^A)^a{}_b D^\mu\Theta^b
      - \overline{D^\mu\Theta^a}(T^A)^a{}_b\Theta^b
    \right),
\end{equation}
which is the standard Yang-Mills equation sourced by the triplet current.

\subsection{Summary}

\begin{center}
\begin{tabular}{lll}
  \hline
  Statement & Status & Notes \\
  \hline
  Global SU(3) invariance established & Proved & \S\ref{sec:global_invariance} \\
  Gauge field $A_\mu$ introduced & Candidate & Def.~\ref{def:gauge_field} \\
  Covariant derivative $D_\mu\boldsymbol{\Theta}$ defined & Candidate & Def.~\ref{def:covariant_derivative} \\
  Local covariance under $U(x)$ & Proved (conditional) & Prop.~\ref{prop:covariance} \\
  Gauge-invariant action & Candidate & Def.~\ref{def:gauged_action} \\
  Physical interpretation of $A_\mu$ & Open & Rem.~\ref{rem:local_gauge_status} \\
  Connection to QCD & Not claimed & Rem.~\ref{rem:local_gauge_status} \\
  \hline
\end{tabular}
\end{center}

% End of local gauge extension section
