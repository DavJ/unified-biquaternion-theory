% UBT Layer 2 Channel Selection Paper Draft
% This is a skeleton for documenting the channel stability lab results

\documentclass[11pt,a4paper]{article}

\usepackage{amsmath,amssymb,amsfonts}
\usepackage{graphicx}
\usepackage{hyperref}
\usepackage{booktabs}
\usepackage{float}
\usepackage{geometry}
\geometry{margin=1in}

\title{Objective Evaluation of UBT Layer 2 Channel Selection Principles:\\
A Statistical Approach to Structural Stability}

\author{UBT Research Team}
\date{\today}

\begin{document}

\maketitle

\begin{abstract}
We present a rigorous statistical framework for evaluating channel stability in the Unified Biquaternion Theory (UBT) Layer 2 formulation. Using four independent stability criteria---spectral robustness (S1), twin prime coherence (S2), energy minimization (S3), and information optimality (S4)---we scan integer-labeled channels in the range $k \in [100, 200]$ to identify structurally stable modes. Bootstrap null hypothesis testing with look-elsewhere correction is applied to assess statistical significance. This blind protocol approach ensures objective evaluation without confirmation bias. Results are presented with full transparency, including null findings.

\textbf{Keywords:} Unified Biquaternion Theory, Channel Selection, Statistical Physics, Prime Numbers, Null Hypothesis Testing
\end{abstract}

\tableofcontents

\section{Introduction}

\subsection{Motivation}

The Unified Biquaternion Theory (UBT) proposes that physical phenomena emerge from a biquaternionic field $\Theta(q, \tau)$ defined over complex time $\tau = t + i\psi$. Layer 2 of UBT introduces discrete channel modes labeled by integers $k$, where certain channels may exhibit enhanced structural stability.

\subsection{Research Question}

\textbf{Central Question:} Do prime numbers, particularly twin prime pairs, naturally emerge as stable channel labels in the UBT Layer 2 framework?

\subsection{Hypothesis}

We test the following falsifiable hypotheses:
\begin{enumerate}
    \item Prime channels exhibit higher spectral robustness than composite channels
    \item Twin prime pairs (e.g., 137-139) show enhanced phase coherence
    \item Stable channels correspond to local energy minima
    \item Prime channels maximize information-theoretic stability
\end{enumerate}

\subsection{Scope}

This paper focuses exclusively on \emph{objective evaluation} of stability criteria. We do not presuppose that any particular channel (including 137) is special. Results are interpreted using pre-registered statistical protocols.

\section{UBT Layer 2: Theoretical Background}

\subsection{Biquaternionic Field Equation}

The fundamental field equation is:
\begin{equation}
\nabla^\dagger \nabla \Theta(q, \tau) = \kappa \mathcal{T}(q, \tau)
\end{equation}
where $\nabla^\dagger$ is the biquaternionic conjugate gradient.

\subsection{Channel Discretization}

Under periodic boundary conditions, the field admits discrete mode solutions:
\begin{equation}
\Theta_k(q, \tau) = \Theta_0 \exp\left(2\pi i k / N\right)
\end{equation}

\subsection{Stability Principle}

Channels are hypothesized to be stable if they satisfy multiple independent criteria (detailed in Section~\ref{sec:criteria}).

\section{Stability Criteria Definitions}
\label{sec:criteria}

\subsection{S1: Spectral Robustness}

\textbf{Definition:}
\begin{equation}
S1(k) = \text{peak\_strength}(k) - \text{local\_noise}(k)
\end{equation}

\textbf{Physical Interpretation:} Measures how strongly a channel resonates compared to its local spectral background.

\textbf{Implementation:} [Details of peak detection and noise window]

\subsection{S2: Twin Prime Coherence}

\textbf{Definition:}
\begin{equation}
S2(k_1, k_2) = S1(k_1) + S1(k_2) + \langle \Theta_{k_1} | \Theta_{k_2} \rangle
\end{equation}

\textbf{Physical Interpretation:} Twin primes may exhibit quantum-like coherence in phase space.

\textbf{Implementation:} [Details of inner product calculation]

\subsection{S3: Energy Criterion}

\textbf{Definition:}
\begin{equation}
E(k) = \int |\nabla \Theta_k|^2 dV
\end{equation}

\textbf{Physical Interpretation:} Local energy minima resist perturbations.

\textbf{Implementation:} [Details of discrete Laplacian approximation]

\subsection{S4: Information Criterion}

\textbf{Definition:}
\begin{equation}
I(k) = H(k) - H(k|\text{context}) = \text{MI}(\Theta_k, \Theta_{\text{neighbors}})
\end{equation}

\textbf{Physical Interpretation:} Measures unique information content.

\textbf{Implementation:} [Details of entropy estimation]

\section{Experimental Setup}

\subsection{Scan Parameters}

\begin{itemize}
    \item \textbf{Range:} $k \in [100, 200]$
    \item \textbf{Resolution:} Integer steps ($\Delta k = 1$)
    \item \textbf{Criteria:} S1, S2, S3, S4 (equal weights)
\end{itemize}

\subsection{Data Source}

% [INSERT: Whether synthetic or observational data was used]

\subsection{Blind Protocol}

To prevent confirmation bias:
\begin{enumerate}
    \item Scan range pre-registered before analysis
    \item Metric definitions locked in configuration file
    \item No manual parameter tuning after seeing results
    \item All channels reported (not cherry-picked)
\end{enumerate}

\section{Statistical Controls}

\subsection{Bootstrap Null Hypothesis Testing}

\textbf{Method:} Monte Carlo permutation with $N_{\text{bootstrap}} = 10{,}000$ iterations.

\textbf{Null Hypothesis:} Channel stability values are indistinguishable from random shuffling.

\subsection{Look-Elsewhere Correction}

Bonferroni correction applied:
\begin{equation}
p_{\text{corrected}} = \min(N_{\text{tests}} \cdot p_{\text{raw}}, 1)
\end{equation}

\subsection{Significance Threshold}

$\alpha = 0.05$ after correction.

\section{Results}

% [INSERT: Tables and figures from analysis/results_summary.py]

\subsection{Overall Ranking}

% Table: Top 10 channels by combined score
\begin{table}[H]
\centering
\caption{Top 10 Stable Channels (Combined Score)}
\begin{tabular}{@{}cccccc@{}}
\toprule
Rank & $k$ & Combined Score & Is Prime & Percentile (\%) & $p_{\text{corrected}}$ \\ \midrule
1    &     &                &          &                 &                        \\
2    &     &                &          &                 &                        \\
...  & ... & ...            & ...      & ...             & ...                    \\ \bottomrule
\end{tabular}
\label{tab:top10}
\end{table}

\subsection{S1: Spectral Robustness}

% Figure: S1 heatmap
% Table: Top S1 channels

\subsection{S2: Twin Prime Coherence}

% Figure: S2 heatmap matrix
% Table: Top twin prime pairs

\subsection{S3: Energy Criterion}

% Figure: Energy distribution
% Table: Local energy minima

\subsection{S4: Information Criterion}

% Figure: Information distribution
% Table: Top information channels

\subsection{Channel 137 Analysis}

% Dedicated subsection for 137 (if in range)
% Report rank, percentile, p-values objectively

\subsection{Twin Pair (137, 139) Analysis}

% Report S2 performance if in range

\subsection{Prime vs Non-Prime Comparison}

% Statistical comparison: means, std, t-test

\section{Discussion}

\subsection{Interpretation of Results}

% [To be filled after results are available]

\subsection{Consistency with UBT Theory}

% [Assess whether results support or challenge Layer 2 hypothesis]

\subsection{Statistical Robustness}

% [Discuss p-values, look-elsewhere effects]

\subsection{Limitations}

\begin{itemize}
    \item Synthetic vs observational data
    \item Finite scan range
    \item Simplified field representations
    \item Model assumptions
\end{itemize}

\subsection{Implications for Channel 137}

% [Objective assessment: does 137 stand out or not?]

\section{Conclusion}

% [Summary of findings]
% [Whether hypotheses are supported, rejected, or inconclusive]
% [Next steps for research]

\section{Future Work}

\begin{itemize}
    \item Expand scan range to $k \in [50, 500]$
    \item Apply to observational datasets (CMB, etc.)
    \item Refine stability criteria based on full biquaternionic field
    \item Test with alternative null models
\end{itemize}

\section*{Acknowledgments}

This research was conducted using an objective, pre-registered protocol to ensure scientific rigor and avoid confirmation bias.

\bibliographystyle{plain}
% \bibliography{references}  % Add references as needed

\appendix

\section{Configuration File}
\label{app:config}

% [Include scan_config.yaml]

\section{Raw Data Tables}
\label{app:data}

% [Include full CSV rankings]

\section{Bootstrap Null Distributions}
\label{app:bootstrap}

% [Include null distribution plots]

\end{document}
