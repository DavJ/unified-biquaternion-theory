\section*{Appendix 7: Maxwellovy rovnice z vektorové části hlavní rovnice}

\subsection*{1. Výchozí rovnice}

Budeme analyzovat vektorovou část hlavní rovnice vaší teorie:
\[
\text{Re}\left[\text{Vec}\left(\mathbf{R}^a_\mu - \tfrac{1}{2} e^a_\mu \, \mathbf{R}\right)\right] = 0
\]
Tato rovnice říká, že reálná vektorová část tenzoru křivosti (po odstranění skalární složky) je nulová.

\subsection*{2. Rozklad na reálnou a imaginární část}

Předpokládáme, že geometrie je určena dvěma nezávislými částmi spinové konexe:
\begin{itemize}
\item $\boldsymbol{\omega}_R$: reálná část (gravitační konexe), jejíž křivost $\mathbf{R}_R$ dává Riemannův tenzor
\item $\boldsymbol{\omega}_I$: imaginární část (elektromagnetická), s $\boldsymbol{\omega}_I^\mu = A^\mu \mathbf{i}$
\end{itemize}

Tenzor křivosti se pak rozpadá jako:
\[
\mathbf{R}_{\mu\nu}^{ab} = \mathbf{R}_{\mu\nu}^{ab (R)} + \mathbf{R}_{\mu\nu}^{ab (I)}
\]

Přičemž:
\[
\mathbf{R}_{\mu\nu}^{ab (I)} = F_{\mu\nu}^{ab} \mathbf{i}
\]

\subsection*{3. Dosazení do hlavní rovnice}

Dosadíme do hlavní rovnice:
\[
\mathbf{R}^a_\mu := e^{\nu b} \mathbf{R}_{\mu\nu}^{ab} = e^{\nu b} \left( \mathbf{R}_{\mu\nu}^{ab (R)} + F_{\mu\nu}^{ab} \mathbf{i} \right)
\]

Nyní extrahujeme vektorovou část výsledného bikvaternionu:
\[
\text{Vec}(\mathbf{R}^a_\mu) = \text{Vec}(\mathbf{R}^{a}_{\mu (R)}) + \text{Vec}(e^{\nu b} F_{\mu\nu}^{ab} \mathbf{i})
\]

Přičemž víme, že $\mathbf{R}_{\mu\nu}^{ab (R)}$ má pouze reálné části (v obecné relativitě), a $\mathbf{i}$ je čistě imaginární kvaternion. Výraz $e^{\nu b} F_{\mu\nu}^{ab} \mathbf{i}$ je tedy imaginární kvaternionový vektor.

Aby reálná část celé rovnice zmizela, musí:

\[
\text{Re}[\text{Vec}(\mathbf{R}^{a}_{\mu (R)})] = - \text{Re}[\text{Vec}(e^{\nu b} F_{\mu\nu}^{ab} \mathbf{i})]
\]

Jelikož levá strana dává Ricciho tenzor, pak:
\[
\text{Ricci}^\mu_a = \epsilon^{\mu\nu\rho\sigma} e^b_\nu F_{\rho\sigma}^{ab}
\]

\subsection*{4. Redukce na Maxwellovy rovnice}

Definujeme:
\[
F_{\mu\nu} := \partial_\mu A_\nu - \partial_\nu A_\mu
\]
a použijeme, že $\boldsymbol{\omega}_I^\mu = A^\mu \mathbf{i} \Rightarrow \mathbf{R}_I = F_{\mu\nu} \mathbf{i}$

Dosazením získáme:
\[
\text{Re}[\text{Vec}(\mathbf{R}^a_\mu)] \supset \text{Re}[\text{Vec}(e^{\nu b} F_{\mu\nu}^{ab} \mathbf{i})]
\Rightarrow \partial^\mu F_{\mu\nu} = 0
\]

\subsection*{5. Závěr}

Ukázali jsme, že vektorová část hlavní bikvaternionové rovnice dává v určitém limitu bez zdrojů právě Maxwellovy rovnice. Tím je elektromagnetismus obsažen v této teorii jako geometrická projekce imaginární části konexe.