
\documentclass{article}
\usepackage{amsmath,amsfonts}
\title{Analytical Derivation of Electron Mass from Electromagnetic Self-Energy}
\author{Unified Biquaternion Theory Team}
\date{\today}
\begin{document}
\maketitle

\section*{Overview}
In this document, we analytically derive the electron mass from its electromagnetic self-energy, based on the hypothesis that the electron is a topological excitation of the $\Theta_1$ field.

\section*{Assumptions and Ansatz}
We assume that the charge distribution of the electron is spherically symmetric and approximated by a Gaussian:
\[
\rho(r) = \frac{e}{\pi^{3/2} R^3} \exp\left(-\frac{r^2}{R^2}\right)
\]
This allows analytical treatment and captures the finite localization scale of the electron.

\section*{Electrostatic Potential}
The electrostatic potential $\phi(r)$ is given by solving Poisson's equation:
\[
\phi(r) = \frac{1}{4\pi\epsilon_0} \int \frac{\rho(r')}{|\vec{r} - \vec{r}'|} \, d^3r'
\]
For the Gaussian source, this results in:
\[
\phi(r) = \frac{e}{4\pi\epsilon_0 r} \operatorname{erf}\left( \frac{r}{R} \right)
\]

\section*{Self-Energy Integral}
The total electromagnetic self-energy is:
\[
\delta m_e c^2 = \frac{1}{2} \int \rho(r) \phi(r) \, d^3r
\]
Evaluating the integral yields:
\[
\delta m_e = \frac{e^2}{\sqrt{\pi} \epsilon_0 R c^2}
\]

\section*{Interpretation}
This result links the electron mass to the scale $R$ of its internal structure, with no new parameters introduced. The remaining task is to derive $R$ from the stress-energy distribution of the $\Theta_1$ Hopfion solution.


\section*{License}
This work is licensed under a Creative Commons Attribution 4.0 International License (CC BY 4.0).

\end{document}