
\section*{Appendix 10: Quantum Electrodynamics (QED) and the Dirac Equation from \(\Theta(q)\)}

\subsection*{Objective}

In this appendix, we show how the structure of the unified field \(\Theta(q)\) gives rise to both:
\begin{itemize}
  \item The classical electromagnetic interaction via a \(U(1)\) gauge connection,
  \item The Dirac equation for spin-\(\frac{1}{2}\) particles as an effective equation for specific excitations of \(\Theta(q)\).
\end{itemize}

This appendix builds the bridge between the geometric formulation of electromagnetism (Appendices 8 and 9)
and the quantum field interpretation that underlies QED.

\subsection*{Gauge Connection from \(\Theta(q)\)}

The imaginary part of the biquaternionic curvature \(\mathbf{R}_{\text{imag}}\) was shown to obey:

\[
D \star \mathbf{R}_{\text{imag}} = 0
\]

This is the geometric form of Maxwell's equations. Now, we define the potential 1-form:

\[
\mathbf{A} := \text{Im}(\omega)
\]

where \(\omega\) is the biquaternion-valued connection 1-form. Then the electromagnetic field strength is:

\[
\mathbf{F} = d\mathbf{A}
\]

and the corresponding action is:

\[
S_{\text{EM}} = \int \frac{1}{2} \mathbf{F} \wedge \star \mathbf{F}
\]

This is the standard form of Maxwell's action, derived from the geometry of \(\Theta(q)\).

\subsection*{Dirac Equation from Spinorial Structure}

Let us now consider the spinor structure encoded within \(\Theta(q)\). We define a subspace of solutions:

\[
\Theta(q) = \psi(q) \otimes \Phi(q)
\]

where:
\begin{itemize}
  \item \(\psi(q)\): a biquaternionic-valued spinor field,
  \item \(\Phi(q)\): a scalar or scalar-spinor modulation field.
\end{itemize}

We postulate that \(\psi(q)\) satisfies a modified Dirac equation:

\[
(i\gamma^\mu D_\mu - m)\psi(q) = 0
\]

where the covariant derivative \(D_\mu\) includes the \(U(1)\) gauge potential \(\mathbf{A}_\mu\):

\[
D_\mu = \partial_\mu + i e \mathbf{A}_\mu
\]

This equation arises naturally when decomposing the unified action of \(\Theta(q)\) into spinor components
and applying the Euler–Lagrange equation under variation with respect to \(\psi(q)\).

\subsection*{Gauge Invariance}

The full action:

\[
S[\psi, \mathbf{A}] = \int \bar{\psi}(i\gamma^\mu D_\mu - m)\psi + \frac{1}{2} \mathbf{F} \wedge \star \mathbf{F}
\]

is invariant under local \(U(1)\) gauge transformations:

\[
\psi \rightarrow e^{i\alpha(x)} \psi, \quad \mathbf{A} \rightarrow \mathbf{A} + d\alpha
\]

which is embedded in the biquaternionic transformation group preserving \(\text{Im}(\omega)\).

\subsection*{Interpretation}

This shows that:

\begin{itemize}
  \item The electromagnetic field emerges as the imaginary part of a connection over \(\Theta(q)\),
  \item The Dirac equation governs spinor excitations of \(\Theta(q)\),
  \item QED is a projection of the general dynamics of \(\Theta(q)\) into the \(U(1)\) gauge sector.
\end{itemize}

\subsection*{Summary}

We have derived QED from the unified field \(\Theta(q)\), including:
\begin{itemize}
  \item Maxwell's equations from biquaternionic curvature,
  \item The Dirac equation from the spinor decomposition of \(\Theta(q)\),
  \item Gauge invariance as an internal symmetry of the field.
\end{itemize}

This completes the quantum formulation of electromagnetism within the Unified Biquaternion Theory framework.
