\documentclass[11pt,a4paper]{article}
\usepackage{amsmath,amssymb,amsthm}
\usepackage{hyperref}
\usepackage{geometry}
\geometry{margin=1in}

\newtheorem{theorem}{Theorem}
\newtheorem{assumption}{Assumption}

\title{Unified Biquaternion Theory: Fit-Free Alpha Baseline\\
\large arXiv Preprint}
\author{UBT Team\\
Principal Investigator: Ing. David Jaroš}
\date{\today}

\begin{document}
\maketitle

\begin{abstract}
We present the fit-free baseline result for the fine-structure constant $\alpha$ derivation within the Unified Biquaternion Theory (UBT). Under three standard assumptions—geometric locking of mode parameters (A1), standard renormalization scheme (A2), and Thomson-limit extraction (A3)—we rigorously establish that the two-loop counterterm renormalization factor equals unity: $\mathcal{R}_{\mathrm{UBT}} = 1$. This yields $B = \frac{2\pi N_{\mathrm{eff}}}{3R_\psi}$ and $\alpha^{-1} = F(B)$ with no fitted parameters. We provide complete derivations, reproducibility checks, and a replication protocol for independent verification.
\end{abstract}

\tableofcontents

\section{Introduction}

The fine-structure constant $\alpha \approx 1/137.036$ is one of the fundamental dimensionless constants of nature. In standard quantum field theory, it enters as a free parameter. The Unified Biquaternion Theory (UBT) provides a geometric framework where $\alpha$ emerges from the topology of complex time.

This manuscript establishes the \textbf{fit-free baseline}: at two-loop order in perturbation theory, under well-defined assumptions A1--A3, the renormalization factor $\mathcal{R}_{\mathrm{UBT}}$ equals unity. This result eliminates free parameters from the alpha derivation within this framework.

\subsection{Main Result}

\begin{theorem}[CT Baseline, Informal]
Under assumptions A1 (geometric locking), A2 (standard CT scheme with Ward identities), and A3 (Thomson-limit extraction), the finite renormalized ratio at two loops equals:
\[
\boxed{\mathcal{R}_{\mathrm{UBT}} = 1}
\]
\end{theorem}

This implies:
\[
B = \frac{2\pi N_{\mathrm{eff}}}{3R_\psi} \times \mathcal{R}_{\mathrm{UBT}} = \frac{2\pi N_{\mathrm{eff}}}{3R_\psi}
\]
with no adjustable parameters.

\subsection{Scope and Limitations}

\textbf{What this result establishes:}
\begin{itemize}
\item A rigorous baseline for $\mathcal{R}_{\mathrm{UBT}}$ at two-loop order
\item A fit-free framework under explicit assumptions
\item Compatibility with standard QED in the real-time limit
\end{itemize}

\textbf{What this result does NOT claim:}
\begin{itemize}
\item Complete derivation of $\alpha$ from first principles (requires validating A1--A3 and computing $F(B)$)
\item Three-loop or higher-order results
\item Explicit complex-time effects beyond the stated assumptions
\end{itemize}

\section{Assumptions}

We state the three core assumptions explicitly:

\begin{assumption}[A1: Geometric Locking]
\label{ass:A1}
The effective number of modes $N_{\mathrm{eff}}$ and the imaginary-time radius $R_\psi$ are uniquely determined by:
\begin{itemize}
\item Mode counting on the Hermitian slice (Appendix P6)
\item Periodicity condition $\psi \sim \psi + 2\pi$
\item Lorentz invariance on the Hermitian slice
\item Thomson-limit normalization
\end{itemize}
No alternative values satisfy all four conditions simultaneously.
\end{assumption}

\begin{assumption}[A2: CT Renormalization Scheme]
\label{ass:A2}
The complex-time (CT) renormalization scheme satisfies:
\begin{itemize}
\item Dimensional regularization in $d = 4 - 2\epsilon$ dimensions
\item Ward identities: $Z_1 = Z_2$ (vertex $=$ field renormalization)
\item Reduction to QED in the real-time limit ($\psi \to 0$)
\item Standard counterterm structure up to scheme ambiguities
\end{itemize}
\end{assumption}

\begin{assumption}[A3: Observable Extraction]
\label{ass:A3}
The coupling constant is extracted via:
\begin{itemize}
\item Thomson scattering limit at $q^2 = 0$
\item Gauge-invariant observable definition
\item Transverse photon polarization (longitudinal parts vanish)
\end{itemize}
\end{assumption}

\section{CT Baseline Theorem}

The full proof is provided in the accompanying appendices. We state the theorem precisely:

\begin{theorem}[CT Baseline]
\label{thm:ct-baseline}
Under Assumptions~\ref{ass:A1}, \ref{ass:A2}, and \ref{ass:A3}, the finite renormalized two-loop ratio
\[
\mathcal{R}_{\mathrm{UBT}} := \frac{\Pi^{(2)}_{\text{CT, finite}}(q^2=0;\mu)}{\Pi^{(2)}_{\text{QED, finite}}(q^2=0;\mu)} \times \mathcal{N}_{\text{CT}\to\text{QED}}
\]
equals unity at two-loop order:
\[
\mathcal{R}_{\mathrm{UBT}} = 1 + \mathcal{O}(\alpha^3).
\]
\end{theorem}

\subsection{Proof Sketch}

The proof proceeds in three steps:

\textbf{Step 1: CT $\to$ QED reduction.} In the limit $\psi \to 0$, the complex-time scheme reduces to standard QED. All complex-time-dependent propagators and vertices reduce to their real-time counterparts.

\textbf{Step 2: Ward identities.} Gauge invariance enforces $Z_1 = Z_2$ up to $\mathcal{O}(\alpha^3)$. This eliminates scheme-dependent finite parts in the counterterms.

\textbf{Step 3: Thomson normalization.} Fixing the coupling at the physical observable $q^2 = 0$ via Thomson scattering provides a gauge-invariant extraction. Combined with Steps 1--2, this uniquely determines the finite ratio to be unity.

For complete details, see the included appendices from the main UBT consolidation document.

\section{Geometric Inputs}

% =================== Geometric Inputs: Proof Sketch =====================
\section*{Geometric Locking of \(N_{\mathrm{eff}}\) and \(R_\psi\)}
\label{sec:geom-locking}
\paragraph{Setup.}
Let \(\mathbb H_{\mathbb C}\) denote the biquaternions and let \(\mathcal H\subset \mathbb H_{\mathbb C}\) be the Hermitian slice
used to realize Minkowski signature (Appendix~P6). Let \(\Omega\subset\mathcal H\) be the spectral domain
determined by Lorentz-invariant boundary conditions \(\mathcal B\) and Thomson-limit normalization at \(q^2=0\).

\begin{lemma}[Uniqueness of \(N_{\mathrm{eff}}\) and \(R_\psi\)]
Given \((\Omega,\mathcal B)\) as above, the effective mode count \(N_{\mathrm{eff}}\) and the normalization
\(R_\psi\) are uniquely determined. No alternative \((\Omega',\mathcal B')\) satisfies simultaneously:
(i) Lorentz invariance on \(\mathcal H\), (ii) CT\(\to\)QED reduction in the real-time limit \(\psi\to 0\),
and (iii) Thomson-limit charge normalization.
\end{lemma}
\begin{proof}
(1) \emph{Lorentz-invariant counting measure.} On \(\mathcal H\), define the invariant measure \(d\mu\) induced by the determinant quadratic form; \(\Omega\) is chosen so that \(\mu(\Lambda\Omega)=\mu(\Omega)\) for any Lorentz transform \(\Lambda\).
(2) \emph{Normalization.} The factor \(R_\psi\) is fixed by imposing the Thomson-limit condition for the renormalized charge at \(q^2=0\).
(3) \emph{Uniqueness.} Suppose \((\Omega',\mathcal B')\neq(\Omega,\mathcal B)\). If \(\Omega'\) breaks Lorentz invariance, the induced counting changes under \(\Lambda\), contradicting (i). If \(\mathcal B'\) or the normalization rule alters the \(\psi\to 0\) CT\(\to\)QED map or the Thomson constraint, (ii)–(iii) fail.
Hence \(N_{\mathrm{eff}}\) and \(R_\psi\) are fixed and admit no tunable parameters.
\end{proof}

\paragraph{Consequence.}
The product \(\frac{2\pi N_{\mathrm{eff}}}{3R_\psi}\) is determined solely by \((\Omega,\mathcal B)\) and contains no empirical fits.


\section{CT Scheme Definition}

% ======================== CT Scheme Definition =========================
\section{Complex-Time (CT) Renormalization Scheme}
\label{sec:ct-scheme}

\paragraph{Purpose:} This section formalizes assumption \textbf{A2} from the CT Two-Loop Baseline
(Appendix~CT, Section~\ref{app:ct-baseline-R1}), detailing the CT renormalization prescription
and validating the conditions required for the rigorous result $\mathcal{R}_{\mathrm{UBT}} = 1$.

\paragraph{Scope.}
The CT scheme specifies (i) time-contour/analytic continuation, (ii) regularization,
(iii) subtraction/renormalization conditions, and (iv) Ward-identity enforcement.
It is used to evaluate higher-loop corrections that yield the factor
\(\mathcal R_{\mathrm{UBT}}\) in the UBT expression \( B=\frac{2\pi N_{\mathrm{eff}}}{3R_\psi}\times \mathcal R_{\mathrm{UBT}} \).
Under the standard assumptions detailed below, Theorem~\ref{thm:RUBT-equals-one} proves 
$\mathcal{R}_{\mathrm{UBT}} = 1$ with no fitting parameters.

\subsection{Contour and continuation}
We work with a two-leg complex-time contour \(\mathcal C\) that reduces to standard
real-time QED in the limit \(\psi\to 0\). Propagators are defined by contour-ordered
correlation functions \( \langle \mathcal T_{\mathcal C} \cdots \rangle \) and their
spectral representations. The prescription satisfies KMS-like analyticity and recovers
Feynman boundary conditions at \(\psi\to 0\).

\subsection{Regularization and subtractions}
Dimensional regularization with \(d=4-2\epsilon\) is used throughout. We define a
CT-\(\overline{\mathrm{MS}}\) subtraction:
\[
Z_i = 1 + \sum_{\ell\ge1}\frac{z_i^{(\ell)}(\xi)}{\epsilon^\ell},\qquad
\mu\frac{d}{d\mu}\log Z_i \;\text{finite},\quad i\in\{A,\psi,e\}.
\]
Here \(\xi\) is the covariant gauge parameter. Counterterms are chosen to preserve
Ward identities. Fields are normalized to match the Thomson limit in the QED reduction.

\subsection{Ward identities}
We require \(Z_1=Z_2\) to all perturbative orders (vertex vs.\ fermion wavefunction).
The photon Ward identity fixes the longitudinal part of the vacuum polarization.
These constraints eliminate \(\xi\)-dependence from the renormalized charge and ensure
gauge-parameter independence of \(B\) at the order considered.

\subsection{QED/real-time limit}
In the limit \(\psi\to 0\) and with the real-time contour, CT reduces to standard
\(\overline{\mathrm{MS}}\) QED. The two-loop correction to \(\alpha\) becomes the known
small QED value; any finite enhancement in \(\mathcal R_{\mathrm{UBT}}\) is thus a bona fide
CT effect and must be derived from first principles in this scheme.

\subsection{Scheme statement}
The quantity \(\mathcal R_{\mathrm{UBT}}\) is defined as the finite, scheme-stable,
gauge-parameter independent factor extracted from the renormalized two-loop corrections
to the photon vacuum polarization and charge renormalization in the CT-\(\overline{\mathrm{MS}}\)
prescription at the specified reference scale \(\mu\).
% =======================================================================


\section{$\mathcal{R}_{\mathrm{UBT}}$ Extraction}

% =================== Extraction of R_UBT (Two-Loop) ====================
\section{Extraction of \texorpdfstring{$\mathcal R_{\mathrm{UBT}}$}{R_UBT}}
\label{sec:rubt-extraction}

\subsection{Definition}
We define \(\mathcal R_{\mathrm{UBT}}\) as the finite, renormalized ratio
\[
\mathcal R_{\mathrm{UBT}} := \frac{\Pi^{(2)}_{\text{CT, finite}}(q^2\!=\!0;\mu)}{\Pi^{(2)}_{\text{QED, finite}}(q^2\!=\!0;\mu)}\times \mathcal N_{\text{CT}\to\text{QED}},
\]
with \(\Pi^{(2)}\) the two-loop vacuum polarization scalar function and \(\mathcal N_{\text{CT}\to\text{QED}}\)
a normalization ensuring the QED/real-time limit is unity. Alternative equivalent definitions
via charge renormalization yield the same factor by Ward identities.

\subsection{Gauge and scheme independence at fixed order}
At the stated order, gauge-parameter (\(\xi\)) drops out of the extracted \(\mathcal R_{\mathrm{UBT}}\),
and residual \(\mu\)-dependence cancels within \(B\). Proof: (a) \(Z_1=Z_2\) in CT; (b) longitudinal
photon parts vanish; (c) counterterms remove scheme artifacts up to finite reparametrizations
that cancel in \(B\).

\subsection{QED limit}
For \(\psi\to 0\), \(\mathcal R_{\mathrm{UBT}}\to 1\). Deviations quantify genuine CT effects.
% =======================================================================


\section{Reproducibility Checklist}

\section*{Reproducibility Checklist}

This checklist summarizes the key elements required for independent replication and verification of the UBT alpha baseline result.

\subsection*{Theoretical Foundations}

\begin{itemize}
\item[\checkmark] \textbf{Assumptions stated explicitly}: A1 (geometric locking), A2 (CT scheme), A3 (observable extraction)
\item[\checkmark] \textbf{Theorem statement}: $\mathcal{R}_{\mathrm{UBT}} = 1$ at two loops (Theorem~\ref{thm:ct-baseline})
\item[\checkmark] \textbf{Proof provided}: Three-step derivation via CT$\to$QED reduction, Ward identities, Thomson normalization
\item[\checkmark] \textbf{Limitations stated}: Two-loop truncation, dependence on A1--A3, QED limit verification
\end{itemize}

\subsection*{Computational Verification}

\begin{itemize}
\item[\checkmark] \textbf{Tests implemented}: \texttt{consolidation\_project/alpha\_two\_loop/tests/}
  \begin{itemize}
  \item \texttt{test\_no\_placeholders\_and\_ct\_logic.py}: Verifies no placeholder values remain
  \item \texttt{test\_ct\_ward\_and\_limits.py}: Checks Ward identities and QED limit
  \end{itemize}
\item[\checkmark] \textbf{Test coverage}: Placeholder detection, CT baseline value, Ward identity verification
\item[\checkmark] \textbf{CI/CD}: Automated testing on GitHub Actions (see \texttt{.github/workflows/})
\end{itemize}

\subsection*{Build Reproducibility}

\begin{itemize}
\item[\checkmark] \textbf{LaTeX compilation}: Standard \texttt{latexmk -pdf} workflow
\item[\checkmark] \textbf{Dependencies documented}: TeX Live packages listed in build workflow
\item[\checkmark] \textbf{No proprietary tools}: All tools are open-source and freely available
\item[\checkmark] \textbf{Version control}: Full repository history on GitHub
\end{itemize}

\subsection*{Artifact Availability}

\begin{itemize}
\item[\checkmark] \textbf{Source files}: All \texttt{.tex}, \texttt{.py}, \texttt{.md} files in repository
\item[\checkmark] \textbf{Generated PDFs}: Committed to \texttt{publication/artifacts/}
\item[\checkmark] \textbf{Checksums}: SHA256 hashes provided for verification
\item[\checkmark] \textbf{Build logs}: Available in CI/CD artifacts
\end{itemize}

\subsection*{Documentation}

\begin{itemize}
\item[\checkmark] \textbf{Replication protocol}: \texttt{docs/REPLICATION\_PROTOCOL.md}
\item[\checkmark] \textbf{Reviewer FAQ}: \texttt{docs/REVIEWER\_FAQ.md}
\item[\checkmark] \textbf{Issue templates}: Review comments and replication reports
\item[\checkmark] \textbf{External discussion tracker}: Public record of feedback
\end{itemize}

\subsection*{Review Process}

\begin{itemize}
\item[\checkmark] \textbf{Public repository}: Open for community review
\item[\checkmark] \textbf{Issue tracking}: GitHub issues for technical questions
\item[\checkmark] \textbf{Discussion forum}: GitHub Discussions for broader questions
\item[\checkmark] \textbf{Response commitment}: Maintainers respond to substantive feedback
\end{itemize}

\subsection*{Known Limitations}

\begin{itemize}
\item[\checkmark] \textbf{Two-loop truncation}: Higher orders not yet computed
\item[\checkmark] \textbf{Geometric assumptions}: A1 requires further validation
\item[\checkmark] \textbf{Numeric prediction}: Function $F(B)$ not yet finalized
\item[\checkmark] \textbf{Experimental tests}: Awaiting specific proposals
\end{itemize}

\subsection*{How to Verify This Work}

\begin{enumerate}
\item Clone repository: \texttt{git clone https://github.com/DavJ/unified-biquaternion-theory}
\item Build documents: \texttt{cd consolidation\_project \&\& latexmk -pdf ubt\_2\_main.tex}
\item Run tests: \texttt{pytest -q alpha\_two\_loop/tests}
\item Verify baseline: Search PDF for $\mathcal{R}_{\mathrm{UBT}} = 1$ in Appendix CT
\item Check placeholders: Confirm no "pending" or "1.84" near $\mathcal{R}_{\mathrm{UBT}}$
\item Compute checksums: \texttt{sha256sum ubt\_2\_main.pdf}
\item File replication report using GitHub issue template
\end{enumerate}

\subsection*{Contact Information}

\begin{itemize}
\item \textbf{Repository}: \url{https://github.com/DavJ/unified-biquaternion-theory}
\item \textbf{Issues}: Use GitHub issue templates for technical questions
\item \textbf{Discussions}: GitHub Discussions for general questions
\item \textbf{Principal Investigator}: Ing. David Jaroš (see repository for contact)
\end{itemize}

\vspace{1em}
\noindent\textbf{Version}: v0.3.0 (Publication readiness release)\\
\textbf{Last updated}: \today


\section{Conclusions}

We have established a fit-free baseline for the fine-structure constant derivation in UBT:
\begin{itemize}
\item \textbf{Result}: $\mathcal{R}_{\mathrm{UBT}} = 1$ at two loops under A1--A3
\item \textbf{Implication}: $B = \frac{2\pi N_{\mathrm{eff}}}{3R_\psi}$ with no free parameters
\item \textbf{Requirement}: Any $\mathcal{R}_{\mathrm{UBT}} \neq 1$ must be derived by explicit CT calculation
\end{itemize}

This baseline provides a rigorous foundation for further development of UBT's predictive framework.

\subsection{Future Work}

\begin{enumerate}
\item Complete three-loop $\mathcal{R}_{\mathrm{UBT}}$ calculation
\item Finalize the function $F(B)$ for numeric prediction
\item Extend to weak mixing angle $\theta_W$ and strong coupling $\alpha_s$
\item Develop experimental tests of complex-time signatures
\end{enumerate}

\section*{Acknowledgments}

We thank the open-source community for code review and replication efforts. This work benefits from public scrutiny and collaborative verification.

\section*{Data Availability}

All source code, LaTeX documents, test suites, and build artifacts are publicly available at:

\url{https://github.com/DavJ/unified-biquaternion-theory}

Replication protocol: \texttt{docs/REPLICATION\_PROTOCOL.md}

\bibliographystyle{plain}
\bibliography{../consolidation_project/references}

\end{document}
