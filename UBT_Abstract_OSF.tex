\documentclass[12pt]{article}
\usepackage{amsmath,amssymb,amsfonts}
\usepackage{geometry}
\usepackage{hyperref}
\geometry{a4paper, margin=1in}

\title{Unified Biquaternion Theory (UBT): Abstract for OSF}
\author{Ing.~David~Jaroš}
\date{November 2025}

\begin{document}
\maketitle

\begin{abstract}
The Unified Biquaternion Theory (UBT) presents a comprehensive framework for unifying General Relativity, Quantum Field Theory, and Standard Model symmetries within a single mathematical structure based on biquaternionic fields over complex time $\tau = t + i\psi$. 

\textbf{Key Features:}
\begin{itemize}
\item \textbf{General Relativity Compatibility}: UBT generalizes Einstein's General Relativity and, in the real-valued limit, exactly reproduces Einstein's field equations for all curvature regimes (including $R \neq 0$), ensuring full compatibility with all experimental confirmations of GR.

\item \textbf{Geometric Gauge Unification}: The Standard Model gauge group $SU(3) \times SU(2) \times U(1)$ emerges naturally from the automorphism group of the biquaternionic structure, providing a geometric origin for all fundamental forces.

\item \textbf{Complex Time Extension}: The complex time coordinate $\tau = t + i\psi$ introduces additional phase-like degrees of freedom that may correspond to dark sector physics and quantum gravitational corrections.

\item \textbf{Theta Function Dynamics}: The phase field $\psi$ evolves according to a drift-diffusion equation whose steady-state solutions are Jacobi theta functions, providing dynamical stability and modular symmetry.

\item \textbf{Dark Sector Physics}: Biquaternionic extensions beyond those required for reproducing GR may provide candidates for dark matter and dark energy through p-adic extensions.

\item \textbf{Fermion Mass Predictions}: The theory derives charged lepton masses from first principles, predicting the electron mass to 0.2\% accuracy (as of v9 update, November 2025).
\end{itemize}

\textbf{Mathematical Foundation}: The core equation is
\begin{equation}
\nabla^\dagger\nabla\Theta(q,\tau) = \kappa \mathcal{T}(q,\tau)
\end{equation}
where $\Theta$ is the unified biquaternionic field, $q \in \mathbb{C}\otimes\mathbb{H}$ represents biquaternion coordinates, and $\kappa$ relates to Newton's constant.

\textbf{Current Status}: UBT is a research framework making its first testable predictions. As of November 2025, the theory has achieved:
\begin{itemize}
\item Mathematical foundations substantially complete
\item Standard Model gauge group rigorously derived from geometry
\item First principles derivation of fermion masses with experimental validation
\item Fine-structure constant geometrically constrained
\item Testable predictions in lepton masses and CMB analysis feasible within 1-2 years
\end{itemize}

\textbf{Note:} Certain speculative sections discuss hyper-spatial or faster-than-light interpretations. 
These remain purely theoretical and are not experimentally verified. The discussions of complex metric solutions, faster-than-light propagation, and hyperdimensional transport are mathematical explorations of the theory's formal structure and should not be interpreted as physical predictions without experimental support.

The theory's connection to consciousness modeling (Complex Consciousness Theory, CCT) is considered interpretive and speculative, and is not part of the core physical results.

\textbf{References}: This work extends approaches by A.H. Chamseddine \cite{chamseddine2025hermitian} and E. Verlinde through a unified complex-time and toroidal topology model.

\end{abstract}

\section*{Keywords}
Biquaternion algebra, complex time, unified field theory, General Relativity, quantum field theory, Standard Model, gauge unification, Hermitian gravity, SU(3) symmetry, theta functions, dark matter, dark energy, fermion masses

\section*{Scientific Classification}
\begin{itemize}
\item \textbf{Primary}: Theoretical Physics, Mathematical Physics
\item \textbf{Secondary}: Quantum Field Theory, General Relativity, Gauge Theory
\item \textbf{Applications}: Dark sector physics, Beyond Standard Model physics
\end{itemize}

\section*{Target Audience}
This work is intended for researchers in theoretical physics, mathematical physics, and related fields with expertise in differential geometry, gauge theory, and quantum field theory.

\bibliographystyle{plain}
\bibliography{references}

\section*{License}
© 2025 Ing. David Jaroš — CC BY-NC-ND 4.0

This work is licensed under a Creative Commons Attribution-NonCommercial-NoDerivatives 4.0 International License (CC BY-NC-ND 4.0).

\textbf{License History:} Earlier drafts (up to v0.3) were released under CC BY 4.0. From v0.4 onward, all material is released under CC BY-NC-ND 4.0 to protect the integrity of the theoretical work during ongoing academic development.

\section*{Contact}
Ing. David Jaroš \\
Email: jdavid.cz@gmail.com

\end{document}
