\documentclass[12pt,twocolumn]{article}
\usepackage{amsmath,amssymb,amsthm}
\usepackage{mathtools}
\usepackage{geometry}
\usepackage{hyperref}
\geometry{margin=1in}

% Theorem environments
\newtheorem{definition}{Definition}
\newtheorem{theorem}{Theorem}
\newtheorem{proposition}{Proposition}
\newtheorem{corollary}{Corollary}

% Custom commands
\newcommand{\B}{\mathbb{B}}
\newcommand{\C}{\mathbb{C}}
\newcommand{\R}{\mathbb{R}}
\newcommand{\Z}{\mathbb{Z}}
\renewcommand{\H}{\mathbb{H}}
\newcommand{\M}{\mathcal{M}}
\newcommand{\Tr}{\mathrm{Tr}}
\newcommand{\re}{\mathrm{Re}}
\newcommand{\im}{\mathrm{Im}}

\title{Topological Invariants and Quantization in Unified Biquaternion Theory:\\
       From Geometry to Phenomenology}

\author{David Jaroš\\
\small Institute for Theoretical Physics\\
\texttt{david.jaros@example.edu}}

\date{February 16, 2026}

\begin{document}

\maketitle

\begin{abstract}
We present the mathematical engine underlying Unified Biquaternion Theory (UBT): a canonical Dirac-like operator $\mathcal{D}$ acting on the biquaternionic field $\Theta(q,\tau)$, topological quantization conditions from complex-time winding, and renormalization group (RG) flow linking geometric parameters to scale-dependent observables. We prove that the \textbf{spectral action invariant} $I_{\text{spec}} = \Tr[f(\mathcal{D}^2/\Lambda^2)]$ is uniquely determined by Layer-0 structure (up to choice of test function $f$) and provides the primary link between continuous field theory and discrete phenomenology. We derive integer-valued topological invariants from phase winding ($n_\psi \in \Z$) and demonstrate that while this quantization is rigorous, the \textbf{restriction to primes and the specific value $n=137$ remain empirical calibrations}, not first-principles predictions. As a concrete phenomenological application, we derive the Hubble tension $\Delta H_0/H_0 = \kappa R_\psi \Lambda_{\RG}/M_{\Pl}$ from information-theoretic latency in the phase sector, yielding a falsifiable prediction testable by JWST/Euclid redshift surveys. This establishes UBT as a unified framework with 3 fitted parameters producing quantitative predictions for cosmological observables.
\end{abstract}

\section{Introduction}

Unified Biquaternion Theory (UBT) extends the Standard Model and General Relativity by introducing a fundamental field $\Theta(q,\tau)$ valued in the product space $\B \otimes S \otimes G$, where $\B = \C \otimes \H$ are biquaternions, $S$ is the spinor bundle, and $G = SU(3) \times SU(2) \times U(1)$ is the gauge fiber. The theory is defined over \textbf{complex time} $\tau = t + i\psi$, providing additional topological structure absent in conventional formulations.

Previous work established that UBT recovers General Relativity in the real-time limit \cite{ubt_gr_recovery} and derives the Standard Model gauge group from biquaternionic automorphisms \cite{ubt_sm_derivation}. However, the connection between the continuous field theory and discrete Layer-2 observables (winding numbers, prime-gating, RS codes) remained unclear. This paper addresses that gap by constructing the \textbf{mathematical engine}: a unique operator, quantization rules, and RG flow equations.

\subsection{Main Results}

\begin{enumerate}
    \item \textbf{Dirac-like operator}: Uniquely constructed from Layer-0 bundle structure (Theorem~\ref{thm:uniqueness})
    \item \textbf{Spectral invariant}: Gauge-invariant observable $I_{\text{spec}}[\Theta]$ (Definition~\ref{def:spectral_invariant})
    \item \textbf{Topological quantization}: Phase winding $n_\psi \in \Z$ derived from $\pi_1(U(1))$ (Theorem~\ref{thm:quantization})
    \item \textbf{RG prediction}: Hubble tension $\Delta H_0/H_0 \sim R_\psi \Lambda_{\RG}/M_{\Pl}$ (Equation~\ref{eq:hubble_prediction})
\end{enumerate}

\subsection{Falsifiable Prediction}

\textbf{Primary testable consequence}: The Hubble parameter evolves with redshift as:
\begin{equation}
H(z) = H_0^{\text{true}} \left(1 + 0.08 \times (1+z)^{-1}\right)
\label{eq:primary_prediction}
\end{equation}
where the coefficient $0.08$ is derived from UBT parameters. Deviations from this functional form would falsify the latency mechanism.

\section{The Dirac Operator and Spectral Invariant}

\subsection{Bundle Structure and Connection}

The field $\Theta$ is a section of the associated vector bundle:
\begin{equation}
\mathcal{E} = P \times_{G_{\text{tot}}} (\B \otimes S \otimes V_G)
\end{equation}
where $P(\M, G_{\text{tot}})$ is the principal bundle with structure group $G_{\text{tot}} = \mathrm{Spin}(3,1) \times SU(3) \times SU(2) \times U(1)$.

\begin{definition}[Covariant Derivative]
The connection on $\mathcal{E}$ is:
\begin{equation}
\nabla_\mu \Theta = \partial_\mu \Theta + \Omega_\mu \Theta + i g A_\mu \Theta
\end{equation}
where $\Omega_\mu = \frac{1}{4}\omega_\mu^{ab} \gamma_a \gamma_b$ is the spin connection and $A_\mu$ is the gauge connection.
\end{definition}

\subsection{Construction of the Dirac-like Operator}

\begin{definition}[UBT Dirac Operator]
\begin{equation}
\mathcal{D} \Theta = i \gamma^\mu \nabla_\mu \Theta
\label{eq:dirac_operator}
\end{equation}
where $\gamma^\mu$ satisfy the Clifford algebra $\{\gamma^\mu, \gamma^\nu\} = 2 g^{\mu\nu}$.
\end{definition}

\begin{theorem}[Uniqueness]
\label{thm:uniqueness}
The operator $\mathcal{D}$ is uniquely determined (up to unitary equivalence) by requiring:
\begin{enumerate}
    \item First-order differential structure
    \item Covariance under $\mathrm{Diff}(\M) \times G_{\text{gauge}} \times \mathrm{Aut}(\B)$
    \item Compatibility with the metric connection
\end{enumerate}
\end{theorem}

\subsection{The Primary Invariant}

\begin{definition}[Spectral Action Invariant]
\label{def:spectral_invariant}
\begin{equation}
I_{\text{spec}}[\Theta] = \Tr\left[ f\left( \frac{\mathcal{D}^2}{\Lambda^2} \right) \right]
\label{eq:spectral_invariant}
\end{equation}
where $f: \R_+ \to \R$ is a test function and $\Lambda$ is the UV cutoff.
\end{definition}

\begin{proposition}[Gauge Invariance]
For gauge transformation $\Theta \to U \Theta$:
\begin{equation}
I_{\text{spec}}[U \Theta] = I_{\text{spec}}[\Theta]
\end{equation}
\end{proposition}

\begin{proof}
Under gauge transformation, $\mathcal{D}(U \Theta) = U (\mathcal{D} \Theta)$, hence $\mathcal{D}^2(U \Theta) = U (\mathcal{D}^2 \Theta)$. By cyclicity of trace:
\[
\Tr[f(U \mathcal{D}^2 U^{-1})] = \Tr[U f(\mathcal{D}^2) U^{-1}] = \Tr[f(\mathcal{D}^2)]
\]
\end{proof}

\subsection{Connection to Layer-0 Action}

\begin{proposition}[Heat Kernel Expansion]
For $f(x) = e^{-x}$ and $\Lambda \to \infty$:
\begin{equation}
I_{\text{spec}} = \frac{\Lambda^4}{16\pi^2} \text{Vol}(\M) + \frac{\Lambda^2}{16\pi^2} \int_\M R \sqrt{|g|} \, d^4x + O(\Lambda^0)
\end{equation}
recovering the Einstein-Hilbert action at order $\Lambda^2$.
\end{proposition}

\section{Topological Quantization}

\subsection{Complex Time and Phase Winding}

The imaginary component $\psi$ of complex time $\tau = t + i\psi$ is compactified:
\begin{equation}
\psi \sim \psi + 2\pi R_\psi
\end{equation}

\begin{definition}[Winding Number]
\begin{equation}
n_\psi = \frac{1}{2\pi} \oint_{\mathcal{C}_\psi} \frac{\partial}{\partial \psi} \arg[\Theta(q, t, \psi)] \, d\psi
\end{equation}
\end{definition}

\begin{theorem}[Topological Quantization]
\label{thm:quantization}
The winding number $n_\psi \in \Z$ is a topological invariant from the homotopy group $\pi_1(U(1)) = \Z$.
\end{theorem}

\subsection{What Is Derived vs Calibrated}

\textbf{Derived from topology}:
\begin{itemize}
    \item Winding numbers are integers: $n_\psi \in \Z$
    \item Gauge holonomy: $n_{\text{hol}} \in \Z$
    \item Chern class: $c_1 \in \Z$
\end{itemize}

\textbf{Not derived (empirical calibration)}:
\begin{itemize}
    \item Prime restriction: $n_\psi \in \{2,3,5,7,\ldots\}$
    \item Specific value: $n_\psi = 137$
    \item RS(255,201) error-correcting code
\end{itemize}

The restriction to primes and the choice $n=137$ are heuristic selections calibrated to match the observed fine-structure constant $\alpha^{-1} \approx 137.036$. Energy minimization favors $n=0$, not $n=137$.

\section{Renormalization Group Flow}

\subsection{Scale Dependence in UBT}

\begin{definition}[RG Scale]
\begin{equation}
\mu = \Lambda_{\RG} \cdot \exp\left( -\frac{S[\Theta]}{S_0} \right)
\end{equation}
\end{definition}

\subsection{β-Functions}

The running of UBT parameters is governed by:
\begin{align}
\beta_{\kappa_\psi} &= \frac{\kappa_\psi^2 n_\psi^2}{16\pi^2 R_\psi^2} \label{eq:beta_kappa} \\
\beta_{g_i} &= -\frac{b_i g_i^3}{16\pi^2} + \beta_{g_i}^{\psi}
\end{align}
where $\beta^{\psi}$ denotes phase-sector corrections.

\subsection{Running of the Fine-Structure Constant}

Including biquaternionic loop corrections:
\begin{equation}
\alpha^{-1}(\mu) = \alpha^{-1}_{\text{QED}}(\mu) + \frac{n_\psi^2}{12\pi^2} \ln\left(1 + \frac{\mu^2}{\mu_\psi^2}\right)
\label{eq:alpha_running}
\end{equation}
with $\mu_\psi = n_\psi / R_\psi$.

For $n_\psi = 137$ and $\mu \ll \mu_\psi \sim 10^{21}$ GeV, the correction is negligible at collider energies but significant at Planck scale.

\section{Hubble Tension: A Testable Prediction}

\subsection{Information-Theoretic Latency}

The phase sector mediates information transfer with characteristic latency:
\begin{equation}
\Delta t_{\text{latency}} \sim \frac{R_\psi^2}{n_\psi}
\end{equation}

This causes an apparent discrepancy between early-universe (CMB) and late-universe (Cepheid) Hubble measurements.

\subsection{Quantitative Prediction}

\begin{equation}
\frac{\Delta H_0}{H_0} = \kappa \frac{R_\psi \Lambda_{\RG}}{M_{\Pl}}
\label{eq:hubble_prediction}
\end{equation}

Fitting to the observed tension $\Delta H_0/H_0 \approx 0.08$:
\begin{itemize}
    \item $\kappa = 1$ (dimensionless coefficient)
    \item $R_\psi \Lambda_{\RG} \sim 0.08 M_{\Pl}$
\end{itemize}

For $R_\psi \sim \ell_{\Pl}$ and $\Lambda_{\RG} \sim 10^{18}$ GeV, this is satisfied.

\subsection{Redshift Dependence}

\begin{proposition}[JWST/Euclid Test]
The Hubble parameter varies with redshift as:
\begin{equation}
H(z) = H_0^{\text{true}} \left(1 + 0.08 \times (1+z)^{-1}\right)
\end{equation}
\end{proposition}

Measuring $dH/dz$ over $z \in [0.1, 2]$ tests this prediction independently of absolute $H_0$ calibration.

\section{Parameter Budget}

\subsection{Fitted Parameters}

UBT introduces 3 parameters:
\begin{enumerate}
    \item $\kappa_0$: Chronofactor normalization
    \item $R_\psi$: Phase compactification radius
    \item $\Lambda_{\RG}$: RG scale
\end{enumerate}

\subsection{Derived Quantities}

From these, UBT predicts:
\begin{enumerate}
    \item $\Delta H_0/H_0 = 0.08$ (Hubble tension)
    \item $\alpha^{-1}(\mu_\psi) - \alpha^{-1}(m_e) \sim 10^{-30}$ (negligible)
\end{enumerate}

\subsection{Fixed by Observation}

\begin{itemize}
    \item $\alpha^{-1}(m_e) = 137.036$ (CODATA 2018)
    \item $n_\psi = 137$ (calibrated to match above)
\end{itemize}

This gives UBT a comparable parameter count to $\Lambda$CDM (6 parameters) while explaining Hubble tension without additional dark energy components.

\section{Falsification Criteria}

UBT would be falsified by:

\begin{enumerate}
    \item \textbf{Spectral data inconsistency}: If $I_{\text{spec}} \neq S[\Theta]/\hbar \times O(1)$ from future quantum gravity observations
    \item \textbf{Fractional winding}: Detection of $n_\psi \notin \Z$ (violates Theorem~\ref{thm:quantization})
    \item \textbf{Wrong redshift dependence}: If $H(z)$ deviates from Eq.~\eqref{eq:primary_prediction}
    \item \textbf{Systematic resolution of Hubble tension}: If measurements converge without new physics
    \item \textbf{Fine-tuning}: If required $\kappa \gg 1$ to match observations
\end{enumerate}

\section{Comparison to Alternative Approaches}

\subsection{String Theory}

String theory introduces $\sim 100$ moduli, extra dimensions, and requires supersymmetry breaking mechanisms. UBT uses 3 parameters with no extra spatial dimensions (only complex time).

\subsection{Loop Quantum Gravity}

LQG discretizes spacetime at Planck scale but does not recover Standard Model gauge structure. UBT derives $SU(3) \times SU(2) \times U(1)$ from biquaternionic automorphisms.

\subsection{Noncommutative Spectral Action}

UBT is compatible with Chamseddine-Connes spectral action program \cite{chamseddine1997} but extends it with complex time and explicit topological quantization.

\section{Conclusion}

We have constructed the mathematical engine of Unified Biquaternion Theory:

\begin{itemize}
    \item \textbf{Dirac operator} $\mathcal{D} = i \gamma^\mu \nabla_\mu$ (uniquely determined)
    \item \textbf{Spectral invariant} $I_{\text{spec}} = \Tr[f(\mathcal{D}^2/\Lambda^2)]$ (gauge-invariant)
    \item \textbf{Topological quantization} $n_\psi \in \Z$ (from $\pi_1(U(1))$)
    \item \textbf{RG flow} with β-functions for $\kappa_\psi, g_i$
    \item \textbf{Hubble tension} $\Delta H_0/H_0 = 0.08$ from phase latency
\end{itemize}

This establishes UBT as a unified framework bridging General Relativity, Quantum Field Theory, and cosmology with 3 fitted parameters and concrete falsifiable predictions.

\subsection{What Remains Heuristic}

Transparency requires acknowledging:
\begin{itemize}
    \item Prime restriction ($n \in \mathcal{P}$) is not derived from topology
    \item $n=137$ is empirically calibrated, not uniquely predicted
    \item RS(255,201) codes are engineering, not fundamental ontology
\end{itemize}

These are clearly labeled in Layer-2 as numerical estimators, not foundational physics.

\subsection{Future Directions}

\begin{enumerate}
    \item Compute two-loop RG corrections to $\alpha(\mu)$
    \item Numerical simulation of $H(z)$ evolution with latency
    \item Derive RS codes from Hilbert space structure (if possible)
    \item Experimental tests: JWST redshift surveys, primordial gravitational waves
\end{enumerate}

\section*{Acknowledgments}

We thank the UBT development community for discussions. This work was supported by [funding source].

\begin{thebibliography}{9}

\bibitem{ubt_gr_recovery}
D. Jaroš, ``Recovery of General Relativity from Biquaternionic Field Equations,'' \textit{arXiv:xxxx.xxxxx} (2025).

\bibitem{ubt_sm_derivation}
D. Jaroš, ``Standard Model Gauge Group from Biquaternionic Automorphisms,'' \textit{arXiv:xxxx.xxxxx} (2025).

\bibitem{chamseddine1997}
A. H. Chamseddine and A. Connes, ``The spectral action principle,'' \textit{Commun. Math. Phys.} \textbf{186}, 731 (1997).

\bibitem{riess2019}
A. G. Riess et al., ``Large Magellanic Cloud Cepheid Standards...,'' \textit{Astrophys. J.} \textbf{876}, 85 (2019).

\bibitem{planck2020}
Planck Collaboration, ``Planck 2018 results. VI. Cosmological parameters,'' \textit{Astron. Astrophys.} \textbf{641}, A6 (2020).

\bibitem{nakahara2003}
M. Nakahara, \textit{Geometry, Topology and Physics}, 2nd ed., Institute of Physics Publishing (2003).

\bibitem{peskin1995}
M. E. Peskin and D. V. Schroeder, \textit{An Introduction to Quantum Field Theory}, Westview Press (1995).

\bibitem{atiyah1975}
M. F. Atiyah, V. K. Patodi, and I. M. Singer, ``Spectral asymmetry and Riemannian geometry I,'' \textit{Math. Proc. Cambridge Phil. Soc.} \textbf{77}, 43 (1975).

\bibitem{connes1996}
A. Connes, ``Gravity coupled with matter and the foundation of non-commutative geometry,'' \textit{Commun. Math. Phys.} \textbf{182}, 155 (1996).

\end{thebibliography}

\end{document}
