% Canonical Biquaternion Algebra Definition
% Version: 1.0
% Date: 2025-11-14
% Status: Canonical - Mathematical Foundation

\section{Biquaternion Algebra}
\label{sec:canonical:biquaternion_algebra}

\subsection{Definition of Biquaternions}

The \textbf{biquaternion algebra} $\mathcal{B}$ is defined as the tensor product of quaternions with complex numbers:

\begin{equation}
\label{eq:canonical:biquat_def}
\mathcal{B} = \mathbb{H} \otimes \mathbb{C}
\end{equation}

where $\mathbb{H}$ is the algebra of quaternions and $\mathbb{C}$ is the field of complex numbers.

\subsubsection{Quaternion Units}

The quaternion algebra $\mathbb{H}$ has basis $\{1, i, j, k\}$ with multiplication rules:

\begin{align}
\label{eq:canonical:quaternion_mult}
i^2 &= j^2 = k^2 = -1 \\
ij &= k, \quad jk = i, \quad ki = j \\
ji &= -k, \quad kj = -i, \quad ik = -j \\
ijk &= -1
\end{align}

\subsubsection{Biquaternion Structure}

A general biquaternion $\Theta \in \mathcal{B}$ can be written as:

\begin{equation}
\label{eq:canonical:biquat_general}
\Theta = \Theta_0 + \Theta_1 i + \Theta_2 j + \Theta_3 k
\end{equation}

where $\Theta_a \in \mathbb{C}$ ($a = 0,1,2,3$) are complex-valued components.

Alternatively, in vector notation:
\begin{equation}
\Theta = \Theta_0 + \vec{\Theta} \cdot \vec{\sigma}
\end{equation}
where $\vec{\Theta} = (\Theta_1, \Theta_2, \Theta_3)$ and $\vec{\sigma} = (i, j, k)$.

\subsection{Biquaternion Operations}

\subsubsection{Addition}

Addition is component-wise:
\begin{equation}
\Theta + \Phi = (\Theta_0 + \Phi_0) + (\Theta_1 + \Phi_1)i + (\Theta_2 + \Phi_2)j + (\Theta_3 + \Phi_3)k
\end{equation}

\subsubsection{Multiplication}

Multiplication follows quaternion algebra rules with complex arithmetic on coefficients:

\begin{equation}
\Theta \cdot \Phi = \Theta_a \Phi_b \, \sigma_a \sigma_b
\end{equation}

where $\sigma_0 = 1$, $\sigma_1 = i$, $\sigma_2 = j$, $\sigma_3 = k$, and Einstein summation is implied.

Explicitly:
\begin{align}
\Theta \cdot \Phi = &(\Theta_0\Phi_0 - \Theta_1\Phi_1 - \Theta_2\Phi_2 - \Theta_3\Phi_3) \\
&+ (\Theta_0\Phi_1 + \Theta_1\Phi_0 + \Theta_2\Phi_3 - \Theta_3\Phi_2)i \\
&+ (\Theta_0\Phi_2 - \Theta_1\Phi_3 + \Theta_2\Phi_0 + \Theta_3\Phi_1)j \\
&+ (\Theta_0\Phi_3 + \Theta_1\Phi_2 - \Theta_2\Phi_1 + \Theta_3\Phi_0)k
\end{align}

\subsubsection{Conjugation}

The biquaternion conjugate combines quaternion and complex conjugation:

\textbf{Quaternion conjugate}:
\begin{equation}
\label{eq:canonical:quat_conj}
\Theta^\dagger = \Theta_0 - \Theta_1 i - \Theta_2 j - \Theta_3 k
\end{equation}

\textbf{Complex conjugate}:
\begin{equation}
\label{eq:canonical:complex_conj}
\bar{\Theta} = \bar{\Theta}_0 + \bar{\Theta}_1 i + \bar{\Theta}_2 j + \bar{\Theta}_3 k
\end{equation}

\textbf{Full biquaternion conjugate} (both operations):
\begin{equation}
\label{eq:canonical:biquat_conj}
\Theta^* = \bar{\Theta}^\dagger = \bar{\Theta}_0 - \bar{\Theta}_1 i - \bar{\Theta}_2 j - \bar{\Theta}_3 k
\end{equation}

\subsection{Norm and Inner Product}

\subsubsection{Biquaternion Norm}

The norm of a biquaternion is defined as:

\begin{equation}
\label{eq:canonical:biquat_norm}
|\Theta|^2 = \Theta^\dagger \Theta = \Theta \Theta^\dagger = |\Theta_0|^2 + |\Theta_1|^2 + |\Theta_2|^2 + |\Theta_3|^2
\end{equation}

This is a non-negative real number (sum of moduli squared of complex components).

\subsubsection{Inner Product}

For two biquaternions $\Theta$ and $\Phi$:

\begin{equation}
\label{eq:canonical:biquat_inner}
\langle \Theta, \Phi \rangle = \text{Re}(\Theta^\dagger \Phi) = \text{Re}(\Theta_0\bar{\Phi}_0 + \Theta_1\bar{\Phi}_1 + \Theta_2\bar{\Phi}_2 + \Theta_3\bar{\Phi}_3)
\end{equation}

\subsection{Matrix Representation}

Biquaternions can be represented as $2 \times 2$ complex matrices or $4 \times 4$ real matrices.

\subsubsection{Standard $2 \times 2$ Complex Representation}

\begin{equation}
\label{eq:canonical:biquat_2x2}
\Theta = \Theta_0 + \Theta_1 i + \Theta_2 j + \Theta_3 k \quad \leftrightarrow \quad
\begin{pmatrix}
\Theta_0 + i\Theta_1 & \Theta_2 + i\Theta_3 \\
-\Theta_2 + i\Theta_3 & \Theta_0 - i\Theta_1
\end{pmatrix}
\end{equation}

\subsubsection{Pauli Matrix Representation}

Using Pauli matrices $\{\sigma_0, \sigma_1, \sigma_2, \sigma_3\}$:

\begin{equation}
\Theta = \sum_{a=0}^{3} \Theta_a \sigma_a
\end{equation}

where:
\begin{equation}
\sigma_0 = \begin{pmatrix} 1 & 0 \\ 0 & 1 \end{pmatrix}, \quad
\sigma_1 = \begin{pmatrix} 0 & 1 \\ 1 & 0 \end{pmatrix}, \quad
\sigma_2 = \begin{pmatrix} 0 & -i \\ i & 0 \end{pmatrix}, \quad
\sigma_3 = \begin{pmatrix} 1 & 0 \\ 0 & -1 \end{pmatrix}
\end{equation}

\subsubsection{Extended $4 \times 4$ Representation}

For UBT field theory, we use the extended representation:

\begin{equation}
\Theta(q,T_B) \in \mathcal{B} \quad \leftrightarrow \quad \Theta \in \mathbb{C}^{4 \times 4}
\end{equation}

This provides 16 complex DOF = 32 real DOF, sufficient for encoding:
\begin{itemize}
    \item Spacetime metric (10 DOF)
    \item Electromagnetic field (6 DOF)
    \item Fermion matter content (remaining DOF)
\end{itemize}

\textbf{Important}: The matrix representation is a \emph{computational tool}. The canonical object is the biquaternion itself.

\subsection{Calculus on Biquaternions}

\subsubsection{Differentiation}

Partial derivatives act component-wise:

\begin{equation}
\partial_\mu \Theta = (\partial_\mu \Theta_0) + (\partial_\mu \Theta_1)i + (\partial_\mu \Theta_2)j + (\partial_\mu \Theta_3)k
\end{equation}

\subsubsection{Gradient and Divergence}

For a biquaternion field $\Theta(x)$:

\textbf{Gradient}:
\begin{equation}
\nabla \Theta = \sum_{\mu=0}^{3} \partial_\mu \Theta \, dx^\mu
\end{equation}

\textbf{D'Alembertian}:
\begin{equation}
\Box \Theta = \partial_\mu \partial^\mu \Theta = g^{\mu\nu} \partial_\mu \partial_\nu \Theta
\end{equation}

\subsection{Physical Significance}

In Unified Biquaternion Theory:

\begin{enumerate}
    \item \textbf{Fundamental field}: $\Theta(q, T_B) \in \mathcal{B}$ is the primary dynamical object
    
    \item \textbf{Coordinates}: $q \in \mathcal{B}$ (biquaternion space coordinates)
    
    \item \textbf{Time}: $T_B = t + i\psi + j\chi + k\xi \in \mathcal{B}$ (biquaternion time)
    
    \item \textbf{Metric emergence}: Biquaternionic metric $\mathcal{G}_{\mu\nu} \in \mathbb{B}$ emerges from the scalar part $\text{Sc}[(\partial_\mu\Theta)(\partial_\nu\Theta^\dagger)]$ (where $\text{Sc}$ denotes the scalar/real component of a biquaternion); the physical spacetime metric is $g_{\mu\nu} := \text{Re}(\mathcal{G}_{\mu\nu})$
    
    \item \textbf{Gauge structure}: Internal biquaternion structure generates $SU(2)$ gauge symmetry
    
    \item \textbf{Fermion content}: Spinor structure naturally embedded in quaternionic degrees of freedom
\end{enumerate}

\subsection{Relationship to Standard Algebras}

The biquaternion algebra $\mathcal{B} = \mathbb{H} \otimes \mathbb{C}$ sits in a hierarchy:

\begin{equation}
\boxed{
\begin{array}{c}
\mathbb{C} \subset \mathbb{H} \subset \mathcal{B} \\
\text{(complex)} \subset \text{(quaternions)} \subset \text{(biquaternions)}
\end{array}
}
\end{equation}

Special cases:
\begin{itemize}
    \item When $\Theta_1 = \Theta_2 = \Theta_3 = 0$: reduces to complex numbers
    \item When all $\Theta_a$ are real: reduces to quaternions
    \item Full structure needed for complete UBT
\end{itemize}

\subsection{Algebraic Properties}

\begin{enumerate}
    \item \textbf{Non-commutative}: $\Theta \Phi \neq \Phi \Theta$ in general
    
    \item \textbf{Associative}: $(\Theta \Phi) \Psi = \Theta (\Phi \Psi)$ for all $\Theta, \Phi, \Psi \in \mathcal{B}$
    
    \item \textbf{Division algebra}: Non-zero biquaternions have inverses
    \begin{equation}
    \Theta^{-1} = \frac{\Theta^\dagger}{|\Theta|^2}
    \end{equation}
    
    \item \textbf{Normed algebra}: $|\Theta \Phi| = |\Theta| |\Phi|$ (when properly defined)
\end{enumerate}

\begin{lemma}[Associativity of $\mathcal{B}$]
\label{lem:biquat_associativity}
The biquaternion algebra $\mathcal{B} = \mathbb{H} \otimes_{\mathbb{R}} \mathbb{C}$ is associative.
\end{lemma}

\begin{proof}[Proof sketch]
$\mathbb{H}$ is associative (the unit quaternions form a group under multiplication, and
the associativity of $\mathbb{H}$ is verified directly from the multiplication rules
$i^2=j^2=k^2=ijk=-1$).  $\mathbb{C}$ is a commutative associative field.  The tensor
product of two associative algebras over a commutative ring is again associative, because
$((\Theta_1 \otimes c_1)(\Theta_2 \otimes c_2))(\Theta_3 \otimes c_3)
= ((\Theta_1\Theta_2)\Theta_3) \otimes (c_1 c_2 c_3)
= \Theta_1(\Theta_2\Theta_3) \otimes (c_1 c_2 c_3)
= (\Theta_1 \otimes c_1)((\Theta_2 \otimes c_2)(\Theta_3 \otimes c_3))$,
where we used associativity of $\mathbb{H}$ and commutativity+associativity of $\mathbb{C}$.
Alternatively, the $2\times2$ complex matrix representation (Eq.~\ref{eq:canonical:biquat_2x2}) is
a ring homomorphism into $M_2(\mathbb{C})$, and matrix multiplication is associative.
\end{proof}

\subsection{Connection to Spin and SU(2)}

The quaternion part of biquaternions is isomorphic to SU(2):

\begin{equation}
\text{Unit quaternions} \cong SU(2) \cong \text{Spin}(3)
\end{equation}

This provides the natural mechanism for:
\begin{itemize}
    \item Fermion spinor structure
    \item Weak isospin $SU(2)_L$
    \item Spin-gravity coupling
\end{itemize}

\textbf{This completes the mathematical foundation for all UBT constructions.}
