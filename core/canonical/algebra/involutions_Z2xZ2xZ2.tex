% © 2025 Ing. David Jaroš — CC BY-NC-ND 4.0
%
% This work is licensed under a Creative Commons Attribution-NonCommercial-NoDerivatives
% 4.0 International License (CC BY-NC-ND 4.0).
%
% License History: Earlier drafts (up to v0.3) were released under CC BY 4.0.
% From v0.4 onward, all material is released under CC BY-NC-ND 4.0 to protect
% the integrity of the theoretical work during ongoing academic development.
%
% Track A: SU(3) Candidate via Z2 x Z2 x Z2 Involutions on B = C tensor H
% Status: Candidate construction, conservative framing.
% Version: 1.0
% Date: 2026-02-28

\section{$\mathbb{Z}_2^3$ Involution Decomposition of $\mathcal{B}$}
\label{sec:algebra:involutions_z2z2z2}

\subsection{Setup}

Let $\mathcal{B} = \mathbb{C} \otimes \mathbb{H}$ be the biquaternion algebra,
viewed as an $8$-dimensional real vector space with ordered basis
\begin{equation}
  \mathcal{B}_\mathrm{basis} = \{1,\, \mathbf{I},\, \mathbf{J},\, \mathbf{K},\,
    i,\, i\mathbf{I},\, i\mathbf{J},\, i\mathbf{K}\},
  \label{eq:inv:basis}
\end{equation}
where $i \in \mathbb{C}$ denotes the complex imaginary unit
($i^2 = -1$, central in $\mathcal{B}$) and
$\mathbf{I}, \mathbf{J}, \mathbf{K} \in \mathbb{H}$ are the quaternion units
satisfying $\mathbf{I}^2 = \mathbf{J}^2 = \mathbf{K}^2 = -1$,
$\mathbf{I}\mathbf{J} = \mathbf{K}$, $\mathbf{J}\mathbf{K} = \mathbf{I}$,
$\mathbf{K}\mathbf{I} = \mathbf{J}$.

\subsection{Three Commuting Involutions}

We define three $\mathbb{R}$-linear maps $P_1, P_2, P_3 : \mathcal{B} \to \mathcal{B}$:

\begin{itemize}
  \item \textbf{$P_1$ (complex conjugation):}
    $P_1(i) = -i$,\; $P_1(\mathbf{I}) = \mathbf{I}$,\;
    $P_1(\mathbf{J}) = \mathbf{J}$,\; $P_1(\mathbf{K}) = \mathbf{K}$.
    Extends by $\mathbb{R}$-linearity: $P_1(i\mathbf{X}) = -i\mathbf{X}$
    for $\mathbf{X} \in \{1, \mathbf{I}, \mathbf{J}, \mathbf{K}\}$.

  \item \textbf{$P_2$ (quaternion conjugation):}
    $P_2(\mathbf{I}) = -\mathbf{I}$,\;
    $P_2(\mathbf{J}) = -\mathbf{J}$,\;
    $P_2(\mathbf{K}) = -\mathbf{K}$,\; $P_2(i) = i$.
    Extends: $P_2(i\mathbf{X}) = i\cdot P_2(\mathbf{X})$.

  \item \textbf{$P_3$ (axis-flip around $\mathbf{I}$):}
    $P_3(\mathbf{I}) = \mathbf{I}$,\;
    $P_3(\mathbf{J}) = -\mathbf{J}$,\;
    $P_3(\mathbf{K}) = -\mathbf{K}$,\; $P_3(i) = i$.
    Equivalently $P_3(x) = \mathbf{I}\, x\, \mathbf{I}^{-1}$ (inner automorphism by $\mathbf{I}$).
\end{itemize}

\begin{lemma}[Involutions]
  \label{lem:involutions}
  Each of $P_1, P_2, P_3$ is an $\mathbb{R}$-linear involution: $P_k^2 = \mathrm{id}$.
\end{lemma}

\begin{proof}
  On every basis element $b \in \mathcal{B}_\mathrm{basis}$, $P_k$ acts by a
  scalar eigenvalue $s_k \in \{+1, -1\}$:
  \[
    P_k(b) = s_k\, b \;\implies\; P_k^2(b) = P_k(s_k b) = s_k\, P_k(b)
    = s_k^2\, b = b,
  \]
  since $s_k^2 = 1$. By $\mathbb{R}$-linearity, $P_k^2 = \mathrm{id}$ on
  all of $\mathcal{B}$.
\end{proof}

\begin{lemma}[Pairwise commutativity]
  \label{lem:commutativity}
  $[P_i, P_j] = 0$ for all $1 \le i < j \le 3$.
\end{lemma}

\begin{proof}
  For any basis element $b$ with eigenvalues $s_i$ and $s_j$:
  \[
    P_i(P_j(b)) = P_i(s_j b) = s_j\, s_i\, b = s_i\, s_j\, b = P_j(P_i(b)).
  \]
  By linearity the result extends to all of $\mathcal{B}$.
\end{proof}

\subsection{Signature Table}

The \emph{signature} of a basis element $b$ is
$\sigma(b) = (s_1, s_2, s_3) \in \{+1,-1\}^3$ where $P_k(b) = s_k b$.

\begin{table}[h]
  \centering
  \caption{Eigenvalue signatures of basis elements under $P_1, P_2, P_3$.}
  \label{tab:inv:signatures}
  \begin{tabular}{ccccc}
    \hline
    Element & $s_1$ & $s_2$ & $s_3$ & Sector \\
    \hline
    $1$         & $+$ & $+$ & $+$ & $(+,+,+)$ \\
    $\mathbf{I}$ & $+$ & $-$ & $+$ & $(+,-,+)$ \\
    $\mathbf{J}$ & $+$ & $-$ & $-$ & $(+,-,-)$ \\
    $\mathbf{K}$ & $+$ & $-$ & $-$ & $(+,-,-)$ \\
    $i$          & $-$ & $+$ & $+$ & $(-,+,+)$ \\
    $i\mathbf{I}$ & $-$ & $-$ & $+$ & $(-,-,+)$ \\
    $i\mathbf{J}$ & $-$ & $-$ & $-$ & $(-,-,-)$ \\
    $i\mathbf{K}$ & $-$ & $-$ & $-$ & $(-,-,-)$ \\
    \hline
  \end{tabular}
\end{table}

\subsection{Projectors and Sector Decomposition}

The projector onto sector $(s_1, s_2, s_3)$ is
\begin{equation}
  \Pi_{(s_1 s_2 s_3)} = \frac{1}{8}
  \prod_{k=1}^{3}\bigl(1 + s_k P_k\bigr).
  \label{eq:inv:projector}
\end{equation}
These satisfy $\Pi^2 = \Pi$ and $\sum \Pi_{(s_1 s_2 s_3)} = \mathrm{id}$.

\subsection{Single-Minus Sectors}
\label{subsec:inv:single_minus}

The three \emph{single-minus} sectors (exactly one negative eigenvalue) are:
\begin{align}
  S_1 &= (-,+,+): & \Pi_{S_1}\mathcal{B} &= \mathrm{span}_\mathbb{R}\{i\},
    & \dim_\mathbb{R} &= 1. \\
  S_2 &= (+,-,+): & \Pi_{S_2}\mathcal{B} &= \mathrm{span}_\mathbb{R}\{\mathbf{I}\},
    & \dim_\mathbb{R} &= 1. \\
  S_3 &= (+,+,-): & \Pi_{S_3}\mathcal{B} &= \{0\},
    & \dim_\mathbb{R} &= 0.
  \label{eq:inv:S3_empty}
\end{align}

\begin{remark}[Emptiness of $S_3$]
  The sector $S_3 = (+,+,-)$ is empty.
  Any element with $P_1 = +1$ (no complex-$i$ factor) and $P_2 = +1$
  (no quaternion-unit factor) must be a real scalar multiple of $1$.
  Since $P_3(1) = +1$ for any algebra map fixing the unit, such an element
  lies in the $(+,+,+)$ sector, not $(+,+,-)$.
  This is a structural consequence of the definitions of $P_1$ and $P_2$,
  and is stated here explicitly as part of the conservative candidate framing.
\end{remark}

\subsection{Candidate 3D Complex Carrier Space}
\label{subsec:inv:carrier_space}

\begin{definition}[Θ-triplet carrier space]
  \label{def:inv:Vc}
  The \emph{candidate triplet carrier space} is the $P_2 = -1$ eigenspace of
  $P_2$ acting on $\mathcal{B}$:
  \begin{equation}
    V_c := \mathrm{span}_\mathbb{C}\{\mathbf{I}, \mathbf{J}, \mathbf{K}\}
         = \{a\mathbf{I} + b\mathbf{J} + c\mathbf{K} \mid a, b, c \in \mathbb{C}\},
    \label{eq:inv:Vc}
  \end{equation}
  where $\mathbb{C}$ acts by left-multiplication by $i$.
\end{definition}

\begin{lemma}[$V_c$ is a complex 3-dimensional space]
  \label{lem:Vc_dim}
  $\dim_\mathbb{C} V_c = 3$.
\end{lemma}

\begin{proof}
  The six real vectors $\{\mathbf{I}, i\mathbf{I}, \mathbf{J}, i\mathbf{J},
  \mathbf{K}, i\mathbf{K}\}$ are distinct basis elements of
  $\mathcal{B}_\mathrm{basis}$, hence $\mathbb{R}$-linearly independent.
  Every element of $V_c$ is of the form $a\mathbf{I} + b\mathbf{J} + c\mathbf{K}$
  with $a, b, c \in \mathbb{C}$, so $\{\mathbf{I}, \mathbf{J}, \mathbf{K}\}$
  is a $\mathbb{C}$-basis and $\dim_\mathbb{C} V_c = 3$.
\end{proof}

The action of $P_3$ on $V_c$ is a $\mathbb{C}$-linear involution with
eigenspaces $\mathrm{span}_\mathbb{C}\{\mathbf{I}\}$ (eigenvalue $+1$) and
$\mathrm{span}_\mathbb{C}\{\mathbf{J}, \mathbf{K}\}$ (eigenvalue $-1$).
The $S_2 = (+,-,+)$ sector corresponds to the real slice of the
$P_3 = +1$ sub-eigenspace of $V_c$.

\subsection{SU(3) Candidate}
\label{subsec:inv:su3}

Define the Hermitian inner product on $V_c$ by
\begin{equation}
  \langle X, Y \rangle := \tfrac{1}{4}\,\mathrm{Tr}(X^\dagger Y),
  \quad X, Y \in V_c,
  \label{eq:inv:inner_product}
\end{equation}
where $X^\dagger$ denotes the biquaternionic adjoint (composed $P_1 \circ P_2$)
and $\mathrm{Tr}$ is the reduced trace.  Under the identification
$V_c \cong \mathbb{C}^3$ via $\{\mathbf{I}, \mathbf{J}, \mathbf{K}\}
\leftrightarrow \{e_1, e_2, e_3\}$, this reduces to the standard Hermitian
inner product on $\mathbb{C}^3$.

\begin{definition}[SU(3) candidate]
  \label{def:inv:SU3}
  The \emph{Track-A SU(3) candidate} is
  \begin{equation}
    \mathrm{SU}(3)_{V_c} := \bigl\{
      U \in \mathrm{GL}(V_c)\;\big|\;
      U\text{ is }\mathbb{C}\text{-linear},\;
      \langle Ux, Uy\rangle = \langle x, y\rangle
      \;\forall\, x,y \in V_c,\;
      \det_\mathbb{C}(U) = 1
    \bigr\}.
  \end{equation}
\end{definition}

This is isomorphic to the standard $\mathrm{SU}(3)$ and acts naturally on
the Θ-triplet extracted from $\mathcal{B}$ by the $P_2$ involution.

\begin{remark}[Conservative framing]
  \label{rem:conservative}
  The group $\mathrm{SU}(3)_{V_c}$ constructed above acts on the subspace
  $V_c \subset \mathcal{B}$; it is \emph{not} a subgroup of
  $\mathrm{Aut}(\mathcal{B})$.  The automorphism group
  $\mathrm{Aut}(\mathbb{C}\otimes\mathbb{H})
  \cong [\mathrm{GL}(2,\mathbb{C})\times\mathrm{GL}(2,\mathbb{C})]/\mathbb{Z}_2$
  does not contain $\mathrm{SU}(3)$ (see \texttt{reports/associative\_su3\_scan.md}).
  The present construction is labelled \emph{Track A, candidate construction},
  and claims only that a natural $\mathrm{SU}(3)$ symmetry acts on the
  Θ-triplet subspace $V_c$ extracted from $\mathcal{B}$.
  No octonions or non-associative structures appear.
\end{remark}
