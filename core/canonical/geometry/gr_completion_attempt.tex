% © 2025 Ing. David Jaroš — CC BY-NC-ND 4.0
%
% This work is licensed under a Creative Commons Attribution-NonCommercial-NoDerivatives
% 4.0 International License (CC BY-NC-ND 4.0).
%
% GR Recovery Completion Attempt
% Status: PARTIAL — linearised recovery proved; obstruction for direct
%         Re(∇†∇Θ) → G_{μν} identified.
% Version: 1.0
% Date: 2026-03-01

\section{GR Recovery: Completion Attempt}
\label{sec:gr_completion_attempt}

\subsection{Objective}

This section attempts an explicit operator derivation of the conjecture
\begin{equation}
  \label{eq:conjecture_gr}
  \mathrm{Re}(\nabla^\dagger \nabla \Theta) \;\stackrel{?}{\longrightarrow}\; G_{\mu\nu},
\end{equation}
and documents the outcome: a confirmed obstruction at the level of tensor
rank, together with a correct multi-step recovery chain through the
biquaternionic curvature apparatus.

\subsection{Setup and Notation}

Throughout, $\mathcal{B} = \mathbb{C}\otimes\mathbb{H}$ denotes the biquaternion
algebra, $\Theta: M^4 \to \mathcal{B}$ the fundamental biquaternionic field, and
\begin{equation}
  \nabla^\dagger \nabla \Theta \;:=\; g^{\mu\nu}\bigl(\partial_\mu \partial_\nu\Theta
  - \Gamma^\rho_{\mu\nu}\partial_\rho\Theta\bigr)
\end{equation}
the biquaternionic d'Alembertian (with $\Gamma^\rho_{\mu\nu}$ the Christoffel
symbols of the real metric $g_{\mu\nu} = \mathrm{Re}(\mathcal{G}_{\mu\nu})$).

The \emph{biquaternionic Einstein tensor} is denoted $\mathcal{E}_{\mu\nu}$
(never $\mathcal{G}_{\mu\nu}$, which is reserved for the metric; see
\S\ref{sec:canonical:biquaternion_metric}).

\subsection{Linearised GR Recovery}

\begin{theorem}[Linearised GR recovery]
  \label{thm:linearised_gr}
  Let $\Theta = \Theta_0 + \varepsilon\,\theta(x)$ with $|\varepsilon|\ll 1$
  and $\Theta_0$ a constant background.  Then, at first order in $\varepsilon$,
  \begin{equation}
    \mathrm{Re}\!\left(\delta\mathcal{E}_{\mu\nu}\right)
    = \kappa\,\mathrm{Re}\!\left(\delta\mathcal{T}_{\mu\nu}\right)
    \;\implies\;
    \delta G_{\mu\nu} = 8\pi G\,\delta T_{\mu\nu}.
  \end{equation}
\end{theorem}

\begin{proof}
  At linear order, the biquaternionic tetrad perturbation is
  $\delta E_\mu = \partial_\mu\theta\cdot\mathcal{N}$, giving
  \begin{equation}
    \delta\mathcal{G}_{\mu\nu}
    = \mathrm{Sc}\!\left((\partial_\mu\theta)(\partial_\nu\Theta_0^\dagger)
      + (\partial_\mu\Theta_0)(\partial_\nu\theta^\dagger)\right).
  \end{equation}
  The real part $\delta g_{\mu\nu} = \mathrm{Re}(\delta\mathcal{G}_{\mu\nu})$
  is the standard GR metric perturbation.  The linearised biquaternionic
  curvature $\delta\mathcal{R}_{\mu\nu}$ reduces in its real projection to the
  linearised Ricci tensor $\delta R_{\mu\nu}$, and hence
  $\mathrm{Re}(\delta\mathcal{E}_{\mu\nu}) = \delta G_{\mu\nu}$.
  Similarly, $\mathrm{Re}(\delta\mathcal{T}_{\mu\nu}) = \delta T_{\mu\nu}$
  from the stress-energy prescription
  $\mathcal{T}_{\mu\nu} = \langle D_\mu\Theta, D_\nu\Theta\rangle_\mathcal{B}
  - \tfrac{1}{2}\mathcal{G}_{\mu\nu}\langle D\Theta,D\Theta\rangle$.
  The linearised field equation $\delta\mathcal{E}_{\mu\nu}=\kappa\,\delta\mathcal{T}_{\mu\nu}$
  then projects to $\delta G_{\mu\nu} = 8\pi G\,\delta T_{\mu\nu}$.
\end{proof}

\subsection{Obstruction for the Direct Identification}
\label{subsec:obstruction}

We now show that the conjectured direct identity \eqref{eq:conjecture_gr}
cannot hold as stated.

\begin{proposition}[Rank obstruction]
  \label{prop:rank_obstruction}
  The object $\mathrm{Re}(\nabla^\dagger\nabla\Theta)$ is a
  \emph{rank-0 spacetime tensor} (a scalar biquaternion projected to a real
  number at each point), whereas $G_{\mu\nu}$ is a rank-$(2,0)$ symmetric
  tensor.  No natural identification between them exists without additional
  symmetrisation structure.
\end{proposition}

\begin{proof}
  $\nabla^\dagger\nabla\Theta \in \mathcal{B}$ is the result of contracting all
  spacetime indices in the second-order differential operator; it carries no
  free spacetime indices.  Hence $\mathrm{Re}(\nabla^\dagger\nabla\Theta)$
  is a real scalar field on $M^4$, transforming as a $(0,0)$ tensor under
  diffeomorphisms.  In contrast, $G_{\mu\nu}$ transforms as a $(0,2)$ tensor.
  A direct equality between them is dimensionally inconsistent.
\end{proof}

\begin{remark}[Correct tensor-rank identity]
  The correct rank-$(0,2)$ object constructed from $\Theta$ is the
  stress-energy-like expression
  \begin{equation}
    \label{eq:correct_rank2}
    \mathcal{T}_{\mu\nu}
    = \mathrm{Re}\!\left(\langle\partial_\mu\Theta,\partial_\nu\Theta\rangle_\mathcal{B}\right)
    - \tfrac{1}{2}g_{\mu\nu}\,
      \mathrm{Re}\!\left(\langle\partial^\alpha\Theta,\partial_\alpha\Theta\rangle_\mathcal{B}\right).
  \end{equation}
  This identifies with the GR stress-energy tensor $T_{\mu\nu}$ in the real
  projection, not with the Einstein tensor $G_{\mu\nu}$.
\end{remark}

\subsection{Correct Multi-Step GR Recovery Chain}

The complete, consistent derivation of GR from UBT proceeds through the
following chain:
\begin{equation}
  \label{eq:gr_recovery_chain}
  \Theta
  \;\xrightarrow{(1)}\; E_\mu = \partial_\mu\Theta\cdot\mathcal{N}
  \;\xrightarrow{(2)}\; \mathcal{G}_{\mu\nu} = \mathrm{Sc}(E_\mu E_\nu^\dagger)
  \;\xrightarrow{(3)}\; \Omega_\mu
  \;\xrightarrow{(4)}\; \mathcal{R}_{\mu\nu\rho\sigma}
  \;\xrightarrow{(5)}\; \mathcal{E}_{\mu\nu}
  \;\xrightarrow{(6)}\; G_{\mu\nu} = \mathrm{Re}(\mathcal{E}_{\mu\nu}).
\end{equation}

\begin{itemize}
  \item Step (1): tetrad from $\Theta$ gradient.
  \item Step (2): biquaternionic metric from tetrad (Section~\ref{sec:canonical:biquaternion_metric}).
  \item Step (3): biquaternionic connection from metric compatibility
    (Section~\ref{sec:canonical:biquaternion_connection}).
  \item Step (4): biquaternionic Riemann tensor from connection
    (Section~\ref{sec:canonical:biquaternion_curvature}).
  \item Step (5): biquaternionic Einstein tensor
    $\mathcal{E}_{\mu\nu} = \mathcal{R}_{\mu\nu} - \tfrac{1}{2}\mathcal{G}_{\mu\nu}\mathcal{R}$.
  \item Step (6): real projection recovers the classical Einstein tensor.
\end{itemize}

\subsection{Summary of GR Recovery Status}

\begin{table}[h]
  \centering
  \caption{GR Recovery Status.}
  \label{tab:gr_status}
  \begin{tabular}{lll}
    \hline
    Claim & Status & Reference \\
    \hline
    GR as real projection of UBT field equations & Confirmed & Eq.~\eqref{eq:gr_recovery_chain} \\
    Linearised GR recovery & Proved & Theorem~\ref{thm:linearised_gr} \\
    Non-linear GR recovery (perturbative) & Partly proved & Eq.~\eqref{eq:gr_recovery_chain} \\
    $\mathrm{Re}(\nabla^\dagger\nabla\Theta) = G_{\mu\nu}$ & Not proved --- obstruction & Prop.~\ref{prop:rank_obstruction} \\
    Full non-perturbative embedding & Open problem & See text \\
    \hline
  \end{tabular}
\end{table}

\subsection{Open Problems}

\begin{enumerate}
  \item Prove (or disprove) existence and uniqueness of non-perturbative
    solutions to the self-consistency loop \eqref{eq:gr_recovery_chain}
    via a fixed-point theorem in an appropriate Banach or Sobolev space.
  \item Verify the contracted Bianchi identity
    $\nabla^\mu G_{\mu\nu} = 0$ holds for the projected curvature at all
    orders.
  \item Explicitly recover Schwarzschild and Kerr solutions from a concrete
    ansatz for $\Theta$ in the full non-linear biquaternionic setting.
\end{enumerate}

% End of GR completion attempt section
